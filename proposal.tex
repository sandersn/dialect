\documentclass[11pt]{article}
% \usepackage{setspace}
\usepackage[all]{xy}
\usepackage{robbib}

\author{Nathan Sanders, Indiana University \\ \tt{ncsander@indiana.edu}}
\title{Syntax Distance for Dialectometry}
\begin{document}
% \doublespacing
\maketitle

% This proposal proposes to continue my existing line of research on
% dialect distance using syntactic features. My previous research has
% focussed on statistical measures as pioneered by \namecite{nerbonne06}
% and extended their features to better capture complex syntactic
% structure.

\section{Introduction}

A reworked introduction will start by identifying quickly what is
dialectometry and a little of the past history, in support of
explaining the research questions of the field. Then it will identify
clearly the questions of my previous research (from the qualifying
paper, I guess) and what I did to answer them.

Then I will explain how the dissertation will address the problems with
the qualifying paper's method and results, and what questions it will
address in addition (if any).

Finally I will identify the hypotheses that I will be testing to
answer these questions.

\subsection{Problems}

Syntax is a relatively undeveloped area in dialectometry. This
dissertation will address some holes in syntactic dialectometry. The
biggest lack in syntactic dialectometry is that there is no generally
accepted syntax measure. In phonology, Levenshtein distance is widely
accepted as the standard against which new phonological distance
measures are measured, and it has been shown to agree well with
linguistic analyses from dialectology \cite{heeringa04}. There is no
equivalent for syntax.

Additionally, analysis of syntactic data poses different challenges
than that of phonological data. Most available phonological data is from
small-scale, carefully collected dialectology corpora. Some such
data is available for syntax; for example \namecite{spruit08} analysed
the Syntactic atlas of the Dutch dialects (TODO:Cite) using the content-agnostic
methods developed by \namecite{goebl06}. However, for most
languages large, unstructured syntactic corpora are the norm;
extraction of individual syntactic features from carefully elicited
sentences is not possible. This means that a syntactic measure for
dialectometry must be able to extract features from an unstructured
corpus: this implies a statistical method. % probabilistic ?

One such method has been proposed by \namecite{nerbonne06}, a simple
statistical measure called $R$, based on work by
\namecite{kessler01}. At present, however, $R$ has not been adequately
shown to detect dialect differences. A small body of work suggests
that it is noticing dialect differences, but as yet there has not been
a satisfying correlation of its results with phonology or, more
importantly, with existing results from the dialectelogy
literature. (This is what really legitimised Leveshtein distance;
correlation of its major boundaries with existing phonological dialect
maps).

The initial paper using $R$ for syntax distance was
\namecite{nerbonne06}. Their experiment compared two generations of
Norwegian immigrants \ldots blah blah. I'm not sure all this stuff
goes here. Anyway, I can get this from elsewhere in the paper if
needed.


\begin{itemize}
\item There is no generally accepted syntax measure in dialectometry.
\item There is no generally accepted syntax measure that works on
  large, unstructured corpora that are common in computational linguistics.
\item R is a syntax measure, but we don't know how well it works.
\item Previous experiments have tried to look at syntax distance with
  respect to existing dialectometry but have failed in one way or another.
\item \namecite{nerbonne06} compared only two populations, and not
  typical dialect populations.
\item \namecite{spruit08} compares syntax using a small dialectology corpus
  of syntactic features, analysed by Goebl's generic distance
  measures.
\item \namecite{sanders08b} compares syntax and phonology of typical
  dialect populations.
\item But the phonology and syntax corpora differ substantially.
\item The informants for the syntax corpus, ICE-GB, are not well distributed across
  England.
\end{itemize}

\subsection{Questions}

The state of syntax measures in dialectometry described above leads to
several research questions. The primary question is whether $R$ is a
sufficient measure of syntax distance. Specifically, have the
ambiguous results of previous research been a shortcoming of $R$, a
disagreement between corpora, or a disagreement between syntactic and
phonological dialect boundaries? This research will eliminate the
corpus variability in \namecite{sanders08b} that confuses those
results.
% If this research eliminates the other causes as an explanation, a new
% syntactic distance measure will need to be developed and tested.

A secondary question, relevant once $R$ is established as a useful
syntax distance measure, is what features provide the best
results. Previous work has shown that leaf-ancestor paths provide a
small advantage over part-of-speech trigrams, presumably by capturing
syntactic structure higher in the parse tree. Additional possible
feature sets include variations on the previously investigated
trigrams and leaf-ancestor paths: various kinds of backoff, such
as coarser node tags or bigrams. Other ways of capturing syntactic
structure may be useful too, such as dependency parses or
supertags. (Should CITE at least supertags because I think they are
not universally understood)

% \begin{itemize}
% \item Is R sufficient?
%   \begin{itemize}
%   \item If not, what should replace it?
%   \item In other words, is there some other measure that can compare
%     arbitrary large corpora without reducing to R?
%     \item Have shortcomings/ambiguities/lack of correlation in the
%       results of previous studies been a result of R or other factors?
%   \end{itemize}
% \item Do different features improve results?
%   \begin{itemize}
%   \item POS trigrams
%   \item POS trigrams with backoff (not likely, but possible)
%   \item leaf-ancestor paths
%   \item dependencies
%   \item supertags
%   \end{itemize}
% \item Is dialectometry on syntax even possible? Worthwhile? What about
%   dialectology as a whole?
% \end{itemize}

\subsection{Hypothesis}
In some styles, this comes between the literature review and the
experiment proposal. I'm not sure about this paper, though; Sandra
wants a better separation between `done' and `to do', but I'm not sure
this is the way to do it.

Given a syntactically annotated corpus collected for dialectology, R
will produce accurate syntax distances between subcorpora. Properties
important to dialectology, such as wide geographic coverage and
variety in the sociological variables of informants, will provide
better results than corpora without these properties.

Given certain features, R will produce more accurate syntax distances
for a given set of subcorpora. These features can be discovered by
comparing their performance on a fixed corpus.

A phonology corpus and syntax corpus based on the same data will
provide better correlation between phonology and syntax distance
measures than a phonology corpus and syntax corpus based on separate
sets of data. This type of corpus will provide the best opportunity to
discover whether phonological and syntactic dialect boundaries
correlate. (Or perhaps whether phonological and syntactic dialect
measures correlate? I'm not sure of this yet)

\subsection{Previous}
% Overview : Goal, Variables, Method
%   Contribution
% Literature Review
%   : (includng theoretical background)
%   Draw hypotheses from earlier studies
% Method
%   :
%   Experiment section as 'Corpus' section

% Goal: To extend existing measurement methods. To measure them
% better. To measure them on more complete data.

% TODO Henrik Rosenkvist seems to
% be the main guy interested in syntactic analysis of dialect distance
% TODO Add citations of some of Marco Rene Spruit's classic/Goebl
% stuff

This dissertation will examine syntax distance in dialectometry using
computational methods as a basis. It is a continuation of my previous
work \cite{sanders07}, \cite{sanders08b} and earlier work by
\namecite{nerbonne06}, the first computational measure of syntax
distance. Dialectometry has existed as a field since
\namecite{seguy73} and is a sub-field of dialectology
\cite{chambers92}; recently, computational methods have come to
dominate dialectometry, but they are limited in focus compared to
previous work; most have explored phonological distance only, while
earlier methods integrated phonological, lexical, and syntactic data.

Despite the flexibility of pre-computational methods, the isogloss
bundles that were their primary product do not adequately capture gradient
generalizations; only boundaries that have near-complete agreement can
be bundled, so these methods missed many generalisations in the data.
In contrast, recent methods are mathematically
sophisticated, capable of identifying dialect gradients. The purpose
of this work is to extend the application of computational syntax
distance to areas that have previously only been analysed by
traditional, non-statistical dialectometrical methods.

Previous work has investigated statistical syntax distance in two
corpora: different generations of Norwegian L2 speakers of English
\cite{nerbonne06}, and the ICE corpus \cite{sanders07},
\cite{sanders08b}, consisting of native English
speakers born in various parts of England \cite{nelson02}.
Nerbonne \& Wiersma used trigrams as features to the syntax distance
classifier; Sanders also used leaf-ancestor paths \cite{sampson00}.
Both Nerbonne \& Wiersma and Sanders used a
permutation test based on Kessler's $R$ \cite{kessler01}. Sanders'
work applied the syntax measure to multiple areas by dividing the
ICE by speaker birthplace. This dissertation will be the first to use
this method on a corpus collected explicitly for syntactic analysis of
dialects. It will also be the first to use dependency structure as
input the syntax distance classifier. Dependency structure is well
suited to Swedish because of its word order, which is freer than
English word order. Its simplicity relative to phrase structure also
eliminates some of the complications present in building leaf-ancestor
paths.

The dialect corpus used will be the SweDiaSyn.
The SweDiaSyn is a transcription of the SweDia 2000, a corpus of
interviews collected throughout Sweden from 1998 to 2000. At each
site, interviewees were one each of a younger man, a younger woman, an
older man and an older woman. Though it does not currently contain
trancriptions of the entire SweDia 2000, the SweDiaSyn will
contain a complete orthographic transcription of the interviews to
standard Swedish and a phonetic transcription where the dialect
differs substantially from standard Swedish. The completed section
currently focusses on older speakers.

Other research is being conducted on SweDiaSyn. Results from traditional
syntactical dialectology will provide a comparison with those from
computational dialectology presented in this dissertation.
% TODO: Cite Rosenkvist? I can't find anything in English and the
% Swedish titles don't seem to be especially indicative of use of
% SweDiaSyn. Maybe I will have to contact him directly.


\section{Previous Work}

\subsection{S\'eguy}

Measurement of linguistic similarity has always been a part of
linguistics. However, until \namecite{seguy73} dubbed a new set of
approaches `dialectometry', these methods lagged behind the rest of
linguistics in formality. S\'eguy's quantitative analysis
of Gascogne French, while not aided by computer, was the predecessor
of more powerful statistical methods that essentially required the use
of computer as well as establishing the field's general dependence on
well-crafted dialect surveys that divide incoming data along
traditional linguistic boundaries: phonology, morphology, syntax, etc.
This makes both collection and analysis easier, although it requires
more work to combine separate analyses to produce a complete picture of dialect
variation.

The project to build the Atlas Linguistique et Ethnographique de la
Gascogne, which S\'eguy directed, collected data in a dialect survey
of Gascogne which asked speakers questions informed by different areas
of linguistics. For example, the pronunciation of `dog' ({\it chien})
was collected to measure phonological variation. It had two common
variants and many other rare ones: [k\~an], [k\~a], as well as [ka],
[ko], [kano], among others. These variants were, for the most part,
% or hat "chapeau": SapEu, kapEt, kapEu (SapE, SapEl, kapEl
known by linguists ahead of time, but their exact geographical
distribution was not.

The atlases, as eventually published, contained not only annotated
maps, but some analysis as well. This analysis was what S\'eguy named
dialectometry. Dialectometry differs from previous attempts to find
dialect boundaries in the way it combines information from the
dialect survey. Previously, dialectologists found isogloss
boundaries for individual items. A dialect boundary was generated when
enough individual isogloss boundaries coincided. However, for any real
corpus, there is so
much individual variation that only major dialect boundaries can
be captured this way.

S\'eguy reverses the process. He first combines survey data to get
a numeric score between each site. Then he posited dialect boundaries
where large distances resulted between sites. The difference is
important, because a single numeric score is easier to
analyze than hundreds of individual boundaries.
Much more subtle dialect boundaries are visible this way; where before
one saw only a jumble of conflicting boundary lines, now one sees
smaller, but consistent, numerical differences separating regions. {Dialectometry
  enables classification of gradient dialect boundaries, since now one
can distinguish weak and strong boundaries. Previously, weak
boundaries were too uncertain.}

However, S\'eguy's method of combination is simple both
linguistically and mathematically. When comparing two sites, any
difference in a response is counted as 1. Only identical
responses count as a distance of 0. Words are not analyzed
phonologically, nor are responses weighted by their relative amount
of variation. Finally, only geographically adjacent sites are
compared. This is a reasonable restriction, but later studies were
able to lift it because of the availability of greater computational
power. Work following S\'eguy's improves on both aspects. In
particular, Hans Goebl developed dialectometry models that are
more mathematically sophisticated.

\subsection{Goebl}

Hans Goebl emerged as a leader in the field of dialectometry,
formalizing the aims and methods of dialectometry. His primary
contribution was development of various methods to combine individual
distances into global distances and global distances into global clusters. These
methods were more sophisticated mathematically than previous
dialectometry and operated on any features extracted from the data. His
analyses have used primarily the Atlas Linguistique de Fran\c{c}ais.

\namecite{goebl06} provides a summary of his work. Most relevant for
this paper are the measures Relative Identity Value and Weighted
Identity Value. They are general methods that are the basis for nearly
all subsequent fine-grained dialectometrical analyses. They have three
important properties. First, they are independent of the source
data. They can operate over any linguistic data for which they are
given a feature set, such as the one proposed by \namecite{gersic71} for
phonology. Second, they can compare data even for items that do not
have identical feature sets, such as Ger\v{s}i\'c's $d$,
which cannot compare consonants and vowels. Third, they can compare
data sets that are missing some entries. This improves on S\'eguy's
analysis by providing a principled way to handle missing survey
responses.

Relative Identity Value, when comparing any two items, counts the
number of features which share the same value and then discounts
(lowers) the importance of the result by the number of unshared
features. The result is a single percentage that indicates
relative similarity. Calculating this distance between all pairs
of items in two regions produces a matrix which can be used for
clustering or other purposes. Note that the presentation below splits
Goebl's original equations into more manageable pieces; the high-level
equation for Relative Identity Value is:

\begin{equation}
  \frac{\textrm{identical}_{jk}} {\textrm{identical}_{jk} - \textrm{unidentical}_{jk}}
\label{riv}
\end{equation}
For some items being compared $j$ and $k$. In this case
\textit{identical} is
\begin{equation}
  \textrm{identical}_{jk} = |f \in \textrm{\~N}_{jk} : f_j = f_k|
\end{equation}
where $\textrm{\~N}_{jk}$ is the set of features shared by  $j$ and
$k$ and $f_j$ and $f_k$ are the value of some feature $f$ for $j$ and
$k$ respectively. \textit{unidentical} is defined similarly, except
that it counts all features N, not just the shared features
$\textrm{\~N}_{jk}$.

\begin{equation}
  \textrm{unidentical}_{jk} = |f \in \textrm{N} : f_j \neq f_k|
\end{equation}

Weighted Identity Value is a refinement of Relative Identity
Value. This measure defines some differences as more
important than others. In particular, feature values that only occur
in a few items give more information than feature values that appear
in a large number of items. This
idea shows up later in the normalization of syntax distance given by
\namecite{nerbonne06}.

The mathematical reasoning behind of this idea is fairly simple. Goebl
is interested in feature values that occur in only a few items. If a
feature has some value that is shared by all of the items, then all
items belong to the same group. This feature value provides {\it no}
useful information for distinguishing the items.  The situation
improves if all but one item share the same value for a feature; at
least there are now two groups, although the larger group is still not
very informative.  The most information is available if each item
being studied has a different value for a feature; the items fall
trivially into singleton groups, one per item.

Equation \ref{wiv-ident} implements this idea by discounting
the \textit{identical} count from equation \ref{riv} by
the amount of information that feature value conveys. The
amount of information, as discussed above, is based on the number of
items that share a particular value for a feature. If all items share
the same value for some feature, then \textit{identical} will be discounted all the
way to zero--the feature conveys no useful information.
Weighted Identical Value's equation for \textit{identical} is
therefore
\begin{equation}
  \textrm{identical} = \sum_f \left\{
  \begin{array}{ll}
    0 & \textrm{if} f_j \neq f_k \\
    1 - \frac{\textrm{agree}f_{j}}{(Ni)w} & \textrm{if} f_j = f_k
  \end{array} \right.
\label{wiv-ident}
\end{equation}

\noindent{}The complete definition of Weighted Identity Value is
\begin{equation} \sum_i \frac{\sum_f \left\{
  \begin{array}{ll}
    0 & \textrm{if} f_j \neq f_k \\
    1 - \frac{\textrm{agree}f_j} {(Ni)w} & \textrm{if} f_j = f_k
\end{array} \right.}
  {\sum_f \left\{
  \begin{array}{ll}
    0 & \textrm{if} f_j \neq f_k \\
    1 - \frac{\textrm{agree}f_j} {(Ni)w} & \textrm{if} f_j = f_k
    \end{array} \right. - |f \in \textrm{N} : f_j \neq f_k|}
  \label{wiv-full}
  \end{equation}

  \noindent{}where $\textrm{agree}f_{j}$ is the number of items that agree
  with item $j$ on feature $f$ and $Ni$ is the total number of
  items ($w$ is the weight, discussed below). Because of the
  piecewise definition of \textit{identical}, this number is always at
  least $1$ because $f_k$ agrees already with $f_j$.
  This equation takes the count of shared features and weights
  them by the size of the sharing group. The features that are shared
  with a large number of other items get a larger fraction of the normal
  count subtracted.

  For example, let $j$ and $k$ be sets of productions for the
  underlying English segment /s/. The allophones of /s/ vary mostly on the feature
  \textit{voice}. Seeing an unvoiced [s] for /s/ is less ``surprising'' than
  seeing a voiced [z], so the discounting process should
  reflect this. For example, assume that an English corpus contains 2000
  underlying /s/ segments. If 500 of them are realized as [z], the
  discounting for \textit{voice} will be as follows:

  \begin{equation}
    \begin{array}{c}
      identical_{/s/\to[z]} = 1 - 500/2000 = 1 - 0.25 = 0.75 \\
      identical_{/s/\to[s]} = 1 - 1500/2000 = 1 - 0.75 = 0.25
    \end{array}
    \label{wiv-voice}
  \end{equation}

  Each time /s/ surfaces as [s], it only receives 1/4 of a point
  toward the agreement score when it matches another [s]. When /s/
  surfaces as [z], it receives three times as much for matching
  another [z]: 3/4 points towards the agreement score. If the
  alternation is even more weighted toward faithfulness, the ratio
  changes even more; if /s/ surfaces as [z] only 1/10 of the time,
  then [z] receives 9 times more value for matching than [s] does.

  The final value, $w$, which is what gives the name ``weighted
  identity value'' to this measure, provides a way to control how much
  is discounted. A high $w$ will subtract more from uninteresting
  groups, so that \textit{voice} might be worth less than
  \textit{place} for /t/ because /t/'s allophones vary more over
  \textit{place}. In equation \ref{wiv-voice}, $w$ is left at 1 to
  facilitate the presentation.

\section{Statistical Methods} % Computational? Mathematical?

It is at this point that the two types of analysis, phonological and
syntactic, diverge. Although Goebl's techniques are general enough to
operate over any set of features that can be extracted, better results
can be obtained by specializing the general measures above to take
advantage of properties of the input.  Specifically, the application
of computational linguistics to dialectometry beginning in the 1990s
introduced methods from other fields. These methods, while generally
giving more accurate results quickly, are tied to the type of data on
which they operate.

% NEW
Currently, the dominant phonological distance measure is Levenshtein
distance. This distance is essentially the count of differing
segments, although various refinements have been tried, such as
inclusion of distinctive features or phonetic
correlates. \namecite{heeringa04} gives an excellent analysis of the
applications and variations of Levenshtein distance. While Levenshtein
distance provides much information as a classifier, it is limited
because it must have a word aligned corpus for comparison. A number of
statistical methods have been proposed that remove this requirement
such as \namecite{hinrichs07} and \namecite{sanders09}, but none have
been as successful on existing dialect resources, which are small and
are already word-aligned. New resources are not easy to develop
because the statistical methods still rely on a phonetic transcription
process.
% end NEW

% \begin{enumerate}
% \item I should really check around to see if there is any new work out
%   there. Surely there is. Course John is free to do whatever works and
%   Wybo may have graduated or something. So there might not be any more
%   work on it.
% \item Explain leaf-ancestor paths, trigrams, dependency `paths' (to be
%   invented).
% \end{enumerate}

\subsection{Syntactic  Distance}

Recently, computational dialectometry has expanded to analysis of
syntax as well. The first work in this area was \quotecite{nerbonne06}
analysis of Finnish L2 learners of English, followed by
\quotecite{sanders07} analysis of British dialect areas. Syntax
distance must be approached quite differently than phonological
distance. Syntactic data is extractable from raw text, so it is much
easier to build a syntactic corpus. But this implies an associated
drop in manual linguistic processing of the data. As a result, the
principle difference between present phonological and syntactic
corpora is that phonology data is word-aligned, while syntax data is
not sentence-aligned.

\subsubsection{Nerbonne and Wiersma}
\label{nerbonne06}

Due to the lack of alignment between the
larger corpora available for syntactic analysis, a statistical
comparison of differences is more appropriate than the simple
symbolic approach possible with the word-aligned corpora used in
phonology. This statistical approach means that a syntactic distance
measure will have to use counting as its basis.

\namecite{nerbonne06} was an early method proposed for syntactic
distance.  It models syntax by part-of-speech (POS) trigrams and uses
differences between trigram type counts in a permutation test of
significance. This method was extended by \namecite{sanders07}, who
used \quotecite{sampson00} leaf-ancestor paths as an alternate basis
for building the model.

The heart of the measure is simple: the difference in type counts
between the combined types of two corpora. \namecite{kessler01}
originally proposed this measure, the {\sc Recurrence}
metric ($R$):

\begin{equation}
R = \Sigma_i |c_{ai} - c_{bi}|
\label{rmeasure}
\end{equation}

\noindent{}Given two corpora $a$ and $b$, $c_a$ and $c_b$ are the type
counts. $i$ ranges over all types, so $c_{ai}$ and $c_{bi}$ are the
type counts of corpora $a$ and $b$ for type $i$.  $R$ is designed to
represent the amount of variation exhibited by the two corpora while
the contribution of individual types remains transparent to aid later
analysis.

To account for differences in corpus size, sampling with replacement is
used. In addition, the samples are normalized to account for
differences in sentence length and complexity.  Unfortunately, even normalized, the
measure doesn't indicate whether its results are significant; a
permutation test is needed for that.

% Other ideas include training a
% model on one area and comparing the entropy (compression) of other
% areas. At this point it's unclear whether this would provide a
% comparable measure, however.

\subsubsection{Language models}
Part-of-speech (POS) trigrams are quite easy to obtain from a syntactically
annotated corpus. \namecite{nerbonne06} argue that POS trigrams
can accurately represent at least the important parts of syntax,
similar to the way chunk parsing can capture the most important
information about a sentence. POS trigrams can either be generated by
a tagger as Nerbonne and Wiersma did, or taken from the leaves of
the trees of a parsed corpus as \namecite{sanders07} did with the
International Corpus of English.

On the other hand, it might be better to directly represent the upper
structure of trees. \quotecite{sampson00} leaf-ancestor paths provide
one way to do this: for each leaf in the
tree, leaf-ancestor paths produce the path from that leaf back to the
root. Generation is
simple as long as every sibling is unique. For example, the parse tree
\[\xymatrix{
  &&\textrm{S} \ar@{-}[dl] \ar@{-}[dr] &&\\
  &\textrm{NP} \ar@{-}[d] \ar@{-}[dl] &&\textrm{VP} \ar@{-}[d]\\
  \textrm{Det} \ar@{-}[d] & \textrm{N} \ar@{-}[d] && \textrm{V} \ar@{-}[d] \\
\textrm{the}& \textrm{dog} && \textrm{barks}\\}
\]
creates the following leaf-ancestor paths:

\begin{itemize}
\item S-NP-Det-The
\item S-NP-N-dog
\item S-VP-V-barks
\end{itemize}

For identical siblings, brackets must be inserted in the path to
disambiguate the first sibling from the second. The process is
described in \namecite{sampson00} or \namecite{sanders07};
in any case identical siblings are somewhat rare.

Sampson originally developed leaf-ancestor paths as an improved
measure of similarity between gold-standard and machine-parsed trees,
to be used in evaluating parsers. The underlying idea of a collection of
features that capture distance between trees transfers quite nicely to
this application. \namecite{sanders07} replaced POS trigrams with
leaf-ancestor paths for the ICE corpus and found improved results on
smaller corpora than Nerbonne and Wiersma had tested. The additional
precision that leaf-ancestor paths provide appears to aid in attaining
significant results.

% Another idea is supertags rather than leaf-ancestor paths. This is
% quite similar but might work better.

For dependency annotations, it is easy to adapt leaf-ancestor paths to
leaf-head paths. Here, each leaf is associated with a leaf-head path,
the path from the leaf to the head of the sentence via the
intermediate heads. For example, the same sentence, ``The dog barks'',
produces the following leaf-head paths.

\begin{itemize}
\item root-V-N-Det-the
\item root-V-N-dog
\item root-V-barks
\end{itemize}

The biggest difference is in the relative length of the paths: long
leaf-ancestor paths indicate deep nesting of structure. Length is a
weaker indicator of deep structure for leaf-head
paths; sometimes a difference in length indicates only a difference in
centrality to the sentence. % or something, this is still kind of
                            % wrong

\[\xymatrix{
& & root \\
NP  \ar@/^2pc/[rr] & DET\ar@/_/[l] & V \ar@{.>}[u]
}
\]

% TODO : This needs to be moved. Up, probably.
\subsubsection{Parsing}

In order to extract the features used in the previous section from
the Swedish dialect corpus SweDiaSyn, or in fact any unparsed corpus,
part-of-speech tagging and parsing will be needed.
The SweDiaSyn is transcribed to standard Swedish
orthography, but will need to be POS tagged and parsed, both for
phrase structures and dependencies.

The Tags 'n' Trigrams (T'n'T) tagger will be used for tagging, with
the POS annotations from Talbanken05 used as training.
After POS tagging, the sentences will be cleaned in preparation for
parsing. Cleaning phrase structure annotations consists of removing
discontinuities of various types, especially disfluencies and
restarts, which may be reparable by a simple top-level strategy. If
more complicated uncrossing is needed, a strategy similar to the split
constituents proposed by \namecite{boyd07} may be needed.

For constituent parsing, Collin's parser will be trained on standard
Swedish, again using Talbanken05. Collin's parser is a probabilistic
context-free grammar parser that is fully lexicalised; every
non-terminal is associated with the lexical item of its head
\cite{collins99}.
% Generation
% occurs in a top-down, head-first manner (not sure about this).
% (TODO: These two descriptions could probably be a LOT longer and
% more complete. It's a pretty short summary right now.)

For dependency parsing, MaltParser will be used with the existing
Swedish model trained on Talbanken05 by Hall, Nilsson and
Nivre. MaltParser is an inductive dependency parser that uses a
machine learning algorithm to guide the parser at choice points
\cite{nivre06b}.
Dependency parsing will proceed similarly to constituency parsing;
the dependency structures of Talbanken05 will be cleaned and
normalised, then used to train a parser that will then parse the
sentences of SweDiaSyn.

% TODO: Find out how much crossing occurs in Swedish corpora, and how
% much of it is from interruptions and self-corrections.

\subsubsection{Permutation test}
\label{permutationtest}

A permutation test detects whether two corpora are significantly
different. It does this on the basis of the R measure described in
section \ref{nerbonne06}. The test first calculates $R$ between
samples of the two corpora. Then the corpora are mixed together and
$R$ is calculated between two samples drawn from the mixed
corpus. If the two corpora are different, $R$ should be larger between
the samples of the original corpora than $R$ from the mixed
corpus: any real differences will be randomly redistributed by the
mixing process, lowering the mixed $R$. Repeating this comparison
enough times will show if the difference is significant. Twenty times
is the minimum needed to detect significance for $p < 0.5$
significance; however, in the experiments, I will repeat the test 1000
times.

To see how this works, for example, assume that $R$ detects real differences between London
and Scotland such that $R(\textrm{London},\textrm{Scotland}) =
100$. The permutation test then mixes London and Scotland to
create LSMixed and splits it into two pieces. Since the real
differences are now mixed between the two shuffled corpora, we
would expect $R(\textrm{LSMixed}_1, \textrm{LSMixed}_2) < 100$,
perhaps around 90 or 95. This should be true at least 95\% of the time if the
differences are significant.

\subsubsection{Normalization}
Afterward, the distance must be normalized to account for two things:
the length of sentences in the corpus and the amount of variety in the
corpus. If sentence length differs too much between corpora, there
will be consistently lower token counts in one corpus, which would
cause a spuriously large $R$. In addition, if one corpus has less
variety than the other, it will have inflated type counts, because
more tokens will be allocated to fewer types. To avoid
this, all tokens are scaled by the average number of types per token
across both corpora: $2n/N$ where $n$ is the type count and $N$ is
the token count. The factor $2$ is necessary because the scaling
occurs based on the token counts of the two corpora combined.

% this next subsection might need to be changed or deleted
\subsection{Cluster Analysis}
\label{cluster-analysis}
The results of the two distance methods can be compared using
binary hierarchical clustering. The resulting dendrogram allows the two
methods to be compared visually.

Correlation is also useful to find out how similar the two method's
predictions are. Because of the connected nature of the inter-region
distances, Mantel's test is necessary to ensure that the correlation
is significant. Mantel's test is a permutation test, much like the
permutation test described for $R$. One distance result set is
permuted repeatedly and at each step correlated with the other
set. The original correlation is significant if the permuted
correlation is lower than the original correlation more than 95\% of
the time.

\subsection{Corpora}

The dialect corpus used in this dissertation will be SweDiaSyn, the
Swedish part of the ScanDiaSyn.
% (CITE SweDiaSyn and ScanDiaSyn,
% except that they don't seem to have any references)
% Here is a citation for ScanDiaSyn if I could track it down and
% translate it
% Vangsnes, �ystein A. 2007. ScanDiaSyn: Prosjektparaplyen Nordisk dialektsyntaks. In T. Arboe (ed.), Nordisk dialektologi og sociolingvistik, Peter Skautrup Centeret for Jysk Dialektforskning, �rhus Universitet. 54-72.
SweDiaSyn is a transcription of SweDia 2000
\cite{bruce99} collected between 1998 and 2000 from 97 locations in Sweden and
10 in Finland. Each location had 12 interviewees: three 30-minute
interviews for each of older male, older female, younger male and
younger female.
However, the SweDiaSyn transcriptions do not yet include all of SweDia
2000; the completed transcriptions currently focus on older
speakers. Currently there are 36713 sentences of dialect speech and 49
sites, an average of 749 sentences per site. The sites may need to be
grouped by province in order to detect significant differences;
previous work on British English used Government Office Regions with
at least 850 sentences per region.

 There are two types of transcription: 
standard Swedish orthography, with glosses for words
not in standard Swedish, and a phonetic transcription for dialects
that differ greatly from standard Swedish. For this dissertation, only
the orthographic/gloss transcription will be used, since phonological
information cannot be easily used in this syntax distance project.

Since SweDiaSyn consists of lexical items only, Talbanken05 will be
used to train a POS tagging and a parser. Talbanken05 is a modernised version of Talbanken76, a Swedish treebank of
roughly 300,000 words \cite{nivre06}. The treebank consists of written
and spoken Swedish. The original treebank was annotated according to a
custom scheme called MAMBA; Talbanken05 updates this to include phrase
structure annotation and dependency annotation using the standard
annotation schemes TIGER-XML and Malt-XML.

\subsection{Previous Experiments}

% TODO: This section could still be a little longer.

\namecite{nerbonne06} were the first to use the syntactic distance
measure described above. They analysed two corpora, both of Norwegian
L2 speakers of English. The first corpus was gathered from speakers
who learned English after childhood and the second was gathered from
speakers who learned English as children. Nerbonne \& Wiersma found a
significant difference between the two corpora. The trigrams that
contributed most to the difference were those in the older corpus that
are unexpected in English. For example, the trigram COP-ADJ-N/COM is
not common in English because a noun phrase following a copula
typically begins with a determiner. Other trigrams indicate
hypercorrection on the part of the older speakers; they appear in the
younger corpus but not as often. Nerbonne \& Wiersma analysed this as
interference from Finnish; the younger learners of English learned it
more completely with less interference from Finnish.

Subsequent work by \namecite{sanders07} and \namecite{sanders08b}
expanded on the Norwegian experiment in two ways. First, it introduced
leaf-ancestor paths as an alternative feature type. Second, it tested
the distance method on a larger set of corpora: Government Office
Regions of England, as well as Scotland and Wales, for a total of
11 corpora. Each was smaller than the Norwegian L2 corpora, so the
permutation test parameters had to be adjusted for some feature
combinations.

The distances between regions were clustered using hierarchical
agglomerative clustering, as described in section \ref{cluster-analysis}. The resulting tree showed a North/South
distinction with some unexpected differences from previously
hypothesised dialect boundaries; for example, the
Northwest region clustered with the Southwest region. This contrasted
with the clustered phonological distances also produced in
\namecite{sanders08b}. In that experiment,
there was no significant correlation between the inter-region
phonological distances and syntactic distances.

There are several possible reasons for this lack of correlation. The
two distance measures may find different dialect boundaries based on
differences between syntax and phonology. Dialect boundaries may have
shifted during the 40 years between the collection of the SED and the
collection of the ICE-GB. One or both methods may be measuring the
wrong thing. However, I will not investigate the relation between
phonology and syntax in this dissertation. The focus will remain on results
of computational syntax distance as compared to traditional syntactic
dialectology.

\bibliographystyle{robbib}
\bibliography{central}
\end{document}
%%% Local Variables: 
%%% mode: latex
%%% TeX-master: t
%%% End: 

