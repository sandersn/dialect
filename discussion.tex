\chapter{Discussion}

This chapter discusses three topics:

\begin{enumerate}
\item Analysis of results: dissertation work on its own.
\item Comparison to Swedish syntactic dialectology.
\item Comparison to Swedish phonological dialectometry.
\item Conclusions: summary of discussion
\item and then summary of dissertation: contribution to dialectometry at large.
\end{enumerate}

However, I can get some of the discussion from my first qualifying
paper; the syntax distance method hasn't changed, it just works a lot
better on a dialect corpus.

A big question is why trigrams are so good. All of the fancier feature
sets do worse than trigrams.

\section{Compared to Previous Work}

In previous work on British English, this method failed to find
agreement between syntactic distance ($R$) and phonological distance
(Levenshtein) distance---there was no
significant correlation between the two methods. Although both showed
something like a North/South distinction in Britain, its orientation was much more
obvious from phonological distance. This lack of agreement was a
preliminary answer to the question of whether multiple
ways of measuring linguistic distance give the same results.

However, there were at least five reasonable explanations for the difference
between the two distance measures.
% First, and
% least satisfying, is the possibility that one of the distances is not
% measuring what it is supposed to. Second, the corpora may not agree
% because of the 40 year difference in age and differing collection
% methodologies. Third, syntactic and phonological dialect markers may
% not share the same boundaries.

\begin{enumerate}
\item One or both of the distances does not measure what it is supposed to.
\item The two corpora may not agree on dialect boundaries because of
  their 40-year difference in age.
\item Place of birth, as recorded in the ICE, may not correlate well
  with spoken dialect, especially given variations in speaker
  education level and place of residence.
\item Dialect boundaries may appear from systematic variation in
  annotation practices rather than the speech.
\item Syntactic and phonological dialect boundaries may be different.
\end{enumerate}

Of these, this dissertation addresses the second, third and fourth
problems directly by using a single corpus, Swedia2000, annotated by a single
person. (TODO: This may not be true for the phonological annotation.)
By finding significant distances between all interview sites of
Swedia2000, it also suggests that $R$ is measuring syntax
distance. PROBABLY.

The last is the most interesting because previous work will not have
exposed this difference. Traditional dialectometry focuses on a strong
agreement among a few features from each collection site. Because
syntactic features are fewer in number than phonological ones, they
are under-represented in this type of analysis. Unfortunately, this
means that the syntactic contribution to isogloss bundles is
correspondingly reduced. In addition, because of isogloss bundles'
insensitivity to rare variations, syntactic features rarely contribute
to isogloss bundles of successful dialect boundaries.
% I really need to CITE this.

In contrast, computational analysis, such as \cite{shackleton07},
captures feature variation precisely using statistical analysis and
sophisticated algorithms. The resulting analysis displays dialects as
gradient phenomena, displaying much more complexity than the
corresponding isogloss analysis. But current specialized computational
methods only apply to phonology. Syntactic data cannot be analyzed
without a syntax-specific method.

This paper attempts to address that lack, and provide some first steps
to show whether syntax and phonology assist each other in establishing
dialects, or whether their dialect regions are unrelated. If they are not
related, and syntactic gradients can be as weak as phonological ones,
then some new dialect regions may become apparent that were not visible in
previous phonology-only analyses.

\subsection{Improvements on British Dialect Experiment}

This dissertation improves on the British experiment in a number of
ways. It addresses the obvious criticism that syntax distance on the
ICE requires so much data that the results are no more informative
syntax those of traditional dialectology---its precision lags
phonological distance methods badly. However, $R$ works with much
smaller corpora when run on Swedia2000. This shows that the problem
with the British experiment is not the distance method, but the
corpus, which fails to capture dialect differences. Most likely is
that the interviewees, mostly in a college setting, actively tried to
suppress dialect differences during the interview.

Another problem with the current study is the 40-year difference in
collection dates between the phonological corpus and the syntactic
corpus. A recent phonological corpus would likely show the same sort
of changes in the North/South divide that show up in the syntactic
corpus. The British population became more mobile during the second
half of the 20th century, and the SED survey explicitly attempted to capture
the dialects that existed before this happened \cite{orton78}.
It would also be nice to have data from the rest of the United Kingdom for
comparison as well, or at least Scotland and Wales as with the ICE.


% TODO: integrate this.
% Alternatively, I could just look at the region pairs that fail to
% achieve significance in the syntactic permutation test and check to
% see if their phonological distance is lower than the other pairs. I
% don't do this (yet).

One interesting question is
how phonological and syntactic distances correlate with geographic
distance---\namecite{gooskens04a} shows that often the correlation is
very good. This would also allow better visualization of dialect areas
than a hierarchical dendrogram.

\section{Future Work}

Try all those smarter variants of $R$.

Include the rest of Nodalida once it is done.

\section{The End}

This is the end of a Highly Impressive Dissertation.
