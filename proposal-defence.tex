\documentclass{beamer}

\usetheme{Ilmenau}
\usepackage[all]{xy}

\title{Syntax Distance for Dialectometry}
\author{Nathan Sanders}
\date{\today}

\begin{document}
\frame{\titlepage}
\section{Overview}
\begin{frame}
  \frametitle{Contents}
  \begin{itemize}
  \item Overview
  \item Principal Hypotheses
  \item Analysis
  \item Writing
\end{itemize}
\end{frame}

\section{Overview}
\begin{frame}
  \frametitle{Overview}
  \begin{itemize}
 \item Define a syntactic distance measure and test it on a dialect corpus.
 \item The distance measure tested is $R$ and the dialect corpus is
  Swediasyn, the transcription of interviews from Swedia2000.
  \end{itemize}
\end{frame}

\begin{frame}
  \frametitle{Terms}
  \begin{itemize}
  \item Dialectology: Study of linguistic variation
  \item Dialectometry: Quantitative analysis of linguistic variation,
    recently dominated by computational methods.
 \end{itemize}
\end{frame}

\begin{frame}
  \frametitle{Terms: Distance Measure}
  A distance measure consists of two functions: a method to produce input
  features and a way to combine those features into a number. (A distance
  is not necessary. A symmetric divergence will suffice.)
\end{frame}

\begin{frame}
  \frametitle{Features}
  \begin{itemize}
  \item Phonology decomposes word into segments or phonological features.
  \item Syntax does not have a standard decomposition of sentences into features.
  \item This dissertation tries several variations: trigrams,
    leaf-ancestor paths, and dependency paths.
  \end{itemize}
\end{frame}

\begin{frame}
  \frametitle{Distance Metric}
  \begin{itemize}
  \item Phonology typically uses Levenshtein distance, which requires
    that the same words be elicited from all interviewees.
  \item Syntax does not have a standard distance metric.
  \item This dissertation tries several variations: primarily $R$ and $R^2$, but
    also Kullbeck-Leibler divergence and Jensen-Shannon divergence.
  \end{itemize}
\end{frame}

\begin{frame}
  \frametitle{Experiment}
  Find syntactic differences between dialects of Swedish
  \begin{enumerate}
  \item Corpus: Swediasyn, unparsed interviews transcribed and glossed
    to standard Swedish
  \item Training corpus: Talbanken, 300 Kwords of parsed spoken and
    written Swedish from the late 70s.
  \item Annotators: TnT, MaltParser and the Berkeley parser, all
    trained on Talbanken.
  \end{enumerate}
\end{frame}
\begin{frame}
  \frametitle{Schedule}
  Need to be done by July 6. OK.
\end{frame}

\section{Analysis}
\begin{frame}
\frametitle{Analyzed}
\begin{itemize}
\item Distance (R)
\item Alternate distances (KL divergence, JS divergence)
\item Alternate features (dependency POS tags taken from Berkeley
  parser)
\item Significance of each combination of distances and features.
\item 5 most heavily weighted features for each comparison.
\item Hierarchical clustering
\item Multi-dimensional scaling
\end{itemize}
\end{frame}

\begin{frame}
\frametitle{To Analyze}
\begin{itemize}
\item Variations on parser parameters and input cleaning
\item More alternate/combined features (particularly, dependency arc labels)
\item Better ways to characterise the most important features of each
  comparison.
\end{itemize}
\end{frame}


\section{Writing}
\begin{frame}
\frametitle{Written}
\begin{itemize}
\item Introduction
\item Methods
\item Questions (a little)
\item Results (a little)
\end{itemize}
\end{frame}

\begin{frame}
\frametitle{To Write}
\begin{itemize}
\item Questions (most)
\item Results (most)
\item Discussion
\end{itemize}
\end{frame}

\end{document}
