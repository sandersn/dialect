\documentclass{beamer}

\usetheme{Ilmenau}
\usepackage[all]{xy}

\title{Syntactic and Phonological Distance in British Dialects}
\author{Nathan Sanders}
\date{\today}

\begin{document}

\frame{\titlepage}

\section[Outline]{}
\frame{\tableofcontents}
\begin{frame}
  Compare syntactic and phonological distance measures from
  dialectometry to see if the two agree, using British English dialect data.
\end{frame}
\section{Background}
\begin{frame}
  \frametitle{Dialectology}
  \begin{definition}
    Dialectology looks for boundaries between languages and lists the
    boundaries' properties.
  \end{definition}
\end{frame}
\begin{frame}
  \frametitle{Dialectology}
  \includegraphics[width=0.9\textwidth]{total-county}
\end{frame}
  % Statistics can tell you if something is significant, but only if
  % you have multiple samples.
  % Once you have multiple samples, you can start telling HOW
  % different something is.
  % Now you get the old SOLID boundaries but also gradients.
  % And you know exactly how important they are.
\begin{frame}
  \frametitle{Dialectometry}

  \begin{definition}
    Dialectometry looks for and quantifies differences between languages.
  \end{definition}
  
\end{frame}
\begin{frame}
  \frametitle{Dialectometry}
  Map from the LAMSAS, generated by RuG-L04
  \begin{center}
  \includegraphics[width=0.5\textwidth]{demo1}
  \end{center}
\end{frame}
\begin{frame}
\begin{columns}
\column[c]{0.5\textwidth}
Pro:\\
  The result is significant and gradient.
\column[c]{0.5\textwidth}
Con: \\
  Methods are harder to develop.
\end{columns}
\vspace{2cm}
Hence: Most dialectometry has worked with phonology.
% good methods have existed for a long time in phonology
% not so in syntax
\end{frame}

\section{Methods}
\subsection{Phonology}
\begin{frame}
\frametitle{Levenshtein Distance}
%     This makes for a pretty equation OR tail-recursive, extraordinarily
%     efficient code implemented entirely in Scheme. You get the
%     equation today.
\begin{definition}
For each character $s_i \in S$ and $t_j \in T$ for any string $S$ and $T$,
\begin{equation}
  levenshtein(s_i,t_j) = min \left(
  \begin{array}{l}
   ins(s_i)+levenshtein(s_{i-1},t_j), \\
 del(t_j)+levenshtein(s_i,t_{j-1}), \\
 sub(s_i,t_j)+levenshtein(s_{i-1},t_{j-1})
   \end{array} \right)
   \label{levequation}
\end{equation}
The total distance between $S$ and $T$ is $levenshtein(S_{|S|},T_{|T|})$.
\end{definition}
\end{frame}
\begin{frame}
%   Don't worry about the details. The important part is how $ins$,
%   $sub$ and $del$ are defined. Here is one of the simplest
%   definitions:
\frametitle{Levenshtein Distance}
\begin{definition}
\begin{equation}
\begin{array}{l}
   ins(t_j) = 1 \\
   del(s_i) = 1 \\
   sub(s_i,t_j) = \left\{
     \begin{array}{ll}
       0 & \textrm{if $s_i=t_j$} \\
       2 & \textrm{otherwise}
     \end{array} \right.

   \end{array}
\end{equation}
\end{definition}
\end{frame}
\begin{frame}
  \frametitle{Levenshtein distance}
  \begin{example}[levenshtein ``art'' ``cat'']
    \begin{tabular}{l|r}
    $_\uparrow$ART &   {\tt start=0} \\
\pause
    C$_\uparrow$ART & {\tt 0+ins(C)=1} \\
\pause
    CA$_\uparrow$RT & {\tt 1+sub(A,A)=1} \\
\pause
    CA$_\uparrow$T & {\tt 1+del(R)=2} \\
\pause
    CAT$_\uparrow$ & {\tt 2+sub(T,T)=2} \\
    \end{tabular}
  \end{example}

\end{frame}
\begin{frame}
\frametitle{Levenshtein distance}
\begin{figure}
\caption{The distance table for ``ART'' to ``CAT''}

\begin{center}
\begin{tabular}{c|c|c|c|c}
%\hline
  &   & A & R & T \\
\hline
  & $\mathbf{0}$ & 1 & 2 & 3 \\
\hline
C & $\mathbf{1}$ & 2 & 3 & 4 \\
\hline
A & 2 & $\mathbf{1}$ & $\mathbf{2}$ & 3 \\
\hline
T & 3 & 2 & 3 & $\mathbf{2}$
% \hline
\end{tabular}

\end{center}

\label{art2cattable}
\end{figure}

\end{frame}
\subsection{Syntax}
\begin{frame}
\frametitle{Syntax}
  A permutation test over $R$ using leaf-ancestor paths as features.
\end{frame}
\begin{frame}
  \frametitle{Leaf-Ancestor Paths}
The parse tree
\begin{columns}
\column[c]{0.5\textwidth}
\[\xymatrix{
  &&\textrm{S} \ar@{-}[dl] \ar@{-}[dr] &&\\
  &\textrm{NP} \ar@{-}[d] \ar@{-}[dl] &&\textrm{VP} \ar@{-}[d]\\
  \textrm{Det} \ar@{-}[d] & \textrm{N} \ar@{-}[d] && \textrm{V} \ar@{-}[d] \\
\textrm{the}& \textrm{dog} && \textrm{barks}\\}
\]
\column[c]{0.5\textwidth}
gives the following leaf-ancestor paths:
\begin{itemize}
\item NP-Det
\item NP-N
\item VP-V
\end{itemize}
\end{columns}
\end{frame}
\begin{frame}
\frametitle{R}
\begin{definition}
\begin{equation}
R = \Sigma_i |c_{ai} - c_{bi}|
\label{rmeasure}
\end{equation}
\end{definition}
\noindent{}Given two corpora $a$ and $b$, $c_a$ and $c_b$ are the type
counts. $i$ ranges over all types, so $c_{ai}$ and $c_{bi}$ are the
type counts for type $i$.
  \begin{example}
    \[a=\{\textrm{NP-Det}:12, \textrm{NP-N}:10, \textrm{VP-V}:5\}\]
    \[b=\{\textrm{NP-Det}:3, \textrm{NP-N}:12, \textrm{VP-V}:20\}\]
    \[R = \{|12-3| + |10-12| + |5-20|\} = 27 \]
  \end{example}
\end{frame}
\begin{frame}
  \frametitle{Permutation Test}
  \begin{enumerate}
  \item $d = R(sample(a),sample(b))$.
  \item At least 20 times:
    \begin{enumerate}
    \item $shuffled = shuffle(a,b)$
    \item $shuffle_a = sample(shuffled)$
    \item $shuffle_b = sample(shuffled)$
    \item $d_{shuffle} = R(shuffle_a,shuffle_b)$
    \item If $d_{shuffle} < d$, there were differences in
      $d$ that were destroyed in $d_{shuffle}$.
  \end{enumerate}
  \item If $d_{shuffle} < d$ more than 95\% of the time, $d$ is significant.
  \end{enumerate}
\end{frame}
\section{Experiment}
\subsection{Corpora}
\begin{frame}
  \frametitle{Corpora}
  \begin{columns}
\column[c]{0.5\textwidth}
  \includegraphics[width=0.95\textwidth]{GB_GOR98_A4}
\column[c]{0.5\textwidth}
  \begin{itemize}
  \item Phonology: \\ Survey of English Dialects (SED)
  \item Syntax: \\ International Corpus of English, Great Britain
    (ICE-GB)
  \end{itemize}
    \end{columns}
\end{frame}
\begin{frame}
  \frametitle{SED}
  \begin{itemize}
  \item Classic dialectology corpus: collected from interviews in the
    1950s, a couple of hundred items.
  \item 55-word subset and phonological features from Shackleton (2007)
  \item The SED tries to capture the older uses of the language.
  \end{itemize}
\end{frame}
\begin{frame}
  \frametitle{ICE-GB}
  \begin{itemize}
  \item Conversation (and writing), syntactically annotated.
  \item Collected in the 1990s, various sources with place-of-birth noted.
  \end{itemize}
\end{frame}
\subsection{Parameters}
\begin{frame}
\frametitle{Phonology}
\begin{columns}
    \column[c]{0.6\textwidth}
Shackleton's feature set
\begin{tabular}{c|lc}
  \hline Vowel & Height & 1.0 - 7.0 \\
  & Backing & 1.0 - 3.0 \\
  & Rounding & 1.0 - 2.0 \\
  & Length & 0.5 - 2.0 \\ \hline
  Consonant & Fricative & 0.0 - 1.0 \\
  & [h]/[wh] & 0.0 - 1.0 \\
  & Glottal Stop & 0.0 - 1.0 \\
  & Velar & 0.0 - 2.0 \\
  & Other & 0.0 - 1.0 \\ \hline
  Rhotic & Place & 1.0 - 3.0 \\
  & Manner & 0.0 - 4.0 \\
\end{tabular}
    \column[c]{0.4\textwidth}
  Determine average value for insertion and deletion--this should be
  half the average substitution cost for arbitrary segments.
\end{columns}
\end{frame}
\begin{frame}
  \frametitle{Syntax}
  \begin{tabular}{l|cr}
  measure& $R$ & $R^2$ \\
  sample size& 500 & 1000 \\
  feature set& POS trigram & leaf-ancestor path \\
  \end{tabular}
\end{frame}
\begin{frame}
  \frametitle{Validity}
  To find out whether parameter settings are valid.
  \begin{enumerate}
  \item Assume that Scotland and London are different.
  \item Make sure that the parameter settings for a distance measure
    agree that Scotland and London are significantly different.
  \end{enumerate}
\end{frame}
\section{Results}
\begin{frame}
  \frametitle{Results}
  \begin{columns}
    \column[c]{0.5\textwidth}
  \includegraphics[width=1.2\textwidth]{sed_dendrogram}
    \column[c]{0.5\textwidth}
  \includegraphics[width=1.2\textwidth]{ice_dendrogram}
\end{columns}
\end{frame}
\begin{frame}
  \frametitle{Results}
  \begin{columns}
    \column[c]{0.5\textwidth}
  \includegraphics[width=1.2\textwidth]{sed_dendrogram}
    \column[c]{0.5\textwidth}
  \includegraphics[width=0.8\textwidth]{GB_GOR98_A4}
\end{columns}
\end{frame}
\begin{frame}
  \frametitle{Results}
  \begin{columns}
    \column[c]{0.5\textwidth}
  \includegraphics[width=0.8\textwidth]{GB_GOR98_A4}
    \column[c]{0.5\textwidth}
  \includegraphics[width=1.2\textwidth]{ice_dendrogram}
\end{columns}
\end{frame}
\begin{frame}
  \frametitle{Results}
  There is no significant correlation between the methods, or between
  $R$ and corpus size.
\end{frame}
\begin{frame}
  \frametitle{Why not?}
  \begin{itemize}
  \item Problem with distance measures
  \item Difference in formality between corpora
  \item Difference in age of corpora
  \item Place of birth may be a bad indicator of dialect in England
  \item Phonology and syntax may disagree
  \end{itemize}
\end{frame}
\begin{frame}
  \frametitle{Future Work}
  \begin{itemize}
  \item Re-run syntax distance on SweDiaSyn, part of ScanDiaSyn.
  \item Compare SweDiaSyn results to phonology results from SweDia
    2000, the source of SweDiaSyn.
  \item Compare results to the growing body of dialectology literature
    based on SweDiaSyn and SweDia 2000.
  \item Revisit top-ranked leaf-ancestor paths as found by $R$.
  \end{itemize}
\end{frame}
\end{document}
