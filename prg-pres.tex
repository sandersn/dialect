\documentclass{beamer}

\usetheme{Ilmenau}
\usepackage[all]{xy}

\title{Syntax Distance for Dialectometry \\ Progress Report : Parsing
  Swedish Badly}
\author{Nathan Sanders}
\date{\today}

\begin{document}

\begin{frame}
  \frametitle{Welcome to Jones!}

  \begin{itemize}
  \item Try Haskell, a Very Nice Functional Language!
  \item We have met the enemy and he is us.
  \item  Who would win? Henry VIII or George III?
  \item  Only the pure of heart can remove the sword from the stone.
  \item With great powder comes great rapidity.
  \item  640K should be enough for anyone.
  \item A spoonful of sugar makes the medicine go down.
  \item Try O'Caml, a Very Nice Functional Language!
  \item Try Scala, a Very Nice Functional Language!
  \item Try Nice, a Very Clean Functional Language!
  \item Try Clean, a Very Nice Functional Language!
  \item Try Nemerle, a Very Nice Functional Language!
  \item Try Miranda\texttrademark, a Functional Language!
  \item Try F\#, a Very Nice Functional Language!
  \item Try Scheme, a Very Nice Functional Language!
  \item Try Clojure, a Very Nice Functional Language!
  \end{itemize}
\end{frame}

\frame{\titlepage}
\section{Experiment}
\begin{frame}
  \frametitle{Experiment}
  Find syntactic differences between dialects of Swedish
  \begin{enumerate}
  \item Corpus: Swediasyn, unparsed interviews transcribed and glossed
    to standard Swedish
  \item Training corpus: Talbanken, 300 Kwords of parsed spoken and
    written Swedish from the late 70s.
  \item Annotators: TnT, MaltParser and the Berkeley parser, all
    trained on Talbanken.
  \end{enumerate}
\end{frame}
\section{Prototype}
\begin{frame}
  \frametitle{Convert Talbanken}
  From TIGER-XML, a flat XML representation in ISO-8859-1 (Latin1)
 \begin{itemize}
  \item To TnT POS (word --- tab --- POS)
  \item To PTB (nested s-exps, which are kind of annoying to produce
    from TIGER-XML)
  \item Swediasyn is UTF-8, and all the Java-based parsers require it,
    so Talbanken must be converted.
  \end{itemize}
\end{frame}
\begin{frame}[fragile]
  \frametitle{Talbanken Example}
\begin{verbatim}
<body>
 <s id="s1">
  <graph root="s1_501">
    <terminals>
     <t id="s1_1" word="ja" pos="POPPHH"/>
     <t id="s1_2" word="tycker" pos="VVPS"/> ...
    </terminals>
    <nonterminals>
     <nt id="s1_503" cat="PP">
      <edge label="PR" idref="s1_3"/>
      <edge label="DT" idref="s1_4"/>
      <edge label="HD" idref="s1_5"/>
     </nt> ...
    </nonterminals></graph></s></body>
\end{verbatim}
\end{frame}
\begin{frame}[fragile]
  \frametitle{TnT Example}
\begin{verbatim}
ja POPPHH
tycker VVPS
för PR
(imagine f\:or above, pdflatex doesn't like UTF-8 either)
min POXPHHGG
del NN
\end{verbatim}
\end{frame}
\begin{frame}[fragile]
  \frametitle{PTB Example}
\begin{verbatim}
(ROOT (S
 (POPPHH ja)
 (VVPS tycker)
 (PP (PR för) (POXPHHGG min) (NN del))
 (S
  (XP (UKAT att) (UKAT att))
  (RJ &gt;&gt;)
  (EHRJPU &gt;&gt;)
  (POOP de)
  (AVPS e)
  (NP (P P (ABZA för) (ABZA mycke) (PR av))
      (EN en)
      (NN monolog)
      (S (PORP som) (NNDDHH prästen) (VVPS håller))))))
\end{verbatim}
\end{frame}
\begin{frame}
  \frametitle{T'n'T}
  \begin{itemize}
  \item Training: No problems yet, default parameters
  \item Classification: Low vocabulary overlap between Talbanken and
    dialects of Swediasyn. (around 15\% unknown tokens, more for small
    corpora)
  \item There were more before converting Talbanken to UTF-8.
  \end{itemize}
\end{frame}
\begin{frame}
  \frametitle{MaltParser}
  \begin{itemize}
  \item Training: Already done and downloadable. Thanks Joakim!
  \item Conversion: POS-tagged Swedia to ConLL is trivial because ConLL is a flat
    format.
  \item Classification: Default parameters, results don't appear to be very good.
  \end{itemize}
\end{frame}
\begin{frame}[fragile]
  \frametitle{Example}
\begin{columns}
\column[c]{0.7\textwidth}
\begin{tabular}{ccccc}
1&       varit&     AVSN&      0&       ROOT\\
2&       v\"aldigt&   AJ&        3&       AA  \\
3&       intresserad&       AJ&        7&       SS\\
4&       av&        PR&        3&       ET  \\
5&       det&       PODP&      4&       PA  \\
6&       h\"ar&       ID&        5&       HD  \\
7&       \aa{}ka&       VVIV&      0&       ROOT\\
8&       ikring&    PR&        0&       ROOT\\
9&       och&       ++OC&      0&       ROOT\\
10&      titta&     VVIV&      0&       ROOT\\
11&      p\aa{}&        PR&        0&       ROOT\\
12&      platser&   NN&        11&      PA  \\
13&      .   &      IP&        1&       IP  \\
\end{tabular}
\column[c]{0.3\textwidth}
\textit{Been interested in going around there and looking at
places}. (translate.google.com)
\end{columns}
\end{frame}
\begin{frame}
  \frametitle{Berkeley Parser}
  \begin{itemize}
  \item Training: Takes 2 GB. No more, no less. So I used
    banks (Don't tell Josh).
  \item Classification: Default parameters, I haven't looked at
    results closely. Takes about 3 days, but less than 1 GB.
  \end{itemize}
\end{frame}
\begin{frame}[fragile]
  \frametitle{Example}
\begin{columns}
\column[c]{0.6\textwidth}
\begin{verbatim}
(S (VVPS AVSN)
   (VNDD AJ)
   (VVIV AJ)
   (CNP (NNDD PR)
        (NNDD PODP)
        (PN__HH ID)
        (NNDDHHGG VVIV))
   (VN__SS PR)
   (VVPS ++OC)
   (NP (NP (NN VVIV)
           (PN__HH PR))
       (PN__HH NN)
       (ID IP)))
\end{verbatim}
\column[c]{0.4\textwidth}
(varit AVN) \\
(v\"a{}ldigt AJ) \\
(intresserad AJ) \\
(av PR) \\
(det PODP) \\
(h\"ar ID) \\
(\aa{}ka VVIV) \\
(ikring PR) \\
(och ++OC) \\
(titta VVIV) \\
(p\aa{} PR) \\
(platser NN)\\
(. IP)
\end{columns}
\end{frame}
\begin{frame}
  \frametitle{Features}
  \begin{itemize}
  \item Trigrams: Trivial, same as before.
  \item Leaf-ancestors: Same as before, except now in Haskell.
  \item Dependency paths: for each leaf, record the path to the root.
 \end{itemize}
\end{frame}
\begin{frame}
  \frametitle{Leaf-Ancestor Paths}

\begin{columns}
\column[c]{0.5\textwidth}
\[\xymatrix{
  &&\textrm{S} \ar@{-}[dl] \ar@{-}[dr] &&\\
  &\textrm{NP} \ar@{-}[d] \ar@{-}[dl] &&\textrm{VP} \ar@{-}[d]\\
  \textrm{Det} \ar@{-}[d] & \textrm{N} \ar@{-}[d] && \textrm{V} \ar@{-}[d] \\
\textrm{the}& \textrm{dog} && \textrm{barks}\\}
\]
\column[c]{0.5\textwidth}
\begin{itemize}
\item S-NP-Det-The
\item S-NP-N-dog
\item S-VP-V-barks
\end{itemize}
\end{columns}

\end{frame}
\begin{frame}
  \frametitle{Dependency Paths}

\begin{columns}
\column[c]{0.5\textwidth}
\[\xymatrix{
& & root \\
DET \ar@/^/[r] & NP\ar@/^/[r] & V \ar@{.>}[u] \\
The & dog & barks
}
\]
\column[c]{0.5\textwidth}
\begin{itemize}
\item root-V-N-Det-the
\item root-V-N-dog
\item root-V-barks
\end{itemize}
\end{columns}

\end{frame}
\begin{frame}[fragile]
  \frametitle{Distance}
 \begin{itemize}
  \item g++ -O2
  \item Same as before: \verb+r = map (uncurry (-) & abs) & sum+
  \item Significance test is only 100 iterations, down from 1000.
  \item May be ghc -O2 soon.
 \end{itemize}
\end{frame}
\end{document}
