
\documentclass{beamer}

\usetheme{Ilmenau}
\usepackage[all]{xy}

\title{Syntactic and Phonological Distance in British Dialects}
\author{Nathan Sanders}
\date{\today}

\begin{document}

\begin{frame}
  \frametitle{Welcome to Jones!}

  \begin{itemize}
  \item You will join an elite, but rapidly dwindling group of brawling immortals. Try to keep your head.
  \item The Norwegians may be watching you. Act casual and pretend to be an American Tourist.
  \item You will kill your father and marry your mother.
  \item  Androids from the future will appear and fight over your destiny.
  \item  The key to defeating the French is to avoid looting.
  \item  Only the pure of heart can remove the sword from the stone.
  \item  Never trust a monkey.
  \item  Family tragedy and a drug overdose will let you see the future.
  \item  USE LINUX, THE FREE OPERATING SYSTEM BY LINUS TORVALDS.
%   \item  You will become indepently wealthy and spend your time
%     exploring ruinous ruins.
  \item  You will age rapidly and fall in love with a wizard.
  \item It's a wonderful life!
  \end{itemize}
\end{frame}

\frame{\titlepage}

\section[Outline]{}
\frame{\tableofcontents}

\section{Methods}
\subsection{Phonology}
\begin{frame}
  \begin{itemize}
  \item Levenshtein distance
  \item You know, string edit distance?
  \end{itemize}
\begin{figure}
\caption{The distance table for ``ART'' to ``CAT''}

\begin{center}
\begin{tabular}{c|c|c|c|c}
%\hline
  &   & A & R & T \\
\hline
  & $\mathbf{0}$ & 1 & 2 & 3 \\
\hline
C & $\mathbf{1}$ & 2 & 3 & 4 \\
\hline
A & 2 & $\mathbf{1}$ & $\mathbf{2}$ & 3 \\
\hline
T & 3 & 2 & 3 & $\mathbf{2}$
% \hline
\end{tabular}

\end{center}

\label{art2cattable}
\end{figure}

\end{frame}
\begin{frame}
    This makes for a pretty equation OR tail-recursive, extraordinarily
    efficient code implemented entirely in Scheme. You get the
    equation today.
\begin{definition}
Now for each character $s_i \in S$ and $t_j \in T$ for any string $S$ and $T$,
\begin{equation}
  levenshtein(s_i,t_j) = min \left(
  \begin{array}{l}
   ins(s_i)+levenshtein(s_{i-1},t_j), \\
 del(t_j)+levenshtein(s_i,t_{j-1}), \\
 sub(s_i,t_j)+levenshtein(s_{i-1},t_{j-1})
   \end{array} \right)
   \label{levequation}
\end{equation}
The total distance between $S$ and $T$ is $levenshtein(S_{|S|},T_{|T|})$.
\end{definition}
\end{frame}
\begin{frame}
  Don't worry about the details. The important part is how $ins$,
  $sub$ and $del$ are defined. Here is one of the simplest
  definitions:
\begin{definition}
\begin{equation}
\begin{array}{l}
   ins(t_j) = 1 \\
   del(s_i) = 1 \\
   sub(s_i,t_j) = \left\{
     \begin{array}{ll}
       0 & \textrm{if $s_i=t_j$} \\
       2 & \textrm{otherwise}
     \end{array} \right.

   \end{array}
\end{equation}
\end{definition}
\end{frame}
\subsection{Syntax}
\begin{frame}
  A permutation test over R over leaf-ancestor paths.
\end{frame}
\begin{frame}
  \includegraphics[width=0.45\textwidth]{GB_GOR98_A4}
\end{frame}
\begin{frame}
  \frametitle{Leaf-Ancestor Paths}
The parse tree
\[\xymatrix{
  &&\textrm{S} \ar@{-}[dl] \ar@{-}[dr] &&\\
  &\textrm{NP} \ar@{-}[d] \ar@{-}[dl] &&\textrm{VP} \ar@{-}[d]\\
  \textrm{Det} \ar@{-}[d] & \textrm{N} \ar@{-}[d] && \textrm{V} \ar@{-}[d] \\
\textrm{the}& \textrm{dog} && \textrm{barks}\\}
\]
creates the following leaf-ancestor paths:

\begin{itemize}
\item S-NP-Det-The
\item S-NP-N-dog
\item S-VP-V-barks
\end{itemize}
\end{frame}
\section{Experiment}
\begin{frame}
  \frametitle{Corpora}
  \begin{itemize}
  \item Phonology: Survey of English Dialects (SED)
  \item Syntax: International Corpus of English, Great Britain
    (ICE-GB)
  \end{itemize}
\end{frame}
\begin{frame}
  \frametitle{SED}
  \begin{itemize}
  \item Classic dialectology corpus: collected from interviews.
  \item NORMs (Non-educated, Old, Rural, Male)
  \item 
  \end{itemize}
\end{frame}
\section{Results}

\section{Practicalities}
\subsection{Useful Code}
\begin{frame}
  \frametitle{\tt iceread}
  \begin{itemize}
  \item The ICE is a nice corpus. Syntactically annotated speech. Not
    every day you find that, eh?
  \item I wrote some Python code that reads the ICE trees and allows you
    to sort and group them by speaker properties.
  \item It's available via subversion here, I guess.
  \end{itemize}
\end{frame}
\subsection{Lessons}
\begin{frame}
  \frametitle{Things I learned}
  \begin{itemize}
  \item Always, {\it always} make a repeatable build script for your experiment.
  \item But {\tt bash} isn't the language to write that script in. %Try Python.
  \item Be {\tt nice}. Watch your program execute the first time.
  \item Python's GC can be unreliable.
  \item Don't be afraid to rewrite your prototype in a faster
    language. It's not as hard as you think.
  \end{itemize}
\end{frame}
\begin{frame}
  \frametitle{A Mystery}
  \begin{itemize}
  \item The only method I could get to work was repeated sampling with
    replacement.
  \item However, shuffling the corpora together and splitting them to
    be the same size should have worked just as well.
  \item Why not? You tell me.
  \item Here's some of the code, I guess.
  \end{itemize}
\end{frame}
\end{document}
%%% Local Variables: 
%%% mode: latex
%%% TeX-master: t
%%% End: 
Read order:
Phono- and tono-genesis:
I think this is boring, but may we should read at least one paper about it. I vote for
Svantesson based purely on the fact that Ken can also throw in his comments from
his paper.

--Near mergers-- and incomplete neutralisation:
well THIS is an explosive topic. Let me see.
I like
Nycs, J. 2005. The dynamics of near-meger in accomodation. (sounds
like it mentions both perception and produciton)
Yu, A C L forthcoming might be interesting, if only to rip on. It
seems to me to be difficult to construct a believable exemplar model.
-- Incomplete neutralisation --
Warner, N, A Jongman, J Sereno, R. Kemps mentions both  production and
perception. And of course one of Bob's papers might be good because we
could bug him about his ideas.
-- Frequency effects --
Jurafsky, D,A Bell and C Girand. 2002. seems better if it talks about
both production and perception. But of course the other one probably
does too.
-- Acquisition --
This is already ordered by interestingness
Bohn, O.S.; Flege, J.E. (1990). Perception and Production of a New Vowel Category by Adult Second Language Learners. In Leather, J. \& James, A. (Eds.) New Sounds 90: Proceedings of the 1990 Amsterdam Symposium on the Acquisition of Second- 
Language Speech.Amsterdam: University of Amsterdam Press. pp. 37-56 
Bradlow, A.R., D.B. Pisoni, R. Akahane-Yamada, & Y. Tohkura (1997).  Training 
Japanese listeners to identify English /r/ and /l/, IV: Some effects of perceptual 
learning on speech production.  Journal of the Acoustical Society of America, 101: 
2299 - 2310.   
Dorman, M. F., Ausberger, C., Bailey, P., & Raphael, L. J. (1978). The relationship 
between speech perception and production in children who subsititute /t/ or /d/ for 
/s/. Journal of the Acoustical Society of America, 64(S1), p. S51 
  THIS one should be under dialect, it sounds so interesting
Flege, J. E., Munro, M. J., MacKay, I. R. A. (1995). Factors affecting strength of 
perceived foreign accent in a second language. Journal of the Acoustical Society of America, 97, 3125--3134 

- especially L2 -
-- General issues --
I thought we already talked about this.
-- Dialect --
I guess that
Brasseur 2006
Clopper and Pisoni
Markham 1999
Niedzielski 2001

-- Accommodation --
Pardo 2006 is still the best
or what about
Smiljanic and Bradlow
Giles was pretty boring and would be expensive to copy
