\documentclass[11pt,letterpaper]{article}
\pdfpagewidth=\paperwidth
\pdfpageheight=\paperheight
\usepackage{times}
\usepackage[all]{xy}
\usepackage{robbib}
\author{Nathan Sanders, Indiana University}
\title{Syntax Distance for Dialectometry}
\begin{document}
\maketitle

This dissertation will examine syntax distance in dialectometry using
computational methods as a basis. It is a continuation of my previous
work \cite{sanders07}, \cite{sanders08b} and earlier work by
\namecite{nerbonne06}, the first computational measure of syntax
distance. Dialectometry has existed as a field since
\namecite{seguy73} and is a sub-field of dialectology
\cite{chambers98}; recently, computational methods have come to
dominate dialectometry, but they are limited in focus compared to
previous work; most have explored phonological distance only, while
earlier methods integrated phonological, lexical, and syntactic data.

Dialectology is the study of linguistic variation.
Its goal is to characterize the linguistic features that separate two
language varieties. Dialectometry is a subfield of dialectology that
uses mathematically sophisticated methods to extract and combine
linguistic features. In recent years it has been associated with
computational linguistic work, most of which has focused on
phonology, starting with \namecite{kessler95} and treated most
comprehensively by \namecite{heeringa04}.

% MERGE
In dialectometry, a distance measure can be defined in two parts:
first, a method of decomposing the linguistic data into minimal,
linguistically meaningful features, and second, a method of combining
the features in a mathematically and linguistically sound way.
Dialectometry has focused on phonological distance measures, while
syntactic measures have remained undeveloped. Unfortunately,
approaching the problem by copying phonology is not a good solution.
Dialectology has traditionally worked with fairly small
corpora, which suffices for phonology, because it is easy to
automatically extract features. For syntax, though, it is not possible to
automatically identify reliable features in small corpora.

% KEEP SOME: (but summarise more?)

A second approach is to define a distance measure using the specific
properties of syntactic data. Syntactic structure is easily decomposed
into many kinds of features, and large syntactic corpora are available
for many languages. Together, these properties mean that large corpora
should provide enough evidence to rank automatically extracted
features. One such method, a simple statistical measure called $R$,
has been proposed by \namecite{nerbonne06} based on work by
\namecite{kessler01}. At present, however, $R$ has not been adequately
shown to detect dialect differences. A small body of work suggests
that it does, but to date there has not been a satisfying correlation
of its results with existing results from the dialectology literature
on syntax.

% MERGE 2
Nerbonne \& Wiersma's first paper used $R$ for syntax distance
together with a test for statistical significance\cite{nerbonne06}.
Their experiment compared two generations of Norwegian L2 speakers of
English, with part-of-speech trigrams as input features.  They found
that the two generations were significantly different. However,
showing that two generations of speakers are significantly different
with respect to $R$ does not necessarily imply that the same will be
true for other types of language varieties. Specifically, for this
dissertation, the success of $R$ on generational differences does not
imply success on dialect differences.

\namecite{sanders08b} addressed this problem by measuring $R$ between
the nine Government Office Regions of England, using the International
Corpus of English Great Britain \cite{nelson02}. Speakers were
classified by birthplace. Sanders found statistically significant
differences between most corpora, using both trigrams and
leaf-ancestor paths as features. However, $R$'s distances were not
significantly correlated with Levenshtein distances. Nor did Sanders
show any qualitative similarities between known syntactic dialect
features and the high-ranked features used by $R$ in producing its
distance. As a result, it is not clear whether the significant $R$
distances correlate with dialectometric phonological distance or with
known features found by dialectologists.

% SUMMARISE Hypotheses here

\bibliographystyle{robbib}
\bibliography{central}
\end{document}
%%% Local Variables: 
%%% mode: latex
%%% TeX-master: t
%%% End: 

E-mail to Steve

Hi Steve, would you be willing to be part of my
dissertation committee? Sandra recommended that I have another member
from linguistics who can give perspective from formal syntax,
since I will be working with computational syntax without much
background in formal syntax.

My proposed dissertation topic is syntax distance for dialectometry,
currently using a Swedish dialect corpus collected from
interviews. I've attached to two-page prospectus and I can also send
you the current draft of the proposal if you want.

E-mail to Alicia


E-mail to Markus

E-mail to Debra/Justin/Rashid pri aikido@indiana.edu
