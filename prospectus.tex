\documentclass[11pt,letterpaper]{article}
\pdfpagewidth=\paperwidth
\pdfpageheight=\paperheight
\usepackage{times}
\usepackage[all]{xy}
\usepackage{robbib}
\author{Nathan Sanders, Indiana University}
\title{Syntax Distance for Dialectometry}
\begin{document}
\maketitle

This dissertation will examine syntax distance in dialectometry using
computational methods as a basis. It is a continuation of my previous
work \cite{sanders07}, \cite{sanders08b} and earlier work by
\namecite{nerbonne06}, the first computational measure of syntax
distance. Dialectometry has existed as a field since
\namecite{seguy73} and is a sub-field of dialectology
\cite{chambers98}; recently, computational methods have come to
dominate dialectometry, but they are limited in focus compared to
previous work; most have explored phonological distance only, while
earlier methods integrated phonological, lexical, and syntactic data.

Dialectology is the study of linguistic variation.
Its goal is to characterize the linguistic features that separate two
language varieties. Dialectometry is a subfield of dialectology that
uses mathematically sophisticated methods to extract and combine
linguistic features. In recent years it has been associated with
computational linguistic work, most of which has focused on
phonology, starting with \namecite{kessler95} and treated most
comprehensively by \namecite{heeringa04}.

% MERGE
In dialectometry, a distance measure can be defined in two parts:
first, a method of decomposing the linguistic data into minimal,
linguistically meaningful features, and second, a method of combining
the features in a mathematically and linguistically sound way.
Dialectometry has focused on phonological distance measures, while
syntactic measures have remained undeveloped. Unfortunately,
approaching the problem by copying phonology is not a good solution.
Dialectology has traditionally worked with fairly small
corpora, which suffices for phonology, because it is easy to
automatically extract features. For syntax, though, it is not possible to
automatically identify reliable features in small corpora.

A better approach is to define a distance measure using the specific
properties of syntactic data. Syntactic structure is easily decomposed
into a variety of features, and large syntactic corpora are available
for many languages. Together, these properties mean that large corpora
should provide enough evidence to rank automatically extracted
features. Based on work by \namecite{kessler01}, \namecite{nerbonne06}
proposed simple measure called $R$ together with a test for
statistical significance. At present, however, $R$ has not been
adequately shown to detect dialect differences. A small body of work
suggests that it does, but to date there has not been a satisfying
correlation of its results with existing results from the dialectology
literature on syntax. For example, \namecite{nerbonne06} found
differences between two generations of L2 English speakers, and I
found differences between most regions of England \cite{sanders08b}.

% SUMMARISE Questions/Hypotheses here

Therefore, the most important question for this
proposal is whether $R$ is a good measure of syntax
distance. Specifically, have the ambiguous results of previous
research been a shortcoming of $R$, differences between phonological
and syntactic corpora, or differences between phonological and
syntactic dialect boundaries? This research will eliminate the corpus
variability in \namecite{sanders08b} that resulted in these
confounding factors. Besides the issue of internal consistency, it is
not yet clear whether $R$ agrees with traditional dialectology results
for syntax. The syntactic features it ranks highest will need to be
enumerated, then compared to the syntactic dialect features found by
dialectologists.

Besides $R$, this dissertation will propose and evaluate alternative
syntactic distance measures. Specifically, $R$ is one way to aggregate
features that are created by decomposing sentences. It treats features
as atomic, not using any internal information. As such, it is not much
different than Goebl's WIV. This may not be a problem if the
decomposition methods used to generate features adequately capture
dialect differences in independent, atomic features. If dialect
differences cannot be captured by independent, atomic features, then a
more syntax-specific method of combination will be needed
instead.
% TODO: WIV, also Kullbeck-Leibler Divergence could work.
% Maybe also k-NN/MBL, HMM binary classifier (?), maybe even a
% neural net

Two secondary questions become relevant once a useful syntax distance
measure is established. First is what features provide the best
results. My previous work has shown that leaf-ancestor paths provide a
small advantage over part-of-speech trigrams, presumably by capturing
syntactic structure higher in the parse tree. Additional feature sets
will be evaluated to find additional ways to represent syntactic
information. Another question is whether $R$ agrees with phonological
distance measures like Levenshtein distance.  Unlike agreement with
traditional dialectology, there is no {\it a priori} reason to expect
agreement between phonology and syntax in delineating dialect
boundaries.


\subsection{Hypotheses}

Hypothesis 1: The features found by dialectologists will agree with
the highly ranked features used by $R$ for classification.  This will
be tested by comparing $R$'s results to the syntactic dialectology
literature. First, and most obviously, the broad regions accepted by
dialectology will be reproduced by the classifier. For example, my
previous research on British English reproduced the well-known
North-South distinction. In addition, $R$ will agree substantially with the
between-region differences of syntactic dialectology--where
dialectologists find differences between regions, $R$ should find a
significant difference as well. What's more, the specifics of the
difference from dialectology should be reflected in the high-ranked
features used by $R$ to distinguish that region from others.

Hypothesis 2: $R$ will produce more accurate syntax distances for some
features than others. The best features can be discovered by
comparing performance of a number of different feature sets on a fixed
corpus. In addition, combinations of the most successful features will
produce even better performance. (as measured \ldots how?)

Hypothesis 3: A phonology corpus and syntax corpus based on the same
corpus will provide better correlation between phonology and syntax
distance measures than a phonology corpus and syntax corpus based on
separate corpora. This type of corpus will provide the best
opportunity to discover whether phonological and syntactic dialect
boundaries correlate. This hypothesis will be tested by comparing results to
previous work on separate corpora.
% (Or perhaps whether phonological and syntactic dialect
% measures correlate? I'm not sure of this yet)

\bibliographystyle{robbib}
\bibliography{central}
\end{document}
%%% Local Variables: 
%%% mode: latex
%%% TeX-master: t
%%% End: 

E-mail to Steve

Hi Steve, would you be willing to be part of my
dissertation committee? Sandra recommended that I have another member
from linguistics who can give perspective from formal syntax,
since I will be working with computational syntax without much
background in formal syntax.

My proposed dissertation topic is syntax distance for dialectometry,
currently using a Swedish dialect corpus collected from
interviews. I've attached to two-page prospectus and I can also send
you the current draft of the proposal if you want.

E-mail to Markus
Hi Markus, would you be willing to be part of my dissertation committee?
Since I am doing computational syntax, you are a logical member.

The current title of my dissertation proposal is Syntax Distance for
Dialectometry. I've attached the prospectus and I can also send you
the current draft of the proposal if you want.

E-mail to Mike
Hi Mike, I am forming my dissertation committee. Would you be willing
to continue on? I am continuing the syntax distance work from my first
qualifying paper.

The current title of my dissertation proposal is Syntax Distance for
Dialectometry. I've attached the prospectus and I can also send you
the current draft of the proposal if you want.
