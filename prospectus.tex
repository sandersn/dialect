\documentclass[11pt]{article}
\author{Nathan Sanders, Indiana University \\ \tt{ncsander@indiana.edu}}
\title{Syntax Distance in Dialectometry : A Two Page Prospectus}
\begin{document}
\maketitle
%
%TODO:Link to Scala table and Solving Cryptograms
% Not About Me
% I am not 34 years old
% I do not speak Japanese
% I was not born in Guam

This proposal proposes to continue my existing line of research on
dialect distance using syntactic features. My previous research has
focussed on statistical measures as pioneered by Nerbonne \& Wiersma
and extended their features to better capture complex syntactic
structure.

\section{Dialectometry}

Dialectometry is the quantitative linguistic study of differences
between languages. It is a subset of dialectology, which looks for
boundaries between languages. However, unlike classic dialectology, it
does not look for strict boundaries (in the form of isogloss bundles)
between languages based on boundaries specified by multiple
variables. Instead, dialectometry first defines a distance measure
that combines information from multiple variables. The result
specifies gradient boundaries, not absolute boundaries.

Dialectology is related to sociolinguistics--the two often work with
the same data. But the questions are different: sociolinguistics asks
Why (and What) while dialectometry asks How Much (and What).
In other words, sociolinguistics analyses the conditions that cause
differences between languages, such as geographic separation, social
separation, code-switching or social climbing. For dialectometry, it
does not matter which dimension is used; the only concern is correlating
the linguistic differences with the areas specified by the
sociological variable. Historically, geographic separation is the most
usual application, but others are certainly possible and some have
been investigated (Sanders, Gooskens, another of John's students)

\section{Previous Work}

Results were inconclusive for British dialects when comparing the
syntactic measure with a well-known phonological measure, Levenshtein
distance. I will address this in several ways, deepening and
complicating the study in the process.

First, a change of corpus. This
has two improvements: more data, and better data. In the ICE, nobody
really talked that much, and everybody was in London College at the
time, so even if they were born out in the country, they were trying
to sound Proper. The Scandinavian Dialect Corpus
(Nordisk dialektkorpus) is recorded mostly on location, in traditional
dialectology style, old people spread somewhat evenly throughout the
country.

Second, an expansion in methods. In the previous work on the ICE
corpus, only trigrams and leaf-ancestor paths are used as a source of
features for the statistical classifier. Only leaf-ancestor paths are
sensitive enough to give significant results. With more data, more
methods should be sensitive enough to produce sensible
results. Therefore, in addition to constituent-based leaf-ancestor
paths, a similar measure will be developed for dependency parses.

This will require some additional preparation relative to experiments
run on the ICE, which is already parsed by a constituent grammar. The
unparsed Scandinavian Dialect Corpus will have to be parsed by
machine. But this provides an opportunity to use multiple parsing
methods. There exist Swedish treebanks parsed by both constituent parsers and
dependency parsers.

\end{document}
%%% Local Variables: 
%%% mode: latex
%%% TeX-master: t
%%% End: 

