\documentclass[11pt]{article}
\begin{document}
\section*{A Two Page Prospectus}
\section*{Prospects for Research}

This proposal proposes to continue my existing line of research on
dialect distance using syntactic features. My previous research has
focussed on statistical measures as pioneered by Nerbonne \& Wiersma
and extended their features to better capture complex syntactic
structure.

Results were inconclusive for British dialects when comparing the
syntactic measure with a well-known phonological measure, Levenshtein
distance. I will address this in several ways, deepening and
complicating the study in the process.

First, a change of corpus. This
has two improvements: more data, and better data. In the ICE, nobody
really talked that much, and everybody was in London College at the
time, so even if they were born out in the country, they were trying
to sound Proper. The Scandinavian Dialect Corpus
(Nordisk dialektkorpus) is recorded mostly on location, in traditional
dialectology style, old people spread somewhat evenly throughout the
country.

Second, an expansion in methods. In the previous work on the ICE
corpus, only trigrams and leaf-ancestor paths are used as a source of
features for the statistical classifier. Only leaf-ancestor paths are
sensitive enough to give significant results. With more data, more
methods should be sensitive enough to produce sensible
results. Therefore, in addition to constituent-based leaf-ancestor
paths, a similar measure will be developed for dependency parses.

This will require some additional preparation relative to experiments
run on the ICE, which is already parsed by a constituent grammar. The
unparsed Scandinavian Dialect Corpus will have to be parsed by
machine. But this provides an opportunity to use multiple parsing
methods. There exist Swedish treebanks parsed by both constituent parsers and
dependency parsers.

\end{document}
%%% Local Variables: 
%%% mode: latex
%%% TeX-master: t
%%% End: 


TODO:Link to Scala table and Solving Cryptograms

-----------
2D and 3D: Commander Keen, Sam \& Max and Half-Life



Review of Commander Keen : Goodbye Galaxy and Sam \& Max 104 (and
Peggle Extreme!)

So I finished Commander Keen : Goodbye Galaxy. Even the for-pay half I
never bought as a kid. I suspect my nostalgia made the free half
rosier, but it really seems superior. The earth, tree and
sand tones are not garish like the purple/orange/black of the machines
in the space station that make up The Armageddon Machine. Also, id
was way too much into episodic gaming back then: even the for-pay half
ends with a teaser for the next episode, which I assume was Aliens Ate
My Babysitter, a game that is <i>not</i> for sale on Steam. I guess
games back then were a lot more blatant about setting up the plot for
a sequel.

At the time, the unique thing about Goodbye Galaxy was the combination
of sprawling levels, low difficulty level (on easy), and multiple
paths. Mario World has multiple paths, but encouraged you to replay
each level multiple times; the effect is one of exploration--looking
for hidden areas. Sonic has multiple paths, but you generally see only
one on your rush to the finish line; the effect is to make each replay
of the whole game unique. With Goodbye Galaxy, like Sonic and unlike
Mario World, you can't replay levels and few of the hidden areas lead
to new levels. But, like Mario, you move quite slowly and there are
many hidden areas. The result is similar to games 5 to 7 years later:
collectionism. Sure, you made it to the end, but did you get all the
points in the whole level?

Unlike later games, starting more or less with Donkey Kong Country,
there's no reward for finding all that stuff. High scores give you
some 1ups and that's it.

The genius of Goodbye Galaxy is in the collection of things,
though. On easy, you really aren't in danger much if your only
ambition is to get to the end. But if you want to collect everything,
you die the death of a thousand cuts. The challenges that are
individually easy will eventually kill you, and you get to start over. I
rushed through the last half of the Armageddon Machine, restraining
myself from collecting the paltry 2000 points guarded by two bots AND
a death bot.

Why do I feel the urge to collect everything in Commander Keen? I
hated what little Banjo/Kazooie I played because of all the required
collectables. Maybe requiring them and quantifying how much I've
missed annoys me; I don't know it when I've missed Super Secret Room
in Commander Keen and so my completionist tendencies are satisfied by
collecting what I can see.

Sam \& Max 104 is a Free Game that ...

Peggle Extreme is a Free Game, too, although really it's a demo for
Peggle Deluxe that is specially Extremified for the HARD CORE gamers
who use Steam. The king chipmunk and his spokesman unicorn talk about
head crabs invading and the backgrounds have storage cubes and the
occasional portal. Although if the music has been Extremified then the
original has got to be a lullaby because the music is classic Drone FM
fodder. OK, maybe not that laid back. But close.

Peggle is weird because the first I heard of it was Stevey, who
compared it to The Price Is Right's version of Pachinko. That's
basically right, or at least it appears that way. The funny thing is
that when I'm paying attention, I can win about 75\% of the time. When
I'm just aiming any which way, I win about 30\% of the time. So there
is clearly some kind of skill involved. I'm just not sure what it is.

For those of you who don't know, Peggle has you aiming little metal balls
from the top of the screen, where they bounce around on little pegs
until they fall off the bottom of the screen. Your job is to hit the
orange pegs.
