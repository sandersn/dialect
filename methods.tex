\chapter{Methods}
It's all about R and running it on Swedish eh.

There is a lot of background information, like other distance
measures and the underlying math of the distance measure I did use.

\section{Previous Work}

\subsection{S\'eguy}

Measurement of linguistic similarity has always been a part of
linguistics. However, until \namecite{seguy73} dubbed a new set of
approaches `dialectometry', these methods lagged behind the rest of
linguistics in formality. S\'eguy's quantitative analysis
of Gascogne French, while not aided by computer, was the predecessor
of more powerful statistical methods that essentially required the use
of computer as well as establishing the field's general dependence on
well-crafted dialect surveys that divide incoming data along
traditional linguistic boundaries: phonology, morphology, syntax, etc.
This makes both collection and analysis easier, although it requires
more work to combine separate analyses to produce a complete picture of dialect
variation.

The project to build the Atlas Linguistique et Ethnographique de la
Gascogne, which S\'eguy directed, collected data in a dialect survey
of Gascogne which asked speakers questions informed by different areas
of linguistics. For example, the pronunciation of `dog' ({\it chien})
was collected to measure phonological variation. It had two common
variants and many other rare ones: [k\~an], [k\~a], as well as [ka],
[ko], [kano], among others. These variants were, for the most part,
% or hat "chapeau": SapEu, kapEt, kapEu (SapE, SapEl, kapEl
known by linguists ahead of time, but their exact geographical
distribution was not.

The atlases, as eventually published, contained not only annotated
maps, but some analyses as well. These analyses were what S\'eguy named
dialectometry. Dialectometry differs from previous attempts to find
dialect boundaries in the way it combines information from the
dialect survey. Previously, dialectologists found isogloss
boundaries for individual items. A dialect boundary was generated when
enough individual isogloss boundaries coincided. However, for any real
corpus, there is so
much individual variation that only major dialect boundaries can
be captured this way.

S\'eguy reversed the process. He first combined survey data to get
a numeric score between each site. Then he posited dialect boundaries
where large distances resulted between sites. The difference is
important, because a single numeric score is easier to
analyze than hundreds of individual boundaries.
Much more subtle dialect boundaries are visible this way; where before
one saw only a jumble of conflicting boundary lines, now one sees
smaller, but consistent, numerical differences separating regions. {Dialectometry
  enables classification of gradient dialect boundaries, since now one
can distinguish weak and strong boundaries. Previously, weak
boundaries were too uncertain.}

However, S\'eguy's method of combination is simple both
linguistically and mathematically. When comparing two sites, any
difference in a response is counted as 1. Only identical
responses count as a distance of 0. Words are not analyzed
phonologically, nor are responses weighted by their relative amount
of variation. Finally, only geographically adjacent sites are
compared. This is a reasonable restriction, but later studies were
able to lift it because of the availability of greater computational
power. Work following S\'eguy's improves on both aspects. In
particular, Hans Goebl developed dialectometry models that are
more mathematically sophisticated.

\subsection{Goebl}

Hans Goebl emerged as a leader in the field of dialectometry,
formalizing the aims and methods of dialectometry. His primary
contribution was development of various methods to combine individual
distances into global distances and global distances into global clusters. These
methods were more sophisticated mathematically than previous
dialectometry and operated on any features extracted from the data. His
analyses have used primarily the Atlas Linguistique de Fran\c{c}ais.

\namecite{goebl06} provides a summary of his work. Most relevant for
this paper are the measures Relative Identity Value and Weighted
Identity Value. They are general methods that are the basis for nearly
all subsequent fine-grained dialectometrical analyses. They have three
important properties. First, they are independent of the source
data. They can operate over any linguistic data for which they are
given a feature set, such as the one proposed by \namecite{gersic71} for
phonology. Second, they can compare data even for items that do not
have identical feature sets, such as Ger\v{s}i\'c's $d$,
which cannot compare consonants and vowels. Third, they can compare
data sets that are missing some entries. This improves on S\'eguy's
analysis by providing a principled way to handle missing survey
responses.

Relative Identity Value, when comparing any two items, counts the
number of features which share the same value and then discounts
(lowers) the importance of the result by the number of unshared
features. The result is a single percentage that indicates
relative similarity. Calculating this distance between all pairs
of items in two regions produces a matrix which can be used for
clustering or other purposes. Note that the presentation below splits
Goebl's original equations into more manageable pieces; the high-level
equation for Relative Identity Value is:

\begin{equation}
  \frac{\textrm{identical}_{jk}} {\textrm{identical}_{jk} - \textrm{unidentical}_{jk}}
\label{riv}
\end{equation}
For some items being compared $j$ and $k$. In this case
\textit{identical} is
\begin{equation}
  \textrm{identical}_{jk} = |f \in \textrm{\~N}_{jk} : f_j = f_k|
\end{equation}
where $\textrm{\~N}_{jk}$ is the set of features shared by  $j$ and
$k$ and $f_j$ and $f_k$ are the value of some feature $f$ for $j$ and
$k$ respectively. \textit{unidentical} is defined similarly, except
that it counts all features N, not just the shared features
$\textrm{\~N}_{jk}$.

\begin{equation}
  \textrm{unidentical}_{jk} = |f \in \textrm{N} : f_j \neq f_k|
\end{equation}

Weighted Identity Value is a refinement of Relative Identity
Value. This measure defines some differences as more
important than others. In particular, feature values that only occur
in a few items give more information than feature values that appear
in a large number of items. This
idea shows up later in the normalization of syntax distance given by
\namecite{nerbonne06}.

The mathematical reasoning behind this idea is fairly simple. Goebl
is interested in feature values that occur in only a few items. If a
feature has some value that is shared by all of the items, then all
items belong to the same group. This feature value provides {\it no}
useful information for distinguishing the items.  The situation
improves if all but one item share the same value for a feature; at
least there are now two groups, although the larger group is still not
very informative.  The most information is available if each item
being studied has a different value for a feature; the items fall
trivially into singleton groups, one per item.

Equation \ref{wiv-ident} implements this idea by discounting
the \textit{identical} count from equation \ref{riv} by
the amount of information that feature value conveys. The
amount of information, as discussed above, is based on the number of
items that share a particular value for a feature. If all items share
the same value for some feature, then \textit{identical} will be discounted all the
way to zero--the feature conveys no useful information.
Weighted Identical Value's equation for \textit{identical} is
therefore
\begin{equation}
  \textrm{identical} = \sum_f \left\{
  \begin{array}{ll}
    0 & \textrm{if} f_j \neq f_k \\
    1 - \frac{\textrm{agree}f_{j}}{(Ni)w} & \textrm{if} f_j = f_k
  \end{array} \right.
\label{wiv-ident}
\end{equation}

\noindent{}The complete definition of Weighted Identity Value is
\begin{equation} \sum_i \frac{\sum_f \left\{
  \begin{array}{ll}
    0 & \textrm{if} f_j \neq f_k \\
    1 - \frac{\textrm{agree}f_j} {(Ni)w} & \textrm{if} f_j = f_k
\end{array} \right.}
  {\sum_f \left\{
  \begin{array}{ll}
    0 & \textrm{if} f_j \neq f_k \\
    1 - \frac{\textrm{agree}f_j} {(Ni)w} & \textrm{if} f_j = f_k
    \end{array} \right. - |f \in \textrm{N} : f_j \neq f_k|}
  \label{wiv-full}
  \end{equation}

  \noindent{}where $\textrm{agree}f_{j}$ is the number of items that agree
  with item $j$ on feature $f$ and $Ni$ is the total number of
  items ($w$ is the weight, discussed below). Because of the
  piecewise definition of \textit{identical}, this number is always at
  least $1$ because $f_k$ agrees already with $f_j$.
  This equation takes the count of shared features and weights
  them by the size of the sharing group. The features that are shared
  with a large number of other items get a larger fraction of the normal
  count subtracted.

  For example, let $j$ and $k$ be sets of productions for the
  underlying English segment /s/. The allophones of /s/ vary mostly on the feature
  \textit{voice}. Seeing an unvoiced [s] for /s/ is less ``surprising'' than
  seeing a voiced [z], so the discounting process should
  reflect this. For example, assume that an English corpus contains 2000
  underlying /s/ segments. If 500 of them are realized as [z], the
  discounting for \textit{voice} will be as follows:

  \begin{equation}
    \begin{array}{c}
      identical_{/s/\to[z]} = 1 - 500/2000 = 1 - 0.25 = 0.75 \\
      identical_{/s/\to[s]} = 1 - 1500/2000 = 1 - 0.75 = 0.25
    \end{array}
    \label{wiv-voice}
  \end{equation}

  Each time /s/ surfaces as [s], it only receives 1/4 of a point
  toward the agreement score when it matches another [s]. When /s/
  surfaces as [z], it receives three times as much for matching
  another [z]: 3/4 points towards the agreement score. If the
  alternation is even more weighted toward faithfulness, the ratio
  changes even more; if /s/ surfaces as [z] only 1/10 of the time,
  then [z] receives 9 times more value for matching than [s] does.

  The final value, $w$, which is what gives the name ``weighted
  identity value'' to this measure, provides a way to control how much
  is discounted. A high $w$ will subtract more from uninteresting
  groups, so that \textit{voice} might be worth less than
  \textit{place} for /t/ because /t/'s allophones vary more over
  \textit{place}. In equation \ref{wiv-voice}, $w$ is left at 1 to
  facilitate the presentation.

\section{Statistical Methods} % Computational? Mathematical?

It is at this point that the two types of analysis, phonological and
syntactic, diverge. Although Goebl's techniques are general enough to
operate over any set of features that can be extracted, better results
can be obtained by specializing the general measures above to take
advantage of properties of the input.  Specifically, the application
of computational linguistics to dialectometry beginning in the 1990s
introduced methods from other fields. These methods, while generally
giving more accurate results quickly, are tied to the type of data on
which they operate.

% NEW
Currently, the dominant phonological distance measure is Levenshtein
distance. This distance is essentially the count of differing
segments, although various refinements have been tried, such as
inclusion of distinctive features or phonetic
correlates. \namecite{heeringa04} gives an excellent analysis of the
applications and variations of Levenshtein distance. While Levenshtein
distance provides much information as a classifier, it is limited
because it must have a word aligned corpus for comparison. A number of
statistical methods have been proposed that remove this requirement
such as \namecite{hinrichs07} and \namecite{sanders09}, but none have
been as successful on existing dialect resources, which are small and
are already word-aligned. New resources are not easy to develop
because the statistical methods still rely on a phonetic transcription
process.
% end NEW

% \begin{enumerate}
% \item I should really check around to see if there is any new work out
%   there. Surely there is. Course John is free to do whatever works and
%   Wybo may have graduated or something. So there might not be any more
%   work on it.
% \item Explain leaf-ancestor paths, trigrams, dependency `paths' (to be
%   invented).
% \end{enumerate}

\subsection{Syntactic Distance}

Recently, computational dialectometry has expanded to analysis of
syntax as well. The first work in this area was \quotecite{nerbonne06}
analysis of Finnish L2 learners of English, followed by
\quotecite{sanders07} analysis of British dialect areas. Syntax
distance must be approached quite differently than phonological
distance. Syntactic data is extractable from raw text, so it is much
easier to build a syntactic corpus. But this implies an associated
drop in manual linguistic processing of the data. As a result, the
principal difference between present phonological and syntactic
corpora is that phonology data is word-aligned, while syntax data is
not sentence-aligned. Automatically constructed syntactic corpora
lead naturally to statistical measures over large amounts of data
rather than more sensitive measures that operate on small corpora.

\subsubsection{Nerbonne and Wiersma}
\label{nerbonne06}

Due to the lack of alignment between the
larger corpora available for syntactic analysis, a statistical
comparison of differences is more appropriate than the simple
symbolic approach possible with the word-aligned corpora used in
phonology. This statistical approach means that a syntactic distance
measure will have to use counting as its basis.

\namecite{nerbonne06} was an early method proposed for syntactic
distance.  It models syntax by part-of-speech (POS) trigrams and uses
differences between trigram type counts in a permutation test of
significance. This method was extended by \namecite{sanders07}, who
used \quotecite{sampson00} leaf-ancestor paths as an alternate basis
for building the model.

The heart of the measure is simple: the difference in type counts
between the combined types of two corpora. \namecite{kessler01}
originally proposed this measure, the {\sc Recurrence}
metric ($R$):

\begin{equation}
R = \Sigma_i |c_{ai} - c_{bi}|
\label{rmeasure}
\end{equation}

\noindent{}Given two corpora $a$ and $b$, $c_a$ and $c_b$ are the type
counts. $i$ ranges over all types, so $c_{ai}$ and $c_{bi}$ are the
type counts of corpora $a$ and $b$ for type $i$.  $R$ is designed to
represent the amount of variation exhibited by the two corpora while
the contribution of individual types remains transparent to aid later
analysis.

To account for differences in corpus size, sampling with replacement is
used. In addition, the samples are normalized to account for
differences in sentence length and complexity.  Unfortunately, even normalized, the
measure doesn't indicate whether its results are significant; a
permutation test is needed for that.

\subsubsection{Normalization}
The distance must be normalized for two kinds of variation:
the length of sentences in the corpus and the amount of variety in the
corpus. If sentence length differs too much between corpora, there
will be consistently lower token counts in one corpus, which would
cause a spuriously large $R$. In addition, if one corpus has less
variety than the other, it will have inflated type counts, because
more tokens will be allocated to fewer types. To avoid
this, all tokens are scaled by the average number of types per token
across both corpora: $2n/N$ where $n$ is the type count and $N$ is
the token count. The additional factor $2$ is necessary because we are
recombining the tokens from the two corpora.

% Other ideas include training a
% model on one area and comparing the entropy (compression) of other
% areas. At this point it's unclear whether this would provide a
% comparable measure, however.

\subsubsection{Language models}
\label{syntactic-features}
\namecite{nerbonne06} argue that POS trigrams can accurately represent
at least the important parts of syntax, similar to the way chunk
parsing can capture the most important information about a
sentence. If this is true, POS trigrams are a good starting point for
a language model; they are simple and easy to obtain in a number of
ways. They can either be generated by a tagger as Nerbonne
and Wiersma did, or taken from the leaves of the trees of a
syntactically annotated corpus as \namecite{sanders07} did with the
International Corpus of English.

On the other hand, it might be better to represent the upper structure
of trees, assuming that syntax is in fact a phenomenon that extends beyond the
lexical. \quotecite{sampson00} leaf-ancestor paths provide one way to
do this: for each leaf in the tree, leaf-ancestor paths produce the
path from that leaf back to the root. Generation is simple as long as
every sibling is unique. For example, the parse tree
\[\xymatrix{
  &&\textrm{S} \ar@{-}[dl] \ar@{-}[dr] &&\\
  &\textrm{NP} \ar@{-}[d] \ar@{-}[dl] &&\textrm{VP} \ar@{-}[d]\\
  \textrm{Det} \ar@{-}[d] & \textrm{N} \ar@{-}[d] && \textrm{V} \ar@{-}[d] \\
\textrm{the}& \textrm{dog} && \textrm{barks}\\}
\]
creates the following leaf-ancestor paths:

\begin{itemize}
\item S-NP-Det-The
\item S-NP-N-dog
\item S-VP-V-barks
\end{itemize}

For identical siblings, brackets must be inserted in the path to
disambiguate the first sibling from the second.
There is one path for each word, and the root appears
in all four. However, there can be ambiguities if some
node happens to have identical siblings. Sampson gives the example
of the two trees
\[\xymatrix{
  &&\textrm{A} \ar@{-}[dl] \ar@{-}[dr] &&&\\
  &\textrm{B} \ar@{-}[d] \ar@{-}[dl] &&\textrm{B} \ar@{-}[d] \ar@{-}[dr] & \\
  \textrm{p} & \textrm{q} && \textrm{r} & \textrm{s} \\
}
\]
and
\[\xymatrix{
  &&\textrm{A} \ar@{-}[d] &&&\\
  &&\textrm{B} \ar@{-}[dll] \ar@{-}[dl] \ar@{-}[dr] \ar@{-}[drr]&&& \\
  \textrm{p} & \textrm{q} && \textrm{r} & \textrm{s} \\
}
\]
which would both produce

  \begin{itemize}
  \item A-B-p
  \item A-B-q
  \item A-B-r
  \item A-B-s
  \end{itemize}

  There is no way to tell from the paths which leaves belong to which
  B node in the first tree, and there is no way to tell the paths of
  the two trees apart despite their different structure. To avoid this
  ambiguity, Sampson uses a bracketing system; brackets are inserted
  at appropriate points to produce
  \begin{itemize}
  \item $[$A-B-p
  \item A-B]-q
  \item A-[B-r
  \item A]-B-s
  \end{itemize}
and
  \begin{itemize}
  \item $[$A-B-p
  \item A-B-q
  \item A-B-r
  \item A]-B-s
  \end{itemize}

Left and right brackets are inserted: at most one
in every path. A left bracket is inserted in a path containing a leaf
that is a leftmost sibling and a right bracket is inserted in a path
containing a leaf that is a rightmost sibling. The bracket is inserted
at the highest node for which the leaf is leftmost or rightmost.

It is a good exercise to derive the bracketing of the previous two trees in deta
il.
In the first tree, with two B
siblings, the first path is A-B-p. Since $p$ is a leftmost child,
a left bracket must be inserted, at the root in this case. The
resulting path is [A-B-p. The next leaf, $q$, is rightmost, so a right
bracket must be inserted. The highest node for which it is rightmost
is B, because the rightmost leaf of A is $s$. The resulting path is
A-B]-q. Contrast this with the path for $q$ in the second tree; here $q$
is not rightmost, so no bracket is inserted and the resulting path is
A-B-q. $r$ is in almost the same position as $q$, but reversed: it is the
leftmost, and the right B is the highest node for which it is the
leftmost, producing A-[B-r. Finally, since $s$ is the rightmost leaf of
the entire sentence, the right bracket appears after A: A]-B-s.

At this point, the alert reader will have
noticed that both a left bracket and right bracket can be inserted for
a leaf with no siblings since it is both leftmost and rightmost. That is,
a path with two brackets on the same node could be produced: A-[B]-c. Because
of this redundancy, single children are
excluded by the bracket markup algorithm. There is still
no ambiguity between two single leaves and a single node with two
leaves because only the second case will receive brackets.

% See for yourself:
% \[\xymatrix{
%   &\textrm{A} \ar@{-}[dl] \ar@{-}[dr] &\\
%   \textrm{B} \ar@{-}[d] &&\textrm{B} \ar@{-}[d] \\
%   \textrm{p} && \textrm{q} \\
% }
% \]

% \[\xymatrix{
%   &\textrm{A} \ar@{-}[d] &\\
%   &\textrm{B} \ar@{-}[dl] \ar@{-}[dr] & \\
%   \textrm{p} && \textrm{q} \\
% }
% \]
% \cite{sampson00} also gives a method for comparing paths to obtain an
% individual path-to-path distance, but this is not necessary for the
% permutation test, which treats paths as opaque symbols.


Sampson originally developed leaf-ancestor paths as an improved
measure of similarity between gold-standard and machine-parsed trees,
to be used in evaluating parsers. The underlying idea of a collection of
features that capture distance between trees transfers quite nicely to
this application. \namecite{sanders07} replaced POS trigrams with
leaf-ancestor paths for the ICE corpus and found improved results on
smaller corpora than Nerbonne and Wiersma had tested. The additional
precision that leaf-ancestor paths provide appears to aid in attaining
significant results.

% Another idea is supertags rather than leaf-ancestor paths. This is
% quite similar but might work better.
\subsubsection{Leaf-Head Paths}
% TODO: This section should probably have a lot more examples and maybe some
% examples of other applications besides my experiment.

For dependency annotations, it is easy to adapt leaf-ancestor paths to
leaf-head paths. Here, each leaf is associated with a leaf-head path,
the path from the leaf to the head of the sentence via the
intermediate heads. For example, the same sentence, ``The dog barks'',
produces the following leaf-head paths.

\begin{itemize}
\item root-V-N-Det-the
\item root-V-N-dog
\item root-V-barks
\end{itemize}

The biggest difference is in the relative length of the paths: long
leaf-ancestor paths indicate deep nesting of structure. Length is a
weaker indicator of deep structure for leaf-head
paths; sometimes a difference in length indicates only a difference in
centrality to the sentence. % or something, this is still kind of
                            % wrong
\[\xymatrix{
& & root \\
DET \ar@/^/[r] & NP\ar@/^/[r] & V \ar@{.>}[u] \\
The & dog & barks
}
\]

\subsection{Previous Experiments}

\namecite{nerbonne06} were the first to use the syntactic distance
measure described above. They analyzed two corpora, both of Norwegian
L2 speakers of English. The first corpus was gathered from speakers
who learned English after childhood and the second was gathered from
speakers who learned English as children. Nerbonne \& Wiersma found a
significant difference between the two corpora. The trigrams that
contributed most to the difference were those in the older corpus that
are unexpected in English. For example, the trigram COP-ADJ-N/COM is
not common in English because a noun phrase following a copula
typically begins with a determiner. Other trigrams indicate
hypercorrection on the part of the older speakers; they appear in the
younger corpus but not as often. Nerbonne \& Wiersma analyzed this as
interference from Finnish; the younger learners of English learned it
more completely with less interference from Finnish.

Subsequent work by \namecite{sanders07} and \namecite{sanders08b}
expanded on the Norwegian experiment in two ways. First, it introduced
leaf-ancestor paths as an alternative feature type. Second, it tested
the distance method on a larger set of corpora: Government Office
Regions of England, as well as Scotland and Wales, for a total of
11 corpora. Each was smaller than the Norwegian L2 corpora, so the
permutation test parameters had to be adjusted for some feature
combinations.

The distances between regions were clustered using hierarchical
agglomerative clustering, as described in section \ref{cluster-analysis}. The resulting tree showed a North/South
distinction with some unexpected differences from previously
hypothesized dialect boundaries; for example, the
Northwest region clustered with the Southwest region. This contrasted
with the clustered phonological distances also produced in
\namecite{sanders08b}. In that experiment,
there was no significant correlation between the inter-region
phonological distances and syntactic distances.

There are several possible reasons for this lack of correlation. The
two distance measures may find different dialect boundaries based on
differences between syntax and phonology. Dialect boundaries may have
shifted during the 40 years between the collection of the SED and the
collection of the ICE-GB. One or both methods may be measuring the
wrong thing. However, I will not investigate the relation between
phonology and syntax in this dissertation. The focus will remain on results
of computational syntax distance as compared to traditional syntactic
dialectology.

\section{Methods}
To investigate the first hypothesis, I need a dialect corpus that can
be syntactically annotated (\ref{syntactically-annotated-corpus}); if
it is not already annotated, it must be possible to annotate it
automatically so I can avoid time-consuming manual annotation.
Automatic annotation will require a syntactically annotated
training corpus (\ref{syntactically-annotated-training}) and a parser
(\ref{parsers}). A distance measure must be defined for the regions
within the dialect corpus (\ref{nerbonne06}), syntactic features must
be extracted for the distance measures (\ref{syntactic-features}), and
the results tested for significance (\ref{permutationtest}) and
clustered (\ref{cluster-analysis}) to determine which dialect regions
are found by the corpus. Finally, the most highly ranked features used
to produce the dialect distances must be enumerated
(\ref{feature-ranking}).

To investigate the second hypothesis, I need a method to combine
different types of features (\ref{combine-feature-sets}) and back off
sparse features (\ref{feature-backoff}). I also need a way to generate
new features that include more information about context
(\ref{alternate-feature-sets}).

If the distance measure $R$ doesn't provide any significant distances
with any combination of features, I will experiment with different distance
measures. For this, there are quite a few possibilities;
Kullbeck-Leibler divergence is one example (\ref{kl-divergence}).

To investigate the third hypothesis, I need a phonological corpus and a method
for calculating phonological dialect distance, then a method to compare
phonological clusters with syntactic clusters. See my qualifying paper
\cite{sanders08b} for details.

\subsection{SweDiaSyn} % This is not a good subsection for the new organization
\label{syntactically-annotated-corpus}
The first hypothesis requires a dialect corpus that can
be syntactically annotated.
The dialect corpus used in this dissertation will be SweDiaSyn, the
Swedish part of the ScanDiaSyn.
% (CITE SweDiaSyn and ScanDiaSyn,
% except that they don't seem to have any references)
% Here is a citation for ScanDiaSyn if I could track it down and
% translate it
% Vangsnes, �ystein A. 2007. ScanDiaSyn: Prosjektparaplyen Nordisk dialektsyntaks. In T. Arboe (ed.), Nordisk dialektologi og sociolingvistik, Peter Skautrup Centeret for Jysk Dialektforskning, �rhus Universitet. 54-72.
SweDiaSyn is a transcription of SweDia 2000 \cite{bruce99} collected
between 1998 and 2000 from 97 locations in Sweden and 10 in
Finland. Each location has 12 interviewees: three 30-minute interviews
for each of older male, older female, younger male and younger female.
However, the SweDiaSyn transcriptions do not yet include all of SweDia
2000; the completed transcriptions currently focus on older
speakers.

Currently there are 36,713 sentences of transcribed speech
from 49 sites, an average of 749 sentences per site.
However, the sites range from 110 to 1780 sentences because some sites
have fewer complete transcriptions than others. In order to detect
significant differences, the sites may need to be grouped by county,
traditional province or EU region; previous work on British English
used EU Government Office Regions with at least 850 sentences per
region. For example, grouping the Swedish corpora into the 25 provinces
boosts the average sentences per province to 1254, excluding provinces
with no transcriptions.

% TODO: Probably switch the second sentence to be first? It's the more
% important but might completely depend on details in the first.
In the SweDiaSyn, there are two types of transcription:
standard Swedish orthography, with glosses for words
not in standard Swedish, and a phonetic transcription for dialects
that differ greatly from standard Swedish. For this dissertation,
the orthographic/gloss transcription will be used so that lexical
items will be comparable across dialects.

\subsection{Talbanken}
\label{syntactically-annotated-training}

Because the first hypothesis requires a syntactically annotated
corpus, and because SweDiaSyn consists of untagged lexical items,
Talbanken05, a syntactically-annotated corpus, will be used to train a
POS tagger and parsers to be used to annotate SweDiaSyn.  Talbanken05
is a treebank of written and transcribed spoken Swedish, roughly
300,000 words in size. It is an updated version of Talbanken76
\cite{nivre06}; Talbanken76's trees are annotated following a custom
scheme called MAMBA; Talbanken05 adds phrase structure annotation and
dependency annotation using the standard annotation formats TIGER-XML
and Malt-XML.  In addition to syntactic annotation, Talbanken is
lexically annotated for morphology and part-of-speech.

% TODO: Should I keep this? It depends on how much detail I want.
% Talbanken's sources are X and Y and Z. It attempts to provide a
% valid sample of the Swedish language, both spoken and written. The
% spoken section is transcribed from conversation, interviews and
% debates, and the written section is taken from high school essays and
% professional prose (TODO:I could probably cite
% Jan Einarsson. 1976. Talbankens skriftspraakskonkordans. Lund
% University: Department of Scandinavian Languages (and
% talspraakskonkordans) IF I could legitimately claim that I got the
% information from there\ldots{} but of course I got it from
% spraakbanken.gu.se/om/eng/index.html actually.

\subsection{Parsing}
\label{parsers}
%% Um, this seems useful, but I'm not sure why I put it here...
%% TODO: Figure out what this is for and use it somewhere?
% In order to investigate hypothesis 1, I will need to produce features
% to give to the classifier. These features should reflect the syntax of
% the speech of the interviewees. Following Nerbonne and Wiersma 2006, I
% will start with parts of speech, then add the leaf-ancestor paths that
% I tried on the ICE-GB, and finally add dependency-ancestor paths that
% are new. Probably one sentence more each on tagging, dependency and
% constituency parsing.
% (NOTE: Insert paragraphs on tagging and dependency and constituency
% parsing before Talbanken discussion)
% :
In order to extract the features used to build the language models
described in the previous methods, SweDiaSyn will need to be POS
tagged and parsed. For this dissertation, both constituency
and dependency features will be provided to the classifier.

The Tags 'n' Trigrams (T'n'T) tagger \cite{brants00} will be used for tagging, with
the POS annotations from Talbanken05 used as training.
After POS tagging, the Talbanken sentences will be cleaned in order to
be usable for training the parsers.
Cleaning Talbanken's constituency annotations consists of removing
discontinuities of various types, especially disfluencies and
restarts, which may be reparable by a simple top-level strategy. If
more complicated uncrossing is needed, a strategy similar to the split
constituents proposed by \namecite{boyd07} may be needed.

For constituency parsing, the Berkeley parser \cite{petrov08} will be
trained on standard Swedish, again from Talbanken05. The Berkeley
parser has shown good performance on languages other than English,
which is not common for constituency parsers.
% TODO: CITE The paper that shows this. Also EXPLAIN it.

For dependency parsing, MaltParser will be used with the existing
Swedish model trained on Talbanken05 by Hall, Nilsson and
Nivre. MaltParser is an inductive dependency parser that uses a
machine learning algorithm to guide the parser at choice points
\cite{nivre06b}.  Dependency parsing will proceed similarly to
constituency parsing; the dependency structures of Talbanken05 will be
cleaned and normalized, then used to train a parser.

% TODO: Find out how much crossing occurs in Swedish corpora, and how
% much of it is from interruptions and self-corrections.

\subsection{Permutation test}
\label{permutationtest}

The first hypothesis requires that the distances produced by a
distance measure be checked for significance; it is possible that
there may not be enough data for two regions to adequately distinguish
them from each other. A permutation test detects whether two corpora are
significantly different on the basis of the $R$ measure
described in section \ref{nerbonne06}. The test first calculates $R$
between samples of the two corpora. Then the corpora are mixed
together and $R$ is calculated between two samples drawn from the
mixed corpus. If the two corpora are different, $R$ should be larger
between the samples of the original corpora than $R$ from the mixed
corpus: any real differences will be randomly redistributed by the
mixing process, lowering the mixed $R$. Repeating this comparison
enough times will show if the difference is significant. Twenty times
is the minimum needed to detect significance for $p < 0.05$
significance; however, in the experiments, I will repeat the test 100
times, enough to detect significance for $p < 0.01$.

To see how this works, for example, assume that $R$ detects real
differences between the two British regions London
and Scotland such that $R(\textrm{London},\textrm{Scotland}) =
100$. The permutation test then mixes London and Scotland to
create LSMixed and splits it into two pieces. Since the real
differences are now mixed between the two shuffled corpora, we
would expect $R(\textrm{LSMixed}_1, \textrm{LSMixed}_2) < 100$.
This should be true at least 95\% of the time for the distance $100$
to be significant.

%% I don't think normalization is important enough to mention if I
%% have to add all the sections from the H2/H3.
% \subsection{Normalization}
% Afterward, the distance must be normalized to account for two things:
% the length of sentences in the corpus and the amount of variety in the
% corpus. If sentence length differs too much between corpora, there
% will be consistently lower token counts in one corpus, which would
% cause a spuriously large $R$. In addition, if one corpus has less
% variety than the other, it will have inflated type counts, because
% more tokens will be allocated to fewer types. To avoid
% this, all tokens are scaled by the average number of types per token
% across both corpora: $2n/N$ where $n$ is the type count and $N$ is
% the token count. The factor $2$ is necessary because the scaling
% occurs based on the token counts of the two corpora combined.

% this next subsection might need to be changed or deleted
\subsection{Cluster Analysis and Correlation}
\label{cluster-analysis}
The first hypothesis requires a clustering method to allow
inter-region distances to be compared more easily. The dendrogram that
binary hierarchical clustering produces allows easy visual comparison
of the most similar regions.

Correlation is also useful to find out how similar the two method's
predictions are. Because of the connected nature of the inter-region
distances, Mantel's test is necessary to ensure that the correlation
is significant. Mantel's test is a permutation test, much like the
permutation test described for $R$. One distance result set is
permuted repeatedly and at each step correlated with the other
set. The original correlation is significant if the permuted
correlation is lower than the original correlation more than 95\% of
the time.

\subsection{Feature Ranking}
\label{feature-ranking}
% TODO: THIS is hypothesis 1B and I left it out!
Feature ranking is needed for the first hypothesis so that the results
of $R$ can be compared qualitatively to the Swedish dialectology
literature; $R$'s most important features should be similar to those
discussed most by dialectologists when comparing regions. Feature
ranking for $R$ is quite simple for one-to-one region comparisons;
each feature's normalized weight is equal to its importance in
determining the distance between the two regions. The most important
features between two sets of regions can be obtained by averaging the
importance of each feature between all (first-set, second-set) region
pairs. This more
% (There is a nice equation lurking in here
% somewhere that I may want to avoid nonetheless.)
complicated technique is needed to relate the results from
the computational distance measures with the features that
dialectologists discuss relative to areas of Sweden larger than
individual provinces or counties.

%Note: All this is speculative. I have no code for this and I'm pretty
%sure the all-pairs average solution is not quite right

\subsection{Combining Feature Sets}
\label{combine-feature-sets}

% maybe use 'kind of feature' instead of 'type of feature'. it's fuzzier
In order to investigate hypothesis 2, I will need a method for
combining features of different types. Here, the obvious approach
of combining feature types linearly should suffice. For example, within a single
type of feature, such as POS tag or leaf-ancestor path,
there is already redundant information about lexical items and tree
structure, so combining the two does not mean that additional
redundancy needs to be taken into account.
% TODO: Last sentence still sucks
% TODO: Notes from last presentation:
% TnT has a 'mark unknown words' option.
% Pass lexical items to Berkeley parser (or make sure both training and test are t
% he same)
% check whether Berkeley parser has POS-tagging option

\subsection{Feature Backoff}
\label{feature-backoff}
% If I can't find a certain frequency of trigrams then backoff to
% bigram
% Thorsten Brants - deleted interpolation

% Martin Volk inter-type backoff (if X type information isn't
% available, then use Y type instead) (2000, 2001 or 2002)
% Scaling Up ...; Exploiting the WWW ...; Combining Unsupervised ...;
In order to investigate hypothesis 2, I will need a framework for
backing off sparse features. For backoff within a single type of
feature, I will use deleted interpolation \cite{jurafskymartin}. Training for the trigram, bigram and unigram counts will
come from the Talbanken.

For backoff between types of features, I will use ranked combinations
of feature sets, based to Martin Volk's system for verb attachment
\cite{volk02}. Volk used an a priori reliability measure for ranking
quality of combined feature types; I will use number of significant
region differences for ranking: the top-ranked feature type will be
the one that produces the highest number of significant distances
between regions. Combinations of feature types will be ranked by
averaging the number of significant distances that the constituent
feature types produce. Then if the distance measure can't find a
significant difference using highest ranked set of features, the
classifier will fall back to the next highest-ranked set of features.

\subsection{Alternate Feature Sets}
\label{alternate-feature-sets}
For hypothesis 2, I will need a way to generate new types of
features. One obvious way to do this is to modify existing feature
types to include more contextual
information. For example, supertags \cite{joshi94} are similar to
leaf-ancestor paths, but include more tree context around the
head. Similarly, dependency paths could be expanded
so that each node on the path includes lexical context, such as
bigrams or trigrams.

\subsection{Alternate Distance Measures}
\label{kl-divergence}
In the case that $R$ does not reach statistical significance, I will
need to experiment with similar but more complicated distance measures
to find a more sensitive one. The obvious choice at this point is
Kullbeck-Leibler divergence, or relative entropy, which is described
in \namecite{manningschutze}. Relative entropy is quite similar to $R$
but more widely used in computational linguistics. Besides this, several variants of
relative entropy exist, such as Jenson-Shannon divergence \cite{lin91}, that lift
various restrictions from the input distributions.

% Another possibility is a return to Goebl's Weighted Identity Value;
% this classifier is similar in some ways to $R$, but has not been
% tested with large corpora, to my knowledge at least. (This is not
% particularly useful and I don't believe that WIV would actually be
% good, so I should probably just drop this.)

More exotic classifiers are of course possible, although I
have not investigated them yet. Examples are k-nearest
neighbor classification or neural nets.
% (maybe it was relative entropy or just normal-kind entropy).
% TODO: WIV, also Kullbeck-Leibler Divergence could work.
% Maybe also k-NN/MBL, HMM binary classifier (?), maybe even a
% neural net

% Make sure the KL divergence couldn't go to infinity
% Jensen-Shannon lifts a couple of restrictions though I think the
% input still has to be a probability distribution

%%% Local Variables: 
%%% mode: latex
%%% TeX-master: "dissertation.tex"
%%% End: 
