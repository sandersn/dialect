\documentclass{beamer}

\usetheme{Ilmenau}
\usepackage[all]{xy}

\title{A Statistical Method for Syntactic Dialectometry}
\author{Nathan Sanders}
\date{\today}
\AtBeginSection[] % Do nothing for \section*
{
\begin{frame}<beamer>
\frametitle{Outline}
\tableofcontents[currentsection]
\end{frame}
}
\begin{document}

\frame{\titlepage}

\section[Outline]{}
\frame{\tableofcontents}
\begin{frame}
  This dissertation establishes the utility and reliability of a
  statistical measure for syntactic dialectometry.
\end{frame}
\section{Introduction}
\begin{frame}
  \frametitle{Dialectology}
\begin{columns}
\column[c]{0.5\textwidth}
  \begin{definition}
    Dialectology is the study of linguistic variation.
  \end{definition}
% One variable at a time, use your in-built intuition to combine
% variables. This requires good intuition and lots of it.
\column[c]{0.5\textwidth}
  \includegraphics[scale=0.22]{dialektboka-karta8}
\end{columns}
\end{frame}
\begin{frame}
  \frametitle{Dialectometry}
\begin{columns}
\column[c]{0.5\textwidth}
 \begin{definition}
    Dialectometry is a subfield that studies variation quantitatively,
    using methods from statistics and information theory.
 \end{definition}
% Combine variables first, use less intuition.
% This requires good combining methods, which is harder than it
% sounds.
% NOTE: You can still manually extract variables by hand. It's a
% separate issue. That's how my work differs from Spruit's.
\column[r]{0.5\textwidth}
\includegraphics[scale=0.25]{Sverigekarta-cluster-5-1000}
\end{columns}
\end{frame}

\begin{frame}
\frametitle{Abstract Distance Measure Model}
\[\xymatrix@C=1pc{
 \textrm{Corpus} \ar@{>}[d]|{} &
  S = s_o,s_1,\ldots
  \ar@{>}[dd]|{f}
  &&
  T = t_o,t_1,\ldots
  \ar@{>}[dd]|{f}
  \\
\textrm{Decomposition}\ar@{>}[dd] &&&\\
 &
 *{\begin{array}{c}
     \left[ + f_o, +f_1 \ldots \right], \\
     \left[ - f_o, +f_1 \ldots \right], \\
     \ldots \\ \end{array}}
 \ar@{>}[dr]
 &&
 *{\begin{array}{c}
     \left[ + f_o, -f_1 \ldots \right], \\
     \left[ + f_o, -f_1 \ldots \right], \\
     \ldots \\ \end{array}}
 \ar@{>}[dl]  \\
 \textrm{Combination} && \textrm{Distance} & \\
} \]
\end{frame}
% % These two are probably too much--more appropriate for a 50 minute
% % talk
% \begin{frame}
%   Feature decomposition
% \end{frame}
% \begin{frame}
%   Distance measure
% \end{frame}


\begin{frame}
  \frametitle{Phonological versus Syntactic Dialectology}
  Unlike phonology, in syntax:
  \begin{itemize}
% because it's easier to introspect on syntax, perhaps because of the
% influence of a writing system, which also leads to standardization
  \item There are fewer syntactic dialect differences.
    % : a larger corpus is needed
  \item There is no accepted measure of syntactic distance.
    % : because corpora have to be larger, but there is no good way to
    % annotate larger corpora collected this way.
  \item There is no accepted way to extract linguistic information
    from syntactic corpora.
   % : without manual annotation, mainstream syntax cannot be used
    % (and in fact NO corpus of mainstream annotated sentences exists,
    % much less a dialect one.)
    % ((Idiot syntacticians don't believe that corpora is useful, even
    % small non-repeating ones. ANYWAY. This is the reason that
    % empiricists don't believe that minimalism can possibly be
    % capturing actual language, because the theory has never been
    % tested on actual language, just miniature tests.))
  \end{itemize}
\end{frame}

\begin{frame}
  \frametitle{Syntax and Dialectology}
  Goebl and Spruit
\begin{itemize}
\item They propose small numbers of features extracted by hand and a
  complex distance measure.
\item This works with smaller corpora, including dialect surveys.
\item Gaps in linguistic knowledge can bias feature selection.
%In other words, you have to do dialectology before you can do dialectometry
\end{itemize}
\end{frame}

\begin{frame}
  \frametitle{Syntax and Dialectology}
  Nerbonne \& Wiersma and Sanders
\begin{itemize}
\item They propose large numbers of automatically extracted features
  and a simple distance measure.
\item This requires larger corpora.
\item But they can be automatically annotated.
\item Until now, there has been no thorough examination of this
  method's ability to reproduce existing dialect knowledge.
\end{itemize}
\end{frame}

\section{Questions}
\begin{frame}
 \frametitle{Question 1}

Do the results of this statistical measure for syntactic dialectometry
agree with dialectology?

%% \includegraphics[scale=0.25]{Sverigekarta-Landskap-consensus-5-1000}
 % oops I need a map of the traditional areas here. I don't have one
 % of those in PDF form I think
 %% \includegraphics[scale=0.25]{Sverigekarta-Landskap-consensus-5-1000}
\end{frame}

\begin{frame}
  \frametitle{Question 2}
 What parameter variations produce the best agreement with
  dialectology?
\end{frame}

\begin{frame}
  \frametitle{Question 3}
 Do the results of this statistical measure for syntax agree with the
  results for a phonological measure of distance on the same data?
\end{frame}

\section{Methods}
\begin{frame}
  \frametitle{Feature Set}
  Capture syntactic information by representing individual units of
  information.
\[\xymatrix@C=1pc{
 \textrm{Corpus} \ar@{>}[d]|{} &
  S = s_o,s_1,\ldots
  \ar@{>}[dd]|{f}
  &&
  T = t_o,t_1,\ldots
  \ar@{>}[dd]|{f}
  \\
\textrm{Decomposition} &&&\\
 &
 *{\begin{array}{c}
     \left[ + f_o, +f_1 \ldots \right], \\
     \left[ - f_o, +f_1 \ldots \right], \\
     \ldots \\ \end{array}}
 &&
 *{\begin{array}{c}
     \left[ + f_o, -f_1 \ldots \right], \\
     \left[ + f_o, -f_1 \ldots \right], \\
     \ldots \\ \end{array}} \\
} \]

\end{frame}
% TODO: Examples in a second frame
\begin{frame}
  \frametitle{Feature Sets}
  \begin{tabular}{c|c}
  Leaf-Ancestor Paths & Nested Structure \\ \hline
  Leaf-Head Paths & \\
  Leaf-Head Paths, based on Timbl training &Long-distance context  \\
  Leaf-Arc Paths & \\ \hline
  Phrase Structure Rule & Internal Structure \\
  Phrase Structure Rule with Grandparent & \\ \hline
  Trigrams & Context and order \\ \hline
  Unigrams & Baseline \\ \hline
  Combined & \\
  \end{tabular}
\end{frame}
\begin{frame}
  \frametitle{Distance Measure}
  Combine two sets of features into a single number.
\[\xymatrix@C=1pc{
\textrm{Decomposition}\ar@{>}[dd] &&&\\
 &
 *{\begin{array}{c}
     \left[ + f_o, +f_1 \ldots \right], \\
     \left[ - f_o, +f_1 \ldots \right], \\
     \ldots \\ \end{array}}
 \ar@{>}[dr]
 &&
 *{\begin{array}{c}
     \left[ + f_o, -f_1 \ldots \right], \\
     \left[ + f_o, -f_1 \ldots \right], \\
     \ldots \\ \end{array}}
 \ar@{>}[dl]  \\
 \textrm{Combination} && \textrm{Distance} & \\
} \]
\end{frame}
% TODO: Definitions in a second frame
\begin{frame}
  \frametitle{Measures}
  \begin{tabular}{c|c}
  $R$ & $\Sigma_i |a_i - b_i|$ \\
  $R^2$ & $\Sigma_i (a_i - b_i)^2$ \\ \hline
  Kullback-Leibler divergence & $\sum_i {a_i \log\frac{a_i}{b_i} + b_i \log\frac{b_i}{a_i}}$ \\
  Jensen-Shannon divergence & $\sum_i {a_i \log\frac{b_i}{\bar{c_i}} + a_i \log\frac{b_i}{\bar{c_i}}}$ \\ \hline
  Cosine similarity & $\cos(a,b)$ \\
  \end{tabular}
\end{frame}
\begin{frame}
  \frametitle{Sampling / Iterations}
  \begin{itemize}
  \item 1000 sentences with replacement
  \item All sentences
 \end{itemize}
  \begin{itemize}
  \item 1 normalization iteration
  \item 5 normalization iterations
 \end{itemize}
\end{frame}
\begin{frame}
  \frametitle{Input Processing: Corpora}
  \begin{columns}
    \column[c]{0.5\textwidth}
    The Swediasyn is a dialect corpus collected in 2000 from more than
    100 villages through Sweden and Swedish-speaking Finland. Four
    speakers from each village were interviewed. The interviews were
    later transcribed, 30 so far.
    \column[c]{0.5\textwidth}
    Talbanken is a mixed newspaper / speech corpus collected in the
    1970s. It is used for training automatic annotators.
  \end{columns}
\end{frame}
\begin{frame}
  \frametitle{Input Processing: Annotators}
  \begin{itemize}
  \item Tags`n'Trigrams for part-of-speech tagging
  \item The Berkeley parser for phrase-structure parsing
  \item MaltParser for dependency parsing
  \end{itemize}
\end{frame}
% \begin{frame}
% \frametitle{Phonology}
%   Determine average value for insertion and deletion--this should be
%   half the average substitution cost for arbitrary segments.
% \end{frame}
\begin{frame}
  \frametitle{Output Processing: Significance}
  \begin{enumerate}
  \item Find $d = R(sample(a),sample(b))$.
  \item Now, at least 20 times:
  \item $shuffled = shuffle(a,b)$
  \item $shuffle_a = sample(shuffled)$
  \item $shuffle_b = sample(shuffled)$
  \item Find $d_{shuffle} = R(shuffle_a,shuffle_b)$
  \item Is $d_{shuffle} < d$ ? It should be; shuffling should destroy any significant differences.
  \item If $d_{shuffle} < d$ more than 95\% of the time, $d$ is significant.
  \end{enumerate}
\end{frame}
\begin{frame}
  \frametitle{Output Processing: Correlation with Travel Distance}
 \begin{itemize}
  \item Dialects generally correlate with distance
  \item Swedish dialects have no sharp boundaries, meaning that
    correlation should be even better.
 \end{itemize}
\end{frame}
\begin{frame}
  \frametitle{Output Processing: Cluster Analysis}
  Produce hierarchical clusters which:
  \begin{itemize}
  \item Put each site in a unique cluster (hard clustering)
  \item Consensus trees improve stability by using only clusters that
    occur in the majority of parameter settings
  \item Composite cluster maps display clusters as boundaries, similar
    to isoglosses.
  \end{itemize}
\end{frame}
\begin{frame}
  \frametitle{Output Processing: Multi-dimensional scaling}
  Produce 3D clusters which:
  \begin{itemize}
  \item Do not put sites in unique clusters (soft clustering)
  \item Maps high dimensional dissimilarity space to 3 dimensional distance
 \end{itemize}
\end{frame}
\begin{frame}
  \frametitle{Output Processing: Feature Ranking}
  \begin{itemize}
  \item Compares the rate of individual features between regions.
  \item Features are characteristic of one or the other of a region
    pair.
  \end{itemize}
\end{frame}
\section{Results}
\begin{frame}
  \frametitle{Organization}
  Results are grouped by sample size and normalization.
  \begin{tabular}{c|c|c|}
  &  1 iteration & 5 iterations \\ \hline
  1000 sentences & 1-1000 & 5-1000 \\ \hline
Full site &    1-full & 5-full \\ \hline
  \end{tabular}
\end{frame}
\begin{frame}
  \frametitle{Significance}
\begin{center}
  \begin{tabular}{c|c|c|}
  &  1 & 5 \\ \hline
  1000 & $\circ$ &  $\circ$\\ \hline
Full & $\circ$ & $\times$\\ \hline
\end{tabular}
\end{center}
  \begin{itemize}
  \item The highest number of significant distances was 1-1000.
  \item However, decreasing significance generally means more informative
    results;
  \item As classifiers become more sensitive, they deal with increased
    noise, which means less significance.
    % This trend holds across feature sets, such that trigrams give
    % the best trade-off between significance and sensitivity.
    % (Unfortunately. This is a pretty low-end tradeoff.)
  \end{itemize}
\end{frame}
\begin{frame}
  \frametitle{Correlation with Travel Distance}
  \begin{center}
  \begin{tabular}{c|c|c|}
  &  1 & 5 \\ \hline
  1000 & 0.24 &  0.22\\ \hline
Full & 0.29 & 0.18 \\ \hline
\end{tabular}
\end{center}
 \begin{itemize}
  % \item Correlation with travel and geographic distance is low but significant.
  % \item Most correlations are 0.2 to 0.3, with a high of 0.37.
  \item This is lower than predicted by the ``boundary free'' view of
    Swedish dialects. % which is the usual, well accepted, view
  \item Correlation with travel distance is slightly better than with
    geographic distance.
    % probably this points to a shortcoming in the distance measure
    % rather than new information about Swedish dialects.
    % also, given the size-geographical distance correlation (0.45**),
    % it all may be epiphenomenal and we learn nothing from looking at
    % correlations.
 \end{itemize}
\end{frame}
\begin{frame}
  \frametitle{Clusters and Consensus Trees}
  \begin{itemize}
  % \item Consensus trees are the easiest way to interpret hierarchical clusters.
 \item Of the previous groups, the 5-1000 setting
    retains the most detailed consensus tree.
  \item Where the others have clusters, they agree with 5-1000's.
 \end{itemize}
\end{frame}

\begin{frame}
  \frametitle{Clusters and Consensus Trees}
  \begin{columns} \column[c]{0.5\textwidth}
    \begin{center}
      \includegraphics[scale=0.3]{Sverigekarta-Landskap-consensus-5-1000}
    \end{center} \column[c]{0.5\textwidth}
    \begin{itemize}
    \item When projected on a map of Sweden, this tree reproduces the
      major dialect areas.
    \item (North, East, West, South)
    % \item South divides into multiple clusters.
    % \item But North and East are part of the same cluster.
    %   maybe an effect of cities? Cities are big these days
    \end{itemize}
  \end{columns}
\end{frame}

\begin{frame}
  \frametitle{Composite Cluster Maps}
\begin{columns}
\column[c]{0.5\textwidth}
\includegraphics[scale=0.3]{Sverigekarta-cluster-5-1000}
\column[c]{0.5\textwidth}
The composite cluster maps reproduce dialectology's
  \begin{itemize}
  \item Weak boundaries between regions.
  \item North-to-south gradient.
\end{itemize}
 Notably, the southern boundary is stronger than the rest, and
    Tors\aa{}s/J\"amshog/\"Ossj\"o are further isolated within this area.
\end{columns}
\end{frame}
\begin{frame}
  \frametitle{Multi-Dimensional Scaling}
  \begin{itemize}
  \item MDS maps are harder to analyze because they cannot be combined.
  \item However, most MDS maps show the same pattern as the other two
    mapping methods.
 \end{itemize}
\end{frame}
\begin{frame}
  \frametitle{Features}
  \begin{itemize}
  \item Overall, feature extraction was biased by the two
    normalization methods.
  \item Without the overuse normalization, the results tend to be
    features that occur in {\it every} dialect; it only highlights
    differences.
  \item With the overuse normalization, the results tend to be quite
    noisy.
  \end{itemize}
\end{frame}
\begin{frame}
  \frametitle{Features}
  Nonetheless, some interesting features appeared:
  \begin{itemize}
  \item The southern clusters had odd positioning of adverbs,
    conjunctions and ends of sentences.
  \item This may point to non-standard placement of adverbs.
  \item The tightest southern cluster showed possible use of double modals.
  \end{itemize}
  % TODO: Pictures here? I guess, I could put some example features.
  % But they are pretty ugly.
\end{frame}
\section{Discussion}
\begin{frame}
  \frametitle{Comparison to Dialectology}
  Good agreement at the levels of
  \begin{itemize}
  \item regions (consensus tree maps, MDS maps)
  \item boundaries (composite cluster maps)
  \item distances (composite cluster maps)
  \end{itemize}
  But not for specific features.
\end{frame}
\begin{frame}
  \frametitle{Comparison to Dialectology Features}
  \begin{itemize}
 \item About a third of the results were positive, a third negative
    and a third inconclusive.
  \item However, none were very strong.
  \item None could be verified as significant, given current methods.
  \end{itemize}
\end{frame}
\begin{frame}
  \frametitle{Comparison to Phonological Dialectometry}
  \begin{itemize}
 \item Agreement with Leinonen's variable maps is good.
 \item All syntactic boundaries have an analogue in a boundary for
   some phonological variable.
  \item Results are preliminary; correlation is needed to know whether
    the similarities are significant.
  \end{itemize}
\end{frame}
\begin{frame}
  \frametitle{Comparison to Syntactic Dialectometry}
  \begin{itemize}
  \item This dissertation shows agreement with dialectology.
  \item Previous work found significant differences but did not
    produce distances, regions or boundaries that agreed with
    dialectology.
  \item The sites used here are smaller than in previous work.
  \item This indicates that corpora specifically collected for dialect
    work still have an advantage.
 \item The parameter variations analyzed here guide future research.
 \end{itemize}
\end{frame}
\section{Conclusion}
\begin{frame}
  \frametitle{Future Work}
  \begin{itemize}
  \item Correlation with phonological dialectometry
 \item Rapid analysis of the rest of Swedia and Nodalida.
 \item Improved feature normalization.
  \item Significance testing for feature-by-feature comparisons.
 \item Improved automatic annotation to reduce feature noise.
  \end{itemize}
\end{frame}
\begin{frame}
  \frametitle{Conclusion}
  This dissertation establishes the utility and reliability of a
  statistical measure for syntactic dialectometry.
\end{frame}

\end{document}
%%% Local Variables: 
%%% mode: latex
%%% TeX-master: t
%%% End: 
