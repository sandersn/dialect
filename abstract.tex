 This dissertation establishes the utility and reliability of a
  statistical distance measure for syntactic dialectometry, expanding
  dialectometry's methods to include syntax as well as phonology and
  the lexicon. It establishes the measure's reliability by comparing
  its results to those of dialectology and phonological dialectometry
  on Swedish dialects, as well as evaluating variant parameter
  settings.

  Dialectology studies varieties of language. Dialectometry is a
  subfield which studies language varieties quantitatively. The
  research questions of this dissertation are (1) whether a
  statistical measure of syntax for dialectometry will reproduce the
  results of syntactic dialectology and phonological dialectometry and
  (2) what parameter settings produce results most similar to
  dialectology's results. Answering these question will establish
  whether a statistical measure of syntax is useful, and if so, how it
  can best be used in future research.

  Statistical dialect distance is defined in two parts: a feature set
  that captures linguistic properties and a measure of dissimilarity
  that combines the features of two sites into a single number.  This
  dissertation uses feature sets from previous work: trigrams
  (Nerbonne \& Wiersma, 2006) and leaf-ancestor paths (Sanders,
  2007). In addition, it introduces two other feature sets: leaf-head
  paths based on dependencies and phrase-structure rules. Previous
  work has relied on the measure $R$ (Nerbonne \& Wiersma
  2006). I introduce measures from information theory:
  Kullback-Leibler and Jensen-Shannon divergences and cosine
  similarity.

  This statistical distance is tested on the Swediasyn, a corpus of
  interviews recorded in villages throughout Sweden. Before measuring
  distance between interview sites, the interview transcriptions
  are annotated with linguistic information needed for each feature
  set. After the distance has been measured, the distances are
  processed so that they can be compared with existing
  dialectology results.

  The results show that, unlike previous work, significant distances
  can be measured between dialect corpora. When these distances are
  mapped to the geography of Sweden, they reproduce the traditional
  dialect regions of Sweden fairly well. As for other indicators of
  success, there is only weak correlation with geographic distance,
  but good agreement between maps produced by dialectometric syntactic
  distance and phonological distance. Searching for specific dialect
  features found by dialectology, however, is inconclusive; better
  feature sets and comparison methods are needed.
