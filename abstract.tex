{\centering A STATISTICAL METHOD FOR SYNTACTIC DIALECTOMETRY
  
}

This dissertation establishes the utility and reliability of a
  statistical distance measure for syntactic dialectometry, expanding
  dialectometry's methods to include syntax as well as phonology and
  the lexicon. It establishes the measure's reliability by comparing
  its results to those of dialectology and phonological dialectometry
  on Swedish dialects, as well as evaluating variant parameter
  settings.
%   Dialectology studies varieties of language. Dialectometry is a
%   subfield which studies language varieties quantitatively.
  The
  research questions of this dissertation are (1) whether a
  statistical measure of syntax for dialectometry will reproduce the
  results of syntactic dialectology and phonological dialectometry and
  (2) what parameter settings produce results most similar to
  dialectology's results.
%   Answering these questions will establish
%   whether a statistical measure of syntax is useful, and if so, how it
%   can best be used in future research.
  
  Statistical dialect distance is defined in two parts: a feature set
  that captures linguistic properties and a measure of dissimilarity
  that combines two sites' features into a single number.  This
  dissertation uses feature sets from previous work: trigrams
  (Nerbonne \& Wiersma, 2006) and leaf-ancestor paths (Sanders,
  2007). In addition, it introduces two other feature sets: leaf-head
  paths based on dependencies and phrase-structure rules. This
  dissertation uses the measure $R$ (Nerbonne \& Wiersma 2006) as well
  as measures from information theory: Kullback-Leibler and
  Jensen-Shannon divergences and cosine similarity.
  This statistical distance is tested on the Swediasyn, a corpus of
  interviews recorded in villages throughout Sweden.
%   Before measuring
%   distance between interview sites, the interview transcriptions were
%   annotated linguistically to support the above feature sets.
  After
  the distance was measured, the distances were processed and then
  compared with existing dialectology results.

  Unlike previous work, significant distances were measured between
  dialect corpora in this dissertation. When these distances are
  mapped to the geography of Sweden, they reproduce the traditional
  dialect regions of Sweden. There is weak correlation with geographic
  distance, but good agreement between dialectometric syntactic and
  phonological distance. Comparing specific dialect features with
  those of dialectology is inconclusive; better comparison methods are
  needed.

\noindent{}\rule{4in}{1pt} \\
\rule{4in}{1pt} \\
\rule{4in}{1pt} \\
\rule{4in}{1pt}

%%% Local Variables: 
%%% mode: latex
%%% TeX-master: "dissertation.tex"
%%% End: 
