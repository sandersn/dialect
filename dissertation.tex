\documentclass{iuthesis}
% macros that I like go here.
\usepackage{graphicx}
\usepackage[all]{xy}
\usepackage{qtree}
\usepackage{robbib}
\includeonly{background,questions,methods,results,discussion}
\title{Syntactic Distance for Dialectometry}
\author{Nathan C. Sanders}
\advisor{Sandra K\"ubler}
\secondreader{Markus Dickinson}
\thirdreader{Steven Franks}
\fourthreader{Michael Gasser}

\department{Linguistics}
\submitdate{10/05/01} % a hopeful guess
\copyrightyear{2010}
\begin{document}

\begin{dedication}
  It's a great day for freedom
\end{dedication}
\begin{acknowledgements}
  I always thought Heeringa's acknowledgements were
  pretty cool
\end{acknowledgements}
\begin{abstract}
  Syntactic distance measures for dialectometry; tested on
  Swedish. It's all right.
\end{abstract}

\frontmatter
\maketitle
\signaturepage
\copyrightpage
\makededication
\makeack
\makeabstract
\tableofcontents
\mainmatter

\chapter{Introduction}

This dissertation establishes the utility and reliability of a
statistical distance measure for syntactic dialectometry, expanding
dialectometry's methods to include syntax as well as phonology and the
lexicon. It is a continuation of my previous work \cite{sanders07},
\cite{sanders08b} and earlier work by \namecite{nerbonne06}, the first
statistical measure of syntax distance. These pioneering studies
explored this measure, but failed to compare it to established results
in dialectology to see if the new method reproduces them. This
dissertation does so, as well as investigating a number of variant
measures and feature sets. It uses Swedish dialect data as a basis for
investigation.

Dialectology is the study of linguistic variation \cite{chambers98}.
% distance / other variables.
Its goal is to characterize the linguistic features that separate
language varieties. The tools that it uses to do this include
isoglosses---geographic descriptions of a particular linguistic
variable--as well as traditional phonological and syntactic analyses
of dialect phenomena. Traditional dialectology predates
sociolinguistics, but has adopted many of its tools so that it has
become in some ways a subfield of sociolinguistics.

Dialectometry is a subfield of dialectology that uses mathematically
sophisticated methods to extract and combine linguistic features
\cite{seguy73}. Its focus is the manipulation of large data sets in a
uniform way, characterizing the differences between regions in a
gradient and statistically sound way. As result, in recent years most
work in the field has been computational linguistic, largely focused
on phonology, starting with \namecite{kessler95}, followed by
\namecite{nerbonne97} and \namecite{nerbonne01}. \namecite{heeringa04}
provides a comprehensive review of phonological distance in
dialectometry as well as some new methods.

This dissertation compares the results of the syntactic distance
measure with syntactic dialectology, both in the form of the traditional
Swedish dialect regions as well as analysis of syntactic dialect
features. It also compares the results to phonological dialectometry's
results on Swedish.

\section{Overview of Dialectometry}

In dialectometry, a distance measure can be defined in two parts:
first, a method of decomposing the data into features that capture
linguistic properties, and second, a method of combining the features
of two corpora, usually by summing them in some way. The decomposition
method can be thought of as ``feature extraction'' for any number of
feature sets, while the combining method can be thought of as the
``distance'', since it combines a large set of features into a single
number. Figure \ref{abstract-distance-measure-model} gives an overview
of how the model works. Input consists of two corpora; each item in
each corpus is decomposed into a set of features extracted by $f$. The
resulting corpora are then compared by $d$, which combines the corpora
into a single number: the distance.

\begin{figure}
\[\xymatrix@C=1pc{
 \textrm{Corpus} \ar@{>}[d]|{} &
  S = s_o,s_1,\ldots
  \ar@{>}[dd]|{f}
  &&
  T = t_o,t_1,\ldots
  \ar@{>}[dd]|{f}
  \\
\textrm{Decomposition}\ar@{>}[dd] &&&\\
 &
 *{\begin{array}{c}
     \left[ + f_o, +f_1 \ldots \right], \\
     \left[ - f_o, +f_1 \ldots \right], \\
     \ldots \\ \end{array}}
 \ar@{>}[ddr]
 &&
 *{\begin{array}{c}
     \left[ + f_o, -f_1 \ldots \right], \\
     \left[ + f_o, -f_1 \ldots \right], \\
     \ldots \\ \end{array}}
 \ar@{>}[ddl]  \\
 \textrm{Combination} && d & \\
 & & \textrm{Distance} & \\
} \]
\label{abstract-distance-measure-model}
\caption{Abstract Distance Measure Model : $d \circ f$}
\end{figure}

Dialectometry has focused on phonological distance measures, while
syntactic measures have remained undeveloped. The most important
reason for this focus is that it is easier to define a distance
measure on phonology. In phonology, it is easy to collect corpora
consisting of identical word sets. Then these words decompose to segments and,
if necessary, segments further decompose to phonological
features. This decomposition is straightforward and based on
\namecite{chomsky68}. For combination, string alignment, or Levenshtein
distance \cite{lev65}, is a well-understood algorithm used for
measuring changes between any two sequences of characters taken from a
common alphabet. Levenshtein distance is simple mathematically, and
has the additional advantage that its intermediate data structures are
easy to interpret as the linguistic processes of epenthesis, deletion and
metathesis.

These things are not possible with syntactic distance: neither matched
sentences nor a single obvious function for decomposition and
combinations. Matched sentences could in theory be collected, but the
number of possible interview responses in syntax is so much larger
that the required time and number of informants would be
correspondingly much greater. For example, in the Survey of English
Dialects \cite{orton63}, phonological items were elicited by asking
the interviewee to answer a question: ``cow'' is the standard English
answer to ``What is the animal that you get milk from?''. This method
avoids priming the interviewee with the interviewer's
pronunciation. However, It does not always have the desired effect,
even for phonology: for the item ``newt'', the responses ``newt'',
``ewt'' and ``eft'' are all comparable phonologically, but a response
like ``salamander'' is not. This problem is exponentially worse for
syntax: an interview question that is sufficiently abstract to avoid
priming a particular structure has a low chance of eliciting that very
structure. For example, a prompt such as ``I took your drink. What do
you say to me?'' has a low chance of eliciting sentences that
exemplify the differing English orders of direct and indirect objects
such as ``give it me'' versus ``give it to me''.
% for example, in many parts of English, the answer will consist
% entirely of expletives.

The specification of functions for decomposition and combination of
syntax faces a different problem. Although many decomposition and
combination methods can be proposed, the standard syntactic theories
% this is what chomsky calls it so shut up. he's always right
cannot be practically used. For example, the parsers in this
dissertation use probabilistic phrase structure rules or dependencies
to represent the grammar of a language. This is typical for parsers
in computational linguistics, but it means that their output, and by
extension the features based on their output, is quite different from
the lexical representations of minimalism. In order to decompose
sentences to minimalist features, a broad-coverage minimalist parser
would be required. Since such a parser does not yet exist, it is
impossible to use the standard linguistic theory in the same way that
phonology does.

%% I'm cutting this paragraph because it makes this section too long
%% and it's more of a pet theory of mine than anything I can back up.
% A secondary reason for dialectometry's focus on phonology is that it
% is inherited from dialectology's focus on phonology.
% % (TODO:Cite?)
% This might be solely due to the history of dialectology as a field, but it is
% likely that more phonological than syntactic differences exist between
% dialects, due to historically greater standardization
% of syntax via the written form of language. Phonological
% dialect features are less likely to be stigmatized and suppressed by a
% standard dialect than syntactic ones.
% % (TODO:Cite, probably
% % Trudgill and Chambers something like '98, maybe where they talk about
% % what aspects of dialects are noticed and stigmatized).
% Whatever the reason, much less dialectology work on syntax is
% available for comparison with new dialectometry results.

\subsection{Syntax and Dialectometry}
% TODO:Or "Gaps in the field" or "Places to improve upon" or something

Because of the preceding two reasons, syntax is a relatively
undeveloped area in dialectometry. Currently, the literature lacks a
generally accepted syntax measure. Unfortunately, approaching the
problem by copying phonology is not a good solution; there are real
differences between syntax and phonology that mean phonological
approaches do not apply. For example, there are fewer differences to
be found in syntax, and they occur more sparsely.  Because
dialectology has traditionally worked with fairly small corpora, and
because of the difficulty of collecting syntactic data, most surveys
cover even fewer syntactic variables than phonological ones. There are
two approaches to remedy this. The first manually enhances the
differences that do exist in small, carefully collected corpora; the
second switches to larger, non-survey corpora and uses statistical
methods to find differences.

The first approach is proposed by \namecite{spruit08} for analyzing
the Syntactic Atlas of the Dutch Dialects \cite{barbiers05}, is to
continue using small dialectology corpora and manually extract
features so that only the most salient features are used. Then a
sophisticated method of combination such as Goebl's Weighted Identity
Value (WIV), described below and by \namecite{goebl06}, can be used to
produce a distance. WIV is more complex mathematically than
Levenshtein distance, and operates on any type of linguistic
feature. However, manual feature extraction is not feasible in
knowledge-poor or time-constrained environments. It is also subject to
bias from the dialectologist. Since the best manually extracted
features are those that capture the difference between two dialects,
the best-known features are most likely to become the best manual
features, passing over the rarely occurring and previously unknown
features that might actually be the best indicators of a particular
dialect.

This first approach ignores the specific properties of the syntax
distance problem. Given a large corpus, manually defined features will
have less coverage than the automatically extracted features used by
the second, statistical approach. Furthermore, automatically extracted
features are easy to define for syntax.  This dissertation covers
part-of-speech trigrams, leaf-ancestor paths, and dependency paths
over nodes, but many variations on these features are possible, such
as lexical trigrams, lexicalized leaf-ancestor paths, or dependency
paths over dependency arc labels. Methods from other syntactic work in
computational linguistics could apply too: supertags \cite{joshi94},
convolution kernels \cite{collins01} or any number of simpler features
such as tree height, number of nodes, or number of words.

The problem for the statistical approach is not defining a feature
set. The problem is defining a good feature set. This is the reason
that the statistical approach uses large corpora: with enough data,
statistically significant comparisons can be made between the
different features; the highest ranked ones can be discovered
automatically rather than manually.  Fortunately, the typical syntactic corpus
is larger than a phonological corpus because the annotation
work is easier; much of the syntactic annotation can be generated
automatically.

Even with a feature set defined, a distance measure still requires a
method of combining features to find a distance. One such method, a
simple statistical measure called $R$, has been proposed by
\namecite{nerbonne06} based on work by \namecite{kessler01}. At
present, however, $R$ has not been adequately shown to detect dialect
differences. A small body of work suggests that it does, but as yet
there has not been a satisfying correlation of its results with
existing results from the dialectology literature on syntax.

Nerbonne \& Wiersma's first paper used part-of-speech trigram features
as a proxy for syntactic information and $R$ for syntax distance
together with a test for statistical significance\cite{nerbonne06}.
Their experiment compared two generations of Norwegian L2 speakers of
English.  They found that the two generations were significantly
different, although they had to normalize the trigram counts to
account for differences in sentence length and complexity. However,
showing that two generations of speakers are significantly different
with respect to $R$ does not necessarily imply that the same will be
true for other types of language varieties. Specifically, for this
dissertation, the success of $R$ on generational differences does not
imply success on dialect differences.

I addressed this problem \cite{sanders08b} by measuring $R$ between
the nine Government Office Regions of England, using the International
Corpus of English Great Britain \cite{nelson02}. Speakers were classified by
birthplace. I also introduced Sampson's leaf-ancestor paths as
a feature set \cite{sampson00}. I found statistically
significant differences between most corpora, using both trigrams and
leaf-ancestor paths as features. However, $R$'s distances were not
significantly correlated with Levenshtein distances. Nor did I
show any qualitative similarities between known syntactic dialect
features and the high-ranked features used by $R$ in producing its
distance. As a result, it is not clear whether the significant $R$ distances
correlate with dialectometric phonological distance or with known
features found by dialectologists.

% NOTE: 2-d stuff is not the primary problem, since we can't compare
% trees to trees anyway. The primary problem is comparing two corpora
% full of differing sentences. A secondary problem arises to make sure
% that the 2-d-extracted features aren't skewed one way or another. I
% guess I need to come up with a general justification for the
% normalizing and smoothing code from Nerbonne & Wiersma

% Additional problems: phonology is 1-dimensional, with one obvious way
% to decompose words into segments and segments into features. Syntax is
% 2-dimensional, so the decomposition must take several more factors
% into account so that the features it produces are
% useful and comparable to each other. And those features are \ldots

% Overview : Goal, Variables, Method
%   Contribution
% Literature Review
%   : (including theoretical background)
%   Draw hypotheses from earlier studies
% Method
%   :
%   Experiment section as 'Corpus' section

% Goal: To extend existing measurement methods. To measure them
% better. To measure them on more complete data.

\section{Overview of the Dissertation}

The problem outlined in the previous section is that dialectometry
lacks a statistical method designed for syntax which does not require
the linguist to specify ad-hoc features manually. This dissertation
addresses the lack directly by applying the method to a dialect
corpus, then comparing the results to existing syntactic dialectology
literature of Swedish, as well as phonological work using established
dialectometry methods. In addition, it tests variations of the
experimental parameters in order to identify the highest-performing
parameters. In summary, this analysis allows future dialectometry
studies to include syntactic as well as phonological analyses, having
an idea of the best method and parameters to use.

There are three research questions that must be answered to determine
the reliability of this measure. They are given in chapter
\ref{questions-chapter}. First, does the measure agree with the
results of dialectology? Previous work has not addressed this
question, but it is crucial that a new measure reproduce the results
from previous linguistic work. To answer this question, the results
Swedish dialect distance results will be processed in a number of ways
so that they are comparable to previous dialect work on Swedish in
multiple ways.

Second, which parameter variations produce the best agreement with
dialectology work? Both the distance measure and feature set can be
varied, as well as a number of other parameter settings, mostly
dealing with controlling for the effects of corpus size. The distance
measures include simple measures like $R$, which is a sum of differences, more
complex variants such as Jensen-Shannon divergence, which is a sum of
logarithmic differences, and cosine similarity, which models each
corpus as a vector in high-dimensional space and finds the angle
between two corpus vectors. Feature sets can be even more varied,
although all the feature sets discussed here assume that the word is the
basic unit of syntactic analysis and that words are naturally grouped
into sentences. Some example feature sets are part-of-speech trigrams,
which are simply triples of parts of speech. Leaf-ancestor paths and
leaf-head paths use the syntactic structure of the sentence, with
leaf-ancestor paths based on constituent grammars (phrase-structure
grammars) and leaf-head paths based on dependency grammars.

Third, does the measure agree with the results of phonological
dialectometry? Agreement is not required; phonological and syntactic
dialect boundaries may disagree, but they are more likely to agree
than disagree, so if the two dialectometric measurements agree, then
this inspires confidence on the new method based on the old method's
reliability.

To answer the three research questions, I start with the statistical
method described in the previous section with the parameter variations
described above in chapter \ref{methods-chapter}. To make sure that
the results are comparable to previous dialectology, I use the dialect
corpus Swediasyn, which is a transcription of interviews recorded
throughout Sweden in various villages. The interviewees were balanced
between older and younger men and women. To generate features from the
Swediasyn, a good deal of processing is required; the corpus is a
transcription with no syntactic annotation. To annotate the Swediasyn,
I use a number of automatic annotators, trained on Talbanken, a corpus
of spoken and written Swedish. However, Talbanken does not include
dialect sources, so some amount of error is expected during the
annotation process. After annotation, feature generation is
straightforward: transformation of parse trees and other annotations.

After measuring distances between the interview sites, a number of
analytic methods are applied to the distances so that they can be
compared to dialectology work. The methods are a test of significance,
a test of correlation, cluster dendrograms and consensus trees,
composite cluster maps, multi-dimensional scaling, and feature
ranking. The tests of significance and correlation represent the
distances' trustworthiness and ability to match dialectology's
assumptions, respectively. The consensus trees, and multi-dimensional
scaling both produce maps. These maps allow the linguist to visually
compare the results with traditional region maps. In the same way,
composite cluster maps allow visual comparison of the results to
isogloss bundles from dialectology. Finally, feature ranking allows
the linguist to view the features that contribute most to separating
two regions. These features can be compared to the dialect phenomena cataloged
by dialectologists.
% Note: There are no isogloss bundles in Swedish dialectology. But
% whatever. (Also I don't really use feature ranking to compare to
% dialectology)

The results in chapter \ref{results-chapter} are presented in the same
order as their corresponding analysis appear in \ref{methods-chapter}.
The dissertation concludes with discussion in chapter
\ref{discussion-chapter}. Here, I compare the results to the dialectology
and phonological dialectometry of Swedish. Then I discuss the relation
of this work to previous work in syntactic dialectology, detailing its
contribution to the field. I finish by presenting avenues for future
work: with a statistical measure of dialect distance, dialectometry
can analyze syntactic features as well as phonological and lexical
ones, producing more complete analyses.

%%% Local Variables: 
%%% mode: latex
%%% TeX-master: "dissertation.tex"
%%% End: 


\chapter{Questions}
\label{questions-chapter}
% along with
% various kinds of backoff, for example, to bigrams or coarser node
% tags.
% TODO: Hey! This is a good idea. I should do it! It's not trivial but
% shouldn't be hard.

The state of syntax measures in dialectometry described above leaves
several research questions unresolved. It is not yet clear whether $R$
is a good measure of syntax distance. Previous results have shown that
it can obtain significant distances, but has either failed to do so
reliably, as in my work on British English \cite{sanders08b}, or has
not compared traditional dialect areas, as in
\namecite{nerbonne06}. Neither study showed that a statistical method
could adequately reproduce existing knowledge about some dialect area,
which is necessary before $R$, and statistical methods as a whole, can
contribute to dialectometry's study of syntax.

This leads to the first question: will the features found by
dialectologists agree with the highly ranked features used by a
statistical method for classification? I will investigate this
question by comparing statistical dialectometry results to the
syntactic dialectology literature on Swedish. A secondary, but related
question is whether the regions of Sweden accepted by dialectology
will be reproduced by a statistical method. For example, my previous
research on British English reproduced the well-known North
England-South England dialect regions. However, this dissertation
eliminates the corpus variability in that research, where a forty-year
gap separated the phonology and syntax corpora, and the syntax corpus
was not collected with dialectology in mind \cite{sanders08b}. With a
corpus collected for the purpose of dialect research, and with a
phonological corpus transcribed from the same interviews, more precise
comparisons should be possible, both between regions and between
syntax and phonology.

A secondary question, relevant once the utility of a statistical
measure for syntax is established, is what variations of the two
functions comprising the measure produce the best results. This
involves variation of both the feature extraction function and the
distance function. Choice of feature set is almost as important as
choice of distance. My previous work on British English showed
that leaf-ancestor paths provide a small advantage over part-of-speech
(POS) trigrams, presumably by capturing syntactic structure higher in
the parse tree. And, whereas development of a statistical distance
measure is difficult, new feature sets can be developed relatively
quickly.
% on average, each committee meeting results in 2.1 new feature sets
% being proposed, and only 0.7 new distance measures.
In this dissertation, I evaluate several feature sets besides POS trigrams and
leaf-ancestor paths, such as phrase structure rules, leaf-head paths,
and lexical trigrams. I also evaluate variants of these feature sets,
for example varying the POS tagger or POS tag set. I also evaluate
combined feature sets.

Feature sets can be evaluated by comparing performance of different
feature sets on a fixed corpus and with a fixed distance
measure. Here, performance is measured using the same criteria as for
distance measures: the number of significant distances between
interview sites and the similarity of the results to those found by
dialectologists.

Besides feature sets, this dissertation evaluates a number of measures
beyond the $R$ of previous work, such as Kullbeck-Leibler divergence
and cosine dissimilarity. $R$ is one way to aggregate features that
are created by decomposing sentences. It treats features as atomic,
and does not manipulate them in any syntax-specific ways. As such, $R$
differs from Goebl's WIV only in being designed for larger feature
sets and larger corpora. Both assume that independent, atomic features
derived from a sentence can adequately capture dialect differences. If
this is not the case, then a more syntax-aware way of comparing
individual features will be needed.

A final question is whether the syntactic dialectometry practiced here
agrees with phonological dialectometry on the same corpus. Unlike the
previous questions, which use agreement between syntactic
dialectometry and dialectology, there is no {\it a priori} reason to
expect syntax/phonology agreement; it is quite possible that
phonological features create one set of boundaries while syntactic
feature create another set. However, agreement between the two would
be further evidence for that statistical methods are useful for
syntactic dialectometry.

\section{Question 1 : Agreement with Dialectology}

The first question is whether a statistical dialectometry measure
agrees with dialectology. On closer inspection, this question covers a
number of more specific questions, each dealing with a specific
comparison to dialectology. First, and most important, is whether the features that
it counts most important are the same as the features discussed in the
dialectology literature. Three other questions are whether
regions, region boundaries, and distances found by this measure agree with
dialectology. Therefore, question 1 has a four-part
answer: agreement between dialectometry and dialectology on regions,
boundaries, distances, and features.

First, however, these terms from dialectology must be defined
precisely. Then the methods used to compare the dialectometry results
with dialectology can be developed.

\subsection{Definition of Dialectology Terms}

Definition of terms from dialectology is appropriate here, along with
an explanation of how they fit together. The basic unit in
dialectology is the feature, such as ``pronunciation of the word
`cow' '' or ``adjective placement in noun phrases''. During analysis,
the linguist may suspect that a certain variant of a feature is
characteristic of a particular region, but more information, usually
from a survey, is needed to make certain.

Given a survey or other source of geographical mapping information, a
boundary for a feature can be drawn. This boundary is called an
isogloss. For simple cases, isoglosses are usually simple to
determine, giving a clear line between two dialects. On the other
hand, complicated cases lead to more complicated geometry; for
example, a few occurrences of a feature variant
can be stranded in the middle of the other variant.

If a number of isoglosses coincide, they form an isogloss bundle,
which separates one region from another. Isogloss bundles are simple
in theory, but in practice they are difficult to find because
isoglosses rarely coincide perfectly. In practice, undisputed isogloss
bundles only occur between well-known dialects, such as the
boundaries between Low and High German or Northern and Southern
English of England. In cases where more precision is required, there
is not usually a sufficient number of coincident isoglosses. Even
though there may be plenty of isoglosses in the area, isoglosses so
rarely coincide that only a few may be construed as forming an
isogloss boundary.

Dialectology does not have a clear equivalent to dialectometry's
distance. The closest analog is size of isogloss bundle; dialect maps
typically indicate size of isogloss bundle by thickness of boundary
line. Additionally, regions that have many specific features known in the
dialectology literature can be inferred to be distant from the rest.

\subsection{Features}

The first aspect of dialectology to compare is the feature. To match the
features of dialectology to the features that a statistical
dialectometric method uses to produce a distance, I first need to find
discussion of Swedish dialect features in the dialectology
literature. For example, \namecite{rosenkvist07} discusses the South
Swedish apparent cleft. Here, the sentence contains an embedded clause
with similar surface appearance to a true cleft. Unlike a true cleft,
however, there is no clefted constituent in the matrix clause. The
apparent cleft appears in southern Sweden, but its precise
distribution is not known; Rosenkvist finds some uses everywhere
except Norrland (northern Sweden), but finds heaviest use in the
former Danish provinces in the south.

%% This stuff makes no sense!
% This feature is best analyzed as a
% single feature; Rosenkvist mentions that it occurs in southern and
% central Swedish and not northern Swedish, but does not give a more
% precise location than ``Svealand and G\"otaland''. So we will compare
% this feature directly to features found in any part of the corpus.

Next the feature should be expressed formally. This
formal description can then be translated to the format representation
used by the dialectometry. Again, these would be the same ideally, but
the dialectology study may not be complete; for example, Rosenkvist's
2007 paper does not yet include a syntactic analysis. And even in
cases where the dialectology gives a formal description, the syntactic
features of dialectometry, for example those described in the next
chapter, are based on more primitive formalisms at present. Therefore
the translation may lose information. Once translated, the features
discussed by dialectologists should appear in the high-ranked features
on which the statistical dialectometry method bases its distance.

In the apparent cleft example, the apparent cleft is realized as an
additional use of the word {\it som}, ordinarily a
complementizer. Typically, the next step is to identify the minimalist
structure for this, but Rosenkvist's 2007 paper does not yet provide
this analysis. Although there is no structure to translate to a
phrase-structure skeleton, his analysis provides enough clues to
produce some features directly. Part-of-speech n-grams are easiest; he
mentions that his corpus search used the strings {\it det \"ar som}
(``It is that'') and {\it det \"ar bara som} ``It's just that''. These
words only need part-of-speech annotation to be n-gram
features. Leaf-head paths can also use these parts of speech for the
local dependencies between {\it det,\"ar}, and {\it som}. Rosenkvist
also mentions some syntactic properties of apparent clefts that are
useful for specifying leaf-head path features: the subject of the {\it
  som}-clause must be a pronoun, so we should expect to see leaf-head
paths of the form {\it ROOT-som-PRON} in the regions that have the
apparent cleft.

Once dialectometric features have been specified from some linguistic
analysis, the analysis consists of the following questions: in what
regions do these features appear? Do these regions match the expected
distribution (if any) from the linguistic analysis? How much do
the features contribute to distance from other regions? If there are
other features that contribute more, what are they?

% TODO: Finish example by using structure from papers on the apparent
% cleft (if it exists) in English and Japanese.

\subsection{Isogloss Boundaries}
%add connective sentences! or paragraph!

Isogloss boundaries are intermediate in complexity between features unspecified
for location and regions demarcated by isogloss bundles. For the
purposes of this dissertation, however, there is not much difference
between a feature with some documented locations and an isogloss
boundary. An isogloss makes the regions of interest clearer, but it is
a difference in degree and not in quality. The real difference in
analysis occurs when dialectology has identified an isogloss bundle.

\subsection{Isogloss Bundles}
%add connective sentences! or paragraph!

Isogloss bundles compare straightforwardly to dialectometry, once
regions have been identified from the dialectometric distances between
sites. There are two primary methods: hierarchical clustering and
multi-dimensional scaling. Neither method is perfect; as with isogloss
bundles, some human input is still needed to determine whether an
inter-region boundary truly exists at some point.

Hierarchical clustering produces well-delineated regions by
recursively merging sites into regions, at the cost of some
uncertainty---the results tend to vary quite a bit from feature set to
feature set. Only clusters that persist between results from multiple
feature sets should be considered valid. Consensus trees aggregate
multiple cluster dendrograms into a stable tree; see figure
\ref{consensus-example-small} for an example. However, because of
the recursive, nested nature of the grouping, there can still be a
question of which level of nesting is appropriate to treat as
a region.

In contrast, multi-dimensional scaling (MDS) is a mathematical
transformation of the high-dimensional space created by measuring
distances between all sites in the corpus; see figure
\ref{mds-example-small} for an example and section \ref{mds} for a
complete discussion. Although MDS does not produce
spurious information, its results are often hard to analyze because it
produces boundaries of varying strength. Very different regions stand
out, but similar regions appear similar even if they contain some
differences. This similarity can make it difficult to decide whether
an area should be considered one region or two.

\begin{figure}
  \includegraphics[scale=0.4]{Sverigekarta-Landskap-consensus-5-1000}
 \caption{Swedia, Consensus Tree Map}
  \label{consensus-example-small}
\end{figure}

\begin{figure}
  \includegraphics[scale=0.4]{Sverigekarta-mds-1-1000-js-trigram-ratio}
 \caption{Swedia, Multi-Dimensional Scaling of Trigrams measured by
    Jensen-Shannon divergence}
  \label{mds-example-small}
\end{figure}

Once both dialectologic and dialectometric regions have been
identified, comparison is straightforward. Each region can be checked
for overlap---regions with a greater overlap area are better matches.

\subsection{Distances}

Although comparing distances from dialectometry to qualitative
research in dialectology is possible, it is not very precise, because
the dialectometric distances must first be translated to something
like the isogloss bundles of dialectometry. Composite cluster maps
provide this translation by drawing dark boundaries when large
distances separate regions; see figure
\ref{composite-example-small} and discussion in section
\ref{methods-composite-clustering}. Alternatively, statements like ``in
general, Southern Swedish is syntactically identical to Standard
Swedish'' \cite{rosenkvist07} can be construed as saying, roughly,
that there is very l ittle distance between Southern and Standard
Swedish. Ultimately, though, the distances from a quantitative
analysis do not have a clear analogue in qualitative analyses.

\begin{figure}
  \includegraphics[scale=0.4]{Sverigekarta-cluster-1-full}
 \caption{Swedia, Composite Cluster Map}
  \label{composite-example-small}
\end{figure}


\section{Question 2 : Variations on the Measure}

The second question of this dissertation reflects the fact that the
distance measures in dialectometry have two
parts. The first part is the function used to extract features from a
corpus and the second is the distance measure that produces a distance
between the features of two corpora. This dissertation investigates a
number of implementations for both functions. The question is which
combination provides the best performance, as measured by agreement
with dialectology.

Specification of feature sets is not difficult; feature sets are
easier to create than distance measure algorithms, as discussion of
distance measures below will show. In addition, feature sets are
easier to combine and to tweak. The real problem is not in
specification of feature sets, but that new feature sets must be
evaluated, since it is not currently possible to produce features
based on a linguistic theory as with phonology's distinctive features.

For example, in previous work, I showed that leaf-ancestor paths have
a small advantage of trigrams \cite{sanders07} in terms of finding
significant distances. Therefore, Question 2 breaks into two smaller
questions: (1) how can new variations be proposed? and (2) how can
they be evaluated? However, before these questions are explored, an
definition of terms related to distance measures is in order.

\subsection{Definition of dialectometry terms}

There are several terms related to distance in mathematics. In order
from least restrictive to most restrictive, they are `divergence',
`dissimilarity' and `distance'. In this dissertation, a `measure' is
used to refer to any of these three functions.  All three kinds of
functions must always return positive numbers, and only return 0 for
corpora that are equal.  A symmetric function returns the same number
whether measuring from point X to point Y or from point Y to point
X. The triangle inequality means that distance from point X to point Y
plus point Y to point Z is at least as long as traveling straight from
point X to point Z.  In other words, it means that it will always be
longer to take the two-leg path than to take the single-leg
path. Equations
\ref{distance-properties-positive}-\ref{distance-properties-triangle}
list the properties formally.

\begin{equation}
  d(x,y) \ge 0
  \label{distance-properties-positive}
\end{equation}

\begin{equation}
 d(x,y) = 0 \textrm{ iff } x=y
 \label{distance-properties-eqq}
\end{equation}

\begin{equation}
  d(x,y) = d(y,x)
\end{equation}

\begin{equation}
  d(x,y) + d(y,z) \ge d(x,z)
\label{distance-properties-triangle}
\end{equation}

A divergence satisfies equations \ref{distance-properties-positive}
and \ref{distance-properties-eqq}: it is always positive and only zero
when two sites are equal. It is less restrictive than the other two
kinds of measures, and is the only one that can capture the common
dialect situation where speakers of dialect X can understand speakers
of dialect Y better than speakers of Y understand those of
X. Unfortunately, the methods used for aggregate comparison in
dialectology, such as hierarchical dendrograms and multi-dimensional
scaling (MDS), require more restrictive measures. Specifically,
dissimilarities are divergences that are in addition
symmetric. Dissimilarities can be used in hierarchical dendrograms and
MDS because multiple dissimilarity comparisons can be mapped into
distance space by situating each pair of sites in its own orthogonal
dimension. This high-dimensionality avoids violating the triangle
inequality. Finally, distances are dissimilarities that additionally
satisfy the triangle inequality without special consideration, so
multiple pairwise comparisons can inhabit the same
dimensions. However, this is not necessary for the analyses in this
dissertation.

Therefore, the measures described in the rest of the dissertation will
be dissimilarities, but not necessarily distances. In the rest of the
dissertation, `distance' will usually be used as a generic term to
refer to a dissimilarity; exceptions where the term `distance' implies
all three properties will be noted. In addition, some of the
dissimilarities have common names that contain other terms. For
example, Kullback-Leibler divergence is augmented here to behave as a
dissimilarity, but it retains its original name when mentioned.

\subsection{Feature Sets}

New feature sets are easy to propose. All that is needed is some way to
condense or divide the information about the sentence into symbols
that can be used as input to a statistical distance
measure. Specifically, the feature sets used in this dissertation use
per-word information, word-order information, and syntactic
information. They attach some information from the constituent tree or
dependency graph to each word, dividing the information according to
the word's position in the sentence. Trigrams attach the leaves to
each word, along with the leaves to the left and right.
Leaf-ancestor paths attach vertical slices of the tree to each
word. Leaf-head paths attach the path to the root to each word.

Feature sets that use other information might also be useful;
convolution kernels give a single number that captures the difference
between two trees \cite{collins01}; a similar feature that captures
aspects of a single tree such as depth, branching degree or
homogeneity might be useful. Besides this, there are numerous simple
features used in other computational linguistic work that attempt to
capture the most important characteristics of a sentence in a simple,
ad-hoc way, such as the first or last $n$ words of a sentence, a
certain number of words surrounding the predicate, or sentence length.

Even before looking at results, it seems that each of these has its
own advantages and disadvantages. Leaf-ancestor paths capture upper
structure of the constituent parse, but no left and right
context. Leaf-head paths capture some of the sentence structure, but
some of the surrounding context as well. Trigrams capture only
immediate left-right context, but include word order information. They
are also less influenced by annotator error since they require only
part-of-speech annotation.

Because evaluation of feature set performance is necessarily evaluation
of the overall combination of feature set and measure, the previously
discussed measures of agreement with dialectology should all be used
as measures of performance. With the distance measure held constant,
the different feature sets can be evaluated against one another.

Before comparison, though, the distances produced for a given
combination of feature set and measure must be checked for
significance. For example, a very sensitive combination could be
inappropriate for small data sets if it can only achieve significance
with large data sets. The significance test ensures that subsequent
evaluation is valid.

\subsection{Distance Measures}

Of the measures considered in this dissertation, $R$ and $R^2$ have
been tested in previous work. $R$ is quite simple; it is a sum of
differences of features. It treats features as opaque symbols; it is
not necessarily limited syntax. Perhaps because of its simplicity, $R$
performs more consistently than other measures tested in this
dissertation: It gives significant results across a larger variety of
feature sets than more complicated measures do.

% The question of measure is more important than feature set because
% measures are harder to construct than feature sets. The measure also
% has a greater effect on the results, and the relation between measure
% and the quality of its results is not as obvious as the same relation
% between feature set and quality of results.

% TODO: The introduction and justification here are worthless. Their
% statements are false or irrelevant. So maybe this section as a whole
% is worthless.

There are two obvious directions to explore when creating a distance
measure to replace $R$. The first direction is to address $R$'s
simplicity by defining a more complex measure that uses sophisticated
ways to measure difference over still-opaque symbolic features. The
second direction is to address $R$'s ignorance of syntax by defining a
measure with specific knowledge of syntax. Finding candidates for the
first direction is easier, given the number of statistical measures
commonly used in computational linguistics. Additionally, the
dialectometric model that divides a measure into distance measure and
feature set is powerful enough that most syntax-specific knowledge can
be represented in terms of features instead of integrated into the
distance measure's algorithm.

Indeed, this makes syntax-aware measures difficult to
specify---they must incorporate knowledge of syntax in a way that
cannot be reified as features. Unlike dialect surveys of phonology,
dialect interviews do not consist of aligned lists of sentences. That
means that pairwise sentence-to-sentence comparison are impossible;
comparison must occur at a lower level. This constraint makes it
difficult to encode any useful awareness of syntax into a syntax-aware
distance measure that cannot be easily represented in the feature set
for a syntax-ignorant measure instead.

It is so difficult to define a useful syntax-aware distance measure
that none are presented in this dissertation. Syntax awareness is
restricted to the feature sets. However, a number of more complicated
statistical measures similar to $R$ are presented. Evaluation of the
distance measures proceeds similarly to evaluation of feature sets;
results for various measures are compared, holding the feature set
constant. The results are checked for significance, then for agreement
with dialectology.

% The hidden structure available for syntax is the parse---whether this
% is a constituent parse, dependency parse or some variant of shallow
% parse. Working by analogy from phonology, segments have hidden
% structure in the form of distinctive features. Segment order in
% phonology corresponds to word order in syntax. Unfortunately, as just
% mentioned, the analogy does not extend to corpus order. However, there
% is one piece of information that is not available to phonology (at
% least pre-autosegmental phonology): the upper structure is
% connected. For the current set of experiments, the upper structure is
% processed, divided and assigned as features attached to an individual
% word in the form of leaf-ancestor paths or dependency paths. This
% approach allows the features to be given to $R$ and treated as if they
% are independent, which is of course not true.

\section{Question 3 : Agreement with Phonological Dialectometry}

Finally, agreement with phonological dialectometry is a useful
indicator of quality. Agreement with phonology indicates a good
feature set, but cannot indicate a bad feature set. Phonological
boundaries need not agree with syntactic boundaries, but it seems {\it
  a priori} likely that they do. Note that agreement with phonology
has the reverse implication of statistical
significance---a test for significance can only indicate a bad
feature set, not prove a good feature set.

There is very little phonological dialectometry for Swedish, so this
comparison may not be valid yet. The only published paper, to my
knowledge, is \namecite{leinonen08}. Leinonen has extended this work
to a dissertation, which is currently unpublished.

%%% Local Variables: 
%%% mode: latex
%%% TeX-master: "dissertation.tex"
%%% End: 


\chapter{Methods}
\label{methods-chapter}

I first discuss previous work that mine is related to but does not
use. Then I discuss previous work, including previous work that I have
done, that I use in this dissertation. Finally, I discuss the methods
used to prepare the data for input and analyze the results.

There is a lot of background information, like other distance
measures and the underlying math of the distance measure I did use.

Here is what Josh misunderstood: he thought that I was taking the top
3 features first from each set of features (which have the set of
features normalized across sites, which I don't do, because to do this
properly would obviate the need for MDS since I'd be working with a
distance instead of a dissimilarity). But the top 3 would be different
for each site pair. THEN he thought I was doing MDS.

\section{Related Work}

\subsection{S\'eguy}

Measurement of linguistic similarity has always been a part of
linguistics. However, until \namecite{seguy73} dubbed a new set of
approaches `dialectometry', these methods lagged behind the rest of
linguistics in formality. S\'eguy's quantitative analysis
of Gascogne French, while not aided by computer, was the predecessor
of more powerful statistical methods that essentially required the use
of computer as well as establishing the field's general dependence on
well-crafted dialect surveys that divide incoming data along
traditional linguistic boundaries: phonology, morphology, syntax, etc.
This makes both collection and analysis easier, although it requires
more work to combine separate analyses to produce a complete picture of dialect
variation.

The project to build the Atlas Linguistique et Ethnographique de la
Gascogne, which S\'eguy directed, collected data in a dialect survey
of Gascogne which asked speakers questions informed by different areas
of linguistics. For example, the pronunciation of `dog' ({\it chien})
was collected to measure phonological variation. It had two common
variants and many other rare ones: [k\~an], [k\~a], as well as [ka],
[ko], [kano], among others. These variants were, for the most part,
% or hat "chapeau": SapEu, kapEt, kapEu (SapE, SapEl, kapEl
known by linguists ahead of time, but their exact geographical
distribution was not.

The atlases, as eventually published, contained not only annotated
maps, but some analyses as well. These analyses were what S\'eguy named
dialectometry. Dialectometry differs from previous attempts to find
dialect boundaries in the way it combines information from the
dialect survey. Previously, dialectologists found isogloss
boundaries for individual items. A dialect boundary was generated when
enough individual isogloss boundaries coincided. However, for any real
corpus, there is so
much individual variation that only major dialect boundaries can
be captured this way.

S\'eguy reversed the process. He first combined survey data to get
a numeric score between each site. Then he posited dialect boundaries
where large distances resulted between sites. The difference is
important, because a single numeric score is easier to
analyze than hundreds of individual boundaries.
Much more subtle dialect boundaries are visible this way; where before
one saw only a jumble of conflicting boundary lines, now one sees
smaller, but consistent, numerical differences separating regions. {Dialectometry
  enables classification of gradient dialect boundaries, since now one
can distinguish weak and strong boundaries. Previously, weak
boundaries were too uncertain.}

However, S\'eguy's method of combination is simple both
linguistically and mathematically. When comparing two sites, any
difference in a response is counted as 1. Only identical
responses count as a distance of 0. Words are not analyzed
phonologically, nor are responses weighted by their relative amount
of variation. Finally, only geographically adjacent sites are
compared. This is a reasonable restriction, but later studies were
able to lift it because of the availability of greater computational
power. Work following S\'eguy's improves on both aspects. In
particular, Hans Goebl developed dialectometry models that are
more mathematically sophisticated.

\subsection{Goebl}

Hans Goebl emerged as a leader in the field of dialectometry,
formalizing the aims and methods of dialectometry. His primary
contribution was development of various methods to combine individual
distances into global distances and global distances into global clusters. These
methods were more sophisticated mathematically than previous
dialectometry and operated on any features extracted from the data. His
analyses have used primarily the Atlas Linguistique de Fran\c{c}ais.

\namecite{goebl06} provides a summary of his work. Most relevant for
this paper are the measures Relative Identity Value and Weighted
Identity Value. They are general methods that are the basis for nearly
all subsequent fine-grained dialectometrical analyses. They have three
important properties. First, they are independent of the source
data. They can operate over any linguistic data for which they are
given a feature set, such as the one proposed by \namecite{gersic71} for
phonology. Second, they can compare data even for items that do not
have identical feature sets, such as Ger\v{s}i\'c's $d$,
which cannot compare consonants and vowels. Third, they can compare
data sets that are missing some entries. This improves on S\'eguy's
analysis by providing a principled way to handle missing survey
responses.

Relative Identity Value, when comparing any two items, counts the
number of features which share the same value and then discounts
(lowers) the importance of the result by the number of unshared
features. The result is a single percentage that indicates
relative similarity. Calculating this distance between all pairs
of items in two regions produces a matrix which can be used for
clustering or other purposes. Note that the presentation below splits
Goebl's original equations into more manageable pieces; the high-level
equation for Relative Identity Value is:

\begin{equation}
  \frac{\textrm{identical}_{jk}} {\textrm{identical}_{jk} - \textrm{unidentical}_{jk}}
\label{riv}
\end{equation}
For some items being compared $j$ and $k$. In this case
\textit{identical} is
\begin{equation}
  \textrm{identical}_{jk} = |f \in \textrm{\~N}_{jk} : f_j = f_k|
\end{equation}
where $\textrm{\~N}_{jk}$ is the set of features shared by  $j$ and
$k$ and $f_j$ and $f_k$ are the value of some feature $f$ for $j$ and
$k$ respectively. \textit{unidentical} is defined similarly, except
that it counts all features N, not just the shared features
$\textrm{\~N}_{jk}$.

\begin{equation}
  \textrm{unidentical}_{jk} = |f \in \textrm{N} : f_j \neq f_k|
\end{equation}

Weighted Identity Value is a refinement of Relative Identity
Value. This measure defines some differences as more
important than others. In particular, feature values that only occur
in a few items give more information than feature values that appear
in a large number of items. This
idea shows up later in the normalization of syntax distance given by
\namecite{nerbonne06}.

The mathematical reasoning behind this idea is fairly simple. Goebl
is interested in feature values that occur in only a few items. If a
feature has some value that is shared by all of the items, then all
items belong to the same group. This feature value provides {\it no}
useful information for distinguishing the items.  The situation
improves if all but one item share the same value for a feature; at
least there are now two groups, although the larger group is still not
very informative.  The most information is available if each item
being studied has a different value for a feature; the items fall
trivially into singleton groups, one per item.

Equation \ref{wiv-ident} implements this idea by discounting
the \textit{identical} count from equation \ref{riv} by
the amount of information that feature value conveys. The
amount of information, as discussed above, is based on the number of
items that share a particular value for a feature. If all items share
the same value for some feature, then \textit{identical} will be discounted all the
way to zero--the feature conveys no useful information.
Weighted Identical Value's equation for \textit{identical} is
therefore
\begin{equation}
  \textrm{identical} = \sum_f \left\{
  \begin{array}{ll}
    0 & \textrm{if} f_j \neq f_k \\
    1 - \frac{\textrm{agree}f_{j}}{(Ni)w} & \textrm{if} f_j = f_k
  \end{array} \right.
\label{wiv-ident}
\end{equation}

\noindent{}The complete definition of Weighted Identity Value is
\begin{equation} \sum_i \frac{\sum_f \left\{
  \begin{array}{ll}
    0 & \textrm{if} f_j \neq f_k \\
    1 - \frac{\textrm{agree}f_j} {(Ni)w} & \textrm{if} f_j = f_k
\end{array} \right.}
  {\sum_f \left\{
  \begin{array}{ll}
    0 & \textrm{if} f_j \neq f_k \\
    1 - \frac{\textrm{agree}f_j} {(Ni)w} & \textrm{if} f_j = f_k
    \end{array} \right. - |f \in \textrm{N} : f_j \neq f_k|}
  \label{wiv-full}
  \end{equation}

  \noindent{}where $\textrm{agree}f_{j}$ is the number of items that agree
  with item $j$ on feature $f$ and $Ni$ is the total number of
  items ($w$ is the weight, discussed below). Because of the
  piecewise definition of \textit{identical}, this number is always at
  least $1$ because $f_k$ agrees already with $f_j$.
  This equation takes the count of shared features and weights
  them by the size of the sharing group. The features that are shared
  with a large number of other items get a larger fraction of the normal
  count subtracted.

  For example, let $j$ and $k$ be sets of productions for the
  underlying English segment /s/. The allophones of /s/ vary mostly on the feature
  \textit{voice}. Seeing an unvoiced [s] for /s/ is less ``surprising'' than
  seeing a voiced [z], so the discounting process should
  reflect this. For example, assume that an English corpus contains 2000
  underlying /s/ segments. If 500 of them are realized as [z], the
  discounting for \textit{voice} will be as follows:

  \begin{equation}
    \begin{array}{c}
      identical_{/s/\to[z]} = 1 - 500/2000 = 1 - 0.25 = 0.75 \\
      identical_{/s/\to[s]} = 1 - 1500/2000 = 1 - 0.75 = 0.25
    \end{array}
    \label{wiv-voice}
  \end{equation}

  Each time /s/ surfaces as [s], it only receives 1/4 of a point
  toward the agreement score when it matches another [s]. When /s/
  surfaces as [z], it receives three times as much for matching
  another [z]: 3/4 points towards the agreement score. If the
  alternation is even more weighted toward faithfulness, the ratio
  changes even more; if /s/ surfaces as [z] only 1/10 of the time,
  then [z] receives 9 times more value for matching than [s] does.

  The final value, $w$, which is what gives the name ``weighted
  identity value'' to this measure, provides a way to control how much
  is discounted. A high $w$ will subtract more from uninteresting
  groups, so that \textit{voice} might be worth less than
  \textit{place} for /t/ because /t/'s allophones vary more over
  \textit{place}. In equation \ref{wiv-voice}, $w$ is left at 1 to
  facilitate the presentation.

\section{Dialectometry}

It is at this point that the two types of analysis, phonological and
syntactic, diverge. Although Goebl's techniques are general enough to
operate over any set of features that can be extracted, better results
can be obtained by specializing the general measures above to take
advantage of properties of the input.  Specifically, the application
of computational linguistics to dialectometry beginning in the 1990s
introduced methods from other fields. These methods, while generally
giving more accurate results quickly, are tied to the type of data on
which they operate.

Currently, the dominant phonological distance measure is Levenshtein
distance. This distance is essentially the count of differing
segments, although various refinements have been tried, such as
inclusion of distinctive features or phonetic
correlates. \namecite{heeringa04} gives an excellent analysis of the
applications and variations of Levenshtein distance. While Levenshtein
distance provides much information as a classifier, it is limited
because it must have a word aligned corpus for comparison. A number of
statistical methods have been proposed that remove this requirement
such as \namecite{hinrichs07} and \namecite{sanders09}, but none have
been as successful on existing dialect resources, which are small and
are already word-aligned. New resources are not easy to develop
because the statistical methods still rely on a phonetic transcription
process.

\subsection{Syntactic Distance}

Recently, computational dialectometry has expanded to analysis of
syntax as well. The first work in this area was \quotecite{nerbonne06}
analysis of Finnish L2 learners of English, followed by
\quotecite{sanders07} analysis of British dialect areas. Syntax
distance must be approached quite differently than phonological
distance. Syntactic data is extractable from raw text, so it is much
easier to build a syntactic corpus. But this implies an associated
drop in manual linguistic processing of the data. As a result, the
principal difference between present phonological and syntactic
corpora is that phonology data is word-aligned, while syntax data is
not sentence-aligned. Automatically constructed syntactic corpora
lead naturally to statistical measures over large amounts of data
rather than more sensitive measures that operate on small corpora.

\namecite{nerbonne06} were the first to use the syntactic distance
measure described below. They analyzed two corpora, both of Finnish
L2 speakers of English. The first corpus was gathered from speakers
who learned English after childhood and the second was gathered from
speakers who learned English as children. Nerbonne \& Wiersma found a
significant difference between the two corpora. The trigrams that
contributed most to the difference were those in the older corpus that
are unexpected in English. For example, the trigram COP-ADJ-N/COM is
not common in English because a noun phrase following a copula
typically begins with a determiner. Other trigrams indicate
hypercorrection on the part of the older speakers; they appear in the
younger corpus but not as often. Nerbonne \& Wiersma analyzed this as
interference from Finnish; the younger learners of English learned it
more completely with less interference from Finnish.

My subsequent work in \cite{sanders07} and \cite{sanders08b}
expanded on the Finnish experiment in two ways. First, it introduced
leaf-ancestor paths as an alternative feature type. Second, it tested
the distance method on a larger set of corpora: Government Office
Regions of England, as well as Scotland and Wales, for a total of
11 corpora. Each was smaller than the Finnish L2 corpora, so the
permutation test parameters had to be adjusted for some feature
combinations.

The distances between regions were clustered using hierarchical
agglomerative clustering, as described in section
\ref{cluster-analysis}. The resulting tree showed a North/South
distinction with some unexpected differences from previously
hypothesized dialect boundaries; for example, the Northwest region
clustered with the Southwest region. This contrasted with the
clustered phonological distances also produced in
\namecite{sanders08b}. In that experiment, there was no significant
correlation between the inter-region phonological distances and
syntactic distances.

There are several possible reasons for this lack of correlation. The
two distance measures may find different dialect boundaries based on
differences between syntax and phonology. Dialect boundaries may have
shifted during the 40 years between the collection of the SED and the
collection of the ICE-GB. One or both methods may be measuring the
wrong thing. In this dissertation, although the focus remains on results
of computational syntax distance as compared to traditional syntactic
dialectology, the discussion compares recent phonological
dialectometry results on Swedish to the results obtained here.

\subsubsection{Nerbonne and Wiersma}
\label{nerbonne06}

Due to the lack of alignment between the larger corpora available for
syntactic analysis, a statistical comparison of differences is more
appropriate than the simple symbolic approach possible with the
word-aligned corpora used in phonology. This statistical approach
means that a syntactic distance measure will have to use counting as
its basis.

\namecite{nerbonne06}'s method models syntax by part-of-speech (POS)
trigrams and uses differences between trigram type counts in a
permutation test of significance. The heart of the measure is simple:
the difference in type counts between the combined features of two
corpora. \namecite{kessler01} originally proposed this measure, the
{\sc Recurrence} metric ($R$):

\begin{equation}
R = \sum_i |c_{ai} - c_{bi}|
\label{rmeasure}
\end{equation}

\noindent{}Given two corpora $a$ and $b$, $c_a$ and $c_b$ are the
feature counts. $i$ ranges over all features, so $c_{ai}$ and $c_{bi}$ are the
counts of corpora $a$ and $b$ for feature $i$. $R$ is designed to
represent the amount of variation exhibited by the two corpora while
the contribution of individual features remains transparent to aid later
analysis. Unfortunately, it doesn't indicate whether its results are significant; a
permutation test is needed for that, described in section
\ref{permutationtest}.

\subsubsection{Dialectometry in British English}

The methods used in this dissertation are an evolution of those in my
previous work on British English: \cite{sanders07} and
\cite{sanders08b}. There, I compared phonological and syntactic
dialectometry as described above. The process is similar to Wiersma's
work in \cite{nerbonne06} and \cite{wiersma09}. Here, it is slightly
simpler and with a few variations.

The input are 30 corpora, one for each interview site (described in
section \ref{syntactically-annotated-corpus}). The sentences in each
corpus have their features generated (the features are described in
\ref{syntactic-features}). Optionally, only 1000 sentences are sampled
with replacement, but the corpus sizes, unlike the British interviews
in my previous work, are fairly similar in size so this is only
necessary for comparison to previous work. Then the features are
counted, producing a mapping of feature types to token counts.

At this point, two corpora are compared based on these feature
counts. The feature counts are first normalized to account for
variation in corpus size and sentence length (described in
the next section). Optionally, they are converted to ratios, meaning
that the counts are scaled relative to the other corpus. For example,
counts of 10 and 30 would produce the ration 1 to 3, as would the
counts 100 and 300. Finally, the distance (described above in
\ref{nerbonne06}) is calculated 10 times and the result is averaged.

The corpus is sampled by sentence rather than by feature because the
intent is to capture syntax, where the composite unit is the
sentence. Similarly, phonology's composite unit is the word---most
processes operate within the word on individual segments; some
processes operate between words but they are fewer. Therefore, the
assumption that words are independent will lose some information but
not the majority. In the same way, the basic unit of syntax is the
sentence; processes operate on the words in the sentence, but
inter-sentence processes are fewer. Because of this, the corpora are
sampled by sentence, combining the sentences of all speakers from an
interview site into a single corpus.

This dissertation skips the per-speaker sampling of
\quotecite{wiersma09} work on Finnish L2 speakers. I assume that,
since discovery of dialect features is the goal of this research, the
sentences of speakers from the same village are independent of the
speaker, at least with respect to dialect features. Although the
motivation is partly theoretical, there is also a difference between
the Swediasyn dialect corpus, with 2-4 speakers for each of 30 sites,
and Wiersma's L2 corpus, with dozens of speakers but only two
groups. Sampling per-speaker would not be feasible for the Swediasyn
because there aren't enough speakers per village.

\subsubsection{Normalization}

The two corpora being compared can differ in size, even if they are
samples with the same number of sentences; if one corpus contains many
long sentences with the other contains many short ones, raw counts
will favor the features extracted from the long sentences simply
because each sentence yields more features. Additionally, the counts can
optionally be converted to ratios to ignore the effect of
frequency---in effect, this ranks features only by how much they differ
between the two corpora, ignoring the question of how often they occur
relative to the other features extracted from the corpora. That is,
a high ratio for a rare feature that happens only ten times in both
corpora is just as important as a high ratio for a common feature that
happens thousands of times.

The first normalization normalizes the counts for each feature within the
pair of vectors $a$ and $b$. The purpose is to normalize the
difference in sentence length, where longer sentences with more words
cause features to be relatively more frequent than corpora with many
short sentences.  Each count $c_i$ in the vector $c$ is converted to a
frequency $f_i$ \[f_i=\frac{c_i}{N} \] where $N$ is the length of
$c$. For two corpora $c_a$ and $c_b$ this produces two frequency
vectors, $f_{a}$ and $f_{b}$.Then the original counts in $c_a$ and
$c_b$ are redistributed according to the frequencies in $f_a$ and $f_b$:
\[\forall j \in a,b : c'_{ji} = \frac{f_{ji}(c_{ai}+c_{bi})}{f_{ai}+f_{bi}}\]
This redistributes the total of a pair from $a$ and $b$ based on
their relative frequencies. In other words, the total for each feature
remains the same:
\[ c_{ai} + c_{bi} = c'_{ai} + c'_{bi} \]
but the values of $c_{ai}$ and $c_{bi}$ are scaled by their frequency
within their respective vectors.

For example, assume that the two corpora have 10 sentences each, with
a corpus $a$ with only 40 words and another, $b$, with 100 words. This
results in $N_a = 40$ and $N_b = 100$. Assume also that there is a
feature $i$ that occurs in both: $c_{ai} = 8$ in $a$ and $c_{bi} = 10$ in
$b$. This means that the relative frequencies are $f_{ai} = 8/40 = 0.2$
and $f_{bi} = 10/100 = 0.1$. The first normalization will redistribute the
total count ($10 + 8 = 18$) according to relative frequencies. So
\[c_{ai}' = \frac{0.2(18)}{0.2+0.1} = 3.6 / 0.3 = 12\] and
\[c_{bi}' = \frac{0.1(18)}{0.2+0.1} = 1.8 / 0.3 = 6\] Now that 8 has
been scaled to 12 and 10 to 6, the fact that corpus $b$ has more words
has been neutralized. This reflects the intuition that something that
occurs 8 of 40 times is more important than something that occurs 10
of 100 times.

% this is the (*2n / N) bit
The second normalization normalizes all values in both permutations
with respect to each other. This is simple: find the average number of
times each feature appears, then divide each scaled count by it. This
produces numbers whose average is 1.0 and whose values are multiples
of the amount that they are greater than the average.  The average
feature count is $N / 2n$, where $N$ is the number of feature
occurrences in both the permutations and $n$ is the number of feature
types. Division by two is necessary since we are multiplying counts
from a single permutation by merged counts from both
permutations. Each entry in the ratio vector now becomes \[\forall j
\in a,b : r_{ji} = \frac{2nc_{ji}'}{N}\]

For example, given the previous example numbers, this second
normalization first finds the average. Assuming 5 unique features for
$a$'s 40 total features and 30 for $b$'s total 100 features gives \[n
= 5 + 30 = 35\] and
\[N = 40 + 100 = 140\]
Therefore, the average feature has $140 / 2(35) = 2$
occurrences in $a$ and $b$ respectively. Dividing $c_{ai}' = 12$ and
$c_{bi}' = 6$ by this average gives $r_{ai} = 6$
and $r_{bi} = 3$. In other words, $r_{ai}$ occurs 6 times more
than the average feature.

\subsection{Syntax Features}
\label{syntactic-features}

In order to investigate hypothesis 1, the distance measure, such as
$R$, needs features that capture the dialect syntax of the interview
corpora given as input. Following Nerbonne and Wiersma 2006, I
start with parts of speech, then add the leaf-ancestor paths that I
tried on the ICE-GB, and finally add dependency-ancestor paths, as
well as variants on these three feature sets. These feature sets each
depend on a different type of automatic annotation, which described in
section \ref{parsers}.

% Hey, these should really be feature *type* sets since they are sets
% of feature types, not sets of observed feature tokens. Maybe I
% should make a note of this and say
% "feature sets" is used in the rest of this dissertation to mean
% "feature type sets" (except maybe in the discussion of the
% normalization, which should refer to vectors anyway)

\namecite{nerbonne06} argue that POS trigrams can accurately represent
at least the important parts of syntax, similar to the way chunk
parsing can capture the most important information about a
sentence. If this is true, POS trigrams are a good starting point for
a language model; they are simple and easy to obtain in a number of
ways. They can either be generated by a tagger as Nerbonne
and Wiersma did, or taken from the leaves of the trees of a
syntactically annotated corpus as I did with the
International Corpus of English \cite{sanders07}.

On the other hand, it might be better to represent the upper structure
of trees in the feature set, assuming that syntax is in fact a
phenomenon that involves hidden structure above the visible words of
the sentence. \quotecite{sampson00} leaf-ancestor paths provide one
way to do this: for each leaf in the parse tree, leaf-ancestor paths produce
the path from that leaf back to the root. Generation is simple as long
as every sibling is unique. For example, the parse tree

\Tree[.S [.NP [.Det the ] [.N dog ] ] [.VP [.V barks ] ] ]

creates the following leaf-ancestor paths:

\begin{itemize}
\item S-NP-Det-The
\item S-NP-N-dog
\item S-VP-V-barks
\end{itemize}

For identical siblings, brackets must be inserted in the path to
disambiguate the first sibling from the second.
There is one path for each word, and the root appears
in all four. However, there can be ambiguities if some
node happens to have identical siblings. Sampson gives the example
of the two trees

\Tree[.A [.B p q ] [.B r s ] ]

and

\Tree[.A [.B p q r s ] ]

which would both produce

  \begin{itemize}
  \item A-B-p
  \item A-B-q
  \item A-B-r
  \item A-B-s
  \end{itemize}

  There is no way to tell from the paths which leaves belong to which
  B node in the first tree, and there is no way to tell the paths of
  the two trees apart despite their different structure. To avoid this
  ambiguity, Sampson uses a bracketing system; brackets are inserted
  at appropriate points to produce
  \begin{itemize}
  \item $[$A-B-p
  \item A-B]-q
  \item A-[B-r
  \item A]-B-s
  \end{itemize}
and
  \begin{itemize}
  \item $[$A-B-p
  \item A-B-q
  \item A-B-r
  \item A]-B-s
  \end{itemize}

Left and right brackets are inserted: at most one
in every path. A left bracket is inserted in a path containing a leaf
that is a leftmost sibling and a right bracket is inserted in a path
containing a leaf that is a rightmost sibling. The bracket is inserted
at the highest node for which the leaf is leftmost or rightmost.

It is a good exercise to derive the bracketing of the previous two trees in detail.
In the first tree, with two B
siblings, the first path is A-B-p. Since $p$ is a leftmost child,
a left bracket must be inserted, at the root in this case. The
resulting path is [A-B-p. The next leaf, $q$, is rightmost, so a right
bracket must be inserted. The highest node for which it is rightmost
is B, because the rightmost leaf of A is $s$. The resulting path is
A-B]-q. Contrast this with the path for $q$ in the second tree; here $q$
is not rightmost, so no bracket is inserted and the resulting path is
A-B-q. $r$ is in almost the same position as $q$, but reversed: it is the
leftmost, and the right B is the highest node for which it is the
leftmost, producing A-[B-r. Finally, since $s$ is the rightmost leaf of
the entire sentence, the right bracket appears after A: A]-B-s.

At this point, the alert reader will have
noticed that both a left bracket and right bracket can be inserted for
a leaf with no siblings since it is both leftmost and rightmost. That is,
a path with two brackets on the same node could be produced: A-[B]-c. Because
of this redundancy, single children are
excluded by the bracket markup algorithm. There is still
no ambiguity between two single leaves and a single node with two
leaves because only the second case will receive brackets.

% See for yourself:
% \[\xymatrix{
%   &\textrm{A} \ar@{-}[dl] \ar@{-}[dr] &\\
%   \textrm{B} \ar@{-}[d] &&\textrm{B} \ar@{-}[d] \\
%   \textrm{p} && \textrm{q} \\
% }
% \]

% \[\xymatrix{
%   &\textrm{A} \ar@{-}[d] &\\
%   &\textrm{B} \ar@{-}[dl] \ar@{-}[dr] & \\
%   \textrm{p} && \textrm{q} \\
% }
% \]
% \cite{sampson00} also gives a method for comparing paths to obtain an
% individual path-to-path distance, but this is not necessary for the
% permutation test, which treats paths as opaque symbols.


Sampson originally developed leaf-ancestor paths as an improved
measure of similarity between gold-standard and machine-parsed trees,
to be used in evaluating parsers. The underlying idea of a collection
of features that capture distance between trees transfers quite nicely
to this application. I replaced POS trigrams with leaf-ancestor paths
for the ICE corpus and found improved results on smaller corpora than
Nerbonne and Wiersma had tested \cite{sanders07}. The additional
precision that leaf-ancestor paths provide appears to aid in attaining
significant results.

\subsubsection{Leaf-Head Paths}
\label{leaf-head-paths}
% TODO: This section should probably have a lot more examples and maybe some
% examples of other applications besides my experiment.

For dependency parses, it is easy to create a variant of leaf-ancestor
paths called ``leaf-head paths''. Like leaf-ancestor paths, each word
in the sentence is associated with a single leaf-head path. The
difference is that the path is from the leaf to the head of the sentence via the
intermediate heads. For example, the same sentence, ``The dog barks'',
produces the following leaf-head paths, given the dependency parse in
figure \ref{example-dep-parse}:

\begin{figure}
\[\xymatrix{
& & root \\
DET \ar@/^/[r]^{DT} & NP\ar@/^/[r]^{SS} & V \ar@{.>}[u] \\
The & dog & barks
}
\]
\caption{Dependency parse for ``The dog barks.''}
\label{example-dep-parse}
\end{figure}

\begin{itemize}
\item root-V-N-Det-the
\item root-V-N-dog
\item root-V-barks
\end{itemize}

The biggest difference between leaf-ancestor paths and leaf-head paths
is the relative length of the paths: long
leaf-ancestor paths indicate deep nesting of structure, while short
ones indicate flatter structure. Length is a
weaker indicator of deep structure for leaf-head
paths; for example, the verb in a nested clause has a much shorter
leaf-head path than leaf-ancestor path, but its dependents have
comparable lengths between the two types of paths. Instead, length of
path measures centrality to the sentence; longer leaf-head paths
indicate less important words.

Leaf-head paths represent a compromise between leaf-ancestor paths and
trigrams. Like trigrams, they capture lexical context, but the context
is based on head dependencies, so long-distance context is
possible. Like leaf-ancestor paths, they capture information about the
nested structure of the sentence, although not as completely or
explicitly.

\subsection{Alternate Feature Sets}
\label{alternate-feature-sets}

This section describes the variants besides the main feature sets
already described above: trigrams, leaf-ancestor paths and leaf-head
paths. Most are variants on these three main sets.

\subsubsection{Phrase Structure Rules}

Phrase structure rules are extracted from the same parses as
leaf-ancestor paths, but instead of capturing a series of parent-child
relations, it captures single-level parent-child-sibling
relations. For example, given the tree in figure
\ref{psg-example-tree} the extracted rules are given in
figure \ref{psg-example}.

\begin{figure}
\Tree[.S [.NP [.Det the ] [.N dog ] ] [.VP [.V barks ] ] ]
 \caption{Example Tree}
  \label{psg-example-tree}
\end{figure}

\begin{figure}
  \begin{tabular}{ccc}
    \Tree[.S NP VP ] & \Tree[.NP Det N ] & \Tree[.VP V ] \\
  \end{tabular}
 \caption{Phrase-Structure Rules Extracted}
  \label{psg-example}
\end{figure}

Phrase structure rules are most similar to leaf-ancestor paths in
emphasizing the upper structure of constituency parse trees. Unlike
leaf-ancestor paths, they capture some context to the left and
right. They also only cover one level in the tree, whereas
leaf-ancestor paths traverse it from leaf to root. Phrase structure
rules have the possibility to be useful in sentences where context is
important, but they also depend on having accurate parses even at the
top of the tree. This is difficult for automatic parsers to achieve.

\subsubsection{Grandparent Phrase Structure Rules}

Grandparent phrase structure rules are a variant of phrase structure
rules that include the grandparent as well. Given the tree in figure
\ref{psg-example-tree}, the extracted features are given in
figure \ref{grand-psg-example}.

\begin{figure}
  \begin{tabular}{ccc}
    \Tree[.ROOT [.S NP VP ] ] & \Tree[.S [.NP Det N ] ] &
    \Tree[.S [.VP V ] ] \\
  \end{tabular}
\caption{Grandparent Phrase-Structure Rules Extracted}
 \label{grand-psg-example}
\end{figure}

Grandparent phrase structure rules add some of the vertical information present in
leaf-ancestor paths, hopefully without introducing data sparseness
problems. However, they retain the advantage over
leaf-ancestor paths of capturing left and right context.

\subsubsection{Arc-Head Paths}

As described in section \ref{leaf-head-paths}, the usual labels for
leaf-head paths are the leaves of the tree: `root-V-N-Det-the' is the
first leaf-head path for ``The dog barks'', which has the parts of
speech ``Det N V''. However, one can
also use the arc labels of the dependency parse to create arc-head
paths. These paths have the same shape as their corresponding
leaf-head-paths, but use the labels of the dependency arcs between
words instead of the parts of speech of the words themselves.

The sentence for the leaf-head example is given in figure
\ref{example-dep-parse}, and the resulting arc-head paths are

\begin{itemize}
\item root-SS-DT-the
\item root-SS-dog
\item root-barks
\end{itemize}

\subsubsection{Tags from Berkeley Parser}

The Berkeley parser, as described in section \ref{parsers}, can
either tag incoming sentences with its own part of speech tagger,
using the same splitting process as the rest of the parser, or with
parts of speech specified externally. In this case, the external
part-of-speech tagger is T'n'T. Although the Berkeley-generated POS
tags are not as good, it may be useful to see how they change the
overall results---although it seems that accurate parts of speech are
required for good features to be generated, it is useful to see how
much the results degrade when given lower quality parts of
speech.
Note that the Berkeley-generated POS tags are used to generate
trigrams and leaf-head paths, by taking the POS tags and feeding them
into MaltParser.

To get the Berkeley parser to generate parts of speech, it is given
the interview corpora directly, skipping the tagging by T'n'T.
Using the same method it uses to parsing the higher structure of the
sentence, it also tags words with parts
of speech. After parsing, these parts of speech are extracted from the
leaves of the parse trees. First, the parts of speech are used to
create trigram features. Second, the Berkeley-tagged words are
converted to CONLL format and given to MaltParser for dependency
parsing. This produces dependency parses based on parts of speech from
the Berkeley parser.

\subsubsection{Dependencies from alternate MaltParser training}

Since MaltParser uses Nivre's oracle-based dependency parsing
algorithm, the default oracle, based on support vector machines, can
be replaced with Timbl, the Tilburg Memory-Based Learner. It is
possible that a memory-based learner improves parsing
because support vector machines depend on large
training corpora to provide good results. In contrast, a memory-based
learner can obtain good results on limited training if the training
happens to be representative and the right combination of parameters
can be found for Timbl.

This is, however, somewhat complicated since Timbl is quite sensitive
to parameter changes and usually requires specific tuning for
particular tasks. To find the best
parameters, I use a manual search across a number of the major
distance measures provided by Timbl, as well as fallback-combinations
from more complicated distance measures to less complicated ones.

Each combination was evaluated with ten-fold cross-validation on
Talbanken. The best combination was Jeffrey divergence with 5 nearest
neighbors, no feature weighting, inverse distance neighbor
weighting, and fallback to the Overlap metric for fewer than two
neighbors. Jeffrey divergence is a symmetric variant of
Kullback-Leibler divergence, also described in section
\ref{kl-divergence}. These parameter settings were used as a basis for
parsing and generation of leaf-ancestor paths.

% \subsubsection{Within-clause Dependency/Leaf-ancestor paths}

% I haven't done this. This is interview data and there are not many
% nested clauses---probably less than 1 in 3 and I don't think it would make much
% difference. Besides, it would be difficult to specify a set of
% criteria for cutting off within-a-clause---simply removing everything
% between the root and the first S would miss some nested clauses.

% All right, so maybe you really could just use parts of speech to
% tell. Even if it worked, I still don't think there would be much
% difference because very few of the important features without
% within-clause cutoff have multiple clauses--most are simple and
% non-nested. Oh, right. Simplifying to ignore clause nesting would
% increase the power of phenomena that happen regardless of nesting
% level but don't occur enough to be visible otherwise.

\subsection{Combining Feature Sets}
\label{combine-feature-sets}

Combining feature sets gives the classifier more information about a
site by combining the information that each feature set
captures. This dissertation uses a simple linear combination. In other
words, all features are counted together with equal
weight. This is easy and should allow the feature ranker
to find a greater variety of features that capture the
same underlying syntactic information.

A more sophisticated method of combining features can be
adapted from Martin Volk's method for deciding prepositional phrase
(PP) attachment \cite{volk02}. Volk uses a number of different feature
sets to decide where to attach PPs. However, he also weights each
feature set differently based on its reliability.

Volk used an a priori reliability measure for ranking
quality of combined feature types; I use number of significant
region differences for ranking: the top-ranked feature set will be
the one that produces the highest number of significant distances
between regions. So, if a feature set finds significant differences
between 90 out of 100 regions, its weight will be 0.90 and will be double
the weight for a feature set that only finds significant differences
between 45/100 regions.

% I don't think I'll do this (I probably won't even do the Volk thing
% since it involves a lot of parameterisation in icecore.h.
% Well, I *could* instead replicate each feature by its weight
% before combining, but this would probably make classification super
% slow because of the huge number of features.)
% ----
% Combinations of feature types will be ranked by
% averaging the number of significant distances that the constituent
% feature types produce.

\subsection{Alternate Distance Measures}
\label{alternate-distance-measures}

There are several reasons to test distance measures besides
$R$. There are a couple of a priori reasons for this: $R$ is fairly
simple, so more complicated variations on it may provide better
sensitivity at the expense of sensitivity to noise. Also, variations
explore the measure space better in case that $R$ is not significant
for some combination of corpus/feature set.

Post-hoc, there are interesting patterns of statistical significance
produced by the combination of distance measure and feature set. These
patterns are not trivially obvious. This is not expected, but may
provide insight into the measure/feature combination, which helps
resolving Hypotheses 1 and 2.

% Another possibility is a return to Goebl's Weighted Identity Value;
% this classifier is similar in some ways to $R$, but has not been
% tested with large corpora, to my knowledge at least. (This is not
% particularly useful and I don't believe that WIV would actually be
% good, so I should probably just drop this.)

% (maybe it was relative entropy or just normal-kind entropy).
% TODO: WIV, also Kullback-Leibler Divergence could work.
% Maybe also k-NN/MBL, HMM binary classifier (?), maybe even a
% neural net

\subsubsection{KL divergence}
\label{kl-divergence}

% TODO: Find original citation for KL divergence

Kullback-Leibler divergence, or relative entropy, is described in
\namecite{manningschutze}.  Relative entropy is similar to $R$ but
more widely used in computational linguistics. The name relative
entropy implies an intuitive interpretation: it is the number of bits
of entropy incurred when compressing a corpus $b$ with the optimal
compression scheme for a second corpus $a$. Unless the two corpora
are identical, the relative entropy $KL(a||b)$ is non-zero because
$a$'s optimal compression scheme will over-compress $b$'s
features that are more common in $a$ than in $b$, whereas it will
under-compress features that are less common in $a$ than in $b$.

For example, assume that corpus $a$ has two features with type counts
\{S-NP-N : 20, S-VP-PP-N : 10\}. An optimal compression scheme for $a$
would compress S-NP-N twice as much as S-VP-PP-N because it occurs
twice as often. However, if this compression scheme is used on a
corpus $b$ with the feature counts \{S-NP-N : 15, S-VP-PP-N : 15\},
efficiency will be worse; S-NP-N and S-VP-PP-N occur the same number
of times in $b$, so the smaller compressed size of S-NP-N will be used
less often than expected, while the larger compressed size of
S-VP-PP-N will be used more. This difference can be measured precisely
for each feature:

\[ c_{ai} \log\frac{c_{ai}}{c_{bi}} \]

where $c_{ai}$ is type count of the $i$th feature in $a$ and $c_{bi}$
is the type count of the $i$th feature in $b$. This measures the
number of bits lost, or entropy, for each feature $i$. Like $R$'s
differences, the per-feature entropy can be summed to find the total
entropy. In the example above, the entropy for S-NP-N is $20
\log\frac{20}{15} = 5.75$.

However, Kullback-Leibler divergence as defined is a divergence: it
measures the divergence of features in the corpus $c_b$ from the
features of corpus $c_a$. A dissimilarity is required for
dialectology, which means that the divergence must additionally be
symmetric. A divergence can be made symmetric by calculating it twice:
the divergence from $c_a$ to $c_b$ added to the one from $c_b$ to
$c_a$. The complete formula is given in equation \ref{klmeasure} and
the complete example is worked in equation \ref{klexample}.

\begin{equation}
KL(a||b) = \sum_i {c_{ai} \log\frac{c_{ai}}{c_{bi}} + c_{bi} \log\frac{c_{bi}}{c_{ai}}}
\label{klmeasure}
\end{equation}

\begin{equation}
 (20 \log\frac{20}{15} + 15 \log\frac{15}{20}) + (10
  \log\frac{10}{15} + 15 \log\frac{15}{10}) = (5.75 - 4.32) + (-4.05 +
  6.08) = 3.46
  \label{klexample}
\end{equation}

\subsubsection{Jensen-Shannon divergence}

Several variants of relative entropy exist that lift various restrictions from the input
distributions. One is Jensen-Shannon divergence \cite{lin91}, which
was designed as a dissimilarity from the start. It uses the same
denominator for both directions: the average of the two
probabilities. That means that each feature's entropy is found using
the following formula:

\[ c_{ai} \log\frac{c_{bi}}{(c_{ai} + c_{bi}) / 2} +
c_{bi} \log\frac{c_{ai}}{(c_{ai} + c_{bi}) / 2} \]

There is a common subexpression in this value: $(c_{ai} + c_{bi}) /
2$: the average of the two features. If we let $\bar{c_i} = (c_{ai} + c_{bi}) /
2$ rewrite the formula to take advantage of this simplification, we
get equation \ref{jsmeasure}.

\begin{equation}
JS = \sum_i {c_{ai} \log\frac{c_{bi}}{\bar{c_i}} + c_{ai}
  \log\frac{c_{bi}}{\bar{c_i}}}
\label{jsmeasure}
\end{equation}

Unlike Kullback-Leibler divergence, Jensen-Shannon divergence does not
require the feature counts to be absolutely continuous; in other
words, that if $c_{ai}$ is non-zero, then $c_{bi}$ has to be non-zero
too. Since the current implementation of Kullback-Leibler divergence
simply skips zero values, this means it ignores features unique to a
particular corpus. Jensen-Shannon divergence avoids this problem.
% In addition, it provides bounds on variational distance
% and the Bayes probability of error.
% TODO: Find out what this means

\subsubsection{Cosine similarity}

Cosine similarity is used in many parts of computational linguistics
and related areas such as information extraction and data
mining. \namecite{nerbonne06} use it as reference point for
comparison to previous work in these areas. Cosine similarity measures
the similarity between two high-dimensional points in space. Each
feature is modeled as a dimension, and the type count from each
corpus is plotted as a point on that dimension. The result is equation
\ref{cosmeasure} or the non-vectorized version in
\ref{cosmeasureiterative}.

\begin{equation}
  \frac{c_{a}\cdot c_{b}} {||c_{ai}||||c_{bi}||}
  \label{cosmeasure}
\end{equation}

\begin{equation}
  \frac{\sum_i {c_{ai}c_{bi}}} {\sqrt{\sum_i {c_{ai}^2}} +
    \sqrt{\sum_i{c_{bi}^2}}}
  \label{cosmeasureiterative}
\end{equation}

Interestingly, the results for this measure are very different from
the other distance measures, possibly because, unlike the others
described, it is not a linear sum.

\section{Input Processing}
To investigate the first question, agreement with dialectology, I need a dialect corpus that can
be syntactically annotated (\ref{syntactically-annotated-corpus}); if
it is not already annotated, it must be possible to annotate it
automatically so I can avoid time-consuming manual annotation.
Automatic annotation requires a syntactically annotated
training corpus (\ref{syntactically-annotated-training}) and parsers
for the models of syntax I use as a basis (\ref{parsers}).

% To investigate the second question, quality of features, I need a
% method to combine different types of features
% (\ref{combine-feature-sets}). I also need a way to generate new
% features that include more information about context
% (\ref{alternate-feature-sets}).

% If the distance measure $R$ doesn't provide any significant distances
% with any combination of features, I will experiment with different
% distance measures (\ref{alternate-distance-measures}). For this, there
% are quite a few possibilities; Kullback-Leibler divergence is one
% example.

\subsection{SweDiaSyn}
\label{syntactically-annotated-corpus}
% (CITE SweDiaSyn and ScanDiaSyn,
% except that they don't seem to have any references)
% Here is a citation for ScanDiaSyn if I could track it down and
% translate it
% Vangsnes, �ystein A. 2007. ScanDiaSyn: Prosjektparaplyen Nordisk dialektsyntaks. In T. Arboe (ed.), Nordisk dialektologi og sociolingvistik, Peter Skautrup Centeret for Jysk Dialektforskning, �rhus Universitet. 54-72.

The first question requires a dialect corpus that can
be syntactically annotated.
The dialect corpus used in this dissertation is SweDiaSyn, the
Swedish part of the ScanDiaSyn.
SweDiaSyn is a transcription of SweDia 2000 \cite{bruce99}. SweDia
2000 was collected
between 1998 and 2000 from 97 locations in Sweden and 10 in
Finland. Each location has 12 interviewees: three 30-minute interviews
for each of older male, older female, younger male and younger female.
However, the SweDiaSyn transcriptions do not yet include all of SweDia
2000; the completed transcriptions currently focus on older
speakers.

Currently there are 36,713 sentences of transcribed speech
from 49 sites, an average of 749 sentences per site.
However, the sites range from 110 to 1780 sentences because some sites
have fewer complete transcriptions than others.

In the SweDiaSyn, there are two types of transcription:
standard Swedish orthography, with glosses for words
not in standard Swedish; and a phonetic transcription for dialects
that differ greatly from standard Swedish. For this dissertation,
the orthographic/gloss transcription is used so that lexical
items are comparable across dialects. However,
only 30 of the 49 sites have been glossed, so the total usable size of
the corpus is 21,004 sentences, with an average of 700 sentences per
site. The sites range from 301 to 1,144 sentences.

\subsection{Talbanken}
\label{syntactically-annotated-training}

Because the first question requires a syntactically annotated
corpus, and because SweDiaSyn consists of unannotated lexical items only,
Talbanken05, a syntactically-annotated corpus, is used to train
the automatic annotators which in turn annotate SweDiaSyn. Talbanken05
is a treebank of written and transcribed spoken Swedish, roughly
300,000 words in size. It is an updated version of Talbanken76
\cite{nivre06}; Talbanken76's trees are annotated following a
scheme called MAMBA; Talbanken05 adds phrase structure annotation and
dependency annotation using the annotation formats TIGER-XML
and Malt-XML. In addition to syntactic annotation, Talbanken is
lexically annotated for morphology and part-of-speech.

% TODO: Should I keep this? It depends on how much detail I want.
% Talbanken's sources are X and Y and Z. It attempts to provide a
% valid sample of the Swedish language, both spoken and written. The
% spoken section is transcribed from conversation, interviews and
% debates, and the written section is taken from high school essays and
% professional prose (TODO:I could probably cite
% Jan Einarsson. 1976. Talbankens skriftspraakskonkordans. Lund
% University: Department of Scandinavian Languages (and
% talspraakskonkordans) IF I could legitimately claim that I got the
% information from there\ldots{} but of course I got it from
% spraakbanken.gu.se/om/eng/index.html actually.

\subsection{Parsing}
\label{parsers}
In order to build the language models described above, SweDiaSyn must
be POS tagged, constituency parsed and dependency parsed. This
allows the features to be extracted for use by the distance measure.

Before annotation, the SweDiaSyn sentences are cleaned in order to
improve the output of the parsers.  Cleaning the sentences consists of
removing stop words and mumbles, as well as restarts, which are marked
in the transcription.

\subsubsection{Tags `n' Trigrams}
The Tags `n' Trigrams (T`n'T) tagger \cite{brants00} is used for tagging, with
the POS annotations from Talbanken05 used as training. T`n'T is an
efficient Markov-model trigram tagger, meaning that it uses only the
previous two words to decide on the part of speech for a word. It
backs off to even less context in the case of sparse data; if a
the trigram composed of the current word and the previous two words has not been seen before, most of the decision will be based
on the current word and one previous word. T`n'T handles unknown words
by a simple form of suffix analysis---unknown words that have similar
endings to known words are more likely to get that tag.

\subsubsection{Berkeley Parser}

For constituency parsing, the Berkeley parser \cite{petrov06} is
trained on Talbanken05. The Berkeley
parser has shown good performance on languages other than English,
which is not common for constituency parsers \cite{petrov07}.
% Note: petrov 06 is a actually a lot better at explaining this.

The Berkeley parser learns a latent annotations, which means that it learns
grammars from training data by assuming that the training gives a
coarser picture than the true grammar, but one that is also too
specialized to the observed sentences. For example, it may start with
the single category NP for noun phrases, which is too coarse to
capture the subject/object distinction. So the category will be split
into NP1 and NP2. However, because the splits are cutoff to a certain
level of frequency, not every random characteristic of the training
data is learned.

\subsubsection{MaltParser}

For dependency parsing, MaltParser will be used with the existing
Swedish model trained on Talbanken05 by Hall, Nilsson and
Nivre. Dependency parsing proceeds similarly to
constituency parsing; the dependency structures of Talbanken05 are
cleaned and normalized, then used to train a parser.

MaltParser is an inductive dependency parser that uses a machine
learning algorithm to guide the parser at choice points
\cite{nivre06b}. This means that the parsing algorithm is
deterministic. Although this sounds impossible for an ambiguous
language, it achieves this by relying on a machine learner to choose
the correct option at points where multiple options exist. The machine
learner is either a memory-based learner or a support vector machine
trained on a history-based model. This model uses the internal state
of the parser while as features for training.

\section{Output Analysis}
\label{output-analysis}
After a distance measure has been defined for the interview sites
within the dialect corpus (see above, section \ref{nerbonne06}) and
syntactic features have been extracted (\ref{syntactic-features}), the
results must tested for significance (\ref{permutationtest}). The
significant results must be analyzed by clustering
(\ref{cluster-analysis}), multi-dimensional scaling (\ref{mds}) and
principal components analysis (\ref{pca}) to determine which dialect
regions are found by the distance measure. Finally, the most highly
ranked features used to produce the dialect distances must be
enumerated (\ref{feature-ranking}).

\subsection{Permutation test}
\label{permutationtest}

However, to find out if the value of $R$ is significant, we
must use a permutation test with a Monte Carlo technique described by
\namecite{good95}, following
closely the same usage by \namecite{nerbonne06}. The intuition behind
the technique is to compare the $R$ of the two corpora with the $R$ of
two random subsets of the combined corpora. If the random subsets' $R$s
are greater than the $R$ of the two actual corpora more than $p$ percent
of the time, then we can reject the null hypothesis that the two were
are actually drawn from the same corpus: that is, we can assume that
the two corpora are different.

The first hypothesis requires that the distances produced by a
distance measure be checked for significance; it is possible that
there may not be enough data for two regions to adequately distinguish
them from each other. A permutation test detects whether two corpora are
significantly different on the basis of the $R$ measure
described in section \ref{nerbonne06}. The test first calculates $R$
between samples of the two corpora. Then the corpora are mixed
together and $R$ is calculated between two samples drawn from the
mixed corpus. If the two corpora are different, $R$ should be larger
between the samples of the original corpora than $R$ from the mixed
corpus: any real differences will be randomly redistributed by the
mixing process, lowering the mixed $R$. Repeating this comparison
enough times will show if the difference is significant. Twenty times
is the minimum needed to detect significance for $p < 0.05$
significance; however, I repeat the test 100
times, enough to detect significance for $p < 0.01$.

To see how this works, for example, assume that $R$ detects real
differences between the two British regions London
and Scotland such that $R(\textrm{London},\textrm{Scotland}) =
100$. The permutation test then mixes London and Scotland to
create LSMixed and splits it into two pieces. Since the real
differences are now mixed between the two shuffled corpora, we
would expect $R(\textrm{LSMixed}_1, \textrm{LSMixed}_2) < 100$.
This should be true at least 95\% of the time for the distance $100$
to be significant.

%% I don't think normalization is important enough to mention if I
%% have to add all the sections from the H2/H3.
% \subsection{Normalization}
% Afterward, the distance must be normalized to account for two things:
% the length of sentences in the corpus and the amount of variety in the
% corpus. If sentence length differs too much between corpora, there
% will be consistently lower token counts in one corpus, which would
% cause a spuriously large $R$. In addition, if one corpus has less
% variety than the other, it will have inflated type counts, because
% more tokens will be allocated to fewer types. To avoid
% this, all tokens are scaled by the average number of types per token
% across both corpora: $2n/N$ where $n$ is the type count and $N$ is
% the token count. The factor $2$ is necessary because the scaling
% occurs based on the token counts of the two corpora combined.

% this next subsection might need to be changed or deleted
\subsection{Cluster Analysis}
\label{cluster-analysis}
The first hypothesis requires a clustering method to allow
inter-region distances to be compared more easily. The dendrogram that
binary hierarchical clustering produces allows easy visual comparison
of the most similar regions. An example is given in figure
\ref{example-dendrogram}.

\begin{figure}
  \Tree[. [. [. A B ] C ] [. D E ] ]
 \caption{Hierarchical Cluster Dendrogram}
  \label{example-dendrogram}
\end{figure}

A clustering algorithm provides understanding of which regions group
together. There are a variety of clustering algorithms, but
hierarchical clustering is the most appropriate method for this
problem because it does not specify the number of groups ahead of
time. Other clustering algorithms such as k-means or expectation
maximization require the number of expected
clusters to be given. Hierarchical clustering creates its clusters implicitly by
grouping items using a binary merge operation. The merge is repeated
until a tree with a single root is formed. The implicit clusters can
be extracted by looking at which speakers share the same subtree, as
well as looking for large differences in internal node heights connecting subtrees.

The initial step for any clustering algorithm is to find distances
between all pairs of regions as described above.
These distances between all pairs of regions result in a set of
high-dimensional spatial relationships. While they could be analyzed
as such, such high-dimensional distances are difficult to
visualize. The job of a clustering algorithm is to reduce
the dimensionality and create a useful visualization of the relative
positions of the speakers. Hierarchical clustering does this by
creating a tree---if there are similarities between the speakers, it
should be obvious by looking at the tree.

There are some complications, however. Because the clustering
algorithm is nothing but repeated merges, it is
not always clear at what level the best clusters are formed. For example,
\namecite{clopper04} used a similar clustering algorithm on perceptual
dialect data from American English speakers and found two distinct
North/South clusters for most features. These two clusters had less
defined sub-clusters as well: the Western speakers of American English
usually grouped with the North cluster but slightly separated from the
other dialects in the North cluster. Of course as the sub-clusters
become smaller, they usually become less distinctive because the
distances are smaller. Ultimately, human judgment is necessary to
determine what is a cluster and what is not.

\subsubsection{Bottom-up hierarchical clustering}

With hierarchical clustering defined, ``bottom-up'' now needs a
definition. As in any tree-building problem, the two obvious ways one
can build a tree are top-down and bottom-up. Bottom-up clustering
works in the following way. First, each speaker is put into its own
group. Then the algorithm determines the distance between each pair of
groups. The closest two groups are merged into a single group and the
process is repeated until a single root group is created. This process
is bottom-up because it creates the terminal nodes of the tree first
and builds up the internal structure of the cluster tree from there.

For example, figures \ref{hierarchical-cluster-1} through
\ref{hierarchical-cluster-5} show the sequence of merges need to
produce the dendrogram in figure \ref{example-dendrogram}. On the
first step A and B are merged (figure \ref{hierarchical-cluster-2}),
followed by D and E (figure \ref{hierarchical-cluster-3}). Then C
merges with the A-B cluster (figure
\ref{hierarchical-cluster-4}). Finally the A-B-C cluster and the D-E
cluster merge to form a single tree (figure
\ref{hierarchical-cluster-5}).

\begin{figure}
  \includegraphics[scale=0.7]{hierarchical-cluster-1}
 \caption{Sites Before Clustering}
  \label{hierarchical-cluster-1}
\end{figure}

\begin{figure}
  \includegraphics[scale=0.7]{hierarchical-cluster-2}
 \caption{Sites After A-B Merge}
  \label{hierarchical-cluster-2}
\end{figure}

\begin{figure}
  \includegraphics[scale=0.7]{hierarchical-cluster-3}
 \caption{Sites After D-E Merge}
  \label{hierarchical-cluster-3}
\end{figure}

\begin{figure}
  \includegraphics[scale=0.7]{hierarchical-cluster-4}
 \caption{Sites After A-B-C Merge}
  \label{hierarchical-cluster-4}
\end{figure}

\begin{figure}
  \includegraphics[scale=0.7]{hierarchical-cluster-5}
 \caption{Sites After Clustering}
  \label{hierarchical-cluster-5}
\end{figure}

To find the two closest groups, \quotecite{ward63} method is used. At
each merge step, this method evaluates every possible binary
merge. Each merge is given a score that minimizes some objective
function---for example, the average of distances between regions in
the new group. The best merge replaces its children and the process
repeats until a singly rooted tree is created. For $n$ regions, this
takes $n-1$ iterations, because at each step, two groups are merged.

\begin{table}
  \begin{tabular}{c|cccc}
    & B & C & D & E \\ \hline
    A & 10 & 20 & 40 & 50 \\
    B & & 20 & 50 & 40 \\
    C &&& 30 & 30\\
    D &&&& 12\\
  \end{tabular}
 \caption{Example dissimilarities}
  \label{cluster-distances}
\end{table}

For example, using the distances in table \ref{cluster-distances} for
the five example regions, the first merge is trivial since each region
starts in its own singleton tree (figure \ref{ward-cluster-1}): the
distance between A and B, 10, is the minimum and thus the best. This
produces the forest in figure \ref{ward-cluster-2}.

\begin{figure}
  \Tree[. A ]
  \Tree[. B ]
  \Tree[. C ]
  \Tree[. D ]
  \Tree[. E ]
 \caption{Ward's method, before clustering}
  \label{ward-cluster-1}
\end{figure}

\begin{figure}
  \Tree[. A B ]
  \Tree[. C ]
  \Tree[. D ]
  \Tree[. E ]
 \caption{Ward's method, after A-B merge}
  \label{ward-cluster-2}
\end{figure}

The distances between the A-B tree and the others are now more
complicated to calculate: the A-B-C merge has a distance of $10+20+20
/ 3$ = 16.6. This is smaller than the A-B-D merge ($10+40+50 / 3 =
33.3$), but larger than the D-E merge ($12 / 1 = 12$) which eventually
turns out to be the smallest merge, producing figure
\ref{ward-cluster-3}.

\begin{figure}
  \Tree[. A B ]
  \Tree[. C ]
  \Tree[. D E ]
 \caption{Ward's method, after D-E merge}
  \label{ward-cluster-3}
\end{figure}

The next merge is primarily concerned with where C will merge, whether
with A-B or D-E; an A-B-D-E merge is much larger at $10 + 40 + 50 + 50
+ 40 + 12 / 6 = 33.6$. As previously calculated, the A-B-C merge is
$16.6$, while a C-D-E merge is $30 + 30 + 12 / 3 = 24$. So the new
merge is A-B-C, producing figure \ref{ward-cluster-4}.

\begin{figure}
  \Tree[. [. A B ] C ]
  \Tree[. D E ]
 \caption{Ward's method, after A-B-C merge}
  \label{ward-cluster-4}
\end{figure}

The two remaining trees are merged. Here, the final value of the
objective function is the average of all distances in the table, that
is $302 / 10 = 30.2$. The final tree is given in figure
\ref{ward-cluster-5}.

\begin{figure}
  \Tree[. [. [. A B ] C ] [. D E ] ]
 \caption{Ward's method, after clustering}
 \label{ward-cluster-5}
\end{figure}

Ward's method is less efficient than other common clustering methods,
but it usually finds small, round clusters, making it worth the extra
computer time. In contrast, single-link distance, for example,
compares only the two closest elements of the two members of a
possible merge. This is faster, but is susceptible to creating thin,
oval groups---even though the bulk of a group may be distant, a single
outlier usually leads to a bad grouping, which recursively leads to
further outliers.

\subsubsection{Consensus Trees}

A weakness of cluster dendrograms is that small variations in
distances can cause large changes in the cluster membership of
sites. Consensus trees circumvent this weakness by combining the
results of multiple related dendrograms. Only the clusters that occur
in the majority of dendrograms appear in the consensus tree. For a
survey of consensus trees, see chapter 6 of \cite{bryant97}.

Consensus trees can be constructed by an algorithm with three primary
steps. First, the algorithm finds the spans of every internal
node in every tree. That is, each non-terminal node $N$ is replaced
with the terminal nodes $w_i \ldots w_j$ that it dominates. Second,
the algorithm counts span types and retains only the spans that occur
in a majority of dendrograms. For example, if there are $9$
dendrograms, the consensus tree will contain only spans that occur in
$5$ or more of them. Third, the spans are reconstructed into a single
tree. Nodes that no longer have a direct parent are added to a higher
ancestor.

\begin{figure}
  \Tree[. A [. [. [. B C ] D ] E ] ]
  \Tree[. A [. [. B [. C D ] ] E ] ]
  \Tree[. A [. B [. [. C D ] E ] ] ]
  \caption{Input cluster dendrograms}
  \label{consensus-example-input}
\end{figure}

\begin{figure}
  \Tree[. A [. [. B C D ] E ] ]
  \caption{Output consensus dendrogram}
  \label{consensus-example-output}
\end{figure}

For example, consider the three hierarchical dendrograms of figure
\ref{consensus-example-input}. They cluster the input set \{A B
C D E\} three different ways. The majority-rule consensus tree for
these three trees is given in figure \ref{consensus-example-output}.
This example is adapted from the one given by \namecite{amenta03}.

The spans for the internal nodes of the three trees are given in
figure \ref{consensus-tree-spans}. There is quite a bit of overlap at
higher levels, but near the leaves, \{B C\} and \{C D\} vary, as
do \{B C D\} and \{C D E\}. As a result, when the spans are
combined and counted, \{B C\} and \{C D E\} only appear once. This
means that they will be dropped from the consensus tree because they
appear in 1/3 of the trees and because 1/3 is less than or equal to 1/2,
they are not majority spans.

\begin{figure}
\begin{tabular}{l|l|l}
\hline
\{B C\}&       \{C D\}    &        \{C D\}           \\
\{B C D\}& \{B C D\}  &          \{C D E\}       \\
\{B C D E\}&  \{B C D E\} &  \{B C D E\}  \\  
\{A B C D E\}& \{A B C D E\}& \{A B C D E\}\\ 
\end{tabular}
\caption{Spans from input trees}
\label{consensus-tree-spans}
\end{figure}

\begin{figure}
  \begin{tabular}{l|ll}
  \{B C\} & 1 / 3 & * \\
  \{C D\} & 2 / 3 \\
  \{B C D\} & 2 / 3 \\
  \{C D E\} & 1 / 3 & * \\
  \{B C D E\} & 3 / 3 \\
  \{A B C D E\} & 3 / 3 \\
  \end{tabular}
  \caption{Span type frequencies (starred rows do not occur in the majority
  of trees)}
  \label{consensus-tree-span-types}
\end{figure}

\begin{figure}
  \begin{tabular}{l}
    \{C D\} \\
    \{B C D\} \\
    \{B C D E\} \\
    \{A B C D E\} \\
  \end{tabular}
  \caption{Majority span types}
  \label{consensus-tree-majority-spans}
\end{figure}

Reconstruction is fairly simple; taken from the top down, both \{A\}
and \{B C D E\} must be children of \{A B C D E\}. Because \{A\} is
not a member of the majority spans, it is added directly as a child of
\{A B C D E\}. Although this occurs in all three original trees, this
is not the case when \{B C D\} adds \{B\} as a child. The result is
that \{B C D\} has ternary branching, which was not present in any of
the original trees.

As can be seen from the example, high degrees of branching in the
consensus tree near the leaves indicate that the original trees do not
agree well. Therefore, it is not safe to draw conclusions from an
original tree in the areas where it disagrees with other original
trees.

\subsubsection{Composite Clustering}

A visual alternative to a consensus tree is composite, or fuzzy,
clustering \cite{kleiweg04}. Instead of removing clusters that do not appear
in the majority of cluster trees, fuzzy clustering plots every cluster
tree on a map completely. However, each tree is plotted
transparently. If a large number of trees agree on a cluster, the
cluster will be plotted many times, creating a dark line. Conversely,
clusters without wide agreement will only get a light line. This provides a
graphical equivalent to consensus trees.

To make this work, hierarchical clustering must change slightly: the
input is a diagonal matrix of distances as before, but the output is
no longer a binary tree but a diagonal matrix of distances, like the
input. The distances between two sites are now the number of clusters
that separate them. % (I think, I'm not sure I understood this part).

You can then draw this directly on a map: put a line equidistant
between each pair of sites, making it darker the more clusters that
separate them.

But this still relies on hierarchical clustering, which is not very
stable. To work around this, I use multiple hierarchical cluster trees
from different combinations of distance measure and feature set. I
combine the weights from the trees and scale the line darkness
accordingly to get a more stable picture of which boundaries are
important.

% TODO: An example map would be nice, but I can't get that until I
% actually do this method.

\subsection{Multi-dimensional scaling}
\label{mds}

Multi-dimensional scaling (MDS) is an alternate approach to making the
high-dimensional dissimilarities more easily interpretable. Instead
of creating a tree, multi-dimensional scaling reduces the
high-dimensional dissimilarities to 3 dimensions, which can
then be represented using (Red,Green,Blue) color triples. When
painted on a map, these colors provide a nice visualization of
the regions that similar sites form as well as how sharp the boundaries are
with other regions.

MDS of dissimilarities uses \quotecite{kruskal64a} method. It reduces
$m$ dissimilarities to an $n$ dimensional space by distorting the
individual dissimilarities by the minimum amount needed to convert
them into distances in $n$ dimensions. Since dissimilarities do not
satisfy the triangle inequality, this means finding the minimum
amount all dissimilarities need to change in order to satisfy it,
turning them into distances.

\begin{table}
  \begin{tabular}{c|cccc}
    & B & C & D & E \\ \hline
    A & 10 & 20 & 40 & 50 \\
    B & & 20 & 50 & 40 \\
    C &&& 30 & 30\\
    D &&&& 12\\
  \end{tabular}
 \caption{Example dissimilarities}
  \label{mds-distances}
\end{table}

\begin{figure}
  \includegraphics[width=0.4\textwidth]{Sverigekarta-Landskap-mds-dep}
  \label{mds-dep}
  \caption{Example of MDS on Swediasyn}
\end{figure}

Kruskal calls this measure of distortion Stress.  An initial stress is
obtained by sorting the dissimilarities by size and measuring how far
each dissimilarity would have to change in order for all to satisfy
the triangle inequality. This initial stress is reduced by a process
of gradient descent as described in \cite{kruskal64b}.

\subsubsection{Principal Components Analysis}
\label{pca}

Principal Components Analysis (PCA) is useful in conjunction with MDS
because the 3 (or so) dimensions that MDS produces can be correlated
with features. Since the features correlated with the first few dimensions are
the most important, they can be used when comparing with dialectology
results.

Each comparison between sites constitutes a new dimension because, for
example, the dissimilarity between site A and B is not comparable to
the dissimilarity between site B and C. This is because
dissimilarities are not distances---they don't satisfy the triangle
inequality. PCA allows us to reduce the number of dimensions to
something that can be visualized, as well as telling us which
dimensions contributed the most to the composite dimensions (the
principal components).

TODO: Research this and get the details of the math.

\subsection{Correlation}

Correlation is interesting on two levels. First, it is useful to find
out how well the results from various measures and feature sets
correlate. Second, it is useful to find out how well the results agree
with outside authorities. In particular, correlation with geographical
distance is strong circumstantial evidence that a distance measure is
doing its job. However, comparison to other work on the same corpus,
such as phonological dialectometry \cite{leinonen08}, is also
informative since it is possible that phonological distance matches
syntactic distance.

TODO: Also explain what kinds of things I actually
correlate:geographic distance and travel distance, which I get from
Bing maps, and corpus size.

However, the significance from Pearson's $R$, the usual measure of
correlation, is overestimated for inter-region distances because of
the way that all regions connect to all other regions. A large part of
the significance found by Pearson's $R$ would arise from the massively
overlapping information in the distances between fully connected
regions. Mantel's test provides tests Pearson's $R$ for significance
between inter-connected sites \cite{mantel67}. Mantel's test is much like
the permutation test for significance described above.  It first finds
the correlation between two sets of distances.  One distance result
set is permuted repeatedly and at each step correlated with the other
set. The original correlation is significant if the permuted
correlation is lower than the original correlation more than 95\% of
the time.


\subsection{Feature Ranking}
\label{feature-ranking}

Feature ranking is needed to compare the results of $R$ qualitatively
to the Swedish dialectology literature; $R$'s most important features
should be similar to those discussed most by dialectologists when
comparing regions. Without feature ranking of some kind, there is no
way to relate the quantitative distances between regions with the
features that contribute most to the distances.

A simple feature ranking for $R$ is easy for
one-to-one region comparisons; each feature's normalized weight is
equal to its importance in determining the distance between the two
regions. See figure \ref{feature-ranking-1-1}: features that appear
more often in one region of the compared pair are negative, while features
that appear more often in the other region are positive. In the
example, the bigram AB-AJ occurs more often in the left-hand-region,
while NN-VB and PN-VB occur more often in the right-hand-region. This
is the same as the first step of $R$ and $R^2$: $R$ then takes the
absolute value of this difference and $R^2$ takes the square.

\begin{figure}
  \includegraphics[scale=0.8]{feature-ranking-1-1}
  \label{feature-ranking-1-1}
  \caption{Feature-ranking 1:1}
\end{figure}

Features can be ranked between a single region and multiple regions by
averaging. For example, in figure \ref{feature-ranking-1-many}, the
binary comparison between the left-hand region and each of the three
right-hand regions produces three sets of features. The features can
be combined by averaging the score for each feature type. NN-VB's
averaged score would be $50 + 20 + 80 / 3 = 50$, for example.

\begin{figure}
  \includegraphics[scale=0.8]{feature-ranking-1-many}
  \label{feature-ranking-1-many}
  \caption{Feature-ranking 1:Many}
\end{figure}

This average can be extended to compare two sets of regions. As can be
seen in figure \ref{feature-ranking-many-many}, each feature on the
right-hand side is no longer a single number, but an average of the
comparison against each region on the left-hand side. Therefore, in
this example, NN-VB's overall average score is

\[ \bigg(\frac{50+30}{2} + \frac{20+30}{2} + \frac{80+30}{2}\bigg) / 3 =
\frac{50+30+20+30+80+30}{6} = 40\]

\begin{figure}
  \includegraphics[scale=0.8]{feature-ranking-many-many}
  \label{feature-ranking-many-many}
  \caption{Feature-ranking Many:Many}
\end{figure}

\subsubsection{Wiersma's Measure of Feature Overuse}

\namecite{wiersma09} uses a method similar to the one described above,
but with an additional normalization intended to show which features
are used relatively more in one or the other of the two regions to be
compared. This normalization is similar to the optional second
normalization: it also removes the effect of frequency. However, this
normalization removes the effect of frequency from the difference
of the two regions, whereas the second removes it from the distance
between two regions. Both normalizations are similar in being of
limited value for the noisier data, automatically annotated. However,
the results could improve if the data are sufficiently clean.
% But how can we know?!

The normalization centers the two counts around 1.0 by dividing each count
by the average count, scaled by the total size of the corpora. The equation
for a feature $i$ with paired counts $c_{ai}$ and $c_{bi}$ is given in
equation \ref{overuse-norm} for corpus $a$. There,
$N_a$ and $N_b$ are the sizes of the corpora $a$ and $b$, and $N$
is the combined size of the two corpora.

\begin{equation}
  o_{ai} = c_{ai} / \frac{(c_{ai} + c_{bi})N_a}{N_a + N_b}
  \label{overuse-norm}
\end{equation}

The equation can be simplified slightly; equations of this form for both corpus $a$
and $b$ are given in equation \ref{overuse-norm-simple}.

\begin{equation}
  o_{ai} = \frac{c_{ai}(N_a + N_b)}{(c_{ai} + c_{bi})N_a} \textrm{  and  }
  o_{bi} = \frac{c_{bi}(N_a + N_b)}{(c_{ai} + c_{bi})N_b}
  \label{overuse-norm-simple}
\end{equation}

For example, the previous one-to-one example shows that NN-VB occurs
20 more times in corpus $b$ than in $a$. Let this arise from the
counts $c_{a\textrm{NN-VB}} = 10$ and $c_{b\textrm{NN-VB}}=30$. Also
let the corpus sizes for $a$ and $b$ be $N_a = 40$ and $N_b = 100$
respectively. For the overuse-normalized numbers, this gives
\[o_{a\textrm{NN-VB}} = \frac{10(100+40)}{(10+30)40} = 1400 / 1600 =
0.875\] and \[o_{a\textrm{NN-VB}} = \frac{30(100+40)}{(10+30)100} = 4200
/ 4000 = 1.05\]

With the frequencies turned into ratios of overuse, it is possible to
see that the additional 20 occurrences in corpus $b$ are not that
important; they only give $o_{b\textrm{NN-VB}}$ a value of
1.05. Unlike frequencies, identical overuses occur between pairs
of the same ratio rather than the same difference. In other words,
$c_{ai}=2,c_{bi}=12$ gives the same overuse, $o_{b}=1.2$, as
$c_{aj}=10,c_{bj}=60$, because in both cases
$c_{b}=6c_{a}$. In contrast, frequency comparison gives a difference
of $c_{bi} - c_{ai} = 10$ and $c_{bj} - c_{aj} = 50$, making the
second difference much more important.

%%% Local Variables: 
%%% mode: latex
%%% TeX-master: "dissertation.tex"
%%% End: 


\chapter{Results}
\label{results-chapter}
% TODO: Many tables are ugly
% TODO: Re-order features in order of importance in all tables
% eg unigram last, preceded by deparc, timbl-dep (redep), etc

These results are meant to answer two main questions: first, how well does
this approach to syntactic dialectometry agree with dialectology?
Second, what combinations of distance measures, feature
sets and other settings produce the best results for linguistic
analyses? Additionally, the results are meant to allow comparison with
phonological dialectometry.

The organization of this chapter mirrors the order of the methods
chapter, particularly the output analysis (section
\ref{output-analysis}). First, there is an overview of the different
parameter settings, the combinations of distance measure and feature
set, as well as other settings (section
\ref{section-parameter-settings}). Then the number of significant
distances for each parameter setting is given (section
\ref{section-significant}), which is followed by the correlation with
geography and travel distance for each parameter setting (section
\ref{section-correlation}). These sections focus mainly on detecting
which settings do not produce valid results, so that they can be
ignored in the rest of the chapter. At a high level, they answer the
question of the suitability of statistical syntactic dialectometry:
whether or not significant results can be found.

Next, the specific dialectological results are examined. First,
cluster dendrograms provide a visualization of which sites the
distance measures find to be similar (section
\ref{section-clusters}). In addition, to improve the reliability of
the dendrograms, consensus trees (section \ref{section-consensus}) and
composite cluster maps are produced (section
\ref{section-composite-cluster}). Next, multi-dimensional scaling
gives a smoother view of similarity than clusters (section
\ref{section-mds}). Finally, features are ranked and extracted from
each cluster in the consensus tree (section \ref{section-features}).

\section{Parameter Settings}
\label{section-parameter-settings}

There are 180 parameter settings investigated in this chapter. This
number arises from the four parameters: measure, feature set, sampling
method and number of normalization iterations. 5 measures, 9 feature
sets, 2 sampling methods and 2 iterations of normalization
gives $5\times 9 \times 2 \times 2=180$ different settings. The
settings are given in table \ref{parameter-settings}.

\begin{table}
\begin{tabular}{|c|} \hline
  Feature Set \\\hline
  Leaf-Ancestor Path \\
  Part-of-speech Trigram \\
  Leaf-Head Path \\
  Phrase Structure Rule \\
  PSR with Grandparent \\
  Part-of-speech Unigram \\
  Leaf-Head Path, based on Timbl training \\
  Leaf-Arc Path \\
  All features combined \\ \hline
\end{tabular}
\begin{tabular}{|c|} \hline
  Measure \\ \hline
  $R$ \\
  $R^2$ \\
  Kullback-Leibler divergence \\
  Jensen-Shannon divergence \\
  cosine dissimilarity\\\hline \hline
  Sampling Method \\ \hline
  1000 sentences \\
  All sentences \\ \hline \hline
  Iterations of normalization \\ \hline
  1 \\
  5 \\ \hline
\end{tabular}
\vspace{5mm}
\caption{Settings for the five parameters tested}
\label{parameter-settings}
\end{table}
% Actually, all this should probably go in methods too, somewhere as a summary.

In addition, the size of each of the 30 interview sites are given in
table \ref{corpus-size}.

\begin{table}
\begin{tabular}{c|cc|c|cc}
      Site & Sentences & Words & Site & Sentences & Words \\\hline
     Ankarsrum &  630 &  7708 & Leksand &  923 &   10676\\
    Anundsjo &  1144 &   11897 &  Loderup &  429 &   7850\\
    Arsunda &  937 &   8933 & Norra Rorum &  546 &  9160\\
     Asby &  693 &   7171 & Orust &  1067 &   11409\\
     Bara &  696 &   10724 & Ossjo &  481 &   12275\\
     Bengtsfors &  663 &   7423 & Segerstad &  837 &   9746\\
    Boda &  1029 &   17425 &  Skinnskatteberg &  730 &   9529\\
     Bredsatra &  360 &   6938 & Sorunda &  768 &   11144\\
     Faro &  659 &   8260 & Sproge &  381 &   4399\\
     Floby &  557 &   6392 & StAnna &  876 &   13156\\
     Fole &  727 &   9920 & Tors\aa{}s &  374 &   9217\\
     Frillesas &  572 &   9634 & Torso &  956 &   15577\\
    Indal &  1126 &   13090 &  Vaxtorp &  903 &   11353\\
     Jamshog &  301 &   8661 & Viby &  431 &   6734\\
     K\"ola &  528 &   10133 & Villberga &  680 &   11479\\
\end{tabular}
  \caption{Size of Interview Sites}
  \label{corpus-size}
\end{table}

\section{Significant Distances}
\label{section-significant}

Significant distances help answer the question whether a syntactic
measure has succeeded in finding reliable distances; the measure will
always return some distance, but if the sites are too small, it may
not be significant. Therefore the results should have few
non-significant distances. In the tables, the total number of
comparisons between all 30 sites is the $435=30(30-1) / 2$. In the
first set, tables \ref{sig-1-1000} -- \ref{sig-1-full}, the results
are shown from one iteration of the normalization step. In the second
set, tables \ref{sig-5-1000} -- \ref{sig-5-full}, the results from
five normalization iterations are shown.

Bold numbers in the tables indicate that fewer than 95\% of the
distances were significant. In table \ref{sig-5-full}, the
5-iteration table that compares full sites, the only combination
with {\it less} than 5\% non-significant results is cosine
dissimilarity with unigram features, marked with italics. Note that
here, 5\% is an arbitrary cutoff point not related to the usual
significance cutoff $p < 0.05$; the basis for these tables are
themselves number of significant distances found.

\begin{table}
\begin{tabular}{l|rrrrr}
  & $R$ & $R^2$ & KL & JS & cos  \\ \hline
  Leaf-Ancestor &0&0&11&0&0 \\
  Trigram &0&0&0&0&0 \\
  Leaf-Head &0&0&0&0&0 \\
  Phrase-Structure Rules &0&0&\textbf{95}&0&0 \\
  Phrase-Structure with Grandparents &0&0&\textbf{273}&0&0 \\
  Unigram &0&0&0&0&0 \\
  Leaf-Head with MaltParser trained by Timbl &0&0&\textbf{47}&0&0 \\
  Leaf-Arc Labels&0&0&0&0&0 \\
  All Features Combined &0&0&0&0&0 \\
\end{tabular}
\caption{Number of non-significant distances for sample size 1000, 1
  normalization}
\label{sig-1-1000}
\end{table}

\begin{table}
\begin{tabular}{l|rrrrr}
& $R$ & $R^2$ & KL & JS & cos  \\ \hline
  Leaf-Ancestor&7&11&12&\textbf{35}&9 \\
  Trigram&4&1&0&\textbf{24}&1 \\
  Leaf-Head&10&12&20&\textbf{44}&19 \\
  Phrase-Structure Rules&\textbf{26}&17&\textbf{24}&\textbf{49}&20 \\
  Phrase-Structure with Grandparents&\textbf{58}&\textbf{35}&\textbf{38}&\textbf{71}&\textbf{33}
   \\
  Unigram&1&2&0&0&2 \\
  Leaf-Head with MaltParser trained by Timbl&11&21&18&\textbf{74}&\textbf{30}
   \\
  Leaf-Arc Labels&14&19&\textbf{37}&\textbf{94}&17 \\
  All Features Combined&0&0&1&8&2 \\
\end{tabular}
 \caption{Number of non-significant distances for complete sites, 1
   normalization}
 \label{sig-1-full}
\end{table}

\begin{table}
\begin{tabular}{l|rrrrr}
& $R$ & $R^2$ & KL & JS & cos  \\ \hline
  Leaf-Ancestor&5 & \textbf{56} & \textbf{34} & 0 & 0\\
  Trigram&3 & 2 & 0 & 0 & 0\\
  Leaf-Head&3 & 14 & 4 & 0 & 0\\
  Phrase-Structure Rules&11 & 4 & \textbf{66} & 1 & 0\\
  Phrase-Structure with Grandparents&18 & 0 & \textbf{109} & 4 & 0\\
  Unigram&\textbf{52} & \textbf{53} & 15 & 17 & 0\\
  Leaf-Head with MaltParser trained by Timbl&7 & 20 & \textbf{45} & 0 & 0\\
  Leaf-Arc Labels&6 & \textbf{54} & 17 & 1 & 0\\
  All Features Combined&0 & 4 & 0 & 0 & 0\\
\end{tabular}
\caption{Number of non-significant distances for sample size 1000, 5
  normalizations}
\label{sig-5-1000}
\end{table}
\begin{table}
\begin{tabular}{l|rrrrr}
& $R$ & $R^2$ & KL & JS & cos  \\ \hline
  Leaf-Ancestor&\textbf{290} & \textbf{284} & \textbf{287} & \textbf{278} & \textbf{204}\\
  Trigram&\textbf{284} & \textbf{283} & \textbf{283} & \textbf{276} & \textbf{196}\\
  Leaf-Head&\textbf{293} & \textbf{286} & \textbf{285} & \textbf{279} & \textbf{211}\\
  Phrase-Structure Rules&\textbf{289} & \textbf{294} & \textbf{286} & \textbf{275} & \textbf{236}\\
  Phrase-Structure with Grandparents&\textbf{285} & \textbf{290} & \textbf{286} & \textbf{270} & \textbf{258}\\
  Unigram&\textbf{297} & \textbf{296} & \textbf{294} & \textbf{293} &
  \textit{9}\\
  Leaf-Head with MaltParser trained by Timbl&\textbf{294} & \textbf{289} & \textbf{288} & \textbf{284} & \textbf{222}\\
  Leaf-Arc Labels&\textbf{294} & \textbf{290} & \textbf{291} & \textbf{293} & \textbf{162}\\
  All Features Combined&\textbf{279} & \textbf{279} & \textbf{279} & \textbf{269} & \textbf{191}\\
\end{tabular}
 \caption{Number of non-significant distances for complete sites, 5
   normalizations}
 \label{sig-5-full}
\end{table}

% TODO: Need to add references to diagrams and also any
% citations

Analysis of the significance of dialect distance provides a measure of
how reliable the distances to be analyzed later in this chapter are. A
distance that does not find significant distances between the 30
sites is not suitable for precise inspection, although small numbers
of non-significant distances will still allow methods to
return interpretable results.

The highest number of significant distances are found in the first
case (table \ref{sig-1-1000}): 1 round of normalization with a
fixed-size sample of 1000 sentences. From there, both full-site
comparisons (table \ref{sig-1-full}) and 5 rounds of normalization
(table \ref{sig-5-1000}) have fewer significant distances, although
the number is still usable. However, the combination of the two, with
5 rounds of normalization over full-site comparisons, has only one
combination with fewer than 5\% of distances that are {\it not}
significant. Although both full-site comparisons and multiple rounds
of normalization may increase the precision of the results, their
combined effect on significance is so detrimental that its results are
useless. For the rest of the analysis, the combination of full-site
comparison and 5 rounds of normalization will be skipped.

\subsection{Significance by Measure}

The distance measures most likely to find significance are, in order,
cosine dissimilarity, Jensen-Shannon divergence and $R$. Each method
had different parameter settings for which it was stronger. For
1000-sentence sampling, tables \ref{sig-1-1000} and \ref{sig-5-1000},
cosine similarity resulted in all significant distances, even for
part-of-speech unigrams, which are intended as the baseline feature
set. Excluding unigrams, Jensen-Shannon divergence has similar
performance. For full-site comparisons, tables \ref{sig-1-full} and
\ref{sig-5-full}, both perform considerably worse; surprisingly, both
perform better on unigram features, Jensen-Shannon so much so that it
is the only feature set for which it finds all significant
distances. $R$, on the other hand, performs decently on all
combinations of parameter settings; its low significance for phrase
structure rules is shared by Kullback-Leibler and Jensen-Shannon
divergences.
% TODO : Maybe more on cosine later. Maybe not.

When comparing the performance of Kullback-Leibler and Jensen-Shannon
divergence it is not surprising that Jensen-Shannon outperforms
Kullback-Leibler on fixed-size sampling. Although both are called
``divergence'', Jensen-Shannon divergence is actually a
dissimilarity. Recall that the divergence from point A to B may differ
from the divergence from point B to A. A divergence like
Kullback-Leibler can be converted to a dissimilarity by measuring
$KL(A,B) + KL(B,A)$. However, this dissimilarity must skip features
unique to a single site in order to avoid division by zero. This
means that for smaller sites Kullback-Leibler loses information that
Jensen-Shannon is able to use.  On the other hand, while this may
explain Kullback-Leibler's improved performance for full-site
comparisons, it doesn't explain Jensen-Shannon's much worse
performance.

\subsection{Significance by Feature Set}

% \item Unigrams do form an adequate baseline; they are bad but not too
%   bad.

% The feature sets most likely to find significance are the combined
% features and unigrams., in order,
% trigrams, all combined features and leaf-head paths (both with
% support-vector-machine training and with Timbl's instance-based
% training). Without ratio normalization, the other feature sets are not
% much worse, but with it included, these three are the best by some
% distance.

For 1 round of normalization, the best feature sets are the simple
ones: trigrams and unigrams, as well all combined features. On the
other hand, trigrams and leaf-head paths (with its variations) are the
best feature sets with 5 rounds of normalization. However, the
variation isn't strong; any feature set can give good results with the
right distance measure. The problem is that no clear patterns emerge.

The relatively high quality of trigrams and unigrams does not make
sense given only the linguistic facts; however, it is likely that the
entirely automatic annotation used here introduces more and more
errors as more annotators run, operating on previous automatic
annotations. Trigrams are the result of only one automatic annotation,
and one for which the state of the art is near human performance. So
the fact that these particular parts of speech are of higher quality
than the corresponding dependencies or constituencies is probably the
deciding factor in their higher number of significant
distances.

% Although it is impossible to tell from my results, I
% predict that a manually annotated dialect corpus would show that
% non-flat syntactic structure is helpful in producing significant
% distances.

Given the above facts, the question should rather be: why do leaf-head
paths perform as well as they do? Better, for example, than the
leaf-ancestor paths on which they're modeled: why does more
normalization hurt leaf-ancestor paths but not leaf-head paths?  It
could be that there is less room for error; many of the common
leaf-head paths are short: short interview sentences with simple
structure make for shorter leaf-head paths than leaf-ancestor
paths. As a result, the important leaf-head paths consist mainly of a
couple of parts-of-speech. This difference in feature length holds for
any length of sentence, but is exaggerated for simple sentences, where
the amount of structure generated for a phrase-structure parse for a
clause is more than for a dependency parse. In general, clauses,
embedded and otherwise, produce the largest difference in amount of
structure between the two, so the feature length differs for deeply
nested sentences as well.

Another reason could be a difference in parsers: MaltParser has been
tested on Swedish by its designers \cite{nivre06b}. Besides English, the Berkeley
parser has been tested prominently on German and Chinese. Therefore,
the difference would better be explained by appealing to the
difference in parsers rather than an unsuitability of Swedish for
constituent analysis.

It is disappointing linguistically that trigrams provide the most
reliable results so far; a linguist would expect that including
syntactic information would make it easier to measure the differences
between sites. If it is, as hypothesized here, an effect of chaining
machine annotators, a study using a manually annotated corpus could
detect this. However, it still means that trigrams are the most useful
feature set from a practical view, because automatic trigram tagging
is very close to human performance with little training. That means
the only required human work is the transcription of interviews in
most cases.

On the other hand, if additional features sets are to be developed for
a corpus, then combining all available features seems to be a
successful strategy. The distance measures seem to be able to use all
available information for finding significant distances.


\section{Correlation}
\label{section-correlation}

In dialectology, the default expectation for dialect distance is that
it correlates with geographic distance \cite{chambers98}. A lack of
correlation does not necessarily mean that a measure is invalid, but
presence of correlation means that the distance measure substantiates the
well-known tendency of dialect distributions to be more or less
smoothly gradient over physical space.

In addition, distance measures are more likely to correlate
significantly with travel distance than with straight-line geographic
distance. This makes sense since the difficulty of moving from place
to place is what influences dialect formation, and taking roads into
account is an improved estimate over straight-line distance.

The tables that present geographic and table correlation,
\ref{cor-1-1000} -- \ref{travel-cor-5-full}, mark significant
correlations with a star for $p < 0.05$, two stars for $p < 0.01$ and
three stars for $p < 0.001$. However, these correlations are only
trustworthy in the case that the underlying distances are
significant. Significant correlations from significant distances (as
cross-referenced from tables \ref{sig-1-1000} -- \ref{sig-5-full}) are
marked by italics.

Besides this, correlation between combinations of measure/feature set
can show how closely related they are--in other words, how similarly
they view the underlying data which remains the same for all. It is
analyzed in section \ref{results-chapter-inter-measure-correlation}.

This is similar to the reasoning behind correlation with
geography---but the assumption is that geography is a factor
underlying dialect formation; while the distance measure measures some
aspect of the language which we hope is dialects, it is indirectly
(even less directly) measuring the geography. Therefore, correlation
with geography should occur.

Third, correlation with corpus size is not predicted and is probably
an undesired defect in sampling or normalization. Correlation with
corpus size is presented in tables \ref{size-cor-1-1000} --
\ref{size-cor-5-full}.

\begin{table}
\begin{tabular}{l|rrrrr}
& $R$ & $R^2$ & KL & JS & cos  \\ \hline
  Leaf-Ancestor&-0.01 & 0.03 & 0.02 & -0.02 & 0.08\\
  Trigram&0.17 & 0.17 & 0.10 & 0.19 & 0.13\\
  Leaf-Head&-0.06 & 0.03 & 0.00 & -0.07 & 0.05\\
  Phrase-Structure Rules&0.01 & \textit{0.18*} & 0.16 & 0.01 & 0.12\\
  Phrase-Structure with Grandparents&0.03 & \textit{0.25*} & 0.21* & 0.03 & 0.12\\
  Unigram&\textit{0.18*} & 0.17 & \textit{0.29**} & \textit{0.30**} & \textit{0.18*}\\
  Dependencies, MaltParser trained by Timbl&-0.07 & 0.02 & -0.00 & -0.08 & 0.05\\
  Arc-Head&-0.07 & 0.06 & -0.06 & -0.09 & 0.00\\
  All Features Combined&-0.02 & 0.03 & 0.01 & -0.02 & 0.07\\
\end{tabular}
 \caption{Geographic correlation for sample size 1000, 1 normalization iteration}
 \label{cor-1-1000}
\end{table}

\begin{table}
\begin{tabular}{l|rrrrr}
& $R$ & $R^2$ & KL & JS & cos  \\ \hline
  Leaf-Ancestor&0.02 & 0.09 & 0.11 & -0.00 & 0.09\\
  Trigram&\textit{0.27*} & \textit{0.26*} & \textit{0.30**} & 0.21* & 0.08\\
  Leaf-Head&-0.03 & 0.12 & 0.14 & -0.06 & 0.02\\
  Phrase-Structure Rules&0.13 & \textit{0.36**} & 0.30** & 0.11 & \textit{0.20*}\\
  Phrase-Structure with Grandparents&0.15 & 0.41** & 0.36** & 0.14 & 0.19*\\
  Unigram&\textit{0.20*} & \textit{0.20*} & \textit{0.33**} & \textit{0.33**} & \textit{0.22*}\\
  Dependencies, MaltParser trained by Timbl&-0.02 & 0.14 & 0.16 & -0.05 & 0.02\\
  Arc-Head&-0.06 & 0.13 & -0.01 & -0.12 & -0.03\\
  All Features Combined&0.03 & 0.11 & 0.16 & -0.00 & 0.04\\
\end{tabular}
 \caption{Geographic correlation for complete sites, 1 normalization iteration}
 \label{cor-1-full}
\end{table}

\begin{table}
\begin{tabular}{l|rrrrr}
& $R$ & $R^2$ & KL & JS & cos  \\ \hline
  Leaf-Ancestor&0.14 & 0.14 & 0.16 & 0.15 & 0.08\\
  Trigram&\textit{0.22*} & 0.17 & \textit{0.22*} & \textit{0.22*} & 0.16\\
  Leaf-Head&0.10 & 0.11 & 0.15 & 0.12 & 0.10\\
  Phrase-Structure Rules&0.14 & 0.10 & 0.14 & 0.15 & 0.06\\
  Phrase-Structure with Grandparents&0.16 & 0.14 & 0.14 & 0.15 & 0.05\\
  Unigram&0.12 & 0.11 & 0.14 & 0.13 & 0.17\\
  Dependencies, MaltParser trained by Timbl&0.09 & 0.12 & 0.16 & 0.11 & 0.11\\
  Arc-Head&0.08 & 0.10 & 0.14 & 0.10 & 0.09\\
  All Features Combined&0.19 & 0.16 & \textit{0.20*} & \textit{0.21*} & 0.11\\
\end{tabular}
 \caption{Geographic correlation for sample size 1000, 5
   normalizations}
 \label{cor-5-1000}
\end{table}

\begin{table}
\begin{tabular}{l|rrrrr}
& $R$ & $R^2$ & KL & JS & cos  \\ \hline
  Leaf-Ancestor&-0.14 & -0.16 & -0.15 & -0.15 & -0.08\\
  Trigram&-0.09 & -0.07 & -0.09 & -0.09 & -0.09\\
  Leaf-Head&-0.22 & -0.21 & -0.18 & -0.22 & -0.10\\
  Phrase-Structure Rules&-0.19 & -0.14 & -0.11 & -0.20 & -0.01\\
  Phrase-Structure with Grandparents&-0.17 & -0.11 & -0.09 & -0.18 & -0.02\\
  Unigram&-0.10 & -0.06 & -0.07 & -0.08 & 0.14\\
  Dependencies, MaltParser trained by Timbl&-0.19 & -0.18 & -0.18 & -0.19 & -0.10\\
  Arc-Head&-0.21 & -0.18 & -0.18 & -0.21 & -0.10\\
  All Features Combined&-0.18 & -0.18 & -0.16 & -0.18 & -0.09\\
\end{tabular}
 \caption{Geographic correlation for complete sites, 5 normalizations}
 \label{cor-5-full}
\end{table}

\begin{table}
\begin{tabular}{l|rrrrr}
& $R$ & $R^2$ & KL & JS & cos  \\ \hline
  Leaf-Ancestor&-0.03 & 0.02 & 0.01 & -0.04 & 0.07\\
  Trigram&0.20 & 0.19 & 0.11 & \textit{0.23*} & 0.14\\
  Leaf-Head&-0.07 & 0.01 & -0.01 & -0.08 & 0.05\\
  Phrase-Structure Rules&0.01 & \textit{0.18*} & 0.17 & 0.00 & 0.14\\
  Phrase-Structure with Grandparents&0.03 & \textit{0.26*} & 0.22* & 0.03 & 0.15\\
  Unigram&\textit{0.20*} & \textit{0.19*} & \textit{0.30**} & \textit{0.31**} & \textit{0.21*}\\
  Dependencies, MaltParser trained by Timbl&-0.08 & 0.02 & -0.01 & -0.09 & 0.05\\
  Arc-Head&-0.08 & 0.05 & -0.06 & -0.10 & 0.00\\
  All Features Combined&-0.03 & 0.03 & 0.01 & -0.03 & 0.06\\
\end{tabular}
 \caption{Travel correlation for sample size 1000, 1 normalization iteration}
 \label{travel-cor-1-1000}
\end{table}

\begin{table}
\begin{tabular}{l|rrrrr}
& $R$ & $R^2$ & KL & JS & cos  \\ \hline
  Leaf-Ancestor&0.02 & 0.08 & 0.11 & 0.00 & 0.08\\
  Trigram&\textit{0.31*} & \textit{0.28*} & \textit{0.32**} & 0.26* & 0.09\\
  Leaf-Head&-0.02 & 0.12 & 0.13 & -0.05 & 0.01\\
  Phrase-Structure Rules&0.15 & \textit{0.37**} & 0.32** & 0.13 & \textit{0.22*}\\
  Phrase-Structure with Grandparents&0.17 & 0.43** & 0.38** & 0.16 & 0.22*\\
  Unigram&\textit{0.22*} & \textit{0.22*} & \textit{0.33**} & \textit{0.34**} & \textit{0.24*}\\
  Dependencies, MaltParser trained by Timbl&-0.01 & 0.14 & 0.17 & -0.04 & 0.02\\
  Arc-Head&-0.06 & 0.12 & -0.02 & -0.12 & -0.03\\
  All Features Combined&0.04 & 0.10 & 0.16 & 0.01 & 0.04\\
\end{tabular}
 \caption{Travel correlation for complete sites, 1 normalization iteration}
 \label{travel-cor-1-full}
\end{table}

\begin{table}
\begin{tabular}{l|rrrrr}
& $R$ & $R^2$ & KL & JS & cos  \\ \hline
  Leaf-Ancestor&0.17 & 0.19* & 0.17* & 0.18 & 0.07\\
  Trigram&\textit{0.24*} & \textit{0.20*} & \textit{0.25*} & \textit{0.26*} & 0.16\\
  Leaf-Head&0.14 & 0.16 & 0.17 & 0.15 & 0.10\\
  Phrase-Structure Rules&0.17 & 0.14 & 0.16* & 0.18 & 0.06\\
  Phrase-Structure with Grandparents&0.19 & \textit{0.18*} & 0.17* & 0.19 & 0.06\\
  Unigram&0.15 & 0.13 & \textit{0.17*} & 0.16 & \textit{0.20*}\\
  Dependencies, MaltParser trained by Timbl&0.12 & 0.16 & 0.18 & 0.14 & 0.11\\
  Arc-Head&0.09 & 0.13 & 0.14 & 0.11 & 0.08\\
  All Features Combined&\textit{0.23*} & \textit{0.20*} & \textit{0.22*} & \textit{0.24*} & 0.11\\
\end{tabular}
 \caption{Travel correlation for sample size 1000, 5 normalizations}
 \label{travel-cor-5-1000}
\end{table}

\begin{table}
\begin{tabular}{l|rrrrr}
& $R$ & $R^2$ & KL & JS & cos  \\ \hline
  Leaf-Ancestor&-0.13 & -0.13 & -0.10 & -0.13 & -0.04\\
  Trigram&-0.06 & -0.04 & -0.05 & -0.06 & -0.05\\
  Leaf-Head&-0.20 & -0.17 & -0.13 & -0.19 & -0.06\\
  Phrase-Structure Rules&-0.15 & -0.08 & -0.05 & -0.15 & 0.04\\
  Phrase-Structure with Grandparents&-0.12 & -0.05 & -0.03 & -0.13 & 0.03\\
  Unigram&-0.07 & -0.03 & -0.04 & -0.05 & \textit{0.18*}\\
  Dependencies, MaltParser trained by Timbl&-0.18 & -0.15 & -0.12 & -0.18 & -0.05\\
  Arc-Head&-0.20 & -0.17 & -0.14 & -0.19 & -0.06\\
  All Features Combined&-0.16 & -0.14 & -0.11 & -0.15 & -0.05\\
\end{tabular}
 \caption{Travel correlation for complete sites, 5 normalizations}
 \label{travel-cor-5-full}
\end{table}

From tables \ref{cor-1-1000} -- \ref{travel-cor-5-full} we see that parameter settings that correlate
significantly do so at rates around 0.2 to 0.3, with a high of 0.37
for phrase-structure-rule features measured by $R^2$, 1 normalization
iteration and comparison of full sites.  The significant
correlations are mostly concentrated in the trigram, unigram and
combined feature sets.

\subsection{Analysis}

As with the number of significant distances, trigrams and unigrams are
the most likely to correlate with geographic and travel distance,
as well as the combined feature set for the 5-normalization parameter
setting.
% As before, a possible explanation is that unigrams are
% simpler, so the type count is a higher than for other measures. With
% more rounds of normalization, more correlations shift over to
% trigrams.
Note that in tables \ref{cor-1-1000} --
\ref{travel-cor-5-full}, the significant correlations are marked with
an asterisk, but only the italicized correlations are based on at
least 95\% significant distances. For example, this means that most of
the significant correlations based on phrase-structure rules are not valid.

It is worthwhile to note, however, that the valid and significant
correlations based on phrase-structure grammars give the highest
correlations: 0.37 for $R^2$ with full-site comparisons and 1 round
of normalization.
The addition of more data and more normalization is interesting in
expanding the correlating parameter settings beyond those that include
unigram features. It may be that this is an instance of the noise/quality tradeoff.
These additions appear to extract more detail from
the data, at the cost of additional interference from noisy data.

% Goes here: Fevered speculation about why travel correlation is *better* with
% the methods that correlate *less*, for 1-full at least.
% OK never mind this isn't true.

\subsection{Inter-measure Correlation}
\label{results-chapter-inter-measure-correlation}

Correlation between measures shows that they produce similar
results. It also suggests that they use similar information to do
so. For example, cosine similarity correlates the least with the
others, which means that its results are the least like the
others. It also implies that cosine similarity uses information from
input features differently than the other measures. Since the
performance of the summed, non-cosine measures is a little better for
this site size, practical use of this distance method should probably
start with them. In other computational linguistic applications,
cosine distance is typically used with larger corpora, so it is
possible that it provides better results with larger corpora, such as
corpora based on entire provinces of Sweden rather than the individual
villages used in this dissertation.

The average correlation between different measures is given in table
\ref{self-correlation-measures}. The correlations are averaged over
the correlations for all combinations of feature set with
1000-sentence samples and with non-significant correlations removed
before averaging.

\begin{table}
  \begin{tabular}{r|cccc}
    & $R^2$ & $KL$ & $JS$ & cos \\ \hline
    $R$ & 0.85 & 0.85 & 0.98 & 0.39\\
    $R^2$&& 0.90 & 0.83 & 0.57\\
    $KL$ &&& 0.88 & 0.67\\
    $JS$ &&&& 0.44
  \end{tabular}
  \caption{Average Inter-measure-correlation of measures}
  \label{self-correlation-measures}
\end{table}

The inter-measure correlation is essentially a summary of the
results from the significance testing and correlations. $R$ and
Jensen-Shannon produce nearly identical results, and also correlate
highly. Cosine similarity is quite different from the other measures,
though the correlation is still higher than with travel distance. This
is expected insofar as the cosine operation at the heart of cosine similarity
differs more from the sums of absolute values or logarithms of other
measures.

\subsection{Correlation with Corpus Size}

As previously stated, correlation with corpus size is not predicted and is probably
an undesired defect in sampling or normalization. Correlation with
corpus size is presented in tables \ref{size-cor-1-1000} --
\ref{size-cor-5-full}.

Corpus size between two sites can be measured in two different ways:
either by the sum of the sites' sizes, or by the difference. Here
the sum is used: a larger sum means more tokens. If there is a
correlation with size, it must arise because higher token counts are
not properly normalized. In other words, two large sites will
have more tokens, leading to higher type counts, which directly leads
to higher distances. Smaller sites will lead to lower distances.

\begin{table}
\begin{tabular}{l|rrrrr}
& $R$ & $R^2$ & KL & JS & cos  \\ \hline
  Leaf-Ancestor&-0.38 & -0.26 & -0.37 & -0.40 & -0.37\\
  Trigram&0.12 & -0.12 & -0.16 & 0.14 & -0.18\\
  Leaf-Head&-0.39 & -0.26 & -0.35 & -0.43 & -0.39\\
  Phrase-Structure Rules&0.06 & 0.15 & 0.00 & 0.03 & -0.10\\
  Phrase-Structure with Grandparents&0.08 & 0.19 & 0.07 & 0.04 & -0.09\\
  Unigram&-0.08 & -0.14 & -0.09 & -0.09 & -0.10\\
  Dependencies, MaltParser trained by Timbl&-0.35 & -0.23 & -0.28 & -0.37 & -0.37\\
  Arc-Head&-0.44 & -0.26 & -0.40 & -0.48 & -0.34\\
  All Features Combined&-0.37 & -0.26 & -0.38 & -0.42 & -0.40\\
\end{tabular}
\caption{Size correlation for sample size 1000, 1 normalization}
\label{size-cor-1-1000}
\end{table}

\begin{table}
\begin{tabular}{l|rrrrr}
& $R$ & $R^2$ & KL & JS & cos  \\ \hline
  Leaf-Ancestor&-0.19 & -0.15 & -0.16 & -0.24 & -0.36\\
  Trigram&\textit{0.30*} & 0.08 & 0.19 & 0.08 & -0.39\\
  Leaf-Head&-0.17 & -0.06 & -0.08 & -0.26 & -0.41\\
  Phrase-Structure Rules&0.52** & \textit{0.40**} & 0.30* & 0.47** & -0.21\\
  Phrase-Structure with Grandparents&0.54** & 0.43** & 0.37** & 0.50** & -0.22\\
  Unigram&-0.09 & -0.13 & -0.11 & -0.13 & -0.13\\
  Dependencies, MaltParser trained by Timbl&-0.08 & 0.02 & 0.09 & -0.14 & -0.39\\
  Arc-Head&-0.32 & -0.16 & -0.26 & -0.40 & -0.35\\
  All Features Combined&-0.15 & -0.11 & -0.10 & -0.25 & -0.42\\
\end{tabular}
\caption{Size correlation for complete sites, 1 normalization}
\label{size-cor-1-full}
\end{table}

\begin{table}
\begin{tabular}{l|rrrrr}
& $R$ & $R^2$ & KL & JS & cos  \\ \hline
  Leaf-Ancestor&\textit{0.35*} & 0.36** & 0.06 & 0.27 & -0.32\\
  Trigram&\textit{0.75**} & \textit{0.63**} & \textit{0.46**} & \textit{0.68**} & -0.24\\
  Leaf-Head&\textit{0.46**} & \textit{0.44**} & 0.14 & \textit{0.38**} & -0.33\\
  Phrase-Structure Rules&\textit{0.85**} & \textit{0.59**} & 0.36** & \textit{0.85**} & -0.34\\
  Phrase-Structure with Grandparents&\textit{0.88**} & \textit{0.66**} & 0.40** & \textit{0.88**} & -0.36\\
  Unigram&0.38** & 0.35** & 0.14 & 0.19 & -0.04\\
  Dependencies, MaltParser trained by Timbl&\textit{0.44**} & \textit{0.41**} & 0.16 & \textit{0.39*} & -0.30\\
  Arc-Head&0.20 & 0.28* & -0.00 & 0.09 & -0.28\\
  All Features Combined&\textit{0.58**} & \textit{0.48**} & 0.21 & \textit{0.47**} & -0.31\\
\end{tabular}
 \caption{Size correlation for sample size 1000, 5 normalizations}
 \label{size-cor-5-1000}
\end{table}

\begin{table}
\begin{tabular}{l|rrrrr}
& $R$ & $R^2$ & KL & JS & cos  \\ \hline
  Leaf-Ancestor&-0.55 & -0.38 & -0.26 & -0.53 & -0.17\\
  Trigram&-0.29 & -0.27 & -0.19 & -0.26 & -0.14\\
  Leaf-Head&-0.61 & -0.43 & -0.27 & -0.58 & -0.18\\
  Phrase-Structure Rules&-0.21 & -0.08 & -0.04 & -0.22 & -0.14\\
  Phrase-Structure with Grandparents&-0.24 & -0.08 & -0.03 & -0.26 & -0.14\\
  Unigram&-0.38 & -0.25 & -0.30 & -0.32 & -0.08\\
  Dependencies, MaltParser trained by Timbl&-0.52 & -0.33 & -0.20 & -0.51 & -0.15\\
  Arc-Head&-0.59 & -0.45 & -0.33 & -0.54 & -0.20\\
  All Features Combined&-0.61 & -0.44 & -0.26 & -0.55 & -0.18\\
\end{tabular}
\caption{Size correlation for complete sites, 5 normalizations}
\label{size-cor-5-full}
\end{table}

In tables \ref{size-cor-1-1000} and \ref{size-cor-1-full}, the
1-normalized correlations, only two correlations are
significant. However, in table \ref{size-cor-5-1000}, 5-normalized
correlations with 1000-sampling, a large number of correlations are
significant. Specifically, the highest performing measures, $R$, $R^2$,
and Jensen-Shannon divergence, correlate significantly with size for
nearly all feature sets. Since this is not a predicted correlation, it
means that these distances may be invalid. However, another piece of
evidence makes this conclusion uncertain: geographics distance also
correlates with corpus size at a rate of 0.31, $p < 0.01$, and travel
distance correlates at 0.32, $p < 0.01$. This correlation is also
unexpected, since there is no reason to expect that distance predicts
corpus size or vice versa. However, it shows that the size correlation of
dialect distance may at least by partly explained here by the
unexpected correlation with geographic and travel distance. Therefore,
5-normalised results will be presented throughout the rest of the
results.

\subsubsection{Analysis}

The correlation of corpus size and dialect distance is a problem. It
is not a predicted as a side effect of the way dialect distance is
measured. The fact that travel distance also correlates with corpus
size at a rate of 0.32 confuses the issue further. Is corpus size the
determining variable? Or is there an unknown variable influencing all
three? One possibility is ``interviewer boundaries'', common in
corpora collected by multiple people \cite{nerbonne03}.  Perhaps a
single interviewer improved with practice and collected longer interviews as
the interview collection progressed. Or perhaps cultural differences
between the interviewer and interviewees caused some participants in
one area to talk more than in another area.

Although the size correlation of the dialect distances may be
explained by the correlation with geographic/travel distance, they are still
somewhat worrying. There is a great enough difference above the
correlation of corpus size and geographic/travel distance that 5-normalized
distances might not be reliable.
However, if 5-normalization introduces a dependency on corpus size,
then the distances from full-corpus comparisons should correlate even
more highly. This is not the case.
% It appears that multiple rounds of
% normalization inadvertently re-introduce a dependency on size.
% TODO: This probably IS a bug in that only Freq norm can be
% iterated. Ratio norm should probably be in a separate loop like so:
% #ifdef RATIO_NORM
%   for(sample::iterator i = ab.begin(); i!=ab.end(); i++) {
%     i->second.first *= 2 * types / tokens;
%     i->second.second *= 2 * types / tokens;
%   }
% #endif

% TODO: I also should write this up when I have time
Alternatively, it is possible that the fixed-size sampling method is not
properly eliminating size differences between interview sites. Future work
should develop a method for normalizing a comparison between two full
sites. It should avoid sampling, but also take the relative number
of sentences into account.

\section{Clusters}
\label{section-clusters}

Cluster dendrograms provide a visualization of which sites the
distance measures find most similar. They are formed in a bottom-up
manner, repeatedly merging the two most similar groups at each step
until only one group remains. The resulting dendrogram usually has
obvious sub-trees which can be treated as clusters. By grouping sites
into clusters, cluster dendrograms allow closer comparison to
dialectology than correlation. These clusters can be compared to the
regions proposed by syntactic dialectology.

The first two dendrograms in this section hold feature set, measure,
and sample size constant at trigrams, Jensen-Shannon, and
1000-sentence samples, respectively. Then they vary the amount of
normalization: figure \ref{cluster-1-js-trigram} has 1 normalization
round, while figure \ref{cluster-5-js-trigram} has 5. These two examples
were chosen because of their high numbers of significant
distances and correlation with travel distance; the highest
correlation of 5-normalized distances with travel distance, 0.26, is
with the Jensen-Shannon measure and trigram features in figure
\ref{cluster-5-js-trigram}.

The third figure, figure \ref{cluster-1-r_sq-psg}, gives the dendrogram
for the parameter settings with the highest travel distance
correlation, phrase-structure rules, 1 normalization, 1000-sentence
samples, and $R^2$ measure. The highest correlation of 1-normalized
distances with travel distance, 0.37, is given by $R^2$ measured over
phrase-structure-rule features, comparing full sites. Those parameter
settings produce the dendrogram in figure
\ref{cluster-1-r_sq-psg}.

% Within the same settings for sampling and number of normalization
% iterations, the clusters based on sentence-length normalization alone are fairly
% similar, regardless of measure and feature set. Changing the sampling
% settings or the number of normalizations substantial reconfiguration.

% For example, the clusters produced by $R$ (figure
% \ref{cluster-1-r-trigram}) and Jensen-Shannon divergence are fairly
% similar (figure \ref{cluster-1-js-trigram}). Both are based on trigram
% features with sentence-length normalization only. Those dendrograms
% differ from their 5-normalized equivalents, figures
% \ref{cluster-5-r-trigram} and \ref{cluster-5-js-trigram}.

% \begin{figure}
%   \includegraphics[width=0.9\textwidth]{dist-1-1000-r-trigram-ratio-clusterward}
%  \caption{Dendrogram With $R$
%     measure and trigram features, 1 normalization, 1000 samples}
%   \label{cluster-1-r-trigram}
% \end{figure}

% TODO: Remove R.app's captions in favour of mine.
% TODO: Remove R.app's x-scale (y-scale) too

\begin{figure}
  \includegraphics[width=0.9\textwidth]{dist-1-1000-js-trigram-ratio-clusterward}
 \caption{Dendrogram With Jensen-Shannon
    measure and trigram features, 1 normalization, 1000 samples}
  \label{cluster-1-js-trigram}
\end{figure}

\begin{figure}
  \includegraphics[width=0.9\textwidth]{dist-5-1000-js-trigram-ratio-clusterward}
 \caption{Dendrogram With Jensen-Shannon
    measure and trigram features, 5 normalizations, 1000 samples}
  \label{cluster-5-js-trigram}
\end{figure}


\begin{figure}
  \includegraphics[width=0.9\textwidth]{dist-1-full-r_sq-psg-ratio-clusterward}
 \caption{Dendrogram With $R^2$ measure and phrase-structure-rule features,
 1 normalization, complete sites}
  \label{cluster-1-r_sq-psg}
\end{figure}


% \begin{figure}
%   \includegraphics[width=0.9\textwidth]{dist-5-1000-r-trigram-ratio-clusterward}
%  \caption{Dendrogram With $R$ measure and trigram features, 5 normalizations, 1000 samples}
%   \label{cluster-5-r-trigram}
% \end{figure}


Unlike the significances, cosine similarity's dendrograms are fairly
similar to those of other features. See for example figure
\ref{cluster-5-cos-trigram}, with cosine, trigram features and
5 iterations of normalization.

However, it is difficult to judge the amount of agreement between
these individual dendrograms. These figures are mostly given as
examples rather than for in-depth comparison. Instead of manually
comparing each to the dialect regions of Sweden, a better option is to
aggregrate them automatically into a single dendrogram, retaining only
the clusters that agree. This is a consensus tree.

\begin{figure}
 \includegraphics[width=0.9\textwidth]{dist-5-1000-cos-trigram-ratio-clusterward}
 \caption{Dendrogram with cosine measure and trigram features, 5
   normalizations}
  \label{cluster-5-cos-trigram}
\end{figure}


\subsection{Consensus Trees}
\label{section-consensus}

Consensus trees combine the results of cluster dendrograms, retaining
only clusters that occur in the majority of dendrograms. When
dendrograms have high agreement, the resulting consensus tree will
retain most of the detail. When dendrograms have low agreement, the
resulting consensus tree will be fairly flat.  This avoids the
dendrograms' problem of instability, where small changes in distances
cause large re-arrangements in the tree. Only dendrograms whose input
distances were at least 95\% significant were used. That is, a
measure/feature set combination had to be non-bold in tables
\ref{sig-1-1000} to \ref{sig-5-full} to be included. The consensus
tree for full-site comparisons and 5 rounds of normalization is not
given because there is only one dendrogram that qualifies.

It's worthwhile to note that more dendrograms were used to build
the consensus tree of figure \ref{consensus-5-1000} than were used in
figures \ref{consensus-1-1000} and \ref{consensus-1-full}. Despite this, figure
\ref{consensus-5-1000} retains much more detail, indicating that its
constituent dendrograms, based on 5 rounds of normalization,
agree more than those with only 1 round of normalization.

The consensus trees are also grouped into clusters, which are then
mapped in figures \ref{map-consensus-1-1000} --
\ref{map-consensus-5-1000}. The outline map of Sweden were provided by
Therese Leinonen and are the same as those in \namecite{leinonen08}. The
L04 package from the University of Groningen was used to map the
consensus trees onto the map of Sweden; the multi-dimensional scaling
maps and composite cluster maps also used L04.
% TODO: CITE this, I think it's a Pieter Kliuweeg paper


\begin{figure}
\includegraphics[scale=0.7]{consensus-1-1000}
% \Tree[. {Villberga\\Viby\\Vaxtorp\\Torso\\Tors\aa{}s\\StAnna\\Sproge\\Sorunda\\Skinnskatteberg\\Segerstad\\Ossjo\\Orust\\Norra Rorum\\Loderup\\Leksand\\K\"ola\\Jamshog\\Indal\\Frillesas\\Fole\\Faro\\Bredsatra\\Boda\\Bara\\Asby\\Arsunda\\Anundsjo\\Ankarsrum} [. {Floby\\Bengtsfors}  ] ]
\caption{Consensus Tree for 1000-samples and 1 normalization}
\label{consensus-1-1000}
\end{figure}

\begin{figure}
\includegraphics[scale=0.7]{consensus-1-full}
% \Tree[. {Villberga\\Viby\\Torso\\Tors\aa{}s\\Sorunda\\Segerstad\\Ossjo\\Orust\\Norra Rorum\\Loderup\\Leksand\\K\"ola\\Indal\\Fole\\Boda\\Bara\\Asby\\Arsunda\\Anundsjo\\Ankarsrum}
%   [. {Vaxtorp\\Skinnskatteberg}  ]
%   [. {StAnna\\Frillesas}  ]
%   [. {Sproge\\Faro}  ]
%   [. {Jamshog\\Bredsatra}  ]
%   [. {Floby\\Bengtsfors}  ] ]
\caption{Consensus Tree for full site comparison and 1 normalization}
\label{consensus-1-full}
\end{figure}

\begin{figure}
\includegraphics[scale=0.7]{consensus-5-1000}
% \Tree[. {Villberga\\Viby\\Vaxtorp\\Torso\\StAnna\\Sproge\\Sorunda\\Skinnskatteberg\\Segerstad\\Orust\\Norra Rorum\\Leksand\\K\"ola\\Indal\\Frillesas\\Fole\\Floby\\Faro\\Boda\\Bengtsfors\\Bara\\Asby\\Arsunda\\Anundsjo\\Ankarsrum} [. {Loderup\\Bredsatra}  ] [. {Tors\aa{}s\\Ossjo\\Jamshog}  ] ]
\caption{Consensus Tree for 1000-samples and 5 normalizations}
\label{consensus-5-1000}
\end{figure}

\begin{figure}
\includegraphics[scale=0.85]{Sverigekarta-Landskap-consensus-1-1000}
\caption{Consensus Tree for 1000-samples and 1 normalization, Mapped}
\label{map-consensus-1-1000}
\end{figure}

\begin{figure}
\includegraphics[scale=0.85]{Sverigekarta-Landskap-consensus-1-full}
\caption{Consensus Tree for full site comparison and 1 normalization, Mapped}
\label{map-consensus-1-full}
\end{figure}

\begin{figure}
\includegraphics[scale=0.85]{Sverigekarta-Landskap-consensus-5-1000}
\caption{Consensus Tree for 1000-samples and 5 normalizations, Mapped}
\label{map-consensus-5-1000}
\end{figure}

% It would still be cool to eliminate only the non-significant distances
% and re-run the clusters. (I can't remember if that's easily possible
% with R though, it may only be a feature of MDS.)

% TODO: Try these two again, excluding cosine. Because I believe cosine sucks
% or at least is a Rogue Element.
% Later: I think I tried this and not much difference.

% Just the ratio ones that are significantly correlated with travel
% distance.
% However: This is even more of a mess than the freq results.
% [. {s0} [. {}
%     [. {} [. {K�la} [. {Ossjo} [. {Tors�s\\Jamshog}  ] ] ]
%           [. {Villberga\\Viby\\Torso\\StAnna\\Sorunda\\Norra Rorum\\Frillesas\\Boda\\Bara}
%              [. {Loderup\\Bredsatra}  ] ] ]
%     [. {Orust\\Leksand\\Indal\\Fole\\Faro\\Asby\\Arsunda\\Anundsjo}
%        [. {Vaxtorp\\Skinnskatteberg}  ]
%        [. {Ankarsrum}
%           [. {Segerstad} 
%              [. {Bengtsfors}
%                 [. {Sproge\\Floby}  ] ] ] ] ] ] ]

\subsubsection{Analysis}

The cluster dendrograms are dangerous to interpret too closely on
their own; the instability of a single dendrogram means that small
clusters cannot be analyzed reliably. For example, in figure
\ref{cluster-5-js-trigram}, a two-way split between the sites on the
top and bottom of the page is obvious, and another in the top cluster
is easy to argue for, but outliers like Anundsj\"o and \AA{}rsunda are
likely to shift from group to group in other dendrograms.

It is safer to analyze the consensus trees; the smoothing effect of
taking the majority rule of each cluster will show where the optimal
cutoff for splitting clusters is removing spurious detail. The three
consensus trees in figures \ref{consensus-1-1000} --
\ref{consensus-5-1000} vary in amount of detail but the trees with
more clusters do not contradict the clusters of the flatter trees.

For 1000-sentence samples and 1 round of normalization, there is one
cluster: Floby and Bengtsfors. Full-site comparison finds
another cluster: J\"amshog, \"Ossj\"o and Tors\aa{}s. Finally,
1000-sentence samples and 5 rounds of normalization finds another
cluster consisting of L\"oderup and Breds\"atra. It also finds
a large two-way split between the sites and adds Sproge to the first
cluster with Floby and Bengtsfors. To aid further analysis, the
clusters are assigned colors, which are detailed in figures
\ref{blue-cluster} -- \ref{orange-cluster}. 

\begin{figure}
\begin{itemize}
\item Floby
\item Bengtsfors
\item Sproge (for 1000-sample, 5-normalization)
\end{itemize}
\caption{Blue Cluster}
\label{blue-cluster}
\end{figure}

\begin{figure}
\begin{itemize}
\item J\"amsh\"og
\item Tors\aa{}s
\item \"Ossj\"o
\end{itemize}
\caption{Red Cluster}
\label{red-cluster}
\end{figure}

\begin{figure}
\begin{itemize}
\item Breds\"atra
\item L\"oderup
\end{itemize}
\caption{Yellow Cluster}
\label{yellow-cluster}
\end{figure}

\begin{figure}
\begin{itemize}
\item Leksand
\item Indal
\item Segerstad
\item Floby
\item Bengtsfors
\item Sproge
\item Skinnskatteberg
\item Orust
\item V\aa{}xtorp
\item F\aa{}r\"o
\item Asby
\item \AA{}rsunda
\item Anundsj\"o
\item Ankarsrum
\item Fole
\end{itemize}
\caption{Cyan Cluster}
\label{cyan-cluster}
\end{figure}

\begin{figure}
\begin{itemize}
\item Viby
\item Bara
\item S:t Anna
\item Frilles\aa{}s
\item J\"amshog
\item Tors\aa{}s
\item \"Ossj\"o
\item K\"ola
\item L\"oderup
\item Breds\"atra
\item Villberga
\item Tors\"o
\item Norra R\"orum
\item Sorunda
\item B\"oda
\end{itemize}
\caption{Orange Cluster}
\label{orange-cluster}
\end{figure}

When these clusters are mapped onto the geography of Sweden, some
patterns are visible. Since figure \ref{consensus-5-1000} is strictly
more complex than the preceding two, it is used as the basis for this
analysis--see figure \ref{map-consensus-5-1000}. The large two-way split
is between the orange and cyan clusters. The orange cluster, which
includes red and yellow clusters, forms two horizontal bands across
Sweden. The centers of the orange cluster appear to be Stockholm and
Malm\"o. Meanwhile, the red and yellow clusters form a boundary along
the northern border of Sk\aa{}ne and Blekinge counties.

Meanwhile, the cyan cluster, which includes the blue cluster, seems to
represent the countryside of Sweden. On the other hand, because the
blue cluster is near G\"oteborg, it might be better characterized
simply as ``non-Stockholm''. This matches the traditional dialect
regions of Sweden, with the exception of of the city/country divide,
and the fact that this hard clustering simplifies the dialect
boundaries, which are traditionally believed to be gradient. Also, the
island Gotland is not put in a separate cluster as predicted by
traditional boundaries. For discussion, see section
\ref{discussion-chapter-dialectology-section}.


\subsection{Composite Cluster Maps}
\label{section-composite-cluster}

Composite cluster maps use an underlying technique similar to
consensus trees--cluster dendrograms, but they combine and present the
information in a very different way. They, too, provide a stabler view
of the groups that sites form when clustered. This view, however,
emphasizes the boundaries between sites. The result looks
much more like the traditional isogloss boundaries of
dialectology.

The two composite cluster maps in figures \ref{map-composite-1-1000}
-- \ref{map-composite-5-1000} are the composite of the same
dendrograms used as input for the consensus trees: all-significant
parameter settings, divided by type of normalization (sentence-length
only or ratio added as well).

\begin{figure}
\includegraphics[scale=0.82]{Sverigekarta-cluster-1-1000}
\caption{Composite Cluster Map for 1000-sample, 1 normalization}
\label{map-composite-1-1000}
\end{figure}

\begin{figure}
\includegraphics[scale=0.82]{Sverigekarta-cluster-1-full}
\caption{Composite Cluster Map for complete sites, 1 normalization}
\label{map-composite-1-full}
\end{figure}

\begin{figure}
\includegraphics[scale=0.82]{Sverigekarta-cluster-5-1000}
\caption{Composite Cluster Map for complete sites, 5 normalizations}
\label{map-composite-5-1000}
\end{figure}

All three composite clusters maps provide a picture similar to the
consensus tree map \ref{map-consensus-5-1000} of the previous
section. The north-to-south gradient is supported by the
weak horizontal boundaries present up and down Sweden.

Of these boundaries, the one between Sk\aa{}ne and the rest of Sweden is
the strongest. Due to the lack of interview sites in the middle of
south Sweden, the boundary is drawn further north than it
traditionally appears, but this is an effect of the software that
produced the figure. Notice that there is also a boundary between the
red cluster, comprised of J\"amshog, Tors\aa{}s, and \"Ossj\"o, and the
other sites, especially visible in figures \ref{map-composite-1-1000} and
\ref{map-composite-5-1000}. Their presence along the northern border
of Sk\aa{}ne is one reason why its boundary with the rest of Sweden is
so strong.

\begin{sloppypar}
  Compared to the consensus tree maps, the composite cluster maps
  cannot support the city/country distinction because there is no way
  to identify distant areas by their color. On the other hand, it is
  possible to detect the relative strength of a boundary. To combine
  these two features, multi-dimensional scaling is needed.
\end{sloppypar}
% But of course MDS maps can't be combined into a consensus...

% However, K\"ola and Frilles\aa{}s still separate fairly well from their
% neighbors. These sites are on the edges of the country and have strong borders
% with surrounding, Like the cluster J\"amshog, Tors\aa{}s and \"Ossj\"o,
% these sites are different from the others. However, they don't have
% any geographic coherence, so it is more likely these are remnants of a
% dialect that was historically wider spread and has since receded.


\section{Multi-Dimensional Scaling}
\label{section-mds}

\begin{sloppypar}
  Multi-dimensional scaling (MDS) plays a similar role to clusters,
  condensing the high-dimensional information into an form that is
  easier to understand. It differs, however, in producing gradient
  numbers, not binary trees: cluster dendrograms put each site into
  one and only one cluster, whereas MDS puts each site into 3D space;
  the clusters are only implicit in the positions.  This also means
  that MDS maps are more stable than dendrograms.
\end{sloppypar}

\quotecite{kruskal64b} MDS works by positioning the dissimilarities in
high-dimensional space, then converting them to true distances in some
lower dimensional space---in this case, three dimensions. It distorts
the dissimilarities equally and by the minimum amount
necessary. Kruskal calls the measure of distortion Stress. Once the
sites are points in 3D space, each sites' x, y, and z co-ordinates can
be mapped onto the colors red, green, and blue, then drawn on a map of
Sweden.

It must be noted that the maps vary in color because of the way that
MDS positions the sites in 3D space, based on the distances between
them. Kruskal's method guarantees that its results are comparable for
equivalent inputs, but this may not always be obvious because the
color equivalence may be difficult to decipher. Equivalent MDS maps
may be rotated with respect to each other in 3D space, and this
rotation is visible in the color selection: if two sites are both blue
in one map and in another map are both orange, then they have the same
relation to each other.


The maps shown here in figures \ref{mds-1-1000-js-trigram} --
\ref{mds-1-full-r_sq-psg} are based on the same parameter settings as the
dendrograms in figures \ref{cluster-1-js-trigram} --
\ref{cluster-1-r_sq-psg}. The first two are Jensen-Shannon divergence
measured over trigrams, with 1 and 5 rounds of normalization,
respectively. The third is $R^2$ measured over phrase-structure rules
with 1 round of normalization.
% 1st is not used
% 3rd is not used
% 5th is not used

% \begin{figure}
% \includegraphics[scale=0.82]{Sverigekarta-mds-1-1000-r-trigram-ratio}
% \caption{$R$ measure with trigram features, 1000-sentence sampling and
%   1 round of normalization}
% \label{mds-1-1000-r-trigram}
% \end{figure}

\begin{figure}
\includegraphics[scale=0.82]{Sverigekarta-mds-1-1000-js-trigram-ratio}
\caption{Jensen-Shannon measure with trigram features, 1000-sentence sampling and
  1 round of normalization}
\label{mds-1-1000-js-trigram}
\end{figure}

% \begin{figure}
% \includegraphics[scale=0.82]{Sverigekarta-mds-1-full-r-trigram-ratio}
% \caption{$R$ measure with trigram features, full-site comparison and
%   1 round of normalization}
% \label{mds-1-full-r-trigram}
% \end{figure}

% \begin{figure}
% \includegraphics[scale=0.82]{Sverigekarta-mds-5-1000-r-trigram-ratio}
% \caption{$R$ measure with trigram features, 1000-sentence sampling and
%   5 rounds of normalization}
% \label{mds-5-1000-r-trigram}
% \end{figure}

\begin{figure}
\includegraphics[scale=0.82]{Sverigekarta-mds-5-1000-js-trigram-ratio}
\caption{Jensen-Shannon measure with trigram features, 1000-sentence sampling and
  5 rounds of normalization}
\label{mds-5-1000-js-trigram}
\end{figure}

\begin{figure}
\includegraphics[scale=0.82]{Sverigekarta-mds-1-full-r_sq-psg-ratio}
\caption{$R^2$ measure with phrase-structure-rule features, full-site comparison and
  1 round of normalization}
\label{mds-1-full-r_sq-psg}
\end{figure}

Despite the differences between MDS and the preceding methods, the
similar results are evident; the maps (figures
\ref{mds-1-1000-js-trigram} -- \ref{mds-1-full-r_sq-psg}) all show
the same patterns as the other methods. That is, there is a general
north-to-south gradience, especially easy to see in map
\ref{mds-1-1000-js-trigram}. There is a strong southern cluster,
visible in all of the diagrams. And there is a general two-way
distinction between city and country.

The main contribution that the MDS maps make is that the
north-to-south gradient is more obviously gradient. In other words, it
is easier to see the gradation from north to south. For example, in
figure \ref{mds-1-full-r_sq-psg}, looking from the north to south, the
colors change quickly close to Stockholm, then fade to green further
south, then transition back to blues and purples further south, in
Sk\aa{}ne.

The Stockholm and Malm\"o areas, which are in the same cluster in the
consensus tree maps, are here seen to be similar without being
identical. For example, in figure \ref{mds-5-1000-js-trigram}, the
Stockholm area is a shade of blue-green while the Malm\"o area is a
shade of blue-grey. Also in figure \ref{mds-5-1000-js-trigram},
Sk\aa{}ne and Blekinge are grey: clearly similar but not identical to
Malm\"o.

\section{Features}
\label{section-features}

Ranked features answer the question of agreement with dialectology
more precisely than the previous two methods. Features are ranked by
their normalized weight, which tells how much weight the distance
measure will give it. This can reveal aggregate differences that may
only be noticeable when counting a large amount of data. Conversely,
with different normalization settings, feature ranking can also point
out rare features that only occur in one site. The first kind of
features are unlikely to be noticed by linguists without the aid of
computers, whereas the second kind are the rare features that are easy
for linguists to notice.

Both kinds of normalizations are shown below; in the first set of
rankings, the normalizations for sentence size are applied, as
described in section \ref{normalization}, whereas the second set of
ranking is normalized for relative overuse, based on
\quotecite{wiersma09} normalization described in
\section{wiersma-normalization}. The overuse normalization shows which
features are used relatively more when comparing two sites. It
normalizes for feature frequency between two sites.

Without the overuse normalization, the top-ranked features will tend to
be the most common ones, those found in almost every sentence in the
interview. These common features tend to highlight gradient
differences: differences in quantity but not in quality. In contrast,
the overuse normalization allows us to see which features happen only
a few times in one side of the comparison and not at all in the
other. This is closer to a traditional linguistic analysis.

In addition, only features that appear in both groups being compared
were ranked; although features that only appear in one or the other
can be interesting, they tend to be noisy in features extracted from
automatically annotated corpora. It is not possible to tell which
unique features are interesting and which are noise, especially when
using the overuse normalization, which makes rarely occurring features
rank similarly to common ones.

These results compare clusters from the consensus trees
based on 1 round of normalization (figures \ref{consensus-1-1000} and
\ref{consensus-1-full}) as well as the consensus tree based on 5
rounds of normalization and a 1000-sentence sample (figure
\ref{consensus-5-1000}). The consensus tree for 5 rounds of
normalization and full-site comparisons only had one tree for input
and was not usable. Given these three consensus trees, the groups in
table \ref{feature-ranking-clusters} are the relevant ones for analysis.

There are four clusters, three small and one large which contains the
remainder of the sites. They are listed in table
\ref{feature-ranking-clusters}. Cluster A, containing Floby and
Bengtsfors, appears in all three consensus trees. Its features are
colored blue in the following figures. Cluster B, containing Jamshog,
Torsas and Ossjo, appears in the second two trees. It features are
colored red. Cluster C, containing Loderup and Bredsatra, appears only
in the third tree. Its features are colored yellow. The remainder of
the sites are in Cluster D; the third consensus tree differs from the
first two in splitting the remainder into two groups, but this
division is ignored here to reduce the number of comparisons. Between
large groups of sites, such comparisons are unlikely to be informative
anyway.

\begin{table}
  \begin{enumerate}
   \item[A] (Blue) Floby, Bengtsfors
    \item[B] (Red) J\"amshog, \"Ossj\"o, Tors\aa{}s
    \item[C] (Yellow) L\"oderup, Breds\"atra
    \item[D] (Cyan) Segerstad, K\"ola, S:t Anna, Sorunda, Norra Rorum,
      Villberga, Torso, Boda, Frilles\aa{}s, Indal, Leksand, Anundsj\"o,
      \AA{}rsunda, Asby, Orust, V\aa{}xtorp, Fole, Sproge, F\aa{}r\"o,
      Ankarsrum, Skinnskatteberg
  \end{enumerate}
  \caption{Clusters discussed}
  \label{feature-ranking-clusters}
\end{table}

For each pair of clusters, I rank and analyze the input features by
comparing feature differences. The features presented here are the ten
highest ranked features for a particular comparison. Although each
feature set has ten features ranked here, they are better thought of
as two sets of five features differences. The top five positive
features are shown as are the top five negative features, scaled such
that the most important feature has the value 1.0.

This has two advantages. It splits the features so that both the
positive and negative evidence are always visible; otherwise, in some
cases, if one side is strong enough, the other would be pushed out of
the top ten. However, it still allows the relative weight of evidence
to be estimated. For example, if some cluster has some idiosyncratic
features, most of the features will be positive, meaning that features
typical of that cluster contribute most to the distance between it and
other clusters. The two-part feature will show this: the five positive
features will have much higher values than the five negative features.


The first subsection, \ref{feature-ranking-complete}, shows all
comparisons between clusters for a single parameter setting: trigram
features, 1000-sentence sampling and sentence-size normalization
only. Besides unigrams, these are the parameters that give the highest
correlation with travel distance for 1000-sentence sampling.
In the next subsection, \ref{feature-ranking-overuse}, the overuse
normalization is added, keeping other parameter settings the same.
The third subsection, \ref{feature-ranking-feature-sets},
a single comparison between cluster A and cluster B is given for
all feature sets.
In the final subsection, \ref{feature-ranking-psg}, the high-ranked
phrase-structure rules are given.
The parts of speech for the features are given in table
\ref{pos-list}. The non-terminals are given in table
\ref{nonterminal-list}.

\begin{table}
\begin{tabular}{c|c||c|c}
  POS & Part of speech & POS & Part of speech \\ \hline
 $++$ & coordinating conjunction&MV & verb ``m\aa{}ste'' (must) \\
 AB & adverb&NN & noun \\
 AJ & adjective&PN & proper name \\
 AN & adjectival noun&PO & pronoun \\
 AV & verb ``vara'' (be)&PR & preposition \\
 BV & verb ``bli(va)'' (become) &QV & verb ``kunna'' (can) \\
EN & indefinite article&RO & numeral \\
 FV & verb ``f\aa{}'' (get) &SP & present participle \\
 GV & verb ``g\"ora'' (do) &SV & verb ``skola'' (shall) \\
 HV & verb ``hava'' (have) &UK & subordinating conjunction \\
 I? & question mark &VN & verbal noun \\
 ID & idiom &VV & other verb \\
 IM & infinitive marker &WV & verb ``vilja'' (want) \\
 IP & period &XX & Unclassifiable \\
MN & meta-noun &YY & Interjection \\ \hline
\end{tabular}
\vspace{5mm}
\caption{List of parts of speech}
\label{pos-list}
\end{table}

\begin{table}
  \begin{tabular}{c|c}
    NT & Non-terminal \\ \hline
     $+$F & Coordination at main clause level \\
     AA & other adverbial \\
     AP & adjective phrase \\
     AVP & adverb phrase \\
     CAP & Coordinated adjective phrase \\
     CNP & Coordinated noun phrase \\
     CONJP & Other coordinated phrase \\
     CS & Coordinated S \\
     ET & Other nominal post-modifier \\
     MS & Macrosyntagm \\
     NAC & Not a constituent \\
     NP & Noun phrase \\
     OA & Object adverbial \\
     PP & Prepositional phrase \\
     RA & Place adverbial \\
     S & Sentence \\
     SS & Other subject \\
     TA & Time adverbial \\ \hline
   \end{tabular}
   \vspace{5mm}
    \caption{List of non-terminal labels}
    \label{nonterminal-list}
\end{table}

\subsection{Trigram Features}
\label{feature-ranking-complete}

The analysis will start with trigram features without the overuse
normalization, since trigrams have the highest rate of significance of
the non-combined feature sets. (The combined feature set is not
presented because the mixed feature types make it difficult to
interpret.)

As mentioned above, the top-ranked trigrams are common, typical of the
core of the sentence. The trigrams typical of cluster A, are formed
most often, for example, from a trigram like \textit{ ``och det
  \"ar''} ``and it's'', followed by an adverb such as
\textit{``v\"al''} ``well''. Figures
\ref{clusterA-clusterB-feat-5-1000-trigram-ratio} --
\ref{clusterA-clusterD-feat-5-1000-trigram-ratio} generalize this
observation by the high rankings of the POS trigrams PO-AV-AB
(pronoun-copula-adverb, figure
\ref{clusterA-clusterB-feat-5-1000-trigram-ratio}), $++$-PO-AV
(conjunction-pronoun-copula, figure
\ref{clusterA-clusterB-feat-5-1000-trigram-ratio}), and PO-VV-AB
(pronoun-verb-adverb, figure
\ref{clusterA-clusterD-feat-5-1000-trigram-ratio}) The same is true of
the other clusters for the most part. Unfortunately, this makes it
hard to say interesting things about the difference in feature
distribution. It does appear that clusters B and C use adverbs and of
conjunctions that differ from the other clusters; for example
$++$-AB-AV in in figure
\ref{clusterA-clusterC-feat-5-1000-trigram-ratio}. The comparison
between cluster A and cluster B highlights the trigram AB-AB-AB
(figure \ref{clusterA-clusterB-feat-5-1000-trigram-ratio}) as
important, but more interesting are the $++$-AB-AV
(conjunction-copula-adverb) and AB-AV-AB (adverb-copula-adverb)
trigrams in the bottom halves of figures
\ref{clusterA-clusterB-feat-5-1000-trigram-ratio} and
\ref{clusterA-clusterC-feat-5-1000-trigram-ratio}. These trigrams
derive from sequences like \textit{``och s\aa{} \"ar''} (``and is
so'') and \textit{``inte \"ar ju''} (``is not now''), which do
not reflect standard Swedish grammar.

% TODO: Get a whole example sentence probably. Ugh. People want so
% much context!

\begin{figure}
  \includegraphics[scale=1.2]{clusterA-clusterB-feat-5-1000-trigram-ratio}
  \caption{cluster A $\Leftrightarrow$ cluster B, trigram features}
  \label{clusterA-clusterB-feat-5-1000-trigram-ratio}
\end{figure}
\begin{figure}
  \includegraphics[scale=1.2]{clusterA-clusterC-feat-5-1000-trigram-ratio}
  \caption{cluster A $\Leftrightarrow$ cluster C, trigram features}
  \label{clusterA-clusterC-feat-5-1000-trigram-ratio}
\end{figure}
\begin{figure}
  \includegraphics[scale=1.2]{clusterA-clusterD-feat-5-1000-trigram-ratio}
  \caption{cluster A $\Leftrightarrow$ cluster D, trigram features}
  \label{clusterA-clusterD-feat-5-1000-trigram-ratio}
\end{figure}


\begin{figure}
  \includegraphics[scale=1.2]{clusterB-clusterC-feat-5-1000-trigram-ratio}
  \caption{cluster B $\Leftrightarrow$ cluster C, trigram features}
\end{figure}
\begin{figure}
  \includegraphics[scale=1.2]{clusterB-clusterD-feat-5-1000-trigram-ratio}
  \caption{cluster B $\Leftrightarrow$ cluster D, trigram features}
\end{figure}
\begin{figure}
  \includegraphics[scale=1.2]{clusterC-clusterD-feat-5-1000-trigram-ratio}
  \caption{cluster C $\Leftrightarrow$ cluster D, trigram features}
\end{figure}

\subsection{Trigrams with Overuse Normalization}
\label{feature-ranking-overuse}

Given this lack of information, there are two dimensions along which
the comparisons can be altered: normalization and feature
set. Starting with normalization, let us add the overuse normalization
technique. Differences appear immediately. First, the balance of
feature weight obviously differs here. For example, in the comparison
between cluster A and cluster B (figure
\ref{clusterA-clusterB-feat-5-1000-trigram-over}), the features of
cluster A are more important in distinguishing the two than the
features of cluster B. The comparison between cluster A and cluster D (figure
\ref{clusterA-clusterD-feat-5-1000-trigram-over})
is so lop-sided that cluster D contributes no features at all.

With the overuse normalization, cluster A has two interesting
patterns. First, the trigrams it overuses are filled with indefinite
articles (EN) and prepositions (PR). Examples include VV-EN-AB
(verb-indefinite-adverb), PR-EN-AB (preposition-indefinite-adverb) and
PR-EN-VN (preposition-indefinite-verbal noun) in figure
\ref{clusterA-clusterB-feat-5-1000-trigram-over}, as well as IM-PR-NN
(infinitive marker-preposition-noun) and PR-ID-PR
(preposition-idiom-preposition) in figure
\ref{clusterA-clusterD-feat-5-1000-trigram-over}. Second, the trigrams
it underuses mostly end with pronouns: 4 of 5 trigrams in the
comparison with cluster B (figure
\ref{clusterA-clusterB-feat-5-1000-trigram-over}) and 4 or 5 in the
comparison with cluster C (figure
\ref{clusterA-clusterC-feat-5-1000-trigram-over}). Even in the
comparison with cluster D (figure
\ref{clusterA-clusterD-feat-5-1000-trigram-over}), 4 of 5 of the
``least overused'' trigrams end with pronouns. (The low values in the
bottom half of the comparison with cluster D are not underused by
cluster A, because cluster D has no unique features here. Instead they
are the ``least overused'' by cluster A.)

Cluster B shows one interesting pattern: overuse of sk\"ola (shall),
including an interesting trigram SV-QV-AB (shall verb-can verb-adverb)
in figure \ref{clusterB-clusterD-feat-5-1000-trigram-over}. Although
this could be a mistake on the part of the tagger, the different forms
of this verb are limited, so this is unlikely: identifying them is not
hard. Instead it points to the possibility of double modals.
%% a quick search suggests that Fennell and Butters (1996) finds
%% evidence in German and Scandinavian languages...but it's a book ro
%% something. Google Scholar has no link, just a wimpy citation.
%% Also:
%% Modals and double modals in the Scandinavian languages
%% Working papers in Scandinavian syntax
%% Thrainsson and Vikner 1995 (but focussing on Danish and Icelandic)

Cluster C doesn't gain any interesting patterns with overuse
normalization in figures
\ref{clusterA-clusterC-feat-5-1000-trigram-over},
\ref{clusterB-clusterC-feat-5-1000-trigram-over}, and
\ref{clusterC-clusterD-feat-5-1000-trigram-over}, except for a
surprising variety in the verbs: g\"ora (do), hava (have), kunna
(get), sk\"ola (shall), vara (be) and vilja (want). Many uses of
adverbs show up as well. It is not clear what either of these patterns
mean linguistically, however.
% I have no idea whether to make that verb singular or plural.
% So like whatever.

Cluster D gives no information whatsoever when the overuse
normalization is added, simply because it has no informative
features. This is expected, given its nature as a combination of many
sites. The tradeoff of more informative features for the smaller
clusters is worthwhile.

\begin{figure}
  \includegraphics[scale=1.2]{clusterA-clusterB-feat-5-1000-trigram-over}
  \caption{cluster A $\Leftrightarrow$ cluster B, trigram features
    with overuse normalization}
  \label{clusterA-clusterB-feat-5-1000-trigram-over}
\end{figure}
\begin{figure}
  \includegraphics[scale=1.2]{clusterA-clusterC-feat-5-1000-trigram-over}
  \caption{cluster A $\Leftrightarrow$ cluster C, trigram features
    with overuse normalization}
  \label{clusterA-clusterC-feat-5-1000-trigram-over}
\end{figure}
\begin{figure}
  \includegraphics[scale=1.2]{clusterA-clusterD-feat-5-1000-trigram-over}
  \caption{cluster A $\Leftrightarrow$ cluster D, trigram features
    with overuse normalization}
  \label{clusterA-clusterD-feat-5-1000-trigram-over}
\end{figure}


\begin{figure}
  \includegraphics[scale=1.2]{clusterB-clusterC-feat-5-1000-trigram-over}
  \caption{cluster B $\Leftrightarrow$ cluster C, trigram features
    with overuse normalization}
  \label{clusterB-clusterC-feat-5-1000-trigram-over}
\end{figure}
\begin{figure}
  \includegraphics[scale=1.2]{clusterB-clusterD-feat-5-1000-trigram-over}
  \caption{cluster B $\Leftrightarrow$ cluster D, trigram features
    with overuse normalization}
  \label{clusterB-clusterD-feat-5-1000-trigram-over}
\end{figure}
\begin{figure}
  \includegraphics[scale=1.2]{clusterC-clusterD-feat-5-1000-trigram-over}
  \caption{cluster C $\Leftrightarrow$ cluster D, trigram features
    with overuse normalization}
  \label{clusterC-clusterD-feat-5-1000-trigram-over}
\end{figure}

\subsection{Variation Across Feature Sets}
\label{feature-ranking-feature-sets}

Moving to other features sets with overuse normalization,
leaf-ancestor paths and leaf-head paths give additional information
about cluster A that lead to the conclusion its defining
characteristic is simple sentences, simpler at least than the other
clusters. Specifically, cluster A's overused leaf-ancestor paths
include few nested sentences (figure
\ref{clusterA-clusterB-feat-5-1000-path-over}). This contrasts sharply
with cluster B and cluster C, which include many nested
sentences. Cluster A does have complex paths, but they feature
prepositional phrases. (Note: NAC stands for ``not a constituent'' and
indicates that the parser could not decide what the correct
constituent was at that point, or that there are crossing branches,
which is less common.)

This characteristic of cluster A appears in the leaf-head paths as
well (figure \ref{clusterA-clusterB-feat-5-1000-dep-over}); cluster
A's paths contain many [adjective]-noun-preposition sequences, but few
verb-verb sequences that indicate nested phrases. Again, cluster B and
cluster C have many of these sequences. Both clusters have a number of
overused adverb features as well, similar to the trigram results. Note
that comparison to cluster D is less interesting. Because it has fewer
unique characteristics, when compared to it, clusters A, B and C show
more generic characteristics. For example, all three clusters show
that their sentences are generally more complex than the general sites
in cluster D.

\begin{figure}
  \includegraphics[scale=1.2]{clusterA-clusterB-feat-5-1000-path-over}
  \caption{cluster A $\Leftrightarrow$ cluster B, leaf-ancestor path features}
  \label{clusterA-clusterB-feat-5-1000-path-over}
\end{figure}
\begin{figure}
  \includegraphics[scale=1.2]{clusterA-clusterB-feat-5-1000-trigram-over}
  \caption{cluster A $\Leftrightarrow$ cluster B, trigram features}
  \label{clusterA-clusterB-feat-5-1000-trigram-over2}
\end{figure}
\begin{figure}
  \includegraphics[scale=1.2]{clusterA-clusterB-feat-5-1000-dep-over}
  \caption{cluster A $\Leftrightarrow$ cluster B, leaf-head features}
  \label{clusterA-clusterB-feat-5-1000-dep-over}
\end{figure}
\begin{figure}
  \includegraphics[scale=1.2]{clusterA-clusterB-feat-5-1000-psg-over}
  \caption{cluster A $\Leftrightarrow$ cluster B, phrase-structure
    rule features}
\end{figure}
\begin{figure}
  \includegraphics[scale=1.2]{clusterA-clusterB-feat-5-1000-grand-over}
  \caption{cluster A $\Leftrightarrow$ cluster B, phrase-structure
    rules features, with grandparent}
\end{figure}
\begin{figure}
  \includegraphics[scale=1.2]{clusterA-clusterB-feat-5-1000-unigram-over}
  \caption{cluster A $\Leftrightarrow$ cluster B, unigram features}
\end{figure}
\begin{figure}
  \includegraphics[scale=1.2]{clusterA-clusterB-feat-5-1000-redep-over}
  \caption{cluster A $\Leftrightarrow$ cluster B, leaf-head features,
MaltParser trained by Timbl}
\end{figure}
\begin{figure}
  \includegraphics[scale=1.2]{clusterA-clusterB-feat-5-1000-deparc-over}
  \caption{cluster A $\Leftrightarrow$ cluster B, leaf-head features
    with arc labels}
\end{figure}
\begin{figure}
  \includegraphics[scale=1.2]{clusterA-clusterB-feat-5-1000-all-over}
  \caption{cluster A $\Leftrightarrow$ cluster B, all combined features}
\end{figure}


\subsection{Phrase-structure rule features}
\label{feature-ranking-psg}

\begin{figure}
  \includegraphics[scale=1.2]{clusterA-clusterB-feat-5-1000-psg-over}
  \caption{cluster A $\Leftrightarrow$ cluster B, phrase-structure
    rule features}
  \label{clusterA-clusterB-feat-5-1000-psg-over2}
\end{figure}
\begin{figure}
  \includegraphics[scale=1.2]{clusterA-clusterC-feat-5-1000-psg-over}
  \caption{cluster A $\Leftrightarrow$ cluster C, phrase-structure rule features}
  \label{clusterA-clusterC-feat-5-1000-psg-over}
\end{figure}
\begin{figure}
  \includegraphics[scale=1.2]{clusterA-clusterD-feat-5-1000-psg-over}
  \caption{cluster A $\Leftrightarrow$ cluster D, phrase-structure rule features}
  \label{clusterA-clusterD-feat-5-1000-psg-over}
\end{figure}


\begin{figure}
  \includegraphics[scale=1.2]{clusterB-clusterC-feat-5-1000-psg-over}
  \caption{cluster B $\Leftrightarrow$ cluster C, phrase-structure rule features}
\end{figure}
\begin{figure}
  \includegraphics[scale=1.2]{clusterB-clusterD-feat-5-1000-psg-over}
  \caption{cluster B $\Leftrightarrow$ cluster D, phrase-structure rule features}
\end{figure}
\begin{figure}
  \includegraphics[scale=1.2]{clusterC-clusterD-feat-5-1000-psg-over}
  \caption{cluster C $\Leftrightarrow$ cluster D, phrase-structure rule features}
\end{figure}

Analysis of the phrase-structure-rule features is difficult because of
all the noise. In figures
\ref{clusterA-clusterB-feat-5-1000-psg-over2} and
\ref{clusterA-clusterD-feat-5-1000-psg-over}, features like
S$\to++$-AB (conjunction-adverb) S$\to$FV-PO-AB-VV (get
verb-pronoun-adverb-verb) are hard to describe as anything but junk
rules created by the parser. On the other hand, there are a lot of
linguistically odd but reasonable rules like S$\to$PO-AV-NP-IP
(pronoun-copula-noun phrase-period) in figure
\ref{clusterA-clusterC-feat-5-1000-psg-over}. Although this is not a
good linguistic decomposition, it is one that a statistical parser
would create when copular sentences are
common enough.

Overall both normalizations leave something to be desired; without
overuse normalization, only very common features appear. These
features convey only basic information, making it hard to identify
characteristics of a cluster. On the other hand, the overuse
normalization is susceptible to noise, especially for more error-prone
feature sets. Even though more detail may be available with this
normalization step, the features must be inspected for general trends
because individual features are not necessarily reliable.

\section{Conclusion}

This chapter provided results and analysis of the results with no
comparison to other work. The next chapter
will compare the results to dialectology, phonological dialectometry,
and previous work in syntactic dialectometry. Even before this
comparison, however, the distance measure seems to be successful at
producing dialect distance.

Quite a few patterns are visible: the significance tests show that the
distance measure is finding significant distances for most parameter
settings; the analysis of correlation shows that some correlate with
geographic and travel distance. The dendrogram maps, composite cluster
maps, and MDS maps all show a picture of with fairly well-defined areas.
Finally, the feature rankings show some interesting patterns but
nothing definitive.


%%% Local Variables: 
%%% mode: latex
%%% TeX-master: "dissertation.tex"
%%% End: 


\chapter{Discussion}

This chapter discusses three topics:

\begin{enumerate}
\item Analysis of results: dissertation work on its own.
TODO: Move this to results chapter.
\item Comparison to Swedish syntactic dialectology.
TODO: More of this to come! (Unless I have already written some)
\item DONE Comparison to Swedish phonological dialectometry. (Based mostly
  on Therese's previous work and personal communication; the
  dissertation won't get here on time)
\item Comparison to syntactic dialectometry.
  That is, previous work. That is, MY previous work, plus
  Wybo's. That's pretty much it you know.
\item Conclusions: summary of discussion
\item and then summary of dissertation: contribution to dialectometry at large.
\end{enumerate}

A big question is why trigrams are so good. All of the fancier feature
sets do worse than trigrams. I should address this in the summary for
sure.

The reason is that the simpler tags are easier to tag. So I recommend
for the real-world analyses with only untagged transcriptions as input
to simply use trigrams.

\section{Analysis of Results}

% TODO: Need to add references to results chapter and also any
% citations

\subsection{Significance of Dialect Distance}

Analysis of the significance of dialect distance provides a measure of
how reliable the distances to be analysed later in this chapter are. A
distance that does not find significant distances between of 30
regions is not suitable for precise inspection, although small numbers
of non-significant distances will still allow less precise methods to
return interpretable results.

The highest number of significant distances are found in the first
case (figure \ref{sig-1-1000}): 1 round of normalisation with a
fixed-size sample of 1000 sentences. From there, both full-corpus
comparisons (figure \ref{sig-1-full}) and 5 rounds of normalisation
(figure \ref{sig-5-1000}) have fewer significant distances, although
the number is still usable. However, the combination of the two, with
5 rounds of normalisation over full-corpus comparisons, has only one
combination with fewer than 5\% of distances that are {\it not}
significant. Although both full-corpus comparisons and multiple rounds
of normalisation may increase the precision of the results, their
combined effect on significance is so detrimental that its results are
useless. For the rest of the analysis, the combination of full-corpus
comparison and 5 rounds of normalisation will be skipped.

\subsubsection{Significance by Measure}

The distance measures most likely to find significance are, in order,
cosine dissimilarity, Jensen-Shannon divergence and $R$. Each method
had different parameter settings for which it was stronger. For
1000-sentence sampling, cosine similarity resulted in all significant
distances, even for part-of-speech unigrams, which are intended as the
baseline feature set. Excluding unigrams, Jensen-Shannon divergence
has similar performance. For full-corpus comparisons, both perform
considerably worse; surprisingly, both perform better on unigram
features, Jensen-Shannon so much so that it's the only feature set for
which it finds all significant distances. $R$, on the other hand,
performs decently on all combinations of parameter settings; its low
significance for phrase structure rules is shared by Kullback-Leibler
and Jensen-Shannon divergences.
% TODO : Maybe more on cosine later. Maybe not.

When comparing the performance of Kullback-Leibler and Jensen-Shannon
divergence it is not surprising that Jensen-Shannon outperforms
Kullback-Leibler on fixed-size sampling. Although both are called
``divergence'', Jensen-Shannon divergence is actually a
dissimilarity. Recall that the divergence from point A to B may differ
from the divergence from point B to A. A divergence like
Kullback-Leibler can be converted to a dissimilarity by measuring
$KL(A,B) + KL(B,A)$. However, this dissimilarity must skip features
unique to a single corpus in order to avoid division by zero. This
means that for smaller corpora Kullback-Leibler loses information that
Jensen-Shannon is able to use.  On the other hand, while this may
explain Kullback-Leibler's improved performance for full-corpus
comparisons, it doesn't explain Jensen-Shannon's much worse
performance.

\subsubsection{Significance by Feature Set}

% \item Unigrams do form an adequate baseline; they are bad but not too
%   bad.

% The feature sets most likely to find significance are the combined
% features and unigrams., in order,
% trigrams, all combined features and leaf-head paths (both with
% support-vector-machine training and with Timbl's instance-based
% training). Without ratio normalisation, the other feature sets are not
% much worse, but with it included, these three are the best by some
% distance.

For 1 round of normalisation, the best feature sets are the simple
ones: trigrams and unigrams, as well all combined features. On the
other hand, trigrams and leaf-head paths (with its variations) are the
best feature sets with 5 rounds of normalisation. However, the
variation isn't strong; any feature set can give good results with the
right distance measure. The problem is that no clear patterns emerge.

The relatively high quality of trigrams and unigrams does not make
sense given only the linguistic facts; however, it is likely that the
entirely automatic annotation used here introduces more and more
errors as more annotators run, operating on previous automatic
annotations. Trigrams are the result of only one automatic annotation,
and one for which the state of the art is near human performance. So
the fact that these particular parts of speech are of higher quality
than the corresponding dependencies or constituencies is probably the
deciding factor in their higher number of significant
distances.

% Although it is impossible to tell from my results, I
% predict that a manually annotated dialect corpus would show that
% non-flat syntactic structure is helpful in producing significant
% distances.

Given the above facts, the question should rather be: why do leaf-head
paths perform as well as they do? Better, for example, than the
leaf-ancestor paths on which they're modelled: why does more
normalisation hurt leaf-ancestor paths but not leaf-head paths?  It
could be that there is less room for error; many of the common
leaf-head paths are short: short interview sentences with simple
structure make for shorter leaf-head paths than leaf-ancestor
paths. As a result, the important leaf-head paths consist mainly of a
couple of parts-of-speech.

Another reason could be a difference in parsers: MaltParser has been
tested before with Swedish (CITE). Besides English, the Berkeley
parser has been tested prominently on German and Chinese. Therefore,
the difference would better be explained by appealing to the
difference in parsers rather than an unsuitability of Swedish for
constituent analysis.

It is disappointing linguistically that trigrams provide the most
reliable results so far; a linguist would expect that including
syntactic information would make it easier to measure the differences
between regions. If it is, as hypothesised here, an effect of chaining
machine annotators, a study using manually annotated corpora could
detect this. However, it still means that trigrams are the most useful
feature set from a practical view, because automatic trigram tagging
is very close to human performance with little training. That means
the only required human work is the transcription of interviews in
most cases.

On the other hand, if additional features sets are to be developed for
a corpus, then combining all available features seems to be a
successful strategy. The distance measures seem to be able to use all
available information for finding significant distances.

\subsection{Correlation}

In dialectology, the default expectation for dialect distance is that
it correlates with geographic distance \cite{chambers98}. A lack of
correlation does not necessarily mean that a measure is useless, but
presence of correlation means that the distance measure substantiates the
well-known tendency of dialect distributions to be more or less
smoothly gradient over physical space.

In addition, the distance measures are more likely to correlate
significantly with travel distance than with straight-line geographic
distance. This makes sense since the difficulty of moving from place
to place is what influences dialect formation, and taking roads into
account is an improved estimate over straight-line distance.

As with the number of significant distances, trigrams and unigrams are
the most likely to to correlate with geographic and travel distance,
as well as the combined feature set for the 5-normalisation parameter
setting.
% As before, a possible explanation is that unigrams are
% simpler, so the type count is a higher than for other measures. With
% more rounds of normalisation, more correlations shift over to
% trigrams.
Note that in figures \ref{cor-1-1000} --
\ref{travel-cor-5-full}, the significant correlations are marked with
an asterisk, but only the italicised correlations are based on at
least 95\% significant distances. For example, this means that most of
the significant correlations based on phrase-structure rules are not valid.

It is worthwile to note, however, that the valid and significant
correlations based on phrase-structure grammars give the highest
correlations: 0.37 for $R^2$ with full-corpus comparisons and 1 round
of normalisation.
The addition of more data and more normalisation is interesting in
expanding the correlating parameter settings beyond those that include
unigram features. It may be that this is an instance of the noise/quality tradeoff.
These additions appear to extract more detail from
the data, at the cost of additional interference from noisy data.

% Goes here: Fevered speculation about why travel correlation is *better* with
% the methods that correlate *less*, for 1-full at least.
% OK never mind this isn't true.

\subsubsection{Inter-measure correlation}

In table \ref{self-correlation-measures}, $R$ and Jensen-Shannon
produce nearly identical results. Also, cosine similarity arrives
at different results than the other measures, though the correlation
is still higher than with travel distance.

\subsubsection{Correlation with Corpus Size}

The correlation of corpus size and dialect distance is a problem. It
is not a predicted as a side effect of the way dialect distance is
measured. The fact that travel distance also correlates with corpus
size at a rate of 0.32 confuses the issue further. Is corpus size the
determining variable? Or is there an unknown variable influencing all
three? Some possibilities are ``interviewer boundaries'', common in
corpora collected by multiple people \cite{chambers98}, or perhaps the
interviewers got better over time and collected longer interviews as
they moved throughout the country, or perhaps cultural differences
between the interviewer and interviewees caused some participants in
one area to talk more than in another area.

Definitely the high correlations between corpus size and the
5-normalisation distances are worrying. They are so much higher than
the correlation of corpus size and travel distance that 5-normalised
distances might not be reliable.
% It appears that multiple rounds of
% normalisation inadvertently re-introduce a dependency on size.
% TODO: This probably IS a bug in that only freq norm can be
% iterated. Ratio norm should probably be in a separate loop like so:
% #ifdef RATIO_NORM
%   for(sample::iterator i = ab.begin(); i!=ab.end(); i++) {
%     i->second.first *= 2 * types / tokens;
%     i->second.second *= 2 * types / tokens;
%   }
% #endif
However, if 5-normalisation introduces a dependency on corpus size,
then the distances from full-corpus comparisons should correlate even
more highly. This is not the case.

% TODO: I also should write this up when I have time
Alternatively, it is possible that the fixed-size sampling method is not doing its
job in eliminating size differences between corpora. Future work
should develop a method for normalising a comparison between two full
corpora. It should avoid sampling, but also take the relative number
of sentences into account. It is not difficult to come up
with a simple method to do so, but it needs some checking to make sure
that the method is valid.

\subsection{Cluster Dendrograms}

The cluster dendrograms in chapter \ref{results-chapter} are dangerous
to interpret too closely on their own; the instability of a single
dendrogram means that small clusters cannot be analysed reliably. For
example, in figure \ref{cluster-5-r-trigram}, a two-way split
between the regions on the top and bottom of the page is
obvious, and a three-way split is easy to argue for, but outliers like
Anundsj\"o and \.Arsunda are likely to shift from group to group in
other dendrograms.

It is safer to analyse the consensus trees; the smoothing effect of
taking the majority rule of each cluster will show where the optimal
cutoff for splitting clusters is. The three consensus trees in figures
\ref{consensus-1-1000} -- \ref{consensus-5-1000} vary in amount of
detail but share most details.

For 1000-sentence samples and 1 round of normalisation, there is one
cluster: Floby and Bengtsfors. Full-corpus comparison finds
another cluster: J\"amshog, \"Ossj\"o and Tors\.as. Finally,
1000-sentence samples and 5 rounds of normalisation finds another
cluster consisting of L\"oderup and Breds\"atra. It also finds
a large two-way split between the regions and adds Sproge to the first
cluster with Floby and Bengtsfors. To aid further analysis, the
clusters are assigned colours, which are detailed in figures
\ref{blue-cluster} -- \ref{orange-cluster}. 

\begin{figure}
\begin{itemize}
\item Floby
\item Bengtsfors
\item Sproge (for 1000-sample, 5-normalisation)
\end{itemize}
\caption{Blue Cluster}
\label{blue-cluster}
\end{figure}

\begin{figure}
\begin{itemize}
\item J\"amsh\"og
\item Tors\.as
\item \"Ossj\"o
\end{itemize}
\caption{Red Cluster}
\label{red-cluster}
\end{figure}

\begin{figure}
\begin{itemize}
\item Breds\"atra
\item L\"oderup
\end{itemize}
\caption{Yellow Cluster}
\label{yellow-cluster}
\end{figure}

\begin{figure}
\begin{itemize}
\item Leksand
\item Indal
\item Segerstad
\item Floby
\item Bengtsfors
\item Sproge
\item Skinnskatteberg
\item Orust
\item V\.axtorp
\item F\.ar\"o
\item Asby
\item \.Arsunda
\item Anundsj\"o
\item Ankarsrum
\item Fole
\end{itemize}
\caption{Cyan Cluster}
\label{cyan-cluster}
\end{figure}

\begin{figure}
\begin{itemize}
\item Viby
\item Bara
\item S:t Anna
\item Frilles\.as
\item J\"amshog
\item Tors\.as
\item \"Ossj\"o
\item K\"ola
\item L\"oderup
\item Breds\"atra
\item Villberga
\item Tors\"o
\item Norra R\"orum
\item Sorunda
\item B\"oda
\end{itemize}
\caption{Orange Cluster}
\label{orange-cluster}
\end{figure}

When these clusters are mapped onto the geography of Sweden, some
patterns are visible. Since figure \ref{consensus-5-1000} is strictly
more complex than the preceding two, it is used as the basis for this
analysis--see map \ref{map-consensus-5-1000}. The large two-way split
is between the orange and cyan clusters. The orange cluster, which
includes red and yellow clusters, forms two horizontal bands across
Sweden. The centres of the orange cluster appear to be Stockholm and
Malm\"o. Meanwhile, the red and yellow clusters form a boundary along
the northern border of Sk\.ane and Blekinge counties.

Meanwhile, the cyan cluster, which includes the blue cluster, seems to
represent the countryside of Sweden. On the other hand, because the
blue cluster is near G\"oteborg, it might be better characterised
simply as ``non-Stockholm''.


\subsection{Composite Cluster Maps}

Composite cluster maps use an underlying technique similar to consensus
trees--cluster dendrograms, but they combine and present the information in
a very different way. The result looks much more like the traditional
isogloss boundaries of dialectology. The composite cluster maps are in
maps\ref{map-composite-1-1000} -- \ref{map-composite-5-1000}.

All three composite clusters maps provide a picture similar to the
consensus tree map \ref{map-consensus-5-1000} of the previous
section. The north-to-south gradient is supported by the
weak horizontal boundaries present up and down Sweden.

Of these boundaries, the one between Sk\.ane and the rest of Sweden is
the strongest. Due to the lack of interview sites in the middle of
south Sweden, the boundary is drawn further north than it
traditionally appears, but this is an effect of the software the
produced the figure. Notice that there is also a boundary between
J\"amshog, Tors\.as, and \"Ossjo\" and the other sites, especially
visible in maps\ref{map-composite-1-1000} and
\ref{map-composite-5-1000}. Their presence along the northern border
of Sk\.ane is one reason why its boundary with the rest of Sweden is
so strong.

Compared to the consensus tree maps, the composite cluster maps cannot
support the city/country distinction because there is no way to
identify distant areas by their colour. On the other hand, it is
possible to detect the relative strength of a boundary. To combine
these two features, multi-dimensional scaling is needed.
% But of course MDS maps can't be combined into a consensus...

% However, K\"ola and Frilles\.as still separate fairly well from their
% neighbours. These sites are on the edges of the country and have strong borders
% with surrounding, Like the cluster J\"amshog, Tors\.as and \"Ossj\"o,
% these sites are different from the others. However, they don't have
% any geographic coherence, so it is more likely these are remnants of a
% dialect that was historically wider spread and has since receded.

\subsection{Multi-Dimensional Scaling}

Multi-dimensional scaling (MDS) operates similarly to cluster
dendrograms in that it reduces the high-dimensional distances to a
lower-dimensionality representation. It differs, however, in producing
gradient numbers, not binary trees. This means that the MDS maps
naturally have gradient borders. Also, because of the way that the
3-dimensional points map to colours, the maps vary. However, they are
still comparable: if two regions are blue in one map and both are
orange in another, then they have the same relation to each other.

Despite the differences between MDS and the preceding methods, the
similar results are evident; the maps (figures
\ref{mds-1-1000-r-trigram} -- \ref{mds-5-1000-js-trigram}) all show
the same patterns as the other methods. That is, there is a general
north-to-south gradience, especially easy to see in map
\ref{mds-1-1000-js-trigram}. There is a strong southern cluster,
visible in all of the diagrams. And there is a general two-way
distinction between city and country.

The main contribution that the MDS maps make is that the
north-to-south gradient is more obviously gradient. In other words, it
is easier to see the gradation from north to south. For example, in
figure \ref{mds-1-full-r_sq-psg}, looking from the north to south, the
colours change quickly close to Stockholm, then fade to green further
south, then transition back to blues and purples further south, in
Sk\.ane.

The Stockholm and Malm\"o areas, which are in the same cluster in the
consensus tree maps, are here seen to be similar without being
identical. For example, in figure \ref{mds-5-1000-js-trigram}, the
Stockholm area is a shade of blue-green while the Malm\"o area is a
shade of blue-grey. Also in figure \ref{mds-5-1000-js-trigram},
Sk\.ane and Blekinge are grey: clearly similar but not identical to
Malm\"o.

\subsection{Features}

%TODO: Put back in the freq/ratio features for $R$ instead of the
%overuse-ranked features. I guess overuse isn't that useful after all.

Next I rank and analyse the input features by comparing their
differences. The features analysed here are the ten highest ranked
features for a particular comparison, based on the consensus tree
clusters of figures \ref{consensus-1-1000} --
\ref{consensus-5-1000}. The clusters are labelled A to D from top to
bottom. The clusters are reproduced in table
\ref{feature-ranking-clusters-again}. Although each feature set has
ten features ranked here, it's actually two sets of five features,
because these are differences. The top five positive features are
shown as are the top five negative features.

\begin{table}
  \begin{enumerate}
   \item[A] (Blue) Floby, Bengtsfors
    \item[B] (Red) J\"amshog, \"Ossj\"o, Tors\.as
    \item[C] (Yellow) L\"oderup, Breds\"atra
    \item[D] (Cyan) Segerstad, K\"ola, S:t Anna, Sorunda, Norra Rorum,
      Villberga, Torso, Boda, Frilles\.as, Indal, Leksand, Anundsj\"o,
      \.Arsunda, Asby, Orust, V\.axtorp, Fola, Sproge, F\.ar\"o,
      Ankarsrum, Skinnskatteberg
  \end{enumerate}
  \caption{Clusters discussed}
  \label{feature-ranking-clusters-again}
\end{table}

This has two advantages. It splits the features so that both the
positive and negative evidence are always visible; otherwise, in some
cases, if one side is strong enough, the other would be pushed out of
the top ten. However, it still allows the relative weight of evidence
to be estimated. For example, if some cluster has some idiosyncratic
features, most of the features will be positive, meaning that most of
the distance comes from features typical of this cluster. The two-part
feature will show this: the five positive features will have much
higher values than the five negative features.

For feature ranking, there is an additonal normalisation used
called overuse normalisation, which ranks rarely used features more
highly.  This can be useful; without it, the top-ranked features will
tend to be the most common ones, those found in almost every sentence
in the interview. These common features tend to highlight
gradient differences: differences in quantity but not in quality. In contrast,
the overuse normalisation allows us to see which features happen only
a few times in one side of the comparison and not at all in the
other. This is closer to a traditional linguistic analysis.

\subsubsection{Analysis}

The analysis will start with trigram features without the overuse
normalisation, since trigrams have the highest rate of significance of
the non-combined feature sets. (The combined feature set is difficult
to read because of the mixing of feature types.)

As mentioned above, the top-ranked trigrams are common, typical
of the core of the sentence. Cluster A's typical trigrams, for
example, typically involve a pronoun or a verb or both: PO-AV-AB
(pronoun-copula-adverb), $++$-PO-AV (conjunction-pronoun-copula) and
PO-VV-AB (pronoun-verb-adverb). The same is of the
other clusters for the most part. Unfortunately, this makes it say
interesting things about the difference in feature distribution. It
does appear that clusters B and C have heavier use of adverbs and of
conjunctions. The comparison between cluster A and cluster B even
highlights the trigram AB-AB-AB as important.

Given this lack of information, there are two dimensions along which
the comparisons can be altered: normalisation and feature
set. Starting with normalisation, let us add the overuse normalisation
technique. Differences appear immediately. First, the balance of
feature weight obviously differs here. For example, in the comparison
between cluster A and cluster B, the features of cluster A are more
important in distinguishing the two than the features of cluster
B. The comparison between cluster A and cluster D is so lop-sided that
cluster D contributes no features at all.

With the overuse normalisation, cluster A has two interesting
patterns. First, the trigrams it overuses are filled with indefinite
articles (EN) and prepositions (PR). Examples include VV-EN-AB
(verb-indefinite-adverb), PR-EN-AB (preposition-indefinite-adverb) and
PR-EN-VN (preposition-indefinite-verbal noun), as well as IM-PR-NN
(infinitive marker-preposition-noun) and PR-ID-PR
(preposition-idiom-preposition). Second, the trigrams it underuses
mostly end with pronouns: 4 of 5 trigrams in the comparison with
cluster B and 4 or 5 in the comparison with cluster C. Even in the
comparison with cluster D, 4 of 5 of the ``least overused'' trigrams
end with pronouns. (The low values in the bottom half of the
comparison with cluster D are not underused by cluster A, because
cluster D has no unique features here. Instead they are the ``least
overused'' by cluster A.)

Cluster B shows one interesting pattern: overuse of sk\"ola (shall),
including an interesting trigram SV-QV-AB (shall verb-can
verb-adverb). Although this could be a mistake on the part of the
tagger, the different forms of this verb are limited, so this is
unlikely: identifying them is not hard. Instead it points to the
possibility of double modals.
%% a quick search suggests that Fennell and Butters (1996) finds
%% evidence in German and Scandinavian languages...but it's a book ro
%% something. Google Scholar has no link, just a wimpy citation.
%% Also:
%% Modals and double modals in the Scandinavian languages
%% Working papers in Scandinavian syntax
%% Thrainsson and Vikner 1995 (but focussing on Danish and Icelandic)

Cluster C doesn't gain any interesting patterns with overuse
normalisation except for a surprising variety in the verbs: g\"ora
(do), hava (have), kunna (get), sk\"ola (shall), vara (be) and vilja
(want). Many uses of adverbs show up as well. It is not clear what
either of these patterns mean linguistically, however.
% I have no idea whether to make that verb singular or plural.
% So like whatever.

Cluster D gives no information whatsoever when the overuse
normalisation is added, simply because it has no informative
features. This is expected, given its nature as a combination of many
sites. The tradeoff of more informative features for the smaller
clusters is worthwhile.

Moving to other features sets with overuse normalisation,
leaf-ancestor paths and leaf-head paths give additional information
about cluster A that lead to the conclusion its defining
characteristic is simple sentences, simpler at least than the other
clusters. Specifically, cluster A's overused leaf-ancestor paths
include few nested sentences. This contrasts sharply with cluster B
and cluster C, which include many nested sentences. Cluster A does
have complex paths, but they feature prepositional phrases. (Note: NAC
stands for ``not a constituent'' and indicates that the parser could
not decide what the correct constituent was at that point.) (Or that
there are crossing branches, which is less common.)

This characteristic of cluster A appears in the leaf-head paths as
well; cluster A's paths contain many [adjective]-noun-preposition
sequences, but few verb-verb sequences that indicate nested
phrases. Again, cluster B and cluster C have many of these
sequences. Both clusters have a number of overused adverb features as
well, similar to the trigram results. Note that comparison to clsuter
D is less interesting. Because it has fewer unique characteristics,
when compared to it, clusters A, B and C show more generic
characteristics. For example, all three clusters show that their
sentences are generally more complex than the general sites in cluster D.

Analysis of the phrase-structure-rule features is difficult because of
all the noise. Features like S$\to++$-AB (conjunction-adverb)
S$\to$FV-PO-AB-VV (get verb-pronoun-adverb-verb) are hard to describe
as anything but junk rules created by the parser. On the other hand,
there are a lot of linguistically odd but reasonable rules like
S$\to$PO-AV-NP-IP (pronoun-copula-noun phrase-period), which makes a
certain kind of sense if you can be persuaded that copular sentences
are special enough to deserve their own rule. (Remember that
statistical parsers trained on interview data are particularly
susceptible to this kind of persuasion.)

Overall both normalisations leave something to be desired; without
overuse normalisation, only very common features appear. These
features convey only basic information, making it hard to identify
characteristics of a cluster. On the other hand, the overuse
normalisation is susceptible to noise, especially for more error-prone
feature sets. Even though more detail may be available with this
normalisation step, the features must be inspected for general trends
because individual features are not necessarily reliable.

\section{Comparison to Syntactic Dialectology}

The comparison to syntactic dialectology covers three main
areas. First, the overall expected pattern of dialects in
Sweden. Second, the specific regions. Third, the individual features.

\subsection{Overall Pattern}

The default expectation of dialect distance is that it should
correlate with geographic distance \cite{goosken04a}. This is
especially true for the Scandinavian languages; \namecite{halland05}
points out that . 

\begin{itemize}
\item Significance -- Does not apply: dialectology only measures
  significance by guessing if there are enough isoglosses in one place
  to form a bundle.
\item Correlation (with geo) -- Just say here that we expect a
  north-to-south gradient, which translates to a high geo
  correlation. We didn't get that oh well.
\item Clusters and Consensus tree and Maps -- remember, these produce a
  city/country divide, a boundary at Jamshog, and something kind of
  near Goteborg.

  Most of this analysis agrees with the existing dialectology
literature; the north-to-south gradient is well-attested in
\namecite{hallberg05}. In the south, it is also well-known that the
boundary with Sk\.ane is stronger; this boundary extends along
Blekinge as well, it seems. However, the difference between city and
countryside is not well attested in the literature, nor is the
possible difference between Stockholm/Malm\"o and G\"oteborg.

The southern section contradicts Hallberg's assertion that the
boundaries do not follow the previously Danish area. However, there
aren't many sites here, so it may just lack precision enough to tell
that the southern boundary is *near* the ex-Danish border. Still, it's
telling that there's such a strong region and boundary down there at
all.

\item Composite Cluster Maps
  
  North-to-south gradient. Yup. 
  
\item MDS
  This is basically the same as the composite cluster maps except that
  you have to agree with my interpretation because the maps are not so
  self-evident.

\item Features
NOTE: This is already done: below
\end{itemize}

The literature for Swedish syntactic dialectology is not large;
although it is not large anywhere, Swedish dialectology is not a very
large field. I will compare my results to two papers,
\namecite{delsing03} and \namecite{rosenkvist07}. The first paper is a
survey of syntactic dialectology from the late 19th and early 20th
centuries. In the same volume, other papers analyse specific phenomena
in more detail; the survey is mostly concerned with the differences
and distributions rather than the syntactic analysis. The second
is an analysis of the South Swedish Apparent Cleft.

\subsection{Delsing's Survey of the Norse Nominal Phrase}

\namecite{delsing03} surveys a number of dialectology studies. These
studies date from the height of the field in Sweden, from circa
1880--1930, which Delsing at times augments with modern data. It is
worth noting that the Swedia data in the comparison was collected
around 2000, so there were likely changes in the dialects in the
intervening 70--120 years. This is particularly true in the northern
dialect areas, where improved travel and communication will have added
a centralising effect.

% The gerunds in this paragraph suck and turn it into a
% single-sentence thing.
However, comparison to the eight phenomena in the survey are too
useful to ignore, so with that caveat in mind, I will investigate each
in turn, starting with a summary of the phenomenon for Swedish
dialects, then identifying the relevant interview sites. Next I
represent the phenomenon in terms of the feature sets used throughout
this dissertation. This is somewhat difficult; the features are mostly
designed for ease of automatic tagging and extraction. When they
capture syntactic differences between two regions, they do so in a way
that can be hard to translate to linguistic analyses. (NOTE: Trigrams
are used therefore because they are least simple, and also their
tagger is more reliable)

With the target sites and features defined, it is straightforward to count the
number of occurrences of each feature in each site and compare the
two. If the predicted dialect phenomenon is reflected in the data,
then the sites associated with the phenomenon will have more
occurrences of the target features than the non-associated sites. This
difference is precisely what the distance measures use.

This method is inadequate for two reasons: first, the translation of
linguistic analysis to feature representation will not be perfect and
may miss some valid instances of the linguistic phenomenon. Second,
more importantly, the differences are not yet checked for statistical
significance. As such, the comparison can only be suggestive;
checking for statistical significance will have to wait for future
work.

% As an aside, much of this missing information IS available to me, so I
% could look manually. But none of it made it through to the distance
% measures, and this analysis compares the way the distance measures
% make the decisions with the way that linguists make their
% decisions. So I have to use only the information that the distance
% measures used.

The maps reproduced here are taken from Delsing's survey as well.

\subsubsection{``Partitive'' Article}

Northern Sweden uses the suffixed article much more than the rest of
Sweden. Delsing says the reason is that some uses of the suffix
article are not definite in the north; they have a partitive function,
similar to the partitive article in French, which is not marked in the
rest of the country. (There is also a use of the suffixed article in
predicative constructions, but this is probably not related). See
figure \ref{partitive-article} for an example.


\begin{figure}
  {\it H\"a finns vattne d\"ari hinken.} \\
  Here found the-water in the-bucket \\
  There is water in the bucket. \\
  \caption{Suffix marking for partitive}
  \label{partitive-article}
\end{figure}

Unfortunately, the part-of-speech tag set used for this
dissertation does not record whether nouns are marked with the
definite suffix. Therefore, there is no way to tell the difference
between suffixed dialect usage and bare standard usage. This feature
must be skipped because it cannot be compared.

\subsubsection{Proper-Noun Articles}

In Northern Scandinavia, first names are preceded by an indefinite
article, and sometimes last names as well. It also includes the
kinship terms that are used as proper names. Standard Swedish does not
include this feature. In Sweden, this feature is found along the border
with Norway as well as Northern Sweden. In our data, this includes the
interview sites K\"ola, Indal, and Anundsj\"o. An example is given in
figure \ref{indefinite-article-proper-noun}

\begin{figure}
  \includegraphics[scale=0.7]{dialektboka-karta3}
  \caption{Proper-Noun Articles}
  \label{indefinite-article-proper-noun-map}
\end{figure}

\begin{figure}
  {\it En Bjurstr\"om ha aff\"arn.} \\
  A Bjurstr\"om has the-store. \\
  Bjurstr\"om has a store. \\
  \caption{Indefinite Article for Proper Nouns: First Names}
  \label{indefinite-article-proper-noun}
\end{figure}

Unlike the partitive article suffix, this feature is easy to detect
with simple features. Specifically, it can be represented as the
bigram EN-PN (indefinite article-proper noun), which can be used as a
search term in the trigram feature set. The same EN-PN sequence is
expected for leaf-head paths, since the indefinite article depends on
proper noun. The phrase-structure-rule features should
look something like NP$\to$EN-PN.

Occurrences of the EN-PN bigram in the trigram feature set for
Leksand, Indal and K\"ola agree with the linguistic analysis: a rate
of 0.00007 versus 0.00006. Unfortunately, this result can hardly be
trusted because the rate of occurrence for both regions is so rare, as
well as so close between the two regions. The only conclusion that can
be drawn is that the hypothesis is not yet disproven.

\subsubsection{Possessives and the article}

In Swedish, and in the other Scandinavian countries, there is a good
deal of variation in the handling of possessives with articles. In
Swedish, normally only one is allowed in a noun phrase: either a
possessive or a determiner, but not both. However, in Danish and the
Danish-influenced areas of Sweden, both are allowed in certain
cases. When the possessive and determiner are separated from the noun
by an adjective, both are allowed. Delsing gives an example from
Danish, copied here in figure \ref{possessive-plus-article-example}.
This pattern is also standard in the southwest corner of Sweden, very
near to Denmark. This includes the interview site Bara. In addition,
it alternates with the standard on the island of Gotland, which
includes the interview sites Fole, F\.ar\"o and Sproge.

\begin{figure}
  \includegraphics[scale=0.7]{dialektboka-karta4}
  \caption{Proper-Noun Articles}
  \label{possessive-plus-article-map}
\end{figure}

\begin{figure}
  {\it naboens den stribede kat} \\
  Neighbours' the striped cat \\
  The neighbours' striped cat.
  \caption{Simultaneous possessive and determiner in noun phrase in
    Danish}
  \label{possessive-plus-article-example}
\end{figure}

This pattern can be detected by looking for the
4-grams PO-PO-AJ-NN, PR-PO-AJ-NN and NN-PO-AJ-NN. The first is the
sequence pronoun-pronoun-adjective-noun, for example {\it mitt det
  gamla huset} ``My the old house-the''. The second starts with a
proper name, such as {\it Pers} ``Per's'', and the third starts with a
noun, such as {\it naboen} ``neighbour's''. These three 4-tag sequence
can be encoded as trigrams by breaking them into two pieces.

In addition to this pattern, there is a second in the north of
Sweden. Here, it is simply that possessive personal pronouns are
allowed both before and after the noun. This pattern includes the
interview sites Indal and Anundsj\"o and is covered in the next
section.

Searching Bara, in the southwest of Sweden, for the previously
mentioned trigram patterns does not find them: the rate of occurrence
is 0.00289 inside Bara but 0.000341 outside. It should be higher in
Bara. However, Delsing also mentions that Scanians he has asked do not
recognise this form either, so it is possible that it has fallen out
of use in the 70 years or so since it was last reported.

Executing a similar search for the Gotland sites (F\.ar\"o, Fole and
Sproge), but with the addition of the standard trigrams PO-AJ-NN,
PR-AJ-NN and NN-AJ-NN, shows similar results: 0.00441 on Gotland,
0.00495 off.

The final region in map \ref{possessive-plus-article-map}, in northern
Sweden, which includes Indal and Andundsj\"o, is actually more
complicated than can be captured by the part-of-speech tags used here;
this region allows possessive proper nouns to occur with
suffix-determiner nouns. But this can occur in either order: for
example, both ``Pers huset'' and ``huset Pers'' is allowed. Although
both ``Pers hus'' and ``Pers huset'' produce identical tags (PR-NN),
trigrams encode order, so the unusual order in ``huset Pers'' can be
searched for.

Searching for the bigrams NN-PN (noun-proper noun) and NN-PO
(noun-pronoun) shows a usage rate of 0.02532 for Indal and Anundsj\"o
and a rate of 0.02438 for the rest of Sweden. This is the expected
direction, but the rate of usage is very similar between the two
regions. The comparison is really too close to make a prediction
because the difference is not likely to be significant.

% \subsubsection{Pronominal Possessives}

% In Swedish, as well all of mainland Scandinavian, another possessive
% construction is the reflexive genitive, which consists of a
% noun-reflexive-noun sequence. An example is given in figure
% \ref{genitive-reflexive-normal-example}. However, this construction
% does not allow pronouns: the sequence noun-reflexive-pronoun is not
% allowed (see figure \ref{genitive-reflexive-pronoun-example}).

% \begin{figure}
%   {\it Per sitt hus} \\
%   Per its house
% ``Per's house'' \\
% \caption{Standard Swedish genitive reflexive construction}
% \label{genitive-reflexive-normal-example}
% \end{figure}

% \begin{figure}
%   {\it han sitt hus} \\
%   his its house
% ``his house'' \\
% \caption{Pronominal genitive reflexive construction}
% \label{genitive-reflexive-pronoun-example}
% \end{figure}

% However, this construction is allowed in NORTHERN SWEDEN.
% Oops, actually I think this whole section is whole throwaway intro to
% something else. Boooooo.

% However, this is not allowed with possessive pronouns:
% *{\it han sitt hus}.

% The prepositional genitive
% behaves the same way: {\it huset till Per} ``Per's
% house'' (gloss: house-the of Per) is legal but *{\it huset till meg}
% ``my house'' (gloss: house-the of me) is not.

% There is an exception for kinship words, which I don't understand
% yet. But somehow ``far min'' is different (maybe just because it's not
% ``min far''?)

% So basically standard Swedish allows trigrams sequences like NN-PO-NN
% ({\it Per sitt hus}) but not PO-PO-NN ({\it han sitt hus}). It also
% allows sequence like NN-PR-NN ({\it huset till Per}) but not NN-PR-PO
% ({\it huset till meg}).

% Does not work (is too close to call): 0.02229 vs 0.2429

% Reversing the bigram, looking for PO-NN in the south gives
% Works (but is still super close): 0.04243 vs 0.03998

% It looks like one set just uses more nouns than the others or
% something. Conclusion: inconclusive, leaning toward no---it looks like
% they're the same.

\subsubsection{Proper Noun Possessives}

In addtion to the post-nominal possessive pattern of the previous
section, there is a variant that is common in Norway. Here, the
sequence is noun-possessive pronoun-proper noun. An example
of this pattern is given in figure \ref{proper-noun-post-possessive}.

\begin{figure}
  {\it Huset hans Per} \\
  The-house his Per \\
  Per's house
  \caption{Possessive formed of Possessive Pronoun and Proper Noun}
  \label{proper-noun-post-possessive}
\end{figure}

This pattern overlaps slightly into Sweden, covering the interview
site K\"ola. The distribution is given in map
\ref{proper-noun-post-possessive-map}. Note that the northern area
with small stripes is the same as in map
\ref{possessive-plus-article-map}. The area of interest is the first,
with larger, thick stripes.

\begin{figure}
  \includegraphics[scale=0.7]{dialektboka-karta6}
  \caption{Proper-Noun Possessives}
  \label{proper-noun-post-possessive-map}
\end{figure}

This phenomenon maps to a trigram NN-PO-PN: noun-pronoun-proper
noun. The occurrence rate of this trigram in K\"ola to the rest of
Sweden is 0 vs 0.00001. This is the wrong direction, and the value is
so low that it is probably noise. There are two possible causes for
this essentially zero result: either neither region has this feature
or there is not sufficient data to tell.

\subsubsection{Noun possessives}

Delsing mentions briefly that central Sweden, including \"Alvdalen and
V\"asterdalarna, uses the dative form of nouns for the
s-genitive. However, the part-of-speech tag set used here makes no
difference between dative and other cases on nouns, so it is not
possible to look for this phenomenon.

\subsubsection{Double indefinite}

In northern Sweden and northern Norway, indefinite articles are used both
before and after adjectives when modifying nouns. Delsing calls this
the ``double indefinite''. See for example figure
\ref{double-indefinite-example}. One indefinite article is used after
each adjective, even for multiple adjectives, so {\it en stor en bil}
(a large car) but also {\it en stor en fin en bil} (a large fine car).

\begin{figure}
  {\it en stor en bil} \\
  a large a car \\
  A large car
  \caption{Double Indefinite}
  \label{double-indefinite-example}
\end{figure}

\begin{figure}
  \includegraphics[scale=0.7]{dialektboka-karta8}
  \caption{Double indefinite (post-adjectival articles)}
  \label{double-indefinite-map}
\end{figure}

In central Sweden, a similar pattern occurs, but the article is not
perceived as independent, but as a suffix of the adjective. In other
words, the above example is perceived as {\it en stor-en bil}
instead. Unfortunately, this pattern appears identical to the ordinary
Swedish case given the course part-of-speech tag set in use.
In constrast, the first pattern is quite easy to represent with
trigrams: we are interested in the 4-gram EN-AJ-EN-NN and the 6-gram
EN-AJ-EN-AJ-EN-NN---alternating series of indefinite articles and
adjectives ended by a noun. The larger n-grams can be broken into the
trigrams EN-AJ-EN and AJ-EN-NN.

This is a northern pattern which includes the interview sites
Anundsj\"o and Indal. When measured, these trigrams occur at a rate of
0.00054 there versus the rest of Sweden, which has a rate of
0.00012. From this we can conclude that this is a rare phenomenon, but
one that happens in the north about 4 times more often than in the
rest of Sweden.

\subsubsection{Double Definite}

Double-definite with adjectives is standard in Sweden and Norway,
where there is a definite article as well as a definite suffix on the
noun (see figure \ref{double-definite-example}). This is not the case
in Denmark (figure \ref{single-definite-example}), where the definite
suffix disappears in case of a definite article, nor in Iceland, where
the definite is suffix-only and there is no article (figure
\ref{single-definite-suffix-example}).

However, in North Sweden, there is a fourth option, where the
adjective combines with the noun into a single word (figure
\ref{adjective-single-definite-suffix-example}). Delsing gives
examples like {\it storhuset} (the big house) and {\it
  stor-svart-gamm-katta} (the big, black, old cat). In Norrland,
Delsing finds that this construction is used almost to the exclusion
of the normal Swedish one. Further south, the two co-exist.

\begin{figure}
  {\it det store huset} \\
  The large the-house \\
  The large house
  \caption{Double definite (Sweden and Norway)}
  \label{double-definite-example}
\end{figure}
\begin{figure}
  {\it Det store hus} \\
  The large house\\
  The large house
  \caption{Single Indefinite (Denmark)}
  \label{single-definite-example}
\end{figure}
\begin{figure}
  {\it gamla h\'usid} \\
  old house-the \\
  The old house
  \caption{Single definite suffix (Iceland)}
  \label{single-definite-suffix-example}
\end{figure}
\begin{figure}
  {\it storhuset} \\
  old-house-the \\
  The old house
  \caption{Single definite suffix with combined adjective (Northern Sweden)}
  \label{adjective-single-definite-suffix-example}
\end{figure}

Therefore, since the annotation scheme does not differentiate between
a combined noun like {\it storhuset} and a normal noun like {\it
  huset}, the better way to detect the region difference is to count
the rate of normal trigrams like PO-AJ-NN (pronoun-adjective-noun);
this is the feature type that occurs rarely or not at all in the
north. If the region division in map \ref{double-definite-map} is
detected, then northern Sweden will have a lower rate of occurrence of
these standard trigrams.

As before, the two northern sites are Indal and Anundsj\"o. The rate
of PO-AJ-NN in this region is 0.00152, compared to 0.00216 for the
rest of Sweden. This difference is in the right direction, and it is
larger than most of the other comparisons here, but of course it has
not been checked for significance so it is currently just suggestive.

\begin{figure}
  \includegraphics[scale=0.7]{dialektboka-karta9}
  \caption{Double definite (and combined adjectives)}
  \label{double-definite-map}
\end{figure}

\subsection{Rosenkvist's Analysis of the South Swedish Apparent Cleft}

\namecite{rosenkvist07} analyses a phenomenon he calls the South
Swedish Apparent Cleft. It involves an embedded clause, similar to a
cleft, but with no clefted constituent. Instead, subordinating
conjunction {\it som} is directly preceded by the verb or an adverb
expressing speaker attitude. The subject of of the {\it som}-clause
must be a pronoun, though Rosenkvist notes that this may be a
pragmatic, not a syntactic, restriction. The two main variants are
given in figures \ref{apparent-cleft-example1} and
\ref{apparent-cleft-example2}, but is also found in yes/no questions
and embedded clauses.

\begin{figure}
  {\it Det \"ar som han har missuppfattat.} \\
  it is {\it som} he has misunderstood \\
  He has misunderstood.
  \caption{Apparent Cleft}
  \label{apparent-cleft-example1}
\end{figure}

\begin{figure}
  {\it Det \"ar bara som han finner p\.a.} \\
  it is only {\it som} he finds-on \\
  He just makes it up.
  \caption{Apparent Cleft with adverb expressing speaker attitude}
  \label{apparent-cleft-example2}
\end{figure}

Unfortunately, Rosenkvist does not give a comprehensive syntactic
analysis of the apparent cleft. This means that a translation to our
feature set based on his description will necessarily be
surface-oriented in the same way this his analysis and results are
surface-oriented.

Accordingly, translating the sequences like {\it Det \"ar som han
  \ldots} gives the 4-gram PO-AV-UK-PO, and {\it Det \"ar bara som han
  \ldots} gives the 5-gram PO-AV-AB-UK-PO (pronoun-be
verb-adverb-subordinating conjunction-pronoun). Although these
part-of-speech sequences can obviously appear in other contexts, they
should appear more in the region that has apparent clefts than in the
region that does not. Converting these sequences to trigrams is
straightforward, producing 5 unique trigrams of interest.

Rosenkvist captures the geographical distribution of the apparent
cleft in two ways. He first consults two collections of Swedish
novels, using the authors' birthplaces as proxies for their
dialect. Second, he uses the results of a questionnaire that he issued
to university students at several Swedish universities: Stockholm,
Gothenburg, Lund and Ume\.a.

Using author birthplace as a proxy for dialect, the apparent cleft is
seen throughout southern and middle Sweden---this includes all the
interview sites except \.Arsunda, Indal and Anundsj\"o. However, based
on the survey results, the apparent cleft is only accepted by speakers
from Halland, Sm\.aland and Sk\.ane. This includes the interview sites
Frilles\.as, V\.axtorp, Ankarsrum, Tors\.as, Bara, L\"oderup, Norra
Rorum and \"Ossj\"o.

Therefore, the test for this comparison is the occurrence rates
for the 5 trigrams based on the two common forms Rosenkvist gives as
examples, with two variations: one region division based on author
birthplaces and one region division based on the student survey. The
southern region in both cases should have more occurrences of the
target trigrams.

For the larger cleft region division based on author birthplaces, the
comparison goes in the expected direction: a rate of 0.02430 in the
south and 0.02427 in the north. But these rates are so close identical
that they should not be regarded as different. For the smaller divison
based on the student survey, the comparison goes in the opposite
direction: 0.02264 in the south and 0.02491 in the north. Again, this
is not much of a difference.

With such a small difference, it is not possible to draw any
conclusions or even suggest whether the distance measures will consistently
notice this difference. One problem is that it hard to capture a
phenomenon like this with trigrams, where the surface form is only
subtly different from that produced by other syntactic structures. A
more complete syntactic analysis of the phenomenon is needed so that
more advanced feature sets from dialectometry can be used to compare
to the results from dialectology.

\subsection{Conclusion}

The dialect constructions surveyed here do not support the agreement
of the new dialectometry results with existing dialectology results
nearly as well as the previous sections which compared the results at
a less detailed level. The larger problem is that no good method yet
exists for doing so; the differences were in some cases large enough
to suggest directions, but without significance testing, it is not
possible to know. In fact, it is possible that the small differences
are significant, and already being used by the distance measures to
distinguish regions; after all, the aggregation of many small
differences is the inherent in the working of the statistical approach
in this dissertation.

\section{Comparison to Phonological Dialectometry}

The comparison to phonological dialectometry is currently difficult
because there has been so little work to date on Swedish. The only
paper at the time of this writing is \namecite{leinonen08}. It uses
factor analysis to characterise the distribution of nine phonological
variables across Sweden, but does not cluster the sites based on these
variables. In other words, it is equivalent to the analysis of feature
rankings presented in chapter \ref{results-chapter}. Since the
syntactic and phonological features can not be compared directly, I will
attempt to compare Leinonen's individual feature maps to my composite
cluster and MDS maps. This is still a comparison of maps based on
single features rather than aggregrates, so it must be taken as
preliminary.

However, Leinonen's dissertation, currently unpublished, will cover
phonological dialectometry of Sweden comprehensively. In future work,
a better comparison should be possible, since both dissertations are
based on the same corpus.

Looking at Leinonen's first two maps, reproduced here as figure
\ref{leinonen-factors-1-2}, we see patterns similar to the
city/country difference from the syntactic results: in the first diagram,
Stockholm and Uppsala differ from the rest of the country, and in the
second Stockholm, Uppsala and Malm\"o areas all differ.

\begin{figure}
  \includegraphics[scale=0.4]{leinonen-factors-1-2}
  \caption{Factors 1 and 2 of Swedish vowels}
  \label{leinonen-factors-1-2}
\end{figure}

In Leinonen's third and fouth maps (figure \ref{leinonen-factors-3-4}),
there is a north/south divide roughly half way between Stockholm and
Malm\"o. This boundary generally reflects the north/south
gradient from my results. However, the phonological boundary is stronger and more
localised than numerous small syntactic ones, such as those seen in
the composite cluster map \ref{map-composite-5-1000}.

\begin{figure}
  \includegraphics[scale=0.4]{leinonen-factors-3-4}
  \caption{Factors 3 and 4 of Swedish vowels}
  \label{leinonen-factors-3-4}
\end{figure}


The fifth map (figure \ref{leinonen-factors-5-6}) is more specific
than the previous four; most of the sites are blue, but there are a
few in the south that are much yellower than the rest. These are the
same three sites that form the red cluster in
figure \label{red-cluster} from the consensus tree
results: J\"amshog, \"Ossj\"o and Tors\.as. The sixth map, however,
shows a clear east/west divide that is not reflected in my data.

\begin{figure}
  \includegraphics[scale=0.4]{leinonen-factors-5-6}
  \caption{Factors 5 and 6 of Swedish vowels}
  \label{leinonen-factors-5-6}
\end{figure}

The level of agreement between the phonological results and syntactic
results is quite high. Of the six variables Leinonen illustrates with
map, all but one reflect some aspect of the combined syntactic
results. The precision with which Leinonen's fifth variable matches
the red cluster from the consensus tree results is surprising for
statistical methods. This discussion is not precise, but it provides
hope that a quantitative comparison between the two result sets will
support high agreement with statistical evidence.

\section{Comparison to Syntactic Dialectometry}

In previous work on British English, this method failed to find
agreement between syntactic distance ($R$) and phonological distance
(Levenshtein) distance---there was no
significant correlation between the two methods. Although both showed
something like a North/South distinction in Britain, its orientation was much more
obvious from phonological distance. This lack of agreement was a
preliminary answer to the question of whether multiple
ways of measuring linguistic distance give the same results.

However, there were at least five reasonable explanations for the difference
between the two distance measures.
% First, and
% least satisfying, is the possibility that one of the distances is not
% measuring what it is supposed to. Second, the corpora may not agree
% because of the 40 year difference in age and differing collection
% methodologies. Third, syntactic and phonological dialect markers may
% not share the same boundaries.

\begin{enumerate}
\item One or both of the distances does not measure what it is supposed to.
\item The two corpora may not agree on dialect boundaries because of
  their 40-year difference in age.
\item Place of birth, as recorded in the ICE, may not correlate well
  with spoken dialect, especially given variations in speaker
  education level and place of residence.
\item Dialect boundaries may appear from systematic variation in
  annotation practices rather than the speech.
\item Syntactic and phonological dialect boundaries may be different.
\end{enumerate}

Of these, this dissertation addresses the second, third and fourth
problems directly by using a single corpus, Swedia2000, annotated by a single
person. (TODO: This may not be true for the phonological annotation.)
By finding significant distances between all interview sites of
Swedia2000, it also suggests that $R$ is measuring syntax
distance. PROBABLY.

The last is the most interesting because previous work will not have
exposed this difference. Traditional dialectometry focuses on a strong
agreement among a few features from each collection site. Because
syntactic features are fewer in number than phonological ones, they
are under-represented in this type of analysis. Unfortunately, this
means that the syntactic contribution to isogloss bundles is
correspondingly reduced. In addition, because of isogloss bundles'
insensitivity to rare variations, syntactic features rarely contribute
to isogloss bundles of successful dialect boundaries.
% I really need to CITE this.

In contrast, computational analysis, such as \cite{shackleton07},
captures feature variation precisely using statistical analysis and
sophisticated algorithms. The resulting analysis displays dialects as
gradient phenomena, displaying much more complexity than the
corresponding isogloss analysis. But current specialized computational
methods only apply to phonology. Syntactic data cannot be analyzed
without a syntax-specific method.

This paper attempts to address that lack, and provide some first steps
to show whether syntax and phonology assist each other in establishing
dialects, or whether their dialect regions are unrelated. If they are not
related, and syntactic gradients can be as weak as phonological ones,
then some new dialect regions may become apparent that were not visible in
previous phonology-only analyses.

\subsection{Improvements on British Dialect Experiment}

This dissertation improves on the British experiment in a number of
ways. It addresses the obvious criticism that syntax distance on the
ICE requires so much data that the results are no more informative
syntax those of traditional dialectology---its precision lags
phonological distance methods badly. However, $R$ works with much
smaller corpora when run on Swedia2000. This shows that the problem
with the British experiment is not the distance method, but the
corpus, which fails to capture dialect differences. Most likely is
that the interviewees, mostly in a college setting, actively tried to
suppress dialect differences during the interview.

Another problem with the current study is the 40-year difference in
collection dates between the phonological corpus and the syntactic
corpus. A recent phonological corpus would likely show the same sort
of changes in the North/South divide that show up in the syntactic
corpus. The British population became more mobile during the second
half of the 20th century, and the SED survey explicitly attempted to capture
the dialects that existed before this happened \cite{orton78}.
It would also be nice to have data from the rest of the United Kingdom for
comparison as well, or at least Scotland and Wales as with the ICE.


% TODO: integrate this.
% Alternatively, I could just look at the region pairs that fail to
% achieve significance in the syntactic permutation test and check to
% see if their phonological distance is lower than the other pairs. I
% don't do this (yet).

One interesting question is
how phonological and syntactic distances correlate with geographic
distance---\namecite{gooskens04a} shows that often the correlation is
very good. This would also allow better visualization of dialect areas
than a hierarchical dendrogram.

\section{Future Work}

Try all those smarter variants of $R$.

Include the rest of Nodalida once it is done.

Better normalisation and feature ranking are needed. It appears that
the current normalisations vary either in favouring differences
only in high-frequency features OR in favouring rare features so much
that the most important appear to be those that only occur in one of
the two regions.

Quantitative comparison to phonology. There is a dissertation out for
it now, based on the same data. So it should be easy.

Better comparison to dialectology.
NOTE for conclusions: But this has taken some time in phonology too. (at least 8-10
years) so this is not such a big problem.

Better parsers!

\section{The End}

This dissertation contributes a better understanding of syntax with
respect to dialectometry. It establishes that statistical methods can
find interesting things, and with not much more data than is expected
of previous dialectometry in other areas. Remember, the majoriy of
interviews used here were less than 1000 sentences. Previous work
pointed the way (Nerbonne \& Wiersma (2006) and Sanders (2008)) but
failed to establish the utility and reliability of these methods
either by lack of dialect application or by lack of consistent
results. This dissertation addresses these shortcomings
comprehensively.

In addition, it points the way toward future work in Swedish; while
the results here are interesting, it is difficult to corraborate them
solidly because of the lack of study on Swedish dialects, both in
dialectology and dialectometry. This gap in the literature is on its
way to being remedied with the work of Leinonen in dialectometry and
X,Y,Z in dialectology.

Like its findings, future directions based on this work are
twofold. In general dialectometry, syntactic investigations should
begin, hopefully extending to languages for which the syntactic
variation is already well-studied.

In Swedish, I hope that this investigation of syntactic dialect
variation will lead to further work in this area; there is little
enough right now, and perhaps a computer-generated overview of the
interesting features will spark some new avenues of investigation for
linguists.

%%%%%%%%% cut two (raw) %%%%%%%%%

This dissertation establishes that statistical methods for syntactic
dialectometry can be useful. The results show that significant
distances can be obtained, which was shown by earlier work and not
much else. They show that dialect distances correlate with geographic
distance for many parameter settings. Clustering dialect distances
reproduces two well-known aspects of Swedish dialects: the Scanian
border, and a north-to-south gradient, as well as a previously unremarked
aspect: differences between the cities and the
countryside. Mult-dimensional scaling produces the same
results. Finally, interesting features have been discovered from these
same clusters, but features in the dialectology literature do not
necessarily appear.

In summary, this dissertation has answered its questions with some
degree of accuracy. Although the more complete verification is still
missing, it points the way for practical studies in the future: many
parameter variations are explored and the most efficacious are pointed
out.

%%% Local Variables: 
%%% mode: latex
%%% TeX-master: "dissertation.tex"
%%% End: 


\backmatter

\begin{center}
\section*{Nathan C. Sanders}
\end{center}
% \begin{tabular}{c|c|c} % re-arrange and maybe put home page on there?
% Electronic Mail & Address & Telephone \\
% \hline
% ncsander@indiana.edu & 4780 E St. Rd. 45 & 812-606-9785 \\
% sanders\_n@yahoo.com & Bloomington, IN 47408  & \\
% \end{tabular}

\subsection*{QUALIFICATIONS}
\begin{description}
  \item[Dialect Classification/Dialectology] Extraction of
    linguistic, human-interpretable features
  \item[Machine learning] Statistical (e.g. unsupervised clustering) \\
    Symbolic (e.g. learning and learnability in Optimality theory)
  \item[Parsing] Adaptation of computer science parsers to learner
    natural language
  \item[Programming] Fluent in Python, Haskell, Java/C\#, and Scheme \\
    Experience in C++, Perl, Visual Basic, Javascript, F\#, and
    Common Lisp \\
    Experience in prototyping and web programming \\
    Three-time competitor, ACM International Collegiate Programming
    Contest
\end{description}
% See http://www.sandersn.com/research.html for papers
% and code in the above areas.
\subsection*{EDUCATION}
\begin{description}
\item [PhD student in Linguistics, minor Computer Science] Indiana
  University, 2010
\item [MA Computational Linguistics] Indiana University, 2006, GPA 3.98
\item[BA Computer Science] minors in French and Spanish,
		College of the Ozarks, summa cum laude May 2004
\end{description}
\subsection*{EXPERIENCE}
\begin{description}
\item[August 2009--November 2009]
  SDET Intern, Microsoft, Bing Search Infrastructure --- Wrote
  software to suggest tests to developers at checkin. Tested
  the distributed computation system forming the infrastructure of
  Bing.
\item[August 2005--May 2009]
  Research Assistant, IU School of Medicine --- Investigated linguistic
  aspects of cochlear implant user development. Developed novel distance
  measure and compared it to existing measures and human judgments.
\item[May 2005--August 2005]
  Application Developer, IU Archives of Traditional Music --- Wrote Java
  application to gather cataloging data for videos. Worked with librarians and
  ethnomusicologists to gather archival metadata requirements.
\item[August 2004--May 2005]
  Web Programmer, IU Overseas Studies --- Maintained database and its
  web interface. Developed new features on request and maintained
  office computers.
\item[Summer 2004, Summer 2003]
  Lead Programmer, Everyware Inc --- Implemented web framework
  for interactive site building. Rewrote existing plug-in
  architecture and extended existing interface.
\item[Summer 2002]
  Intern Programmer, SIL International --- Programmed import
  process for translation editor in C++, with a pre-processor in Python.
\item[Spring 2001--Spring 2004]
  Lab Assistant, College of the Ozarks Foreign Language Lab ---
  Tutored students in Spanish and French. Maintained lab
  computers and created tracking databases.
\end{description}
\subsection*{PUBLICATIONS}

% no idea why I have to indent *twice*
\indent\indent Sanders, N. C. 2009. Phonological distance measures. \emph{Journal
    of Quantitative Linguistics}, 43:96--114.

  Sanders, N. C. 2008. Cluster analysis of phonological distance
  measures of cochlear implant users. In \emph{Proceedings of the
    Tenth International Conference on Cochlear Implants and Other
    Implantable Auditory Technologies}, 113.

 Sanders, N. C. 2007. Measuring Syntactic Difference in British
  English. In \emph{Proceeding of the ACL 2007 Student Research
    Workshop}, 1--6, Prague, Czech Republic, June.

  Sanders, N. C. and Chin, S. B. 2006. Phonological distance measures for cochlear implant users. \emph{Wiener Medizinische Wochenschrift 156}, [Suppl 119] 8

 Sanders, N. C.  2004. Compiler error detection techniques applied to natural language processing. In \emph{Proceedings of the Thirty-fifth SIGCSE Technical Symposium on Computer Science Education}. Association for Computing Machinery, 515

\bibliographystyle{robbib}
\bibliography{central}
\end{document}
