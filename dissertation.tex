\documentclass{iuthesis}
% macros that I like go here.
\usepackage{graphicx}
\usepackage[all]{xy}
\usepackage{qtree}
\usepackage{robbib}
\includeonly{background,questions,methods,results,discussion}
\title{Syntactic Distance for Dialectometry}
\author{Nathan C. Sanders}
\advisor{Sandra K\"ubler}
\secondreader{Markus Dickinson}
\thirdreader{Steven Franks}
\fourthreader{Michael Gasser}

\department{Linguistics}
\submitdate{10/05/01} % a hopeful guess
\copyrightyear{2010}
\begin{document}

\begin{dedication}
  It's a great day for freedom
\end{dedication}
\begin{acknowledgements}
  I always thought Heeringa's acknowledgements were
  pretty cool
\end{acknowledgements}
\begin{abstract}
  Syntactic distance measures for dialectometry; tested on
  Swedish. It's all right.
\end{abstract}

\frontmatter
\maketitle
\signaturepage
\copyrightpage
\makededication
\makeack
\makeabstract
\tableofcontents
\mainmatter

\chapter{Introduction}
% TODO: Probably just copy the abstract+conclusion in here and merge
% the two...
This dissertation examines syntax distance in dialectometry using
computational methods as a basis. It is a continuation of my previous
work \cite{sanders07}, \cite{sanders08b} and earlier work by
\namecite{nerbonne06}, the first computational measure of syntax
distance. Dialectometry has existed as a field since
\namecite{seguy73} and is a sub-field of dialectology
\cite{chambers98}; recently, computational methods have come to
dominate dialectometry, but they are limited in focus compared to
previous work; most have explored phonological distance only, while
earlier methods integrated phonological, lexical, and syntactic data.

Dialectology is the study of linguistic variation. % in space / over
% distance / other variables.
Its goal is to characterize the linguistic features that
separate two language varieties. Dialectometry is a subfield of
dialectology that uses mathematically sophisticated methods to extract
and combine linguistic features. In recent years it has
been associated with computational linguistic work, most of which
has focused on phonology, starting with
\namecite{kessler95}, followed by \namecite{nerbonne97} and
\namecite{nerbonne01}. \namecite{heeringa04} provides a comprehensive
review of phonological distance in dialectometry as well as some new
methods.

In dialectometry, a distance measure can be defined in two parts:
first, a method of decomposing the linguistic data into minimal,
linguistically meaningful features, and second, a method of combining
the features in a mathematically and linguistically sound way. Figure
\ref{abstract-distance-measure-model} gives an overview of how the
model works. Input consists of two corpora; each item in each corpus
is decomposed into a set of features extracted by $f(s)$. The
resulting corpora are then compared by $d(S,T)$, which combines the
corpora into a single number: the distance.

\begin{figure}
\[\xymatrix@C=1pc{
 \textrm{Corpus} \ar@{>}[d]|{f(s)} &
  S = s_o,s_1,\ldots
  \ar@{>}[d] % \ar@<2ex>[d] \ar@<-2ex>[d]
  &&
  T = t_o,t_1,\ldots
  \ar@{>}[d] % \ar@<2ex>[d] \ar@<-2ex>[d]
  \\
 *\txt{Decomposition} \ar@{>}[d]|{d(S,T)} &
 *{\begin{array}{c}
     \left[ + f_o, +f_1 \ldots \right], \\
     \left[ - f_o, +f_1 \ldots \right], \\
     \ldots \\ \end{array}}
 \ar@{>}[dr]
 &&
 *{\begin{array}{c}
     \left[ + f_o, -f_1 \ldots \right], \\
     \left[ + f_o, -f_1 \ldots \right], \\
     \ldots \\ \end{array}}
 \ar@{>}[dl]  \\
 \textrm{Combination} &
 & \textrm{Distance} & \\
} \]
\label{abstract-distance-measure-model}
\caption{Abstract Distance Measure Model : $f \circ d$}
\end{figure}

Dialectometry has focused on phonological distance measures, while
syntactic measures have remained undeveloped. The most important
reason for this focus is that it is easier to define a distance
measure on phonology. In phonology, it is easy to collect corpora
consisting of identical word sets. Then these words decompose to segments and,
if necessary, segments further decompose to phonological
features. This decomposition is straightforward and based on
\namecite{chomsky68}. For combination, string alignment, or Levenshtein
distance \cite{lev65}, is a well-understood algorithm used for
measuring changes between any two sequences of characters taken from a
common alphabet. Levenshtein distance is simple mathematically, and
has the additional advantage that its intermediate data structures are
easy to interpret as the linguistic processes of epenthesis, deletion and
metathesis. These things are not possible
with syntactic distance: neither matched sentence collections nor
straightfoward functions for decomposition and combinations.

A secondary reason for dialectometry's focus on phonology is that it
is inherited from dialectology's focus on phonology.
% (TODO:Cite?)
This might be solely due to the history of dialectology as a field, but it is
likely that more phonological than syntactic differences exist between
dialects, due to historically greater standardization
of syntax via the written form of language. Phonological
dialect features are less likely to be stigmatized and suppressed by a
standard dialect than syntactic ones.
% (TODO:Cite, probably
% Trudgill and Chambers something like '98, maybe where they talk about
% what aspects of dialects are noticed and stigmatized).
Whatever the reason, much less dialectology work on syntax is
available for comparison with new dialectometry results.

\subsection{Problems}

Because of the preceding two reasons, syntax is a relatively
undeveloped area in dialectometry. Currently, the literature lacks a
generally accepted syntax measure. Unfortunately, approaching the
problem by copying phonology is not a good solution; there are real
differences between syntax and phonology that mean phonological
approaches do not apply. For example, there are fewer differences to be
found in syntax, and they occur more sparsely.
% (TODO: Back this up either with reasoning or citation).
However, dialectology has traditionally worked with fairly small
corpora. This suffices for phonology, because
it is easy to extract good features and there are many
consistent differences between corpora. For syntax, though, it is not possible
% (TODO: Weasel a bit)
to identify reliable features in small corpora.

There are two approaches that have been proposed to remedy this. The
first, proposed by \namecite{spruit08} for analyzing the Syntactic
Atlas of the Dutch Dialects \cite{barbiers05}, is to continue using
small dialectology corpora and manually extract features so that only
the most salient features are used. Then a sophisticated method of
combination such as Goebl's Weighted Identity Value (WIV), described
below and by \namecite{goebl06}, can be used to produce a
distance. WIV is more complex mathematically than Levenshtein
distance, and operates on any type of linguistic feature. However, manual feature
extraction is not feasible in knowledge-poor or time-constrained
environments. It is also subject to bias from the
dialectologist. Since the best manually extracted features are those that capture
the difference between two dialects, the best-known features are most
likely to become the best manual features, passing over the rarely
occurring and obscure features that might actually be the best
indicators of a particular dialect.

This approach ignores the specific properties of the syntax distance
problem. It is easy to define features for syntactic structure. This
proposal covers part-of-speech trigrams, leaf-ancestor paths, and
dependency paths over nodes, but many variations on these features are
possible, such as lexical trigrams, lexicalized leaf-ancestor paths,
or dependency paths over dependency arc labels. Methods from other
syntactic work in computational linguistics could apply too: supertags
\cite{joshi94}, convolution kernels \cite{collins01} or any number of
simpler features such as tree height, number of nodes, or number of
words. The problem is not finding a feature set. The problem is
finding a good feature set. Small corpora hamper this search by making
statistical significance difficult to achieve, especially since
syntactic dialect differences are expected to be less frequent than
phonological ones. Fortunately, syntactic corpora are typically larger
than phonological corpora because the annotation work is easier; much
of the syntactic annotation can be generated automatically and then
corrected manually.

Even with a feature set defined, a distance measure still requires a
method of combining features. One such method, a simple statistical
measure called $R$, has been proposed by \namecite{nerbonne06} based
on work by \namecite{kessler01}. At present, however, $R$ has not been
adequately shown to detect dialect differences. A small body of work
suggests that it does, but as yet there has not been a satisfying
correlation of its results with phonology or, as with phonological
distance, with existing results from the dialectology literature on
syntax.

Nerbonne \& Wiersma's first paper used part-of-speech trigram features
as a proxy for syntactic information and $R$ for syntax distance
together with a test for statistical significance\cite{nerbonne06}.
Their experiment compared two generations of Norwegian L2 speakers of
English.  They found that the two generations were significantly
different, although they had to normalize the trigram counts to
account for differences in sentence length and complexity. However,
showing that two generations of speakers are significantly different
with respect to $R$ does not necessarily imply that the same will be
true for other types of language varieties. Specifically, for this
dissertation, the success of $R$ on generational differences does not
imply success on dialect differences.

I addressed this problem \cite{sanders08b} by measuring $R$ between
the nine Government Office Regions of England, using the International
Corpus of English Great Britain \cite{nelson02}. Speakers were classified by
birthplace. I also introduced Sampson's leaf-ancestor paths as
features \cite{sampson00}. I found statistically
significant differences between most corpora, using both trigrams and
leaf-ancestor paths as features. However, $R$'s distances were not
significantly correlated with Levenshtein distances. Nor did I
show any qualitative similarities between known syntactic dialect
features and the high-ranked features used by $R$ in producing its
distance. As a result, it is not clear whether the significant $R$ distances
correlate with dialectometric phonological distance or with known
features found by dialectologists.

% NOTE: 2-d stuff is not the primary problem, since we can't compare
% trees to trees anyway. The primary problem is comparing two corpora
% full of differing sentences. A secondary problem arises to make sure
% that the 2-d-extracted features aren't skewed one way or another. I
% guess I need to come up with a general justification for the
% normalizing and smoothing code from Nerbonne & Wiersma

% Additional problems: phonology is 1-dimensional, with one obvious way
% to decompose words into segments and segments into features. Syntax is
% 2-dimensional, so the decomposition must take several more factors
% into account so that the features it produces are
% useful and comparable to each other. And those features are \ldots

% % TODO Henrik Rosenkvist seems to
% % be the main guy interested in syntactic analysis of dialect distance

% Overview : Goal, Variables, Method
%   Contribution
% Literature Review
%   : (including theoretical background)
%   Draw hypotheses from earlier studies
% Method
%   :
%   Experiment section as 'Corpus' section

% Goal: To extend existing measurement methods. To measure them
% better. To measure them on more complete data.

\section{Dissertation Overview}

How I will solve these problems and test these hypotheses.

Blah blah a few sentences on methods: $R$ and feature extraction and
analysis blah blah.

As regards to Swedish: I use the Swediasyn, Swedish section, to test
the hypotheses on Swedish. [[Describes Swediasyn]]
There has been some syntactic dialectology on
Swedish. [[So far I know of a couple of papers by Rosenkvist.]] It is
not clear whether syntactic features follow the same boundaries as
other features. [[There isn't a whole lot of literature on any
boundaries actually.]]

%%% Local Variables: 
%%% mode: latex
%%% TeX-master: "dissertation.tex"
%%% End: 


\chapter{Questions}
% TODO: Rewrite and merge the following question/hypothesis paragraph pairs
% H1 - organization is all wrong still
The state of syntax measures in dialectometry described above leaves
several research questions unresolved. The most important for this
proposal is whether $R$ is a good measure of syntax
distance. Specifically, have the ambiguous results of previous
research been a shortcoming of $R$, differences between phonological
and syntactic corpora, or differences between phonological and
syntactic dialect boundaries?

To investigate this, I propose Hypothesis 1: the features found by
dialectologists will agree with the highly ranked features used by $R$
for classification. I will test Hypothesis 1 by comparing $R$'s
results to the syntactic dialectology literature on Swedish. In
addition, Hypothesis 1B states that the regions of Sweden accepted by
dialectology will be reproduced by $R$. For example, my
previous research on British English reproduced the well-known North
England-South England dialect regions. However, this research will eliminate the
corpus variability in that research \cite{sanders08b} that resulted in
the confounding factors mentioned above, meaning that more precise
results, such as specific identifying features, should be detectable as well.

Finally, if $R$ is found to be a bad measure of syntax distance, this
dissertation will propose and evaluate alternative syntactic distance
measures. Specifically, $R$ is one way to aggregate features that are
created by decomposing sentences. It treats features as atomic, and
does not manipulate them in any syntax-specific ways. As such, $R$ is
not much different than Goebl's WIV. This may not be a problem if the
decomposition methods used to generate features adequately capture
dialect differences in independent, atomic features. If dialect
differences cannot be captured by independent, atomic features, then a
more syntax-specific method of combining features will be needed
instead. Alternatively, a more complex statistical measure may be
useful, taking the basic idea of $R$ and increasing its
sensitivity. For example, Kullbeck-Leibler divergence, like $R$,
provides a dissimilarity that is intuitively similar to distance.

%H2 - Dad didn't understand that this is other features to be fed into
%R not replacement of R entire.
A secondary question, relevant once a useful syntax distance measure
is established, is what input features cause $R$ to produce the best
results.  Previous work has shown that leaf-ancestor paths provide a
small advantage over part-of-speech trigrams, presumably by capturing
syntactic structure higher in the parse tree. Additional possible
feature sets include variations on the previously investigated
trigrams and leaf-ancestor paths, along with various kinds of backoff,
for example, to bigrams or coarser node tags. Features from dependency
parses may be useful, too, in capturing non-local dependencies that
can be captured neither by trigrams nor leaf-ancestor paths.

Therefore, I propose Hypothesis 2: better input features
for $R$ will produce more accurate syntax
distances. These features can be discovered by comparing performance
of a number of different feature sets on a fixed corpus. In addition,
combinations of successful features will produce even better
performance.
% This sentence is either redundant or should appear earlier.
The quality of a set of features can be
measured by its sensitivity---the number of significant distances it
finds---and the similarity of the highly ranked features $R$ produces
to those found by dialectologists.

\section{Question 1 : Agreement with Dialectology}

At a high level, the first question is whether $R$, as a measure from
dialectometry, agrees with dialectology. On closer inspection, this
question covers a number of more specific questions, each dealing with
a specific comparison to dialectology. The first is whether the
features that $R$ counts most important are the same as the features
discussed in the dialectology literature. The other questions are
whether boundaries, regions and distances found by $R$ agree with
dialectology.

Feature agreement is the most important and most difficult
question. It is the most important for two reasons. First, it offers
the most precise explanation and is the easiest to disprove. Second,
dialectology starts with individual features, so the most work for
comparison is available at the feature level.

\subsection{Dialectology}

Definition of terms from dialectology is appropriate here, along with
an explanation of how they fit together. The basic unit in
dialectology is the feature, which corresponds to a feature given to
$R$. During analysis, the linguist may suspect that a feature is
characteristic of a particular region, but more information, usually
from a survey, is needed to make certain.

Given a survey or other source of geographical mapping information, a
boundary for a feature can be delineated. This boundary is called an
isogloss. With the right information, isoglosses are usually simple to
determine. One complicated case has a few occurrences of a feature
variant stranded in the middle of the other variant. This complicates
the geometry of the isogloss.

If a number of isoglosses coincide, they form an isogloss bundle,
which separates one region from another. Isogloss bundles are simple
in theory, but in practise they are difficult to find because
isoglosses rarely coincide perfectly. In practise, undisputed isogloss
bundles only occur between well-known dialects, such as the
boundaries between Low and High German or Northern and Southern
English of England. In cases where more precision is required, there
is not usually a sufficient number of coincident isoglosses. Even
though there may be plenty of isoglosses in the area, isoglosses so
rarely coincide that only a few may be construed as forming an
isogloss boundary.

Dialectology does not have a clear equivalent to dialectometry's
distance. The closest analogue is size of isogloss bundle; to reflect
this, dialect maps typically indicate size of isogloss bundle by
thickness of boundary line. Additionally, some regions are known to differ from
others much more than the rest of the regions. This usually mirrors
isogloss bundles, surfacing in the literature as a list of well-known
dialect features of a region.

\subsection{Features}

To match the features of dialectology to the features that $R$ uses to
produce a distance, I first need to find discussion of Swedish dialect
features in the dialectology literature. For example,
\namecite{rosenkvist07} discusses the South-Swedish apparent
cleft. This feature is best analysed as a single feature; Rosenkvist
mentions that it occurs in southern and central Swedish and not
northern Swedish, but does not give a more precise location than
``Svealand and G\"otaland''. So we will compare this feature directly
to features found in any part of the corpus.

Next I convert the feature to the formalism used for annotation.
Typically this requires conversion of a minimalist structure to a
simpler structure based on a phrase-structure grammar. This produces a
skeleton parse or parse fragment. Then the skeletal structure can be
broken into features in the same way that parses of corpus sentences
are. Once converted, the feature should appearch in the list of
important features for regions in southern Sweden.

In the apparent cleft example, the apparent cleft is realised as an
additional use of the word {\it som}, ordinarily a
complementiser. Typically, the next step is to identify the minimalist
structure for this, but Rosenkvist's 2007 paper does not yet provide
this analysis. Although there is no structure to translate to a
phrase-structure skeleton, his analysis provides enough clues to
produce some features directly. Trigrams are easiest; he mentions that
his corpus search used the strings {\it det \"ar som} (``It is that'')
and {\it det \"ar bara som} ``It's just that''. These words only
part-of-speech annotation to be trigram features. Dependency
paths can also use these parts of speech for the local dependencies
between {\it det,\"ar} and {\it som}. Rosenkvist also mentions
some syntactic properties of apparent clefts that are useful for
specifying dependency path features: the subject of the {\it
  som}-clause must be a pronoun, so we should expect to see
dependency paths of the form {\it ROOT-som-PRON} in the regions that
have the apparent cleft.

Once dialectometric features have been specified from some linguistic
analysis, the analysis consists of the following questions: in what
regions do these features appear? Do these regions match the expected
distribution (if any) from the linguistic analysis? How much do
the features contribute to distance from other regions? If there are
other features that contribute more, what are they?

TODO: Finish example by using structure from papers on the apparent
cleft (if it exists) in English and Japanese.

\subsection{Isogloss Boundaries}

Isogloss boundaries are intermediate in complexity between features unspecified
for location and regions demarcated by isogloss bundles. For the
purposes of this dissertation, however, there is not much difference
between a feature with some documented locations and an isogloss
boundary. An isogloss makes the regions of interest clearer, but it is
a difference in degree and not in quality. The real difference in
analysis occurs when dialectology has identified an isogloss bundle.

\subsection{Isogloss Bundles}

Isogloss bundles compare straightforwardly to dialectometry, insofar as
dialectometric methods produce regions. There are two primary ways for
identifying regions from these methods: hierarchical clustering and
multi-dimensional scaling. Neither method is perfect; as with isogloss
bundles, some human input is still needed to determine whether an
inter-region boundary truly exists at some point.

Hierarchical clustering
produces well-delineated regions, at the cost of some
uncertainty---the results tend to vary quite a bit from feature set to
feature set. Only clusters that persist between results from multiple
feature sets should be considered valid.

In contrast, multi-dimensional scaling (MDS) is a straightforward
mathematical transformation of the high-dimensional space created by
measuring distances between all regions in the corpus. Although MDS
does not produce spurious information, its results are often hard to
analyse because it produces boundaries of varying strength. Very
different regions stand out, but similar regions appear similar even
if they contain some differences. This similarity can make it
difficult to decide whether an area should be considered one region or
two.

\begin{figure}
  \includegraphics[width=0.32\textwidth]{Sverigekarta-Landskap-Swedia}
  \label{agree-clusters-small}
  \caption{Swedia, Clusters Common to All 3 Methods}
\end{figure}

\begin{figure}
  \includegraphics[width=0.5\textwidth]{Sverigekarta-Landskap-mds-dep}
  \label{mds-dep-small}
  \caption{Swedia, Multi-Dimensional Scaling of Depedency Path Distance}
\end{figure}

Once both dialectologic and dialectometric regions have been
identified, comparison is straightforward. Each region can be checked
for overlap---regions with a greater overlap area are better matches.

The real problem in identifying regions is that there has not been
enough dialectology work on Swedish syntax yet: numerous identified
isogloss bundles are indicative of a well understood area of
dialectology. In Swedish syntax, few regions are well-known and
well-documented enough to have isogloss bundles. It is possible that
dialectometry will lead the way in providing answers in this
area.

\subsection{Distances}

Although comparing distances from dialectometry to qualitative
research in dialectology is possible, it is unlikely to be useful in
this dissertation because of the previously mentioned lack of
developed Swedish dialectology in syntax. When it is possible,
one either looks for differences in size of isogloss bundle, or (more
weakly) statements like ``in general, Southern
Swedish is syntactically identical to Standard Swedish''
\cite{rosenkvist07} or ``there are numerous differences between
dialect X and the standard language''.

\section{Alternate Distance Measures}

Of the measures considered in this dissertation, $R$ is the
simplest---it's the sum of difference in feature counts. It is almost
the simplest statistical measure possible, given no knowledge of the
features to be used. Despite this, R appears to perform better, or
at least more consistently, than other measures that have been tested
in this dissertation. It gives significant results across a larger
variety of feature sets than more complicated measures do.

The question of measure is more important than feature set because
measures are harder to construct than feature sets, and even harder to
combine compared to feature sets. However, the distance measures
tested so far do not have a greater effect on significance also have a
greater effect on significance.

TODO: The introduction and justification here are worthless. Their
statements are false or irrelevant. So maybe this section as a whole
is worthless.

There are basically two directions to explore when creating an
distance measure to replace $R$. The first direction is to address
$R$'s simplicity by finding a more complex measure. The second
direction is to address $R$'s ignorance of syntax by finding a measure
with specific knowledge of syntax. The first direction is easier,
given the statistical bent of the corpus used here and the number of
statistical measures commonly used in computational
linguistics. A number of these measures are described in the Methods
chapter after the description of $R$.
Additionally, the statistical framework is powerful
enough that most syntax-specific knowledge can be represented in terms
of features instead of integrated into the distance measure's
algorithm.

TODO: ``Statistical'' is not quite the right word here.

Indeed, it is so difficult to think of syntax-specific algorithms that
none are presented in this dissertation. However, a number of more
complicated statistical measures are presented and some are tested. Goebl's
Weighted Identity Value (originally, Gewichteter Identit\"atswert) is
the earliest to be used in dialectometry, but others such as
Kullback-Leibler divergence and its variants are a good fit because of
their similarity to $R$.

\subsection{Syntax-specific distance measures}

As mentioned above, syntax-specific measures are more difficult to
specify---they must somehow incorporate knowledge of syntax in the
measure itself, in such a way that cannot be reified as features that
could then be processed by a generic statistical measure. This is
difficult when working with syntactic corpora; the amount of
information about each sentence is much lower than the amount of
information about each word in a traditional dialectology corpus of
phonology. Specifically, there is no alignment between sentences, nor is
there a predefined list of sentences to be elicited. In addition,
there is no well-accepted model equivalent the one set out by
\namecite{chomsky68}, which can be further augmented by later work in
phonology, such as \quotecite{goldsmith76} autosegmental
phonology. Dialectology work on phonology has relaxed these
assumptions, but switching to statistical analysis of unordered
corpora \cite{sanders06} and \cite{hinrichs07} has not
been as successful as abandoning distinctive features \cite{heeringa04}.

The hidden structure available for syntax is the parse---whether this
is a constituent parse, dependency parse or some variant of shallow
parse. Working by analogy from phonology, segments have hidden
structure in the form of distinctive features. Segment order in
phonology corresponds to word order in syntax. Unfortunately, as just
mentioned, the analogy does not extend to corpus order. However, there
is one piece of information that is not available to phonology (at
least pre-autosegmental phonology): the upper structure is
connected. For the current set of experiments, the upper structure is
processed, divided and assigned as features attached to an individual
word in the form of leaf-ancestor paths or dependency paths. This
approach allows the features to be given to $R$ and treated as if they
are independent, which is of course not true.

However, even taking advantage of this connected upper structure only
provides additional features per sentence. These features can be
profitably added to the others when using a statistical measure, but
they do not require a new measure.

\subsection{Important note on terms}

In the preceding section, there are several terms related to distance
in use. In order from least restrictive to most restrictive, they are
`divergence', `dissimilarity' and `distance'. In this dissertation, a
`measure' is any of these three functions. A divergence is a measure
that is not required to be either symmetric or to satisfy the triangle
inequality. A dissimilarity is symmetric, but is not required to
satisfy the triangle inequality. A distance satisfies both
properties. For this dissertation, the measure only needs to be a
dissimilarity; a true distance is not needed.

All three kinds of functions must always return positive numbers, and
only return 0 for corpora that are equal.  A symmetric function
returns the same number whether measuring from point X to point Y or
from point Y to point X. The triangle inequality means that distance
from point X to point Y plus point Y to point Z is at least as long as
travelling straight from point X to point Z.  If this is true, it
means that the function is measuring the same thing between all
corpora and therefore that the shortest path between two points can be
shortened by substituting a path with fewer intermediate
points. Equations
\ref{distance-properties-positive}-\ref{distance-properties-triangle}
list the properties formally.

\begin{equation}
  d(x,y) \ge 0
  \label{distance-properties-positive}
\end{equation}

\begin{equation}
 d(x,y) = 0 \textrm{ iff } x=y
\end{equation}

\begin{equation}
  d(x,y) = d(y,x)
\end{equation}

\begin{equation}
  d(x,y) + d(y,z) \ge d(x,z)
\label{distance-properties-triangle}
\end{equation}

\section{Question 2: Best Features}

The second question of this dissertation reflects the fact that the
overall question of syntax distance for dialectometry has two
parts. The first question deals with whether a distance measure like
$R$ can be found that works with features extracted from a
large unaligned corpus. Therefore, the second question deals with the
feature sets used as input: how to specify them, how to generate them,
and how to evaluate them.

TODO:DONE to here

This is an easier question, because feature sets are easier to dream
up. They are easier to combine (although I haven't done so yet.)
They are easier to tweak slightly for a great improvement in
results. The real problem with variation of feature sets is that there
no standard from syntax that can be easily adapted to
dialectometry. Phonology has order (mentioned above) and distinctive
features.

Previous work has shown that leaf-ancestor paths own a small advantage
over trigrams, and dependency paths seem to own advantages for
different measures than leaf-ancestor paths. So this question breaks
down into two pieces. How do you propose new feature sets, and how do
you tell the good ones?

\subsection{Proposing Feature Sets}

New feature sets are pretty easy to propose. Basically, chop up the
tree or dependency graph in some way, or condense the information down
in some way.  Trigrams take only the leaves, three at a
time. Leaf-ancestor paths take vertical slices of the tree. Dependency
paths take ``vertical'' slices of the graph (either arcs or nodes;
though arcs don't really work).

More lossy ways of dividing the parse might also be useful;
convolution kernels sum the differences between two trees. Oh wait, so
that wouldn't work. But something similar might. Maybe. Besides
convolution kernels, there are methods that try to identify the most
important features of the sentence manually. For example, the first or
last $n$ words in the sentence, words surrounding the predicate. Or
even lossier things like sentence length or tree height or number of
internal nodes, which captures the relative syntactic complexity of a
sentence.

Each of these has its advantages and disadvantages; leaf-ancestor
paths are good at capturing upper (hidden) structure. On the other
hand, dependency paths capture similar information but with more
emphasis on long-distance relations between words. Convolution
kernel-esque thing that I define tries to sum up the whole sentence in
a single symbol/number, as do tree height and node count. Other
specific features like first $n$ words intend to capture some aspect
of processing that will be reflected in real differences between
dialects.

\subsection{Evaluating Feature Sets}

The ideal feature set provides distances that match closely to the
real dialect differences between regions. Since these are not
precisely know, a number of proxies are necessary. One important proxy
is geographical distance; although there is no {\it a priori} reason
for geographical distance to match dialect distance, the dialectology
literature overwhelmingly shows that it does. Other proxies are
specific boundary features (mentioned above, for evaluating distance
measure). Also phonological distance.

Also, statistical significance. This is important but doesn't tell you
if you have a {\it good} set of features, only if you have a {\it bad}
set.

\subsection{Total Quality}

Total quality of the syntax distance is the multiple of distance measure
and feature set. Current results show inconsistency between measures
and sets; some combinations of measure and feature set work well, but
changing one makes it perform badly. In other words, one feature set
does not perform overwhelmingly better than the others; each feature
set has at least one measure for which it performs badly. (The
current patterns are odd.)


%%% Local Variables: 
%%% mode: latex
%%% TeX-master: "dissertation.tex"
%%% End: 


\chapter{Methods}
It's all about R and running it on Swedish eh.

There is a lot of background information, like other distance
measures and the underlying math of the distance measure I did use.

\section{Previous Work}

\subsection{S\'eguy}

Measurement of linguistic similarity has always been a part of
linguistics. However, until \namecite{seguy73} dubbed a new set of
approaches `dialectometry', these methods lagged behind the rest of
linguistics in formality. S\'eguy's quantitative analysis
of Gascogne French, while not aided by computer, was the predecessor
of more powerful statistical methods that essentially required the use
of computer as well as establishing the field's general dependence on
well-crafted dialect surveys that divide incoming data along
traditional linguistic boundaries: phonology, morphology, syntax, etc.
This makes both collection and analysis easier, although it requires
more work to combine separate analyses to produce a complete picture of dialect
variation.

The project to build the Atlas Linguistique et Ethnographique de la
Gascogne, which S\'eguy directed, collected data in a dialect survey
of Gascogne which asked speakers questions informed by different areas
of linguistics. For example, the pronunciation of `dog' ({\it chien})
was collected to measure phonological variation. It had two common
variants and many other rare ones: [k\~an], [k\~a], as well as [ka],
[ko], [kano], among others. These variants were, for the most part,
% or hat "chapeau": SapEu, kapEt, kapEu (SapE, SapEl, kapEl
known by linguists ahead of time, but their exact geographical
distribution was not.

The atlases, as eventually published, contained not only annotated
maps, but some analyses as well. These analyses were what S\'eguy named
dialectometry. Dialectometry differs from previous attempts to find
dialect boundaries in the way it combines information from the
dialect survey. Previously, dialectologists found isogloss
boundaries for individual items. A dialect boundary was generated when
enough individual isogloss boundaries coincided. However, for any real
corpus, there is so
much individual variation that only major dialect boundaries can
be captured this way.

S\'eguy reversed the process. He first combined survey data to get
a numeric score between each site. Then he posited dialect boundaries
where large distances resulted between sites. The difference is
important, because a single numeric score is easier to
analyze than hundreds of individual boundaries.
Much more subtle dialect boundaries are visible this way; where before
one saw only a jumble of conflicting boundary lines, now one sees
smaller, but consistent, numerical differences separating regions. {Dialectometry
  enables classification of gradient dialect boundaries, since now one
can distinguish weak and strong boundaries. Previously, weak
boundaries were too uncertain.}

However, S\'eguy's method of combination is simple both
linguistically and mathematically. When comparing two sites, any
difference in a response is counted as 1. Only identical
responses count as a distance of 0. Words are not analyzed
phonologically, nor are responses weighted by their relative amount
of variation. Finally, only geographically adjacent sites are
compared. This is a reasonable restriction, but later studies were
able to lift it because of the availability of greater computational
power. Work following S\'eguy's improves on both aspects. In
particular, Hans Goebl developed dialectometry models that are
more mathematically sophisticated.

\subsection{Goebl}

Hans Goebl emerged as a leader in the field of dialectometry,
formalizing the aims and methods of dialectometry. His primary
contribution was development of various methods to combine individual
distances into global distances and global distances into global clusters. These
methods were more sophisticated mathematically than previous
dialectometry and operated on any features extracted from the data. His
analyses have used primarily the Atlas Linguistique de Fran\c{c}ais.

\namecite{goebl06} provides a summary of his work. Most relevant for
this paper are the measures Relative Identity Value and Weighted
Identity Value. They are general methods that are the basis for nearly
all subsequent fine-grained dialectometrical analyses. They have three
important properties. First, they are independent of the source
data. They can operate over any linguistic data for which they are
given a feature set, such as the one proposed by \namecite{gersic71} for
phonology. Second, they can compare data even for items that do not
have identical feature sets, such as Ger\v{s}i\'c's $d$,
which cannot compare consonants and vowels. Third, they can compare
data sets that are missing some entries. This improves on S\'eguy's
analysis by providing a principled way to handle missing survey
responses.

Relative Identity Value, when comparing any two items, counts the
number of features which share the same value and then discounts
(lowers) the importance of the result by the number of unshared
features. The result is a single percentage that indicates
relative similarity. Calculating this distance between all pairs
of items in two regions produces a matrix which can be used for
clustering or other purposes. Note that the presentation below splits
Goebl's original equations into more manageable pieces; the high-level
equation for Relative Identity Value is:

\begin{equation}
  \frac{\textrm{identical}_{jk}} {\textrm{identical}_{jk} - \textrm{unidentical}_{jk}}
\label{riv}
\end{equation}
For some items being compared $j$ and $k$. In this case
\textit{identical} is
\begin{equation}
  \textrm{identical}_{jk} = |f \in \textrm{\~N}_{jk} : f_j = f_k|
\end{equation}
where $\textrm{\~N}_{jk}$ is the set of features shared by  $j$ and
$k$ and $f_j$ and $f_k$ are the value of some feature $f$ for $j$ and
$k$ respectively. \textit{unidentical} is defined similarly, except
that it counts all features N, not just the shared features
$\textrm{\~N}_{jk}$.

\begin{equation}
  \textrm{unidentical}_{jk} = |f \in \textrm{N} : f_j \neq f_k|
\end{equation}

Weighted Identity Value is a refinement of Relative Identity
Value. This measure defines some differences as more
important than others. In particular, feature values that only occur
in a few items give more information than feature values that appear
in a large number of items. This
idea shows up later in the normalization of syntax distance given by
\namecite{nerbonne06}.

The mathematical reasoning behind this idea is fairly simple. Goebl
is interested in feature values that occur in only a few items. If a
feature has some value that is shared by all of the items, then all
items belong to the same group. This feature value provides {\it no}
useful information for distinguishing the items.  The situation
improves if all but one item share the same value for a feature; at
least there are now two groups, although the larger group is still not
very informative.  The most information is available if each item
being studied has a different value for a feature; the items fall
trivially into singleton groups, one per item.

Equation \ref{wiv-ident} implements this idea by discounting
the \textit{identical} count from equation \ref{riv} by
the amount of information that feature value conveys. The
amount of information, as discussed above, is based on the number of
items that share a particular value for a feature. If all items share
the same value for some feature, then \textit{identical} will be discounted all the
way to zero--the feature conveys no useful information.
Weighted Identical Value's equation for \textit{identical} is
therefore
\begin{equation}
  \textrm{identical} = \sum_f \left\{
  \begin{array}{ll}
    0 & \textrm{if} f_j \neq f_k \\
    1 - \frac{\textrm{agree}f_{j}}{(Ni)w} & \textrm{if} f_j = f_k
  \end{array} \right.
\label{wiv-ident}
\end{equation}

\noindent{}The complete definition of Weighted Identity Value is
\begin{equation} \sum_i \frac{\sum_f \left\{
  \begin{array}{ll}
    0 & \textrm{if} f_j \neq f_k \\
    1 - \frac{\textrm{agree}f_j} {(Ni)w} & \textrm{if} f_j = f_k
\end{array} \right.}
  {\sum_f \left\{
  \begin{array}{ll}
    0 & \textrm{if} f_j \neq f_k \\
    1 - \frac{\textrm{agree}f_j} {(Ni)w} & \textrm{if} f_j = f_k
    \end{array} \right. - |f \in \textrm{N} : f_j \neq f_k|}
  \label{wiv-full}
  \end{equation}

  \noindent{}where $\textrm{agree}f_{j}$ is the number of items that agree
  with item $j$ on feature $f$ and $Ni$ is the total number of
  items ($w$ is the weight, discussed below). Because of the
  piecewise definition of \textit{identical}, this number is always at
  least $1$ because $f_k$ agrees already with $f_j$.
  This equation takes the count of shared features and weights
  them by the size of the sharing group. The features that are shared
  with a large number of other items get a larger fraction of the normal
  count subtracted.

  For example, let $j$ and $k$ be sets of productions for the
  underlying English segment /s/. The allophones of /s/ vary mostly on the feature
  \textit{voice}. Seeing an unvoiced [s] for /s/ is less ``surprising'' than
  seeing a voiced [z], so the discounting process should
  reflect this. For example, assume that an English corpus contains 2000
  underlying /s/ segments. If 500 of them are realized as [z], the
  discounting for \textit{voice} will be as follows:

  \begin{equation}
    \begin{array}{c}
      identical_{/s/\to[z]} = 1 - 500/2000 = 1 - 0.25 = 0.75 \\
      identical_{/s/\to[s]} = 1 - 1500/2000 = 1 - 0.75 = 0.25
    \end{array}
    \label{wiv-voice}
  \end{equation}

  Each time /s/ surfaces as [s], it only receives 1/4 of a point
  toward the agreement score when it matches another [s]. When /s/
  surfaces as [z], it receives three times as much for matching
  another [z]: 3/4 points towards the agreement score. If the
  alternation is even more weighted toward faithfulness, the ratio
  changes even more; if /s/ surfaces as [z] only 1/10 of the time,
  then [z] receives 9 times more value for matching than [s] does.

  The final value, $w$, which is what gives the name ``weighted
  identity value'' to this measure, provides a way to control how much
  is discounted. A high $w$ will subtract more from uninteresting
  groups, so that \textit{voice} might be worth less than
  \textit{place} for /t/ because /t/'s allophones vary more over
  \textit{place}. In equation \ref{wiv-voice}, $w$ is left at 1 to
  facilitate the presentation.

\section{Statistical Methods} % Computational? Mathematical?

It is at this point that the two types of analysis, phonological and
syntactic, diverge. Although Goebl's techniques are general enough to
operate over any set of features that can be extracted, better results
can be obtained by specializing the general measures above to take
advantage of properties of the input.  Specifically, the application
of computational linguistics to dialectometry beginning in the 1990s
introduced methods from other fields. These methods, while generally
giving more accurate results quickly, are tied to the type of data on
which they operate.

% NEW
Currently, the dominant phonological distance measure is Levenshtein
distance. This distance is essentially the count of differing
segments, although various refinements have been tried, such as
inclusion of distinctive features or phonetic
correlates. \namecite{heeringa04} gives an excellent analysis of the
applications and variations of Levenshtein distance. While Levenshtein
distance provides much information as a classifier, it is limited
because it must have a word aligned corpus for comparison. A number of
statistical methods have been proposed that remove this requirement
such as \namecite{hinrichs07} and \namecite{sanders09}, but none have
been as successful on existing dialect resources, which are small and
are already word-aligned. New resources are not easy to develop
because the statistical methods still rely on a phonetic transcription
process.
% end NEW

% \begin{enumerate}
% \item I should really check around to see if there is any new work out
%   there. Surely there is. Course John is free to do whatever works and
%   Wybo may have graduated or something. So there might not be any more
%   work on it.
% \item Explain leaf-ancestor paths, trigrams, dependency `paths' (to be
%   invented).
% \end{enumerate}

\subsection{Syntactic Distance}

Recently, computational dialectometry has expanded to analysis of
syntax as well. The first work in this area was \quotecite{nerbonne06}
analysis of Finnish L2 learners of English, followed by
\quotecite{sanders07} analysis of British dialect areas. Syntax
distance must be approached quite differently than phonological
distance. Syntactic data is extractable from raw text, so it is much
easier to build a syntactic corpus. But this implies an associated
drop in manual linguistic processing of the data. As a result, the
principal difference between present phonological and syntactic
corpora is that phonology data is word-aligned, while syntax data is
not sentence-aligned. Automatically constructed syntactic corpora
lead naturally to statistical measures over large amounts of data
rather than more sensitive measures that operate on small corpora.

\subsubsection{Nerbonne and Wiersma}
\label{nerbonne06}

Due to the lack of alignment between the
larger corpora available for syntactic analysis, a statistical
comparison of differences is more appropriate than the simple
symbolic approach possible with the word-aligned corpora used in
phonology. This statistical approach means that a syntactic distance
measure will have to use counting as its basis.

\namecite{nerbonne06} was an early method proposed for syntactic
distance.  It models syntax by part-of-speech (POS) trigrams and uses
differences between trigram type counts in a permutation test of
significance. This method was extended by \namecite{sanders07}, who
used \quotecite{sampson00} leaf-ancestor paths as an alternate basis
for building the model.

The heart of the measure is simple: the difference in type counts
between the combined types of two corpora. \namecite{kessler01}
originally proposed this measure, the {\sc Recurrence}
metric ($R$):

\begin{equation}
R = \Sigma_i |c_{ai} - c_{bi}|
\label{rmeasure}
\end{equation}

\noindent{}Given two corpora $a$ and $b$, $c_a$ and $c_b$ are the type
counts. $i$ ranges over all types, so $c_{ai}$ and $c_{bi}$ are the
type counts of corpora $a$ and $b$ for type $i$.  $R$ is designed to
represent the amount of variation exhibited by the two corpora while
the contribution of individual types remains transparent to aid later
analysis.

To account for differences in corpus size, sampling with replacement is
used. In addition, the samples are normalized to account for
differences in sentence length and complexity.  Unfortunately, even normalized, the
measure doesn't indicate whether its results are significant; a
permutation test is needed for that.

\subsubsection{Normalization}
The distance must be normalized for two kinds of variation:
the length of sentences in the corpus and the amount of variety in the
corpus. If sentence length differs too much between corpora, there
will be consistently lower token counts in one corpus, which would
cause a spuriously large $R$. In addition, if one corpus has less
variety than the other, it will have inflated type counts, because
more tokens will be allocated to fewer types. To avoid
this, all tokens are scaled by the average number of types per token
across both corpora: $2n/N$ where $n$ is the type count and $N$ is
the token count. The additional factor $2$ is necessary because we are
recombining the tokens from the two corpora.

% Other ideas include training a
% model on one area and comparing the entropy (compression) of other
% areas. At this point it's unclear whether this would provide a
% comparable measure, however.

\subsubsection{Language models}
\label{syntactic-features}
\namecite{nerbonne06} argue that POS trigrams can accurately represent
at least the important parts of syntax, similar to the way chunk
parsing can capture the most important information about a
sentence. If this is true, POS trigrams are a good starting point for
a language model; they are simple and easy to obtain in a number of
ways. They can either be generated by a tagger as Nerbonne
and Wiersma did, or taken from the leaves of the trees of a
syntactically annotated corpus as \namecite{sanders07} did with the
International Corpus of English.

On the other hand, it might be better to represent the upper structure
of trees, assuming that syntax is in fact a phenomenon that extends beyond the
lexical. \quotecite{sampson00} leaf-ancestor paths provide one way to
do this: for each leaf in the tree, leaf-ancestor paths produce the
path from that leaf back to the root. Generation is simple as long as
every sibling is unique. For example, the parse tree
\[\xymatrix{
  &&\textrm{S} \ar@{-}[dl] \ar@{-}[dr] &&\\
  &\textrm{NP} \ar@{-}[d] \ar@{-}[dl] &&\textrm{VP} \ar@{-}[d]\\
  \textrm{Det} \ar@{-}[d] & \textrm{N} \ar@{-}[d] && \textrm{V} \ar@{-}[d] \\
\textrm{the}& \textrm{dog} && \textrm{barks}\\}
\]
creates the following leaf-ancestor paths:

\begin{itemize}
\item S-NP-Det-The
\item S-NP-N-dog
\item S-VP-V-barks
\end{itemize}

For identical siblings, brackets must be inserted in the path to
disambiguate the first sibling from the second.
There is one path for each word, and the root appears
in all four. However, there can be ambiguities if some
node happens to have identical siblings. Sampson gives the example
of the two trees
\[\xymatrix{
  &&\textrm{A} \ar@{-}[dl] \ar@{-}[dr] &&&\\
  &\textrm{B} \ar@{-}[d] \ar@{-}[dl] &&\textrm{B} \ar@{-}[d] \ar@{-}[dr] & \\
  \textrm{p} & \textrm{q} && \textrm{r} & \textrm{s} \\
}
\]
and
\[\xymatrix{
  &&\textrm{A} \ar@{-}[d] &&&\\
  &&\textrm{B} \ar@{-}[dll] \ar@{-}[dl] \ar@{-}[dr] \ar@{-}[drr]&&& \\
  \textrm{p} & \textrm{q} && \textrm{r} & \textrm{s} \\
}
\]
which would both produce

  \begin{itemize}
  \item A-B-p
  \item A-B-q
  \item A-B-r
  \item A-B-s
  \end{itemize}

  There is no way to tell from the paths which leaves belong to which
  B node in the first tree, and there is no way to tell the paths of
  the two trees apart despite their different structure. To avoid this
  ambiguity, Sampson uses a bracketing system; brackets are inserted
  at appropriate points to produce
  \begin{itemize}
  \item $[$A-B-p
  \item A-B]-q
  \item A-[B-r
  \item A]-B-s
  \end{itemize}
and
  \begin{itemize}
  \item $[$A-B-p
  \item A-B-q
  \item A-B-r
  \item A]-B-s
  \end{itemize}

Left and right brackets are inserted: at most one
in every path. A left bracket is inserted in a path containing a leaf
that is a leftmost sibling and a right bracket is inserted in a path
containing a leaf that is a rightmost sibling. The bracket is inserted
at the highest node for which the leaf is leftmost or rightmost.

It is a good exercise to derive the bracketing of the previous two trees in deta
il.
In the first tree, with two B
siblings, the first path is A-B-p. Since $p$ is a leftmost child,
a left bracket must be inserted, at the root in this case. The
resulting path is [A-B-p. The next leaf, $q$, is rightmost, so a right
bracket must be inserted. The highest node for which it is rightmost
is B, because the rightmost leaf of A is $s$. The resulting path is
A-B]-q. Contrast this with the path for $q$ in the second tree; here $q$
is not rightmost, so no bracket is inserted and the resulting path is
A-B-q. $r$ is in almost the same position as $q$, but reversed: it is the
leftmost, and the right B is the highest node for which it is the
leftmost, producing A-[B-r. Finally, since $s$ is the rightmost leaf of
the entire sentence, the right bracket appears after A: A]-B-s.

At this point, the alert reader will have
noticed that both a left bracket and right bracket can be inserted for
a leaf with no siblings since it is both leftmost and rightmost. That is,
a path with two brackets on the same node could be produced: A-[B]-c. Because
of this redundancy, single children are
excluded by the bracket markup algorithm. There is still
no ambiguity between two single leaves and a single node with two
leaves because only the second case will receive brackets.

% See for yourself:
% \[\xymatrix{
%   &\textrm{A} \ar@{-}[dl] \ar@{-}[dr] &\\
%   \textrm{B} \ar@{-}[d] &&\textrm{B} \ar@{-}[d] \\
%   \textrm{p} && \textrm{q} \\
% }
% \]

% \[\xymatrix{
%   &\textrm{A} \ar@{-}[d] &\\
%   &\textrm{B} \ar@{-}[dl] \ar@{-}[dr] & \\
%   \textrm{p} && \textrm{q} \\
% }
% \]
% \cite{sampson00} also gives a method for comparing paths to obtain an
% individual path-to-path distance, but this is not necessary for the
% permutation test, which treats paths as opaque symbols.


Sampson originally developed leaf-ancestor paths as an improved
measure of similarity between gold-standard and machine-parsed trees,
to be used in evaluating parsers. The underlying idea of a collection of
features that capture distance between trees transfers quite nicely to
this application. \namecite{sanders07} replaced POS trigrams with
leaf-ancestor paths for the ICE corpus and found improved results on
smaller corpora than Nerbonne and Wiersma had tested. The additional
precision that leaf-ancestor paths provide appears to aid in attaining
significant results.

% Another idea is supertags rather than leaf-ancestor paths. This is
% quite similar but might work better.
\subsubsection{Leaf-Head Paths}
% TODO: This section should probably have a lot more examples and maybe some
% examples of other applications besides my experiment.

For dependency annotations, it is easy to adapt leaf-ancestor paths to
leaf-head paths. Here, each leaf is associated with a leaf-head path,
the path from the leaf to the head of the sentence via the
intermediate heads. For example, the same sentence, ``The dog barks'',
produces the following leaf-head paths.

\begin{itemize}
\item root-V-N-Det-the
\item root-V-N-dog
\item root-V-barks
\end{itemize}

The biggest difference is in the relative length of the paths: long
leaf-ancestor paths indicate deep nesting of structure. Length is a
weaker indicator of deep structure for leaf-head
paths; sometimes a difference in length indicates only a difference in
centrality to the sentence. % or something, this is still kind of
                            % wrong
\[\xymatrix{
& & root \\
DET \ar@/^/[r] & NP\ar@/^/[r] & V \ar@{.>}[u] \\
The & dog & barks
}
\]

\subsection{Previous Experiments}

\namecite{nerbonne06} were the first to use the syntactic distance
measure described above. They analyzed two corpora, both of Norwegian
L2 speakers of English. The first corpus was gathered from speakers
who learned English after childhood and the second was gathered from
speakers who learned English as children. Nerbonne \& Wiersma found a
significant difference between the two corpora. The trigrams that
contributed most to the difference were those in the older corpus that
are unexpected in English. For example, the trigram COP-ADJ-N/COM is
not common in English because a noun phrase following a copula
typically begins with a determiner. Other trigrams indicate
hypercorrection on the part of the older speakers; they appear in the
younger corpus but not as often. Nerbonne \& Wiersma analyzed this as
interference from Finnish; the younger learners of English learned it
more completely with less interference from Finnish.

Subsequent work by \namecite{sanders07} and \namecite{sanders08b}
expanded on the Norwegian experiment in two ways. First, it introduced
leaf-ancestor paths as an alternative feature type. Second, it tested
the distance method on a larger set of corpora: Government Office
Regions of England, as well as Scotland and Wales, for a total of
11 corpora. Each was smaller than the Norwegian L2 corpora, so the
permutation test parameters had to be adjusted for some feature
combinations.

The distances between regions were clustered using hierarchical
agglomerative clustering, as described in section \ref{cluster-analysis}. The resulting tree showed a North/South
distinction with some unexpected differences from previously
hypothesized dialect boundaries; for example, the
Northwest region clustered with the Southwest region. This contrasted
with the clustered phonological distances also produced in
\namecite{sanders08b}. In that experiment,
there was no significant correlation between the inter-region
phonological distances and syntactic distances.

There are several possible reasons for this lack of correlation. The
two distance measures may find different dialect boundaries based on
differences between syntax and phonology. Dialect boundaries may have
shifted during the 40 years between the collection of the SED and the
collection of the ICE-GB. One or both methods may be measuring the
wrong thing. However, I will not investigate the relation between
phonology and syntax in this dissertation. The focus will remain on results
of computational syntax distance as compared to traditional syntactic
dialectology.

\section{Methods}
To investigate the first hypothesis, I need a dialect corpus that can
be syntactically annotated (\ref{syntactically-annotated-corpus}); if
it is not already annotated, it must be possible to annotate it
automatically so I can avoid time-consuming manual annotation.
Automatic annotation will require a syntactically annotated
training corpus (\ref{syntactically-annotated-training}) and a parser
(\ref{parsers}). A distance measure must be defined for the regions
within the dialect corpus (\ref{nerbonne06}), syntactic features must
be extracted for the distance measures (\ref{syntactic-features}), and
the results tested for significance (\ref{permutationtest}) and
clustered (\ref{cluster-analysis}) to determine which dialect regions
are found by the corpus. Finally, the most highly ranked features used
to produce the dialect distances must be enumerated
(\ref{feature-ranking}).

To investigate the second hypothesis, I need a method to combine
different types of features (\ref{combine-feature-sets}) and back off
sparse features (\ref{feature-backoff}). I also need a way to generate
new features that include more information about context
(\ref{alternate-feature-sets}).

If the distance measure $R$ doesn't provide any significant distances
with any combination of features, I will experiment with different distance
measures. For this, there are quite a few possibilities;
Kullbeck-Leibler divergence is one example (\ref{kl-divergence}).

To investigate the third hypothesis, I need a phonological corpus and a method
for calculating phonological dialect distance, then a method to compare
phonological clusters with syntactic clusters. See my qualifying paper
\cite{sanders08b} for details.

\subsection{SweDiaSyn} % This is not a good subsection for the new organization
\label{syntactically-annotated-corpus}
The first hypothesis requires a dialect corpus that can
be syntactically annotated.
The dialect corpus used in this dissertation will be SweDiaSyn, the
Swedish part of the ScanDiaSyn.
% (CITE SweDiaSyn and ScanDiaSyn,
% except that they don't seem to have any references)
% Here is a citation for ScanDiaSyn if I could track it down and
% translate it
% Vangsnes, �ystein A. 2007. ScanDiaSyn: Prosjektparaplyen Nordisk dialektsyntaks. In T. Arboe (ed.), Nordisk dialektologi og sociolingvistik, Peter Skautrup Centeret for Jysk Dialektforskning, �rhus Universitet. 54-72.
SweDiaSyn is a transcription of SweDia 2000 \cite{bruce99} collected
between 1998 and 2000 from 97 locations in Sweden and 10 in
Finland. Each location has 12 interviewees: three 30-minute interviews
for each of older male, older female, younger male and younger female.
However, the SweDiaSyn transcriptions do not yet include all of SweDia
2000; the completed transcriptions currently focus on older
speakers.

Currently there are 36,713 sentences of transcribed speech
from 49 sites, an average of 749 sentences per site.
However, the sites range from 110 to 1780 sentences because some sites
have fewer complete transcriptions than others. In order to detect
significant differences, the sites may need to be grouped by county,
traditional province or EU region; previous work on British English
used EU Government Office Regions with at least 850 sentences per
region. For example, grouping the Swedish corpora into the 25 provinces
boosts the average sentences per province to 1254, excluding provinces
with no transcriptions.

% TODO: Probably switch the second sentence to be first? It's the more
% important but might completely depend on details in the first.
In the SweDiaSyn, there are two types of transcription:
standard Swedish orthography, with glosses for words
not in standard Swedish, and a phonetic transcription for dialects
that differ greatly from standard Swedish. For this dissertation,
the orthographic/gloss transcription will be used so that lexical
items will be comparable across dialects.

\subsection{Talbanken}
\label{syntactically-annotated-training}

Because the first hypothesis requires a syntactically annotated
corpus, and because SweDiaSyn consists of untagged lexical items,
Talbanken05, a syntactically-annotated corpus, will be used to train a
POS tagger and parsers to be used to annotate SweDiaSyn.  Talbanken05
is a treebank of written and transcribed spoken Swedish, roughly
300,000 words in size. It is an updated version of Talbanken76
\cite{nivre06}; Talbanken76's trees are annotated following a custom
scheme called MAMBA; Talbanken05 adds phrase structure annotation and
dependency annotation using the standard annotation formats TIGER-XML
and Malt-XML.  In addition to syntactic annotation, Talbanken is
lexically annotated for morphology and part-of-speech.

% TODO: Should I keep this? It depends on how much detail I want.
% Talbanken's sources are X and Y and Z. It attempts to provide a
% valid sample of the Swedish language, both spoken and written. The
% spoken section is transcribed from conversation, interviews and
% debates, and the written section is taken from high school essays and
% professional prose (TODO:I could probably cite
% Jan Einarsson. 1976. Talbankens skriftspraakskonkordans. Lund
% University: Department of Scandinavian Languages (and
% talspraakskonkordans) IF I could legitimately claim that I got the
% information from there\ldots{} but of course I got it from
% spraakbanken.gu.se/om/eng/index.html actually.

\subsection{Parsing}
\label{parsers}
%% Um, this seems useful, but I'm not sure why I put it here...
%% TODO: Figure out what this is for and use it somewhere?
% In order to investigate hypothesis 1, I will need to produce features
% to give to the classifier. These features should reflect the syntax of
% the speech of the interviewees. Following Nerbonne and Wiersma 2006, I
% will start with parts of speech, then add the leaf-ancestor paths that
% I tried on the ICE-GB, and finally add dependency-ancestor paths that
% are new. Probably one sentence more each on tagging, dependency and
% constituency parsing.
% (NOTE: Insert paragraphs on tagging and dependency and constituency
% parsing before Talbanken discussion)
% :
In order to extract the features used to build the language models
described in the previous methods, SweDiaSyn will need to be POS
tagged and parsed. For this dissertation, both constituency
and dependency features will be provided to the classifier.

The Tags 'n' Trigrams (T'n'T) tagger \cite{brants00} will be used for tagging, with
the POS annotations from Talbanken05 used as training.
After POS tagging, the Talbanken sentences will be cleaned in order to
be usable for training the parsers.
Cleaning Talbanken's constituency annotations consists of removing
discontinuities of various types, especially disfluencies and
restarts, which may be reparable by a simple top-level strategy. If
more complicated uncrossing is needed, a strategy similar to the split
constituents proposed by \namecite{boyd07} may be needed.

For constituency parsing, the Berkeley parser \cite{petrov08} will be
trained on standard Swedish, again from Talbanken05. The Berkeley
parser has shown good performance on languages other than English,
which is not common for constituency parsers.
% TODO: CITE The paper that shows this. Also EXPLAIN it.

For dependency parsing, MaltParser will be used with the existing
Swedish model trained on Talbanken05 by Hall, Nilsson and
Nivre. MaltParser is an inductive dependency parser that uses a
machine learning algorithm to guide the parser at choice points
\cite{nivre06b}.  Dependency parsing will proceed similarly to
constituency parsing; the dependency structures of Talbanken05 will be
cleaned and normalized, then used to train a parser.

% TODO: Find out how much crossing occurs in Swedish corpora, and how
% much of it is from interruptions and self-corrections.

\subsection{Permutation test}
\label{permutationtest}

The first hypothesis requires that the distances produced by a
distance measure be checked for significance; it is possible that
there may not be enough data for two regions to adequately distinguish
them from each other. A permutation test detects whether two corpora are
significantly different on the basis of the $R$ measure
described in section \ref{nerbonne06}. The test first calculates $R$
between samples of the two corpora. Then the corpora are mixed
together and $R$ is calculated between two samples drawn from the
mixed corpus. If the two corpora are different, $R$ should be larger
between the samples of the original corpora than $R$ from the mixed
corpus: any real differences will be randomly redistributed by the
mixing process, lowering the mixed $R$. Repeating this comparison
enough times will show if the difference is significant. Twenty times
is the minimum needed to detect significance for $p < 0.05$
significance; however, in the experiments, I will repeat the test 100
times, enough to detect significance for $p < 0.01$.

To see how this works, for example, assume that $R$ detects real
differences between the two British regions London
and Scotland such that $R(\textrm{London},\textrm{Scotland}) =
100$. The permutation test then mixes London and Scotland to
create LSMixed and splits it into two pieces. Since the real
differences are now mixed between the two shuffled corpora, we
would expect $R(\textrm{LSMixed}_1, \textrm{LSMixed}_2) < 100$.
This should be true at least 95\% of the time for the distance $100$
to be significant.

%% I don't think normalization is important enough to mention if I
%% have to add all the sections from the H2/H3.
% \subsection{Normalization}
% Afterward, the distance must be normalized to account for two things:
% the length of sentences in the corpus and the amount of variety in the
% corpus. If sentence length differs too much between corpora, there
% will be consistently lower token counts in one corpus, which would
% cause a spuriously large $R$. In addition, if one corpus has less
% variety than the other, it will have inflated type counts, because
% more tokens will be allocated to fewer types. To avoid
% this, all tokens are scaled by the average number of types per token
% across both corpora: $2n/N$ where $n$ is the type count and $N$ is
% the token count. The factor $2$ is necessary because the scaling
% occurs based on the token counts of the two corpora combined.

% this next subsection might need to be changed or deleted
\subsection{Cluster Analysis and Correlation}
\label{cluster-analysis}
The first hypothesis requires a clustering method to allow
inter-region distances to be compared more easily. The dendrogram that
binary hierarchical clustering produces allows easy visual comparison
of the most similar regions.

Correlation is also useful to find out how similar the two method's
predictions are. Because of the connected nature of the inter-region
distances, Mantel's test is necessary to ensure that the correlation
is significant. Mantel's test is a permutation test, much like the
permutation test described for $R$. One distance result set is
permuted repeatedly and at each step correlated with the other
set. The original correlation is significant if the permuted
correlation is lower than the original correlation more than 95\% of
the time.

\subsection{Feature Ranking}
\label{feature-ranking}
% TODO: THIS is hypothesis 1B and I left it out!
Feature ranking is needed for the first hypothesis so that the results
of $R$ can be compared qualitatively to the Swedish dialectology
literature; $R$'s most important features should be similar to those
discussed most by dialectologists when comparing regions. Feature
ranking for $R$ is quite simple for one-to-one region comparisons;
each feature's normalized weight is equal to its importance in
determining the distance between the two regions. The most important
features between two sets of regions can be obtained by averaging the
importance of each feature between all (first-set, second-set) region
pairs. This more
% (There is a nice equation lurking in here
% somewhere that I may want to avoid nonetheless.)
complicated technique is needed to relate the results from
the computational distance measures with the features that
dialectologists discuss relative to areas of Sweden larger than
individual provinces or counties.

%Note: All this is speculative. I have no code for this and I'm pretty
%sure the all-pairs average solution is not quite right

\subsection{Combining Feature Sets}
\label{combine-feature-sets}

% maybe use 'kind of feature' instead of 'type of feature'. it's fuzzier
In order to investigate hypothesis 2, I will need a method for
combining features of different types. Here, the obvious approach
of combining feature types linearly should suffice. For example, within a single
type of feature, such as POS tag or leaf-ancestor path,
there is already redundant information about lexical items and tree
structure, so combining the two does not mean that additional
redundancy needs to be taken into account.
% TODO: Last sentence still sucks
% TODO: Notes from last presentation:
% TnT has a 'mark unknown words' option.
% Pass lexical items to Berkeley parser (or make sure both training and test are t
% he same)
% check whether Berkeley parser has POS-tagging option

\subsection{Feature Backoff}
\label{feature-backoff}
% If I can't find a certain frequency of trigrams then backoff to
% bigram
% Thorsten Brants - deleted interpolation

% Martin Volk inter-type backoff (if X type information isn't
% available, then use Y type instead) (2000, 2001 or 2002)
% Scaling Up ...; Exploiting the WWW ...; Combining Unsupervised ...;
In order to investigate hypothesis 2, I will need a framework for
backing off sparse features. For backoff within a single type of
feature, I will use deleted interpolation \cite{jurafskymartin}. Training for the trigram, bigram and unigram counts will
come from the Talbanken.

For backoff between types of features, I will use ranked combinations
of feature sets, based to Martin Volk's system for verb attachment
\cite{volk02}. Volk used an a priori reliability measure for ranking
quality of combined feature types; I will use number of significant
region differences for ranking: the top-ranked feature type will be
the one that produces the highest number of significant distances
between regions. Combinations of feature types will be ranked by
averaging the number of significant distances that the constituent
feature types produce. Then if the distance measure can't find a
significant difference using highest ranked set of features, the
classifier will fall back to the next highest-ranked set of features.

\subsection{Alternate Feature Sets}
\label{alternate-feature-sets}
For hypothesis 2, I will need a way to generate new types of
features. One obvious way to do this is to modify existing feature
types to include more contextual
information. For example, supertags \cite{joshi94} are similar to
leaf-ancestor paths, but include more tree context around the
head. Similarly, dependency paths could be expanded
so that each node on the path includes lexical context, such as
bigrams or trigrams.

\subsection{Alternate Distance Measures}
\label{kl-divergence}
In the case that $R$ does not reach statistical significance, I will
need to experiment with similar but more complicated distance measures
to find a more sensitive one. The obvious choice at this point is
Kullbeck-Leibler divergence, or relative entropy, which is described
in \namecite{manningschutze}. Relative entropy is quite similar to $R$
but more widely used in computational linguistics. Besides this, several variants of
relative entropy exist, such as Jenson-Shannon divergence \cite{lin91}, that lift
various restrictions from the input distributions.

% Another possibility is a return to Goebl's Weighted Identity Value;
% this classifier is similar in some ways to $R$, but has not been
% tested with large corpora, to my knowledge at least. (This is not
% particularly useful and I don't believe that WIV would actually be
% good, so I should probably just drop this.)

More exotic classifiers are of course possible, although I
have not investigated them yet. Examples are k-nearest
neighbor classification or neural nets.
% (maybe it was relative entropy or just normal-kind entropy).
% TODO: WIV, also Kullbeck-Leibler Divergence could work.
% Maybe also k-NN/MBL, HMM binary classifier (?), maybe even a
% neural net

% Make sure the KL divergence couldn't go to infinity
% Jensen-Shannon lifts a couple of restrictions though I think the
% input still has to be a probability distribution

%%% Local Variables: 
%%% mode: latex
%%% TeX-master: "dissertation.tex"
%%% End: 


\chapter{Results}
\label{results-chapter}
% TODO: Many tables are ugly
% TODO: Re-order features in order of importance in all tables
% eg unigram last, preceded by deparc, timbl-dep (redep), etc

These results are meant to answer two main questions: first, how well does
this approach to syntactic dialectometry agree with dialectology?
Second, what combinations of distance measures, feature
sets and other settings produce the best results for linguistic
analyses? Additionally, the results are meant to allow comparison with
phonological dialectometry.

The organization of this chapter mirrors the order of the methods
chapter, particularly the output analysis (section
\ref{output-analysis}). First, there is an overview of the different
parameter settings, the combinations of distance measure and feature
set, as well as other settings. Then the number of significant
distances for each parameter setting is given, which is followed by
the correlation with geography and travel distance for each parameter
setting. These sections focus mainly on detecting which settings do not
produce valid results, so that they can be ignored in the rest of the
chapter. At a high level, they answer the question of the suitability
of statistical syntactic dialectometry: whether or not significant
results can be found.

Next, the specific dialectological results are examined. First,
cluster dendrograms provide a visualization of which regions the
distance measures find to be similar. In addition, to improve the
reliability of the dendrograms, consensus trees and composite cluster
maps are produced. Next, multi-dimensional scaling gives an smoother
view of similarity than clusters. Finally, features are ranked and
extracted from each cluster in the consensus tree.

\section{Parameter Settings}

There are 180 parameter settings investigated in this chapter. This
number arises from the four parameters: measure, feature set, sampling
method and number of normalization iterations. 5 measures, 9 feature
sets, 2 sampling methods and 2 numbers of normalization iterations
gives $5\times 9 \times 2 \times 2=180$ different settings. The
settings are given in table \ref{parameter-settings}.

\begin{table}
\begin{tabular}{|c|} \hline
  Feature Set \\\hline
  Leaf-Ancestor Path \\
  Part-of-speech Trigram \\
  Leaf-Head Path \\
  Phrase Structure Rule \\
  PSR with Grandparent \\
  Part-of-speech Unigram \\
  Leaf-Head Path, based on Timbl training \\
  Leaf-Arc Path \\
  All features combined \\ \hline
\end{tabular}
\begin{tabular}{|c|} \hline
  Measure \\ \hline
  $R$ \\
  $R^2$ \\
  Kullback-Leibler divergence \\
  Jensen-Shannon divergence \\
  cosine dissimilarity\\\hline \hline
  Sampling Method \\ \hline
  1000 sentences \\
  All sentences \\ \hline \hline
  Iterations of normalization \\ \hline
  1 \\
  5 \\ \hline
\end{tabular}
\caption{Settings for the five parameters tested}
\label{parameter-settings}
\end{table}
% Actually, all this should probably go in methods too, somewhere as a summary.

In addition, the size of each of the 30 interview sites are given in
table \ref{corpus-size}.

\begin{table}
\begin{tabular}{c|cc|c|cc}
      Site & Sentences & Words & Site & Sentences & Words \\\hline
     Ankarsrum &  630 &  7708 & Leksand &  923 &   10676\\
    Anundsjo &  1144 &   11897 &  Loderup &  429 &   7850\\
    Arsunda &  937 &   8933 & Norra Rorum &  546 &  9160\\
     Asby &  693 &   7171 & Orust &  1067 &   11409\\
     Bara &  696 &   10724 & Ossjo &  481 &   12275\\
     Bengtsfors &  663 &   7423 & Segerstad &  837 &   9746\\
    Boda &  1029 &   17425 &  Skinnskatteberg &  730 &   9529\\
     Bredsatra &  360 &   6938 & Sorunda &  768 &   11144\\
     Faro &  659 &   8260 & Sproge &  381 &   4399\\
     Floby &  557 &   6392 & StAnna &  876 &   13156\\
     Fole &  727 &   9920 & Tors\.as &  374 &   9217\\
     Frillesas &  572 &   9634 & Torso &  956 &   15577\\
    Indal &  1126 &   13090 &  Vaxtorp &  903 &   11353\\
     Jamshog &  301 &   8661 & Viby &  431 &   6734\\
     K\"ola &  528 &   10133 & Villberga &  680 &   11479\\
\end{tabular}
  \caption{Corpus Size of Interview Sites}
  \label{corpus-size}
\end{table}

\section{Significant Distances}

Significant distances help answer the question whether a syntactic
measure has succeeded in finding reliable distances. The measures
should have zero non-significant distances, or at least a small
number. In the tables, the total number of comparisons between all 30
regions is the $435=30(30-1) / 2$. In the first set,
tables \ref{sig-1-1000} -- \ref{sig-1-full}, the results are shown
from one iteration of the normalization step. In the second set,
tables \ref{sig-5-1000} -- \ref{sig-5-full}, the results
from five normalization iterations are shown.

Bold numbers in the tables indicate that more than 5\% of the
distances were not significant. In table \ref{sig-5-full}, the
5-iteration table that compares full corpora, the only combination
with {\it less} than 5\% non-significant results is cosine
dissimilarity with unigram features, marked with italics. Note that
here, 5\% is an arbitrary cutoff point not related to the usual
significance cutoff $p < 0.05$; the basis for these tables are
themselves number of significant distances found.

\begin{table}
\begin{tabular}{l|rrrrr}
  & $R$ & $R^2$ & KL & JS & cos  \\ \hline
  Leaf-Ancestor &0&0&11&0&0 \\
  Trigram &0&0&0&0&0 \\
  Leaf-Head &0&0&0&0&0 \\
  Phrase-Structure Rules &0&0&\textbf{95}&0&0 \\
  Phrase-Structure with Grandparents &0&0&\textbf{273}&0&0 \\
  Unigram &0&0&0&0&0 \\
  Leaf-Head with MaltParser trained by Timbl &0&0&\textbf{47}&0&0 \\
  Leaf-Arc Labels&0&0&0&0&0 \\
  All Features Combined &0&0&0&0&0 \\
\end{tabular}
\caption{Number of significant distances for sample size 1000, 1
  normalization}
\label{sig-1-1000}
\end{table}

\begin{table}
\begin{tabular}{l|rrrrr}
& $R$ & $R^2$ & KL & JS & cos  \\ \hline
  Leaf-Ancestor&7&11&12&\textbf{35}&9 \\
  Trigram&4&1&0&\textbf{24}&1 \\
  Leaf-Head&10&12&20&\textbf{44}&19 \\
  Phrase-Structure Rules&\textbf{26}&17&\textbf{24}&\textbf{49}&20 \\
  Phrase-Structure with Grandparents&\textbf{58}&\textbf{35}&\textbf{38}&\textbf{71}&\textbf{33}
   \\
  Unigram&1&2&0&0&2 \\
  Leaf-Head with MaltParser trained by Timbl&11&21&18&\textbf{74}&\textbf{30}
   \\
  Leaf-Arc Labels&14&19&\textbf{37}&\textbf{94}&17 \\
  All Features Combined&0&0&1&8&2 \\
\end{tabular}
 \caption{Number of significant distances for complete regions, 1
   normalization}
 \label{sig-1-full}
\end{table}

\begin{table}
\begin{tabular}{l|rrrrr}
& $R$ & $R^2$ & KL & JS & cos  \\ \hline
  Leaf-Ancestor&5 & \textbf{56} & \textbf{34} & 0 & 0\\
  Trigram&3 & 2 & 0 & 0 & 0\\
  Leaf-Head&3 & 14 & 4 & 0 & 0\\
  Phrase-Structure Rules&11 & 4 & \textbf{66} & 1 & 0\\
  Phrase-Structure with Grandparents&18 & 0 & \textbf{109} & 4 & 0\\
  Unigram&\textbf{52} & \textbf{53} & 15 & 17 & 0\\
  Leaf-Head with MaltParser trained by Timbl&7 & 20 & \textbf{45} & 0 & 0\\
  Leaf-Arc Labels&6 & \textbf{54} & 17 & 1 & 0\\
  All Features Combined&0 & 4 & 0 & 0 & 0\\
\end{tabular}
\caption{Number of significant distances for sample size 1000, 5
  normalizations}
\label{sig-5-1000}
\end{table}
\begin{table}
\begin{tabular}{l|rrrrr}
& $R$ & $R^2$ & KL & JS & cos  \\ \hline
  Leaf-Ancestor&\textbf{290} & \textbf{284} & \textbf{287} & \textbf{278} & \textbf{204}\\
  Trigram&\textbf{284} & \textbf{283} & \textbf{283} & \textbf{276} & \textbf{196}\\
  Leaf-Head&\textbf{293} & \textbf{286} & \textbf{285} & \textbf{279} & \textbf{211}\\
  Phrase-Structure Rules&\textbf{289} & \textbf{294} & \textbf{286} & \textbf{275} & \textbf{236}\\
  Phrase-Structure with Grandparents&\textbf{285} & \textbf{290} & \textbf{286} & \textbf{270} & \textbf{258}\\
  Unigram&\textbf{297} & \textbf{296} & \textbf{294} & \textbf{293} &
  \textit{9}\\
  Leaf-Head with MaltParser trained by Timbl&\textbf{294} & \textbf{289} & \textbf{288} & \textbf{284} & \textbf{222}\\
  Leaf-Arc Labels&\textbf{294} & \textbf{290} & \textbf{291} & \textbf{293} & \textbf{162}\\
  All Features Combined&\textbf{279} & \textbf{279} & \textbf{279} & \textbf{269} & \textbf{191}\\
\end{tabular}
 \caption{Number of significant distances for complete regions, 5
   normalizations}
 \label{sig-5-full}
\end{table}

% TODO: Need to add references to diagrams and also any
% citations

Analysis of the significance of dialect distance provides a measure of
how reliable the distances to be analyzed later in this chapter are. A
distance that does not find significant distances between of 30
regions is not suitable for precise inspection, although small numbers
of non-significant distances will still allow methods to
return interpretable results.

The highest number of significant distances are found in the first
case (figure \ref{sig-1-1000}): 1 round of normalization with a
fixed-size sample of 1000 sentences. From there, both full-corpus
comparisons (figure \ref{sig-1-full}) and 5 rounds of normalization
(figure \ref{sig-5-1000}) have fewer significant distances, although
the number is still usable. However, the combination of the two, with
5 rounds of normalization over full-corpus comparisons, has only one
combination with fewer than 5\% of distances that are {\it not}
significant. Although both full-corpus comparisons and multiple rounds
of normalization may increase the precision of the results, their
combined effect on significance is so detrimental that its results are
useless. For the rest of the analysis, the combination of full-corpus
comparison and 5 rounds of normalization will be skipped.

\subsubsection{Significance by Measure}

The distance measures most likely to find significance are, in order,
cosine dissimilarity, Jensen-Shannon divergence and $R$. Each method
had different parameter settings for which it was stronger. For
1000-sentence sampling, cosine similarity resulted in all significant
distances, even for part-of-speech unigrams, which are intended as the
baseline feature set. Excluding unigrams, Jensen-Shannon divergence
has similar performance. For full-corpus comparisons, both perform
considerably worse; surprisingly, both perform better on unigram
features, Jensen-Shannon so much so that it's the only feature set for
which it finds all significant distances. $R$, on the other hand,
performs decently on all combinations of parameter settings; its low
significance for phrase structure rules is shared by Kullback-Leibler
and Jensen-Shannon divergences.
% TODO : Maybe more on cosine later. Maybe not.

When comparing the performance of Kullback-Leibler and Jensen-Shannon
divergence it is not surprising that Jensen-Shannon outperforms
Kullback-Leibler on fixed-size sampling. Although both are called
``divergence'', Jensen-Shannon divergence is actually a
dissimilarity. Recall that the divergence from point A to B may differ
from the divergence from point B to A. A divergence like
Kullback-Leibler can be converted to a dissimilarity by measuring
$KL(A,B) + KL(B,A)$. However, this dissimilarity must skip features
unique to a single corpus in order to avoid division by zero. This
means that for smaller corpora Kullback-Leibler loses information that
Jensen-Shannon is able to use.  On the other hand, while this may
explain Kullback-Leibler's improved performance for full-corpus
comparisons, it doesn't explain Jensen-Shannon's much worse
performance.

\subsubsection{Significance by Feature Set}

% \item Unigrams do form an adequate baseline; they are bad but not too
%   bad.

% The feature sets most likely to find significance are the combined
% features and unigrams., in order,
% trigrams, all combined features and leaf-head paths (both with
% support-vector-machine training and with Timbl's instance-based
% training). Without ratio normalization, the other feature sets are not
% much worse, but with it included, these three are the best by some
% distance.

For 1 round of normalization, the best feature sets are the simple
ones: trigrams and unigrams, as well all combined features. On the
other hand, trigrams and leaf-head paths (with its variations) are the
best feature sets with 5 rounds of normalization. However, the
variation isn't strong; any feature set can give good results with the
right distance measure. The problem is that no clear patterns emerge.

The relatively high quality of trigrams and unigrams does not make
sense given only the linguistic facts; however, it is likely that the
entirely automatic annotation used here introduces more and more
errors as more annotators run, operating on previous automatic
annotations. Trigrams are the result of only one automatic annotation,
and one for which the state of the art is near human performance. So
the fact that these particular parts of speech are of higher quality
than the corresponding dependencies or constituencies is probably the
deciding factor in their higher number of significant
distances.

% Although it is impossible to tell from my results, I
% predict that a manually annotated dialect corpus would show that
% non-flat syntactic structure is helpful in producing significant
% distances.

Given the above facts, the question should rather be: why do leaf-head
paths perform as well as they do? Better, for example, than the
leaf-ancestor paths on which they're modeled: why does more
normalization hurt leaf-ancestor paths but not leaf-head paths?  It
could be that there is less room for error; many of the common
leaf-head paths are short: short interview sentences with simple
structure make for shorter leaf-head paths than leaf-ancestor
paths. As a result, the important leaf-head paths consist mainly of a
couple of parts-of-speech.

Another reason could be a difference in parsers: MaltParser has been
tested before with Swedish (CITE). Besides English, the Berkeley
parser has been tested prominently on German and Chinese. Therefore,
the difference would better be explained by appealing to the
difference in parsers rather than an unsuitability of Swedish for
constituent analysis.

It is disappointing linguistically that trigrams provide the most
reliable results so far; a linguist would expect that including
syntactic information would make it easier to measure the differences
between regions. If it is, as hypothesized here, an effect of chaining
machine annotators, a study using manually annotated corpora could
detect this. However, it still means that trigrams are the most useful
feature set from a practical view, because automatic trigram tagging
is very close to human performance with little training. That means
the only required human work is the transcription of interviews in
most cases.

On the other hand, if additional features sets are to be developed for
a corpus, then combining all available features seems to be a
successful strategy. The distance measures seem to be able to use all
available information for finding significant distances.


\section{Correlation}

In dialectology, the default expectation for dialect distance is that
it correlates with geographic distance \cite{chambers98}. A lack of
correlation does not necessarily mean that a measure is useless, but
presence of correlation means that the distance measure substantiates the
well-known tendency of dialect distributions to be more or less
smoothly gradient over physical space.

In addition, distance measures are more likely to correlate
significantly with travel distance than with straight-line geographic
distance. This makes sense since the difficulty of moving from place
to place is what influences dialect formation, and taking roads into
account is an improved estimate over straight-line distance.

The tables that present geographic and table correlation,
\ref{cor-1-1000} -- \ref{travel-cor-5-full}, mark significant
correlations with a star for $p < 0.05$, two stars for $p < 0.01$ and
three stars for $p < 0.001$. However, these correlations are only
trustworthy in the case that the underlying distances are
significant. Significant correlations from significant distances (as
cross-referenced from tables \ref{sig-1-1000} -- \ref{sig-5-full}) are
marked by italics.

Besides this, correlation between combinations of measure/feature set
can show how closely related they are--in other words, how similarly
they view the underlying data which remains the same for all.

This is similar to the reasoning behind correlation with
geography---but the assumption is that geography is a factor
underlying dialect formation; while the distance measure measures some
aspect of the language which we hope is dialects, it is indirectly
(even less directly) measuring the geography. Therefore, correlation
with geography should occur.

Third, correlation with corpus size is not predicted and is probably
an undesired defect in sampling or normalization. Correlation with
corpus size is presented in tables \ref{size-cor-1-1000} --
\ref{size-cor-5-full}.

\begin{table}
\begin{tabular}{l|rrrrr}
& $R$ & $R^2$ & KL & JS & cos  \\ \hline
  Leaf-Ancestor&-0.01 & 0.03 & 0.02 & -0.02 & 0.08\\
  Trigram&0.17 & 0.17 & 0.10 & 0.19 & 0.13\\
  Dependency&-0.06 & 0.03 & 0.00 & -0.07 & 0.05\\
  Phrase-Structure Rules&0.01 & \textit{0.18*} & 0.16 & 0.01 & 0.12\\
  Phrase-Structure with Grandparents&0.03 & \textit{0.25*} & 0.21* & 0.03 & 0.12\\
  Unigram&\textit{0.18*} & 0.17 & \textit{0.29**} & \textit{0.30**} & \textit{0.18*}\\
  Dependencies, MaltParser trained by Timbl&-0.07 & 0.02 & -0.00 & -0.08 & 0.05\\
  Dependency, Arc Labels&-0.07 & 0.06 & -0.06 & -0.09 & 0.00\\
  All Features Combined&-0.02 & 0.03 & 0.01 & -0.02 & 0.07\\
\end{tabular}
 \caption{Geographic correlation for sample size 1000, 1 normalization iteration}
 \label{cor-1-1000}
\end{table}

\begin{table}
\begin{tabular}{l|rrrrr}
& $R$ & $R^2$ & KL & JS & cos  \\ \hline
  Leaf-Ancestor&0.02 & 0.09 & 0.11 & -0.00 & 0.09\\
  Trigram&\textit{0.27*} & \textit{0.26*} & \textit{0.30**} & 0.21* & 0.08\\
  Dependency&-0.03 & 0.12 & 0.14 & -0.06 & 0.02\\
  Phrase-Structure Rules&0.13 & \textit{0.36**} & 0.30** & 0.11 & \textit{0.20*}\\
  Phrase-Structure with Grandparents&0.15 & 0.41** & 0.36** & 0.14 & 0.19*\\
  Unigram&\textit{0.20*} & \textit{0.20*} & \textit{0.33**} & \textit{0.33**} & \textit{0.22*}\\
  Dependencies, MaltParser trained by Timbl&-0.02 & 0.14 & 0.16 & -0.05 & 0.02\\
  Dependency, Arc Labels&-0.06 & 0.13 & -0.01 & -0.12 & -0.03\\
  All Features Combined&0.03 & 0.11 & 0.16 & -0.00 & 0.04\\
\end{tabular}
 \caption{Geographic correlation for complete corpora, 1 normalization iteration}
 \label{cor-1-full}
\end{table}

\begin{table}
\begin{tabular}{l|rrrrr}
& $R$ & $R^2$ & KL & JS & cos  \\ \hline
  Leaf-Ancestor&0.14 & 0.14 & 0.16 & 0.15 & 0.08\\
  Trigram&\textit{0.22*} & 0.17 & \textit{0.22*} & \textit{0.22*} & 0.16\\
  Dependency&0.10 & 0.11 & 0.15 & 0.12 & 0.10\\
  Phrase-Structure Rules&0.14 & 0.10 & 0.14 & 0.15 & 0.06\\
  Phrase-Structure with Grandparents&0.16 & 0.14 & 0.14 & 0.15 & 0.05\\
  Unigram&0.12 & 0.11 & 0.14 & 0.13 & 0.17\\
  Dependencies, MaltParser trained by Timbl&0.09 & 0.12 & 0.16 & 0.11 & 0.11\\
  Dependency, Arc Labels&0.08 & 0.10 & 0.14 & 0.10 & 0.09\\
  All Features Combined&0.19 & 0.16 & \textit{0.20*} & \textit{0.21*} & 0.11\\
\end{tabular}
 \caption{Geographic correlation for sample size 1000, 5
   normalizations}
 \label{cor-5-1000}
\end{table}

\begin{table}
\begin{tabular}{l|rrrrr}
& $R$ & $R^2$ & KL & JS & cos  \\ \hline
  Leaf-Ancestor&-0.14 & -0.16 & -0.15 & -0.15 & -0.08\\
  Trigram&-0.09 & -0.07 & -0.09 & -0.09 & -0.09\\
  Dependency&-0.22 & -0.21 & -0.18 & -0.22 & -0.10\\
  Phrase-Structure Rules&-0.19 & -0.14 & -0.11 & -0.20 & -0.01\\
  Phrase-Structure with Grandparents&-0.17 & -0.11 & -0.09 & -0.18 & -0.02\\
  Unigram&-0.10 & -0.06 & -0.07 & -0.08 & 0.14\\
  Dependencies, MaltParser trained by Timbl&-0.19 & -0.18 & -0.18 & -0.19 & -0.10\\
  Dependency, Arc Labels&-0.21 & -0.18 & -0.18 & -0.21 & -0.10\\
  All Features Combined&-0.18 & -0.18 & -0.16 & -0.18 & -0.09\\
\end{tabular}
 \caption{Geographic correlation for complete corpora, 5 normalizations}
 \label{cor-5-full}
\end{table}

\begin{table}
\begin{tabular}{l|rrrrr}
& $R$ & $R^2$ & KL & JS & cos  \\ \hline
  Leaf-Ancestor&-0.03 & 0.02 & 0.01 & -0.04 & 0.07\\
  Trigram&0.20 & 0.19 & 0.11 & \textit{0.23*} & 0.14\\
  Dependency&-0.07 & 0.01 & -0.01 & -0.08 & 0.05\\
  Phrase-Structure Rules&0.01 & \textit{0.18*} & 0.17 & 0.00 & 0.14\\
  Phrase-Structure with Grandparents&0.03 & \textit{0.26*} & 0.22* & 0.03 & 0.15\\
  Unigram&\textit{0.20*} & \textit{0.19*} & \textit{0.30**} & \textit{0.31**} & \textit{0.21*}\\
  Dependencies, MaltParser trained by Timbl&-0.08 & 0.02 & -0.01 & -0.09 & 0.05\\
  Dependency, Arc Labels&-0.08 & 0.05 & -0.06 & -0.10 & 0.00\\
  All Features Combined&-0.03 & 0.03 & 0.01 & -0.03 & 0.06\\
\end{tabular}
 \caption{Travel correlation for sample size 1000, 1 normalization iteration}
 \label{travel-cor-1-1000}
\end{table}

\begin{table}
\begin{tabular}{l|rrrrr}
& $R$ & $R^2$ & KL & JS & cos  \\ \hline
  Leaf-Ancestor&0.02 & 0.08 & 0.11 & 0.00 & 0.08\\
  Trigram&\textit{0.31*} & \textit{0.28*} & \textit{0.32**} & 0.26* & 0.09\\
  Dependency&-0.02 & 0.12 & 0.13 & -0.05 & 0.01\\
  Phrase-Structure Rules&0.15 & \textit{0.37**} & 0.32** & 0.13 & \textit{0.22*}\\
  Phrase-Structure with Grandparents&0.17 & 0.43** & 0.38** & 0.16 & 0.22*\\
  Unigram&\textit{0.22*} & \textit{0.22*} & \textit{0.33**} & \textit{0.34**} & \textit{0.24*}\\
  Dependencies, MaltParser trained by Timbl&-0.01 & 0.14 & 0.17 & -0.04 & 0.02\\
  Dependency, Arc Labels&-0.06 & 0.12 & -0.02 & -0.12 & -0.03\\
  All Features Combined&0.04 & 0.10 & 0.16 & 0.01 & 0.04\\
\end{tabular}
 \caption{Travel correlation for complete corpora, 1 normalization iteration}
 \label{travel-cor-1-full}
\end{table}

\begin{table}
\begin{tabular}{l|rrrrr}
& $R$ & $R^2$ & KL & JS & cos  \\ \hline
  Leaf-Ancestor&0.17 & 0.19* & 0.17* & 0.18 & 0.07\\
  Trigram&\textit{0.24*} & \textit{0.20*} & \textit{0.25*} & \textit{0.26*} & 0.16\\
  Dependency&0.14 & 0.16 & 0.17 & 0.15 & 0.10\\
  Phrase-Structure Rules&0.17 & 0.14 & 0.16* & 0.18 & 0.06\\
  Phrase-Structure with Grandparents&0.19 & \textit{0.18*} & 0.17* & 0.19 & 0.06\\
  Unigram&0.15 & 0.13 & \textit{0.17*} & 0.16 & \textit{0.20*}\\
  Dependencies, MaltParser trained by Timbl&0.12 & 0.16 & 0.18 & 0.14 & 0.11\\
  Dependency, Arc Labels&0.09 & 0.13 & 0.14 & 0.11 & 0.08\\
  All Features Combined&\textit{0.23*} & \textit{0.20*} & \textit{0.22*} & \textit{0.24*} & 0.11\\
\end{tabular}
 \caption{Travel correlation for sample size 1000, 5 normalizations}
 \label{travel-cor-5-1000}
\end{table}

\begin{table}
\begin{tabular}{l|rrrrr}
& $R$ & $R^2$ & KL & JS & cos  \\ \hline
  Leaf-Ancestor&-0.13 & -0.13 & -0.10 & -0.13 & -0.04\\
  Trigram&-0.06 & -0.04 & -0.05 & -0.06 & -0.05\\
  Dependency&-0.20 & -0.17 & -0.13 & -0.19 & -0.06\\
  Phrase-Structure Rules&-0.15 & -0.08 & -0.05 & -0.15 & 0.04\\
  Phrase-Structure with Grandparents&-0.12 & -0.05 & -0.03 & -0.13 & 0.03\\
  Unigram&-0.07 & -0.03 & -0.04 & -0.05 & \textit{0.18*}\\
  Dependencies, MaltParser trained by Timbl&-0.18 & -0.15 & -0.12 & -0.18 & -0.05\\
  Dependency, Arc Labels&-0.20 & -0.17 & -0.14 & -0.19 & -0.06\\
  All Features Combined&-0.16 & -0.14 & -0.11 & -0.15 & -0.05\\
\end{tabular}
 \caption{Travel correlation for complete corpora, 5 normalizations}
 \label{travel-cor-5-full}
\end{table}

From the tables we see that parameter settings that correlate
significantly do so at rates around 0.2 to 0.3, with a high of 0.37
for phrase-structure-rule features measured by $R^2$, 1 normalization
iteration and comparison of full corpora.  The significant
correlations are mostly concentrated in the trigram, unigram and
combined feature sets.

\subsection{Analysis}

As with the number of significant distances, trigrams and unigrams are
the most likely to to correlate with geographic and travel distance,
as well as the combined feature set for the 5-normalization parameter
setting.
% As before, a possible explanation is that unigrams are
% simpler, so the type count is a higher than for other measures. With
% more rounds of normalization, more correlations shift over to
% trigrams.
Note that in figures \ref{cor-1-1000} --
\ref{travel-cor-5-full}, the significant correlations are marked with
an asterisk, but only the italicized correlations are based on at
least 95\% significant distances. For example, this means that most of
the significant correlations based on phrase-structure rules are not valid.

It is worthwhile to note, however, that the valid and significant
correlations based on phrase-structure grammars give the highest
correlations: 0.37 for $R^2$ with full-corpus comparisons and 1 round
of normalization.
The addition of more data and more normalization is interesting in
expanding the correlating parameter settings beyond those that include
unigram features. It may be that this is an instance of the noise/quality tradeoff.
These additions appear to extract more detail from
the data, at the cost of additional interference from noisy data.

% Goes here: Fevered speculation about why travel correlation is *better* with
% the methods that correlate *less*, for 1-full at least.
% OK never mind this isn't true.

\subsection{Inter-measure Correlation}

Correlation between measures simply indicates that they are using
similar information from the corpus to do their classification. This
is expected since most measures are quite similar. The one that
differs the most, cosine similarity, also correlates the least with
the others. The average correlation between different measures is
given in table \ref{self-correlation-measures}. The correlations are
averaged over the correlations for all combinations of feature set
with 1000-sentence samples and with non-significant correlations
removed before averaging.

\begin{table}
  \begin{tabular}{r|cccc}
 & $R^2$ & $KL$ & $JS$ & cos \\ \hline
  $R$ & 0.85 & 0.85 & 0.98 & 0.39\\
  $R^2$&& 0.90 & 0.83 & 0.57\\
  $KL$ &&& 0.88 & 0.67\\
  $JS$ &&&& 0.44
\end{tabular}
\caption{Average Inter-measure-correlation of measures}
 \label{self-correlation-measures}
\end{table}

The inter-measure correlation mostly says what is visible from the
significance testing and correlations. $R$ and Jensen-Shannon produce
nearly identical results, and correlate highly. Cosine similarity is
quite different from the other measures, though the correlation is
still higher than with travel distance. This makes some sense in that
the cosine at the heart of cosine similarity differs more from the
sums of absolute values and logarithms of other measures.

\subsection{Correlation with Corpus Size}

As previously stated, correlation with corpus size is not predicted and is probably
an undesired defect in sampling or normalization. Correlation with
corpus size is presented in tables \ref{size-cor-1-1000} --
\ref{size-cor-5-full}.

Corpus size between two regions can be measured in two different ways:
either by the sum of the regions' sizes, or by the difference. Here
the sum is used: a larger sum means more tokens. If there is a
correlation with size, it must arise because higher token counts are
not properly normalized. In other words, two large regions will
have more tokens, leading to higher type counts, which directly leads
to higher distances. Smaller regions will lead to lower distances.

\begin{table}
\begin{tabular}{l|rrrrr}
& $R$ & $R^2$ & KL & JS & cos  \\ \hline
  Leaf-Ancestor&-0.38 & -0.26 & -0.37 & -0.40 & -0.37\\
  Trigram&0.12 & -0.12 & -0.16 & 0.14 & -0.18\\
  Dependency&-0.39 & -0.26 & -0.35 & -0.43 & -0.39\\
  Phrase-Structure Rules&0.06 & 0.15 & 0.00 & 0.03 & -0.10\\
  Phrase-Structure with Grandparents&0.08 & 0.19 & 0.07 & 0.04 & -0.09\\
  Unigram&-0.08 & -0.14 & -0.09 & -0.09 & -0.10\\
  Dependencies, MaltParser trained by Timbl&-0.35 & -0.23 & -0.28 & -0.37 & -0.37\\
  Dependency, Arc Labels&-0.44 & -0.26 & -0.40 & -0.48 & -0.34\\
  All Features Combined&-0.37 & -0.26 & -0.38 & -0.42 & -0.40\\
\end{tabular}
\caption{Size correlation for sample size 1000, 1 normalization}
\label{size-cor-1-1000}
\end{table}

\begin{table}
\begin{tabular}{l|rrrrr}
& $R$ & $R^2$ & KL & JS & cos  \\ \hline
  Leaf-Ancestor&-0.19 & -0.15 & -0.16 & -0.24 & -0.36\\
  Trigram&\textit{0.30*} & 0.08 & 0.19 & 0.08 & -0.39\\
  Dependency&-0.17 & -0.06 & -0.08 & -0.26 & -0.41\\
  Phrase-Structure Rules&0.52** & \textit{0.40**} & 0.30* & 0.47** & -0.21\\
  Phrase-Structure with Grandparents&0.54** & 0.43** & 0.37** & 0.50** & -0.22\\
  Unigram&-0.09 & -0.13 & -0.11 & -0.13 & -0.13\\
  Dependencies, MaltParser trained by Timbl&-0.08 & 0.02 & 0.09 & -0.14 & -0.39\\
  Dependency, Arc Labels&-0.32 & -0.16 & -0.26 & -0.40 & -0.35\\
  All Features Combined&-0.15 & -0.11 & -0.10 & -0.25 & -0.42\\
\end{tabular}
\caption{Size correlation for complete corpora, 1 normalization}
\label{size-cor-1-full}
\end{table}

\begin{table}
\begin{tabular}{l|rrrrr}
& $R$ & $R^2$ & KL & JS & cos  \\ \hline
  Leaf-Ancestor&\textit{0.35*} & 0.36** & 0.06 & 0.27 & -0.32\\
  Trigram&\textit{0.75**} & \textit{0.63**} & \textit{0.46**} & \textit{0.68**} & -0.24\\
  Dependency&\textit{0.46**} & \textit{0.44**} & 0.14 & \textit{0.38**} & -0.33\\
  Phrase-Structure Rules&\textit{0.85**} & \textit{0.59**} & 0.36** & \textit{0.85**} & -0.34\\
  Phrase-Structure with Grandparents&\textit{0.88**} & \textit{0.66**} & 0.40** & \textit{0.88**} & -0.36\\
  Unigram&0.38** & 0.35** & 0.14 & 0.19 & -0.04\\
  Dependencies, MaltParser trained by Timbl&\textit{0.44**} & \textit{0.41**} & 0.16 & \textit{0.39*} & -0.30\\
  Dependency, Arc Labels&0.20 & 0.28* & -0.00 & 0.09 & -0.28\\
  All Features Combined&\textit{0.58**} & \textit{0.48**} & 0.21 & \textit{0.47**} & -0.31\\
\end{tabular}
 \caption{Size correlation for sample size 1000, 5 normalizations}
 \label{size-cor-5-1000}
\end{table}

\begin{table}
\begin{tabular}{l|rrrrr}
& $R$ & $R^2$ & KL & JS & cos  \\ \hline
  Leaf-Ancestor&-0.55 & -0.38 & -0.26 & -0.53 & -0.17\\
  Trigram&-0.29 & -0.27 & -0.19 & -0.26 & -0.14\\
  Dependency&-0.61 & -0.43 & -0.27 & -0.58 & -0.18\\
  Phrase-Structure Rules&-0.21 & -0.08 & -0.04 & -0.22 & -0.14\\
  Phrase-Structure with Grandparents&-0.24 & -0.08 & -0.03 & -0.26 & -0.14\\
  Unigram&-0.38 & -0.25 & -0.30 & -0.32 & -0.08\\
  Dependencies, MaltParser trained by Timbl&-0.52 & -0.33 & -0.20 & -0.51 & -0.15\\
  Dependency, Arc Labels&-0.59 & -0.45 & -0.33 & -0.54 & -0.20\\
  All Features Combined&-0.61 & -0.44 & -0.26 & -0.55 & -0.18\\
\end{tabular}
\caption{Size correlation for complete corpora, 5 normalizations}
\label{size-cor-5-full}
\end{table}

A large number of parameter settings that include 5 iterations of
normalization correlate with corpus size. However, corpus size also
correlates with geographic distance at a rate of 0.31 for $p < 0.01$
and travel distance at a rate of 0.32 for $p < 0.01$. This makes the
conclusion that these distances are invalid difficult to defend. Therefore,
results with 5 normalizations will be presented in the rest of this
chapter.

\subsubsection{Analysis}

The correlation of corpus size and dialect distance is a problem. It
is not a predicted as a side effect of the way dialect distance is
measured. The fact that travel distance also correlates with corpus
size at a rate of 0.32 confuses the issue further. Is corpus size the
determining variable? Or is there an unknown variable influencing all
three? Some possibilities are ``interviewer boundaries'', common in
corpora collected by multiple people \cite{chambers98}, or perhaps the
interviewers got better over time and collected longer interviews as
they moved throughout the country, or perhaps cultural differences
between the interviewer and interviewees caused some participants in
one area to talk more than in another area.

The high correlations between corpus size and the 5-normalization
distances are definitely worrying. They are so much higher than the
correlation of corpus size and travel distance that 5-normalized
distances might not be reliable.
% It appears that multiple rounds of
% normalization inadvertently re-introduce a dependency on size.
% TODO: This probably IS a bug in that only Fred norm can be
% iterated. Ratio norm should probably be in a separate loop like so:
% #ifdef RATIO_NORM
%   for(sample::iterator i = ab.begin(); i!=ab.end(); i++) {
%     i->second.first *= 2 * types / tokens;
%     i->second.second *= 2 * types / tokens;
%   }
% #endif
However, if 5-normalization introduces a dependency on corpus size,
then the distances from full-corpus comparisons should correlate even
more highly. This is not the case.

% TODO: I also should write this up when I have time
Alternatively, it is possible that the fixed-size sampling method is not doing its
job in eliminating size differences between corpora. Future work
should develop a method for normalizing a comparison between two full
corpora. It should avoid sampling, but also take the relative number
of sentences into account. It is not difficult to come up
with a simple method to do so, but it needs some checking to make sure
that the method is valid.

\section{Clusters}

Cluster dendrograms provide a visualization of which regions the
distance measures find to be similar.  Clusters answer the question of
whether $R$ is useful for dialectometry more precisely than correlation by
inducing grouping regions. These groups can be compared to
regions proposed by syntactic dialectology.

Within the same settings for sampling and number of normalization
iterations, the clusters based on sentence-length normalization alone are fairly
similar, regardless of measure and feature set. Changing the sampling
settings or the number of normalizations substantial reconfiguration.

For example, the clusters produced by $R$ (figure
\ref{cluster-1-r-trigram}) and Jensen-Shannon divergence are fairly
similar (figure \ref{cluster-1-js-trigram}). Both are based on trigram
features with sentence-length normalization only. Those dendrograms
differ from their 5-normalized equivalents, figures
\ref{cluster-5-r-trigram} and \ref{cluster-5-js-trigram}.

\begin{figure}
  \includegraphics[width=0.9\textwidth]{dist-1-1000-r-trigram-ratio-clusterward}
 \caption{Dendrogram With $R$
    measure and trigram features, 1 normalization, 1000 samples}
  \label{cluster-1-r-trigram}
\end{figure}

\begin{figure}
  \includegraphics[width=0.9\textwidth]{dist-1-1000-js-trigram-ratio-clusterward}
 \caption{Dendrogram With Jensen-Shannon
    measure and trigram features, 1 normalization, 1000 samples}
  \label{cluster-1-js-trigram}
\end{figure}

\begin{figure}
  \includegraphics[width=0.9\textwidth]{dist-1-full-r_sq-psg-ratio-clusterward}
 \caption{Dendrogram With $R^2$ measure and phrase-structure-rule features,
 1 normalization, complete corpora}
  \label{cluster-1-r_sq-psg}
\end{figure}


\begin{figure}
  \includegraphics[width=0.9\textwidth]{dist-5-1000-r-trigram-ratio-clusterward}
 \caption{Dendrogram With $R$ measure and trigram features, 5 normalizations, 1000 samples}
  \label{cluster-5-r-trigram}
\end{figure}

\begin{figure}
  \includegraphics[width=0.9\textwidth]{dist-5-1000-js-trigram-ratio-clusterward}
 \caption{Dendrogram With Jensen-Shannon
    measure and trigram features, 5 normalizations, 1000 samples}
  \label{cluster-5-js-trigram}
\end{figure}

The highest correlation of 1-normalized distances with travel
distance, 0.37, is given by $R^2$ measured over phrase-structure-rule
features, comparing full corpora. Those parameter settings produce the
dendrogram in figure \ref{cluster-1-r_sq-psg}. The highest
correlation of 5-normalized distances with travel distance, 0.26, is
given by the Jensen-Shannon measure and trigram features, comparing
1000-sentence samples of corpora. Its dendrogram is in figure
\ref{cluster-5-js-trigram}.

Unlike the significances, cosine similarity's dendrograms are fairly
similar to those of other features. See for example figure
\ref{cluster-5-cos-trigram}, with cosine, trigram features and
5 iterations of normalization.

\begin{figure}
 \includegraphics[width=0.9\textwidth]{dist-5-1000-cos-trigram-ratio-clusterward}
 \caption{Dendrogram with cosine measure and trigram features, 5
   normalizations}
  \label{cluster-5-cos-trigram}
\end{figure}


\subsection{Consensus Trees}

Consensus trees combine the results of cluster dendrograms, which
avoids the dendrograms' problem of instability, where small changes
in distances cause large re-arrangements in the tree. Only dendrograms
whose input distances were at least 95\% significant were used. That is, a
measure/feature set combination had to be non-bold in tables
\ref{sig-1-1000} to \ref{sig-5-full} to be included. The consensus
tree for full-corpora comparisons and 5 rounds of normalization is not
given because there is only one dendrogram that qualifies.

In addition, more dendrograms were used to build
the consensus tree of figure \ref{consensus-5-1000} than were used in
figures \ref{consensus-1-1000} and \ref{consensus-1-full}. Despite this, figure
\ref{consensus-5-1000} retains much more detail, indicating that its
constituent dendrograms, based on 5 rounds of normalization,
agree more than those with only 1 round of normalization.

The consensus trees are also grouped into clusters, which are then
mapped in figures \ref{map-consensus-1-1000} --
\ref{map-consensus-5-1000}. The maps of Sweden were provided by
Therese Leinonen and are the same those in \namecite{leinonen08}.  The
outline maps are used by permission of Therese Leinonen. The
multi-dimensional scaling and composite cluster maps were both
generated by the L04 package from the University of Groningen.
% TODO: CITE this, I think it's a Pieter Kliuweeg paper


\begin{figure}
\includegraphics[scale=0.7]{consensus-1-1000}
% \Tree[. {Villberga\\Viby\\Vaxtorp\\Torso\\Tors\.as\\StAnna\\Sproge\\Sorunda\\Skinnskatteberg\\Segerstad\\Ossjo\\Orust\\Norra Rorum\\Loderup\\Leksand\\K\"ola\\Jamshog\\Indal\\Frillesas\\Fole\\Faro\\Bredsatra\\Boda\\Bara\\Asby\\Arsunda\\Anundsjo\\Ankarsrum} [. {Floby\\Bengtsfors}  ] ]
\caption{Consensus Tree for 1000-samples and 1 normalization}
\label{consensus-1-1000}
\end{figure}

\begin{figure}
\includegraphics[scale=0.7]{consensus-1-full}
% \Tree[. {Villberga\\Viby\\Torso\\Tors\.as\\Sorunda\\Segerstad\\Ossjo\\Orust\\Norra Rorum\\Loderup\\Leksand\\K\"ola\\Indal\\Fole\\Boda\\Bara\\Asby\\Arsunda\\Anundsjo\\Ankarsrum}
%   [. {Vaxtorp\\Skinnskatteberg}  ]
%   [. {StAnna\\Frillesas}  ]
%   [. {Sproge\\Faro}  ]
%   [. {Jamshog\\Bredsatra}  ]
%   [. {Floby\\Bengtsfors}  ] ]
\caption{Consensus Tree for full corpus comparison and 1 normalization}
\label{consensus-1-full}
\end{figure}

\begin{figure}
\includegraphics[scale=0.7]{consensus-5-1000}
% \Tree[. {Villberga\\Viby\\Vaxtorp\\Torso\\StAnna\\Sproge\\Sorunda\\Skinnskatteberg\\Segerstad\\Orust\\Norra Rorum\\Leksand\\K\"ola\\Indal\\Frillesas\\Fole\\Floby\\Faro\\Boda\\Bengtsfors\\Bara\\Asby\\Arsunda\\Anundsjo\\Ankarsrum} [. {Loderup\\Bredsatra}  ] [. {Tors\.as\\Ossjo\\Jamshog}  ] ]
\caption{Consensus Tree for 1000-samples and 5 normalizations}
\label{consensus-5-1000}
\end{figure}

\begin{figure}
\includegraphics[scale=0.85]{Sverigekarta-Landskap-consensus-1-1000}
\caption{Consensus Tree for 1000-samples and 1 normalization, Mapped}
\label{map-consensus-1-1000}
\end{figure}

\begin{figure}
\includegraphics[scale=0.85]{Sverigekarta-Landskap-consensus-1-full}
\caption{Consensus Tree for full corpus comparison and 1 normalization, Mapped}
\label{map-consensus-1-full}
\end{figure}

\begin{figure}
\includegraphics[scale=0.85]{Sverigekarta-Landskap-consensus-5-1000}
\caption{Consensus Tree for 1000-samples and 5 normalizations, Mapped}
\label{map-consensus-5-1000}
\end{figure}

% It would still be cool to eliminate only the non-significant distances
% and re-run the clusters. (I can't remember if that's easily possible
% with R though, it may only be a feature of MDS.)

% TODO: Try these two again, excluding cosine. Because I believe cosine sucks
% or at least is a Rogue Element.
% Later: Probably not worth it.

% Just the ratio ones that are significantly correlated with travel
% distance.
% However: This is even more of a mess than the freq results.
% [. {s0} [. {}
%     [. {} [. {K�la} [. {Ossjo} [. {Tors�s\\Jamshog}  ] ] ]
%           [. {Villberga\\Viby\\Torso\\StAnna\\Sorunda\\Norra Rorum\\Frillesas\\Boda\\Bara}
%              [. {Loderup\\Bredsatra}  ] ] ]
%     [. {Orust\\Leksand\\Indal\\Fole\\Faro\\Asby\\Arsunda\\Anundsjo}
%        [. {Vaxtorp\\Skinnskatteberg}  ]
%        [. {Ankarsrum}
%           [. {Segerstad} 
%              [. {Bengtsfors}
%                 [. {Sproge\\Floby}  ] ] ] ] ] ] ]

\subsubsection{Analysis}

The cluster dendrograms are dangerous
to interpret too closely on their own; the instability of a single
dendrogram means that small clusters cannot be analyzed reliably. For
example, in figure \ref{cluster-5-r-trigram}, a two-way split
between the regions on the top and bottom of the page is
obvious, and a three-way split is easy to argue for, but outliers like
Anundsj\"o and \.Arsunda are likely to shift from group to group in
other dendrograms.

It is safer to analyze the consensus trees; the smoothing effect of
taking the majority rule of each cluster will show where the optimal
cutoff for splitting clusters is. The three consensus trees in figures
\ref{consensus-1-1000} -- \ref{consensus-5-1000} vary in amount of
detail but share most details.

For 1000-sentence samples and 1 round of normalization, there is one
cluster: Floby and Bengtsfors. Full-corpus comparison finds
another cluster: J\"amshog, \"Ossj\"o and Tors\.as. Finally,
1000-sentence samples and 5 rounds of normalization finds another
cluster consisting of L\"oderup and Breds\"atra. It also finds
a large two-way split between the regions and adds Sproge to the first
cluster with Floby and Bengtsfors. To aid further analysis, the
clusters are assigned colors, which are detailed in figures
\ref{blue-cluster} -- \ref{orange-cluster}. 

\begin{figure}
\begin{itemize}
\item Floby
\item Bengtsfors
\item Sproge (for 1000-sample, 5-normalization)
\end{itemize}
\caption{Blue Cluster}
\label{blue-cluster}
\end{figure}

\begin{figure}
\begin{itemize}
\item J\"amsh\"og
\item Tors\.as
\item \"Ossj\"o
\end{itemize}
\caption{Red Cluster}
\label{red-cluster}
\end{figure}

\begin{figure}
\begin{itemize}
\item Breds\"atra
\item L\"oderup
\end{itemize}
\caption{Yellow Cluster}
\label{yellow-cluster}
\end{figure}

\begin{figure}
\begin{itemize}
\item Leksand
\item Indal
\item Segerstad
\item Floby
\item Bengtsfors
\item Sproge
\item Skinnskatteberg
\item Orust
\item V\.axtorp
\item F\.ar\"o
\item Asby
\item \.Arsunda
\item Anundsj\"o
\item Ankarsrum
\item Fole
\end{itemize}
\caption{Cyan Cluster}
\label{cyan-cluster}
\end{figure}

\begin{figure}
\begin{itemize}
\item Viby
\item Bara
\item S:t Anna
\item Frilles\.as
\item J\"amshog
\item Tors\.as
\item \"Ossj\"o
\item K\"ola
\item L\"oderup
\item Breds\"atra
\item Villberga
\item Tors\"o
\item Norra R\"orum
\item Sorunda
\item B\"oda
\end{itemize}
\caption{Orange Cluster}
\label{orange-cluster}
\end{figure}

When these clusters are mapped onto the geography of Sweden, some
patterns are visible. Since figure \ref{consensus-5-1000} is strictly
more complex than the preceding two, it is used as the basis for this
analysis--see map \ref{map-consensus-5-1000}. The large two-way split
is between the orange and cyan clusters. The orange cluster, which
includes red and yellow clusters, forms two horizontal bands across
Sweden. The centers of the orange cluster appear to be Stockholm and
Malm\"o. Meanwhile, the red and yellow clusters form a boundary along
the northern border of Sk\.ane and Blekinge counties.

Meanwhile, the cyan cluster, which includes the blue cluster, seems to
represent the countryside of Sweden. On the other hand, because the
blue cluster is near G\"oteborg, it might be better characterized
simply as ``non-Stockholm''.


\subsection{Composite Cluster Maps}

Composite cluster maps use an underlying technique similar to
consensus trees--cluster dendrograms, but they combine and present the
information in a very different way. They, too, provide a stabler view
of the groups that regions form when clustered. This view, however,
emphasizes the boundaries between regions. The result looks
much more like the traditional isogloss boundaries of
dialectology.

The two composite cluster maps in figures \ref{map-composite-1-1000}
-- \ref{map-composite-5-1000} are the composite of the same
dendrograms used as input for the consensus trees: all-significant
parameter settings, divided by type of normalization (sentence-length
only or ratio added as well).

\begin{figure}
\includegraphics[scale=0.82]{Sverigekarta-cluster-1-1000}
\caption{Composite Cluster Map for 1000-sample, 1 normalization}
\label{map-composite-1-1000}
\end{figure}

\begin{figure}
\includegraphics[scale=0.82]{Sverigekarta-cluster-1-full}
\caption{Composite Cluster Map for complete corpora, 1 normalization}
\label{map-composite-1-full}
\end{figure}

\begin{figure}
\includegraphics[scale=0.82]{Sverigekarta-cluster-5-1000}
\caption{Composite Cluster Map for complete corpora, 5 normalizations}
\label{map-composite-5-1000}
\end{figure}

All three composite clusters maps provide a picture similar to the
consensus tree map \ref{map-consensus-5-1000} of the previous
section. The north-to-south gradient is supported by the
weak horizontal boundaries present up and down Sweden.

Of these boundaries, the one between Sk\.ane and the rest of Sweden is
the strongest. Due to the lack of interview sites in the middle of
south Sweden, the boundary is drawn further north than it
traditionally appears, but this is an effect of the software the
produced the figure. Notice that there is also a boundary between
J\"amshog, Tors\.as, and \"Ossjo\" and the other sites, especially
visible in maps\ref{map-composite-1-1000} and
\ref{map-composite-5-1000}. Their presence along the northern border
of Sk\.ane is one reason why its boundary with the rest of Sweden is
so strong.

\begin{sloppypar}
  Compared to the consensus tree maps, the composite cluster maps
  cannot support the city/country distinction because there is no way
  to identify distant areas by their color. On the other hand, it is
  possible to detect the relative strength of a boundary. To combine
  these two features, multi-dimensional scaling is needed.
\end{sloppypar}
% But of course MDS maps can't be combined into a consensus...

% However, K\"ola and Frilles\.as still separate fairly well from their
% neighbors. These sites are on the edges of the country and have strong borders
% with surrounding, Like the cluster J\"amshog, Tors\.as and \"Ossj\"o,
% these sites are different from the others. However, they don't have
% any geographic coherence, so it is more likely these are remnants of a
% dialect that was historically wider spread and has since receded.


\section{Multi-Dimensional Scaling}

\begin{sloppypar}
  Multi-dimensional scaling (MDS) plays a similar role to clusters,
  condensing the high-dimensional information into an
  easier-to-understand form.  It differs, however, in producing
  gradient numbers, not binary trees: only enough scaling is done to
  produce the desired number of dimensions, and there is no exclusive
  membership in a single group. This also means that MDS maps are more
  stable than dendrograms.  Also, because of the way that the
  3-dimensional points map to colors, the maps vary. However, they
  are still comparable: if two regions are blue in one map and both
  are orange in another, then they have the same relation to each
  other.
\end{sloppypar}


The maps shown here in figures \ref{mds-1-1000-js-trigram} --
\ref{mds-5-1000-js-trigram} are from the same parameter settings as the
dendrograms, with the addition of the all-combined feature set. This
provides a combined MDS view somewhat analogous to the consensus trees or
composite clusters for the cluster dendrograms.

\begin{figure}
\includegraphics[scale=0.82]{Sverigekarta-mds-1-1000-r-trigram-ratio}
\caption{$R$ measure with trigram features, 1000-sentence sampling and
  1 round of normalization}
\label{mds-1-1000-r-trigram}
\end{figure}

\begin{figure}
\includegraphics[scale=0.82]{Sverigekarta-mds-1-1000-js-trigram-ratio}
\caption{Jensen-Shannon measure with trigram features, 1000-sentence sampling and
  1 round of normalization}
\label{mds-1-1000-js-trigram}
\end{figure}

\begin{figure}
\includegraphics[scale=0.82]{Sverigekarta-mds-1-full-r-trigram-ratio}
\caption{$R$ measure with trigram features, full-corpus comparison and
  1 round of normalization}
\label{mds-1-full-r-trigram}
\end{figure}

\begin{figure}
\includegraphics[scale=0.82]{Sverigekarta-mds-1-full-r_sq-psg-ratio}
\caption{$R^2$ measure with phrase-structure-rule features, full-corpus comparison and
  1 round of normalization}
\label{mds-1-full-r_sq-psg}
\end{figure}

\begin{figure}
\includegraphics[scale=0.82]{Sverigekarta-mds-5-1000-r-trigram-ratio}
\caption{$R$ measure with trigram features, 1000-sentence sampling and
  5 rounds of normalization}
\label{mds-5-1000-r-trigram}
\end{figure}

\begin{figure}
\includegraphics[scale=0.82]{Sverigekarta-mds-5-1000-js-trigram-ratio}
\caption{Jensen-Shannon measure with trigram features, 1000-sentence sampling and
  5 rounds of normalization}
\label{mds-5-1000-js-trigram}
\end{figure}

Despite the differences between MDS and the preceding methods, the
similar results are evident; the maps (figures
\ref{mds-1-1000-r-trigram} -- \ref{mds-5-1000-js-trigram}) all show
the same patterns as the other methods. That is, there is a general
north-to-south gradience, especially easy to see in map
\ref{mds-1-1000-js-trigram}. There is a strong southern cluster,
visible in all of the diagrams. And there is a general two-way
distinction between city and country.

The main contribution that the MDS maps make is that the
north-to-south gradient is more obviously gradient. In other words, it
is easier to see the gradation from north to south. For example, in
figure \ref{mds-1-full-r_sq-psg}, looking from the north to south, the
colors change quickly close to Stockholm, then fade to green further
south, then transition back to blues and purples further south, in
Sk\.ane.

The Stockholm and Malm\"o areas, which are in the same cluster in the
consensus tree maps, are here seen to be similar without being
identical. For example, in figure \ref{mds-5-1000-js-trigram}, the
Stockholm area is a shade of blue-green while the Malm\"o area is a
shade of blue-grey. Also in figure \ref{mds-5-1000-js-trigram},
Sk\.ane and Blekinge are grey: clearly similar but not identical to
Malm\"o.

\section{Features}

Ranked features answer the question of agreement with dialectology
more precisely than the previous two sections. Feature ranking has two
advantages in precision: first, it can reveal aggregate differences that may
not be noticeable without counting a large corpus; second, it can
point out rare features that only occur in one kind of corpus. The
first kind of features are unlikely to be noticed by linguists without
the aid of computers, whereas the second kind are the rare features
that are easy for linguists to notice.

There are two sets of rankings on display here; the first set is the
previous normalization for sentence size, whereas the second is
normalized for relative overuse, based on \quotecite{wiersma09}
normalization.  Without the overuse normalization, the top-ranked
features will tend to be the most common ones, those found in almost
every sentence in the interview. These common features tend to
highlight gradient differences: differences in quantity but not in
quality. In contrast, the overuse normalization allows us to see which
features happen only a few times in one side of the comparison and not
at all in the other. This is closer to a traditional linguistic
analysis.

In addition, only features that appear in both groups were ranked;
although features that only appear in one or the other can be
interesting, they tend to be noisy in features extracted from
automatically annotated corpora. It is not possible to tell which
unique features are interesting and which are noise, especially when
using the overuse normalization, which makes rarely occurring features
rank similarly to common ones.

These results compare clusters from the consensus trees
based on 1 round of normalization (figures \ref{consensus-1-1000} and
\ref{consensus-1-full}) as well as the consensus tree based on 5
rounds of normalization and a 1000-sentence sample (figure
\ref{consensus-5-1000}. The consensus tree for 5 rounds or
normalization and full-corpus comparisons only had one tree for input
and was not usable. Given these three consensus trees, the groups in
table \ref{feature-ranking-clusters} are the relevant ones for analysis.

There are four clusters, three small and one large which contains the remainder of
the sites. Cluster A, containing Floby and Bengtsfors, appears in all
three consensus trees. Its features are colored blue in the following
figures. Cluster B, containing Jamshog, Torsas and Ossjo, appears in the
second two trees. It features are colored red. Cluster C, containing
Loderup and Bredsatra, appears only in the third tree. Its features
are colored yellow. The remainder of the sites are in Cluster D;
the third consensus tree differs from the first two in splitting the
remainder into two groups, but this division is ignored here to reduce
the number of comparisons. Between large groups of sites, such
comparisons are unlikely to be informative anyway.

\begin{table}
  \begin{enumerate}
   \item[A] (Blue) Floby, Bengtsfors
    \item[B] (Red) J\"amshog, \"Ossj\"o, Tors\.as
    \item[C] (Yellow) L\"oderup, Breds\"atra
    \item[D] (Cyan) Segerstad, K\"ola, S:t Anna, Sorunda, Norra Rorum,
      Villberga, Torso, Boda, Frilles\.as, Indal, Leksand, Anundsj\"o,
      \.Arsunda, Asby, Orust, V\.axtorp, Fole, Sproge, F\.ar\"o,
      Ankarsrum, Skinnskatteberg
  \end{enumerate}
  \caption{Clusters discussed}
  \label{feature-ranking-clusters}
\end{table}

For each pair of clusters, I rank and analyze the input features by
comparing feature differences. The features presented here are the ten
highest ranked features for a particular comparison. Although each
feature set has ten features ranked here, they are better thought of
as two sets of five features differences. The top five positive
features are shown as are the top five negative features.

This has two advantages. It splits the features so that both the
positive and negative evidence are always visible; otherwise, in some
cases, if one side is strong enough, the other would be pushed out of
the top ten. However, it still allows the relative weight of evidence
to be estimated. For example, if some cluster has some idiosyncratic
features, most of the features will be positive, meaning that most of
the distance comes from features typical of this cluster. The two-part
feature will show this: the five positive features will have much
higher values than the five negative features.


The first subsection, \ref{feature-ranking-complete}, shows all
comparisons between regions for a single parameter setting: trigram
features, 1000-sentence sampling and sentence-size normalization
only. Besides unigrams, these are the parameters that give the highest
correlation with travel distance for 1000-sentence sampling.

In the next subsection, \ref{feature-ranking-overuse}, the overuse
normalization is added, keeping other parameter settings the same.

The third subsection, \ref{feature-ranking-feature-sets},
a single comparison between cluster A and cluster B is given for
all feature sets.

In the final subsection, \ref{feature-ranking-psg}, the high-ranked
phrase-structure rules are given.

The most common parts of speech are given below. The complete list is
given in Appendix X. (TODO: Move most of this list to an
appendix).

\begin{itemize}
\item $++$ = coordinating conjunction
\item AB = adverb
\item AJ = adjective
\item AN = adjectival noun
\item AV = verb ``vara'' (be)
\item BV = verb ``bli(va)'' (become) %
% \item EH = hesitation
\item EN = indefinite article
\item FV = verb ``f\.a'' (get) %
\item GV = verb ``g\"ora'' (do) %
\item HV = verb ``hava'' (have) %
\item I? = question mark %
\item IC = quotation mark %
\item ID = idiom
\item IG = other punctuation mark %
\item IK = comma, correction
\item IM = infinitive marker
\item IP = period
% \item IQ = colon
% \item IR = parenthesis
% \item IS = semicolon
% \item IT = dash
\item IU = exclamation mark
\item KV = ``komma at'' (periphrastic future)
\item MN = meta-noun
\item MV = verb ``m\.aste'' (must)
\item NJ = falling juncture
\item NN = noun
\item PN = proper name
\item PO = pronoun
\item PR = preposition
\item PU = list item
% \item QQ = ?
\item QV = verb ``kunna'' (can)
% \item RJ = level juncture
\item RO = numeral
\item SP = present participle
\item SV = verb ``skola'' (shall)
\item TP = perfect participle
% \item UJ = rising juncture
\item UK = subordinating conjunction
% \item UU = exclamation
\item VN = verbal noun
\item VV = other verb
\item WV = verb ``vilja'' (want)
\item YY = Interjection
\item XX = Unclassifiable
\end{itemize}

%%% END FORK

\subsection{Trigram Features}
\label{feature-ranking-complete}

The analysis will start with trigram features without the overuse
normalization, since trigrams have the highest rate of significance of
the non-combined feature sets. (The combined feature set is not
presented because the mixed feature types make it difficult to read.)

As mentioned above, the top-ranked trigrams are common, typical of the
core of the sentence. Cluster A's typical trigrams, for example,
typically involve a pronoun or a verb or both: PO-AV-AB
(pronoun-copula-adverb), $++$-PO-AV (conjunction-pronoun-copula) and
PO-VV-AB (pronoun-verb-adverb). The same is of the other clusters for
the most part. Unfortunately, this makes it hard to say interesting
things about the difference in feature distribution. It does appear
that clusters B and C have heavier use of adverbs and of
conjunctions. The comparison between cluster A and cluster B even
highlights the trigram AB-AB-AB as important.

\begin{figure}
  \includegraphics[scale=1.2]{clusterA-clusterB-feat-5-1000-trigram-ratio}
  \caption{cluster A $\Leftrightarrow$ cluster B, trigram features}
\end{figure}
\begin{figure}
  \includegraphics[scale=1.2]{clusterA-clusterC-feat-5-1000-trigram-ratio}
  \caption{cluster A $\Leftrightarrow$ cluster C, trigram features}
\end{figure}
\begin{figure}
  \includegraphics[scale=1.2]{clusterA-clusterD-feat-5-1000-trigram-ratio}
  \caption{cluster A $\Leftrightarrow$ cluster D, trigram features}
\end{figure}


\begin{figure}
  \includegraphics[scale=1.2]{clusterB-clusterC-feat-5-1000-trigram-ratio}
  \caption{cluster B $\Leftrightarrow$ cluster C, trigram features}
\end{figure}
\begin{figure}
  \includegraphics[scale=1.2]{clusterB-clusterD-feat-5-1000-trigram-ratio}
  \caption{cluster B $\Leftrightarrow$ cluster D, trigram features}
\end{figure}
\begin{figure}
  \includegraphics[scale=1.2]{clusterC-clusterD-feat-5-1000-trigram-ratio}
  \caption{cluster C $\Leftrightarrow$ cluster D, trigram features}
\end{figure}

\subsection{Trigrams with Overuse Normalization}
\label{feature-ranking-overuse}

Given this lack of information, there are two dimensions along which
the comparisons can be altered: normalization and feature
set. Starting with normalization, let us add the overuse normalization
technique. Differences appear immediately. First, the balance of
feature weight obviously differs here. For example, in the comparison
between cluster A and cluster B, the features of cluster A are more
important in distinguishing the two than the features of cluster
B. The comparison between cluster A and cluster D is so lop-sided that
cluster D contributes no features at all.

With the overuse normalization, cluster A has two interesting
patterns. First, the trigrams it overuses are filled with indefinite
articles (EN) and prepositions (PR). Examples include VV-EN-AB
(verb-indefinite-adverb), PR-EN-AB (preposition-indefinite-adverb) and
PR-EN-VN (preposition-indefinite-verbal noun), as well as IM-PR-NN
(infinitive marker-preposition-noun) and PR-ID-PR
(preposition-idiom-preposition). Second, the trigrams it underuses
mostly end with pronouns: 4 of 5 trigrams in the comparison with
cluster B and 4 or 5 in the comparison with cluster C. Even in the
comparison with cluster D, 4 of 5 of the ``least overused'' trigrams
end with pronouns. (The low values in the bottom half of the
comparison with cluster D are not underused by cluster A, because
cluster D has no unique features here. Instead they are the ``least
overused'' by cluster A.)

Cluster B shows one interesting pattern: overuse of sk\"ola (shall),
including an interesting trigram SV-QV-AB (shall verb-can
verb-adverb). Although this could be a mistake on the part of the
tagger, the different forms of this verb are limited, so this is
unlikely: identifying them is not hard. Instead it points to the
possibility of double modals.
%% a quick search suggests that Fennell and Butters (1996) finds
%% evidence in German and Scandinavian languages...but it's a book ro
%% something. Google Scholar has no link, just a wimpy citation.
%% Also:
%% Modals and double modals in the Scandinavian languages
%% Working papers in Scandinavian syntax
%% Thrainsson and Vikner 1995 (but focussing on Danish and Icelandic)

Cluster C doesn't gain any interesting patterns with overuse
normalization except for a surprising variety in the verbs: g\"ora
(do), hava (have), kunna (get), sk\"ola (shall), vara (be) and vilja
(want). Many uses of adverbs show up as well. It is not clear what
either of these patterns mean linguistically, however.
% I have no idea whether to make that verb singular or plural.
% So like whatever.

Cluster D gives no information whatsoever when the overuse
normalization is added, simply because it has no informative
features. This is expected, given its nature as a combination of many
sites. The tradeoff of more informative features for the smaller
clusters is worthwhile.

\begin{figure}
  \includegraphics[scale=1.2]{clusterA-clusterB-feat-5-1000-trigram-over}
  \caption{cluster A $\Leftrightarrow$ cluster B, trigram features
    with overuse normalization}
\end{figure}
\begin{figure}
  \includegraphics[scale=1.2]{clusterA-clusterC-feat-5-1000-trigram-over}
  \caption{cluster A $\Leftrightarrow$ cluster C, trigram features
    with overuse normalization}
\end{figure}
\begin{figure}
  \includegraphics[scale=1.2]{clusterA-clusterD-feat-5-1000-trigram-over}
  \caption{cluster A $\Leftrightarrow$ cluster D, trigram features
    with overuse normalization}
\end{figure}


\begin{figure}
  \includegraphics[scale=1.2]{clusterB-clusterC-feat-5-1000-trigram-over}
  \caption{cluster B $\Leftrightarrow$ cluster C, trigram features
    with overuse normalization}
\end{figure}
\begin{figure}
  \includegraphics[scale=1.2]{clusterB-clusterD-feat-5-1000-trigram-over}
  \caption{cluster B $\Leftrightarrow$ cluster D, trigram features
    with overuse normalization}
\end{figure}
\begin{figure}
  \includegraphics[scale=1.2]{clusterC-clusterD-feat-5-1000-trigram-over}
  \caption{cluster C $\Leftrightarrow$ cluster D, trigram features
    with overuse normalization}
\end{figure}

\subsection{Variation Across Feature Sets}
\label{feature-ranking-feature-sets}

Moving to other features sets with overuse normalization,
leaf-ancestor paths and leaf-head paths give additional information
about cluster A that lead to the conclusion its defining
characteristic is simple sentences, simpler at least than the other
clusters. Specifically, cluster A's overused leaf-ancestor paths
include few nested sentences. This contrasts sharply with cluster B
and cluster C, which include many nested sentences. Cluster A does
have complex paths, but they feature prepositional phrases. (Note: NAC
stands for ``not a constituent'' and indicates that the parser could
not decide what the correct constituent was at that point.) (Or that
there are crossing branches, which is less common.)

This characteristic of cluster A appears in the leaf-head paths as
well; cluster A's paths contain many [adjective]-noun-preposition
sequences, but few verb-verb sequences that indicate nested
phrases. Again, cluster B and cluster C have many of these
sequences. Both clusters have a number of overused adverb features as
well, similar to the trigram results. Note that comparison to cluster
D is less interesting. Because it has fewer unique characteristics,
when compared to it, clusters A, B and C show more generic
characteristics. For example, all three clusters show that their
sentences are generally more complex than the general sites in cluster D.

\begin{figure}
  \includegraphics[scale=1.2]{clusterA-clusterB-feat-5-1000-path-over}
  \caption{cluster A $\Leftrightarrow$ cluster B, leaf-ancestor path features}
\end{figure}
\begin{figure}
  \includegraphics[scale=1.2]{clusterA-clusterB-feat-5-1000-trigram-over}
  \caption{cluster A $\Leftrightarrow$ cluster B, trigram features}
\end{figure}
\begin{figure}
  \includegraphics[scale=1.2]{clusterA-clusterB-feat-5-1000-dep-over}
  \caption{cluster A $\Leftrightarrow$ cluster B, dependency features}
\end{figure}
\begin{figure}
  \includegraphics[scale=1.2]{clusterA-clusterB-feat-5-1000-psg-over}
  \caption{cluster A $\Leftrightarrow$ cluster B, phrase-structure
    rule features}
\end{figure}
\begin{figure}
  \includegraphics[scale=1.2]{clusterA-clusterB-feat-5-1000-grand-over}
  \caption{cluster A $\Leftrightarrow$ cluster B, phrase-structure
    rules features, with grandparent}
\end{figure}
\begin{figure}
  \includegraphics[scale=1.2]{clusterA-clusterB-feat-5-1000-unigram-over}
  \caption{cluster A $\Leftrightarrow$ cluster B, unigram features}
\end{figure}
\begin{figure}
  \includegraphics[scale=1.2]{clusterA-clusterB-feat-5-1000-redep-over}
  \caption{cluster A $\Leftrightarrow$ cluster B, dependency features,
MaltParser trained by Timbl}
\end{figure}
\begin{figure}
  \includegraphics[scale=1.2]{clusterA-clusterB-feat-5-1000-deparc-over}
  \caption{cluster A $\Leftrightarrow$ cluster B, dependency features
    with arc labels}
\end{figure}
\begin{figure}
  \includegraphics[scale=1.2]{clusterA-clusterB-feat-5-1000-all-over}
  \caption{cluster A $\Leftrightarrow$ cluster B, all combined features}
\end{figure}


\subsection{Phrase-structure rule features}
\label{feature-ranking-psg}

\begin{figure}
  \includegraphics[scale=1.2]{clusterA-clusterB-feat-5-1000-psg-ratio}
  \caption{cluster A $\Leftrightarrow$ cluster B, phrase-structure
    rule features}
\end{figure}
\begin{figure}
  \includegraphics[scale=1.2]{clusterA-clusterC-feat-5-1000-psg-ratio}
  \caption{cluster A $\Leftrightarrow$ cluster C, phrase-structure rule features}
\end{figure}
\begin{figure}
  \includegraphics[scale=1.2]{clusterA-clusterD-feat-5-1000-psg-ratio}
  \caption{cluster A $\Leftrightarrow$ cluster D, phrase-structure rule features}
\end{figure}


\begin{figure}
  \includegraphics[scale=1.2]{clusterB-clusterC-feat-5-1000-psg-ratio}
  \caption{cluster B $\Leftrightarrow$ cluster C, phrase-structure rule features}
\end{figure}
\begin{figure}
  \includegraphics[scale=1.2]{clusterB-clusterD-feat-5-1000-psg-ratio}
  \caption{cluster B $\Leftrightarrow$ cluster D, phrase-structure rule features}
\end{figure}
\begin{figure}
  \includegraphics[scale=1.2]{clusterC-clusterD-feat-5-1000-psg-ratio}
  \caption{cluster C $\Leftrightarrow$ cluster D, phrase-structure rule features}
\end{figure}

Analysis of the phrase-structure-rule features is difficult because of
all the noise. Features like S$\to++$-AB (conjunction-adverb)
S$\to$FV-PO-AB-VV (get verb-pronoun-adverb-verb) are hard to describe
as anything but junk rules created by the parser. On the other hand,
there are a lot of linguistically odd but reasonable rules like
S$\to$PO-AV-NP-IP (pronoun-copula-noun phrase-period), which makes a
certain kind of sense if you can be persuaded that copular sentences
are special enough to deserve their own rule. (Remember that
statistical parsers trained on interview data are particularly
susceptible to this kind of persuasion.)

Overall both normalizations leave something to be desired; without
overuse normalization, only very common features appear. These
features convey only basic information, making it hard to identify
characteristics of a cluster. On the other hand, the overuse
normalization is susceptible to noise, especially for more error-prone
feature sets. Even though more detail may be available with this
normalization step, the features must be inspected for general trends
because individual features are not necessarily reliable.


%%% Local Variables: 
%%% mode: latex
%%% TeX-master: "dissertation.tex"
%%% End: 


\chapter{Discussion}
\label{discussion-chapter}

This chapter compares the dissertation's results to three areas. It
compares the results to dialectology, starting with the traditional
dialect regions of Sweden and moving to individual dialect phenomena. Then it
compares the results to phonological dialectometry, which uses many of
the same analytical techniques on phonology data. Finally, it compares
this work to previous work in the field of syntactic dialectometry and
summarizes its improvements. The chapter ends with a summary of the
work and its contributions to dialectology at large and Swedish
dialectology in particular.

% A big question is why trigrams are so good. All of the fancier feature
% sets do worse than trigrams. I should address this in the summary for
% sure.

% The reason is that the simpler tags are easier to tag. So I recommend
% for the real-world analyses with only untagged transcriptions as input
% to simply use trigrams.

\section{Comparison to Syntactic Dialectology}
\label{discussion-chapter-dialectology-section}

The comparison to syntactic dialectology consists of three
sections. The first section looks at the general expectations of
dialectology with respect to correlation with geographic distance. The
second section compares the traditional dialect regions of Sweden to
the ones found by the statistical dialect measure. The third section
finishes by comparing specific phenomena of dialect regions to the
corresponding features from interview sites.

\subsection{General Expectations}

The default expectation of dialect distance is that it should
correlate with geographic distance, see \namecite{chambers98} and
\namecite{gooskens04a}. The principal places that geographic distance
fails to correlate with dialect distance are where dialect boundaries
that exist between adjacent sites; here, a small geographic distance
is paired with large dialect distance. For non-adjacent sites, in
contrast, a large geographical distance may be paired with a small
dialect distance. This can occur, for example, with relic dialects,
where the innovative dialect expands from the center, leaving similar
dialects isolated on the edges. However, neither of these cases holds
for the Scandinavian languages; \namecite{hallberg05} points out that
Swedish dialect areas form a continuous gradient without any strong
boundaries. This means a particularly strong correlation between
geography and dialect. Therefore, the first step is to compare the
correlation of geographic distance with dialect distance as measured
here.

Unfortunately, the correlations between geographic distance and
dialect distance are uniformly low, even when they do attain
significance. The highest correlation is 0.36. Correlating dialect
distance with travel distance rather than geographic distance gives
0.37, which is an improvement, albeit a small one. However, as
Gooskens point out, time and distance required to travel between two
points at the beginning of the 21st century is considerably less than
it was one hundred years ago or more. Measuring travel time between
sites at some point in the past as she does might provide an even
better correlation with dialect distance.

Nonetheless, the overall pattern agrees with Hallberg's analysis;
there is a north-to-south gradient that is fairly smooth; the
composite cluster maps (figure \ref{map-composite-5-1000} in chapter
\ref{results-chapter}, for example) show this pattern best, but the
consensus tree and MDS maps do as well. The exceptions to this
gradient are the areas surrounding Stockholm and Malm\"o, as well as
the whole of the southern provinces Sk\aa{}ne and Blekinge. It may be
that modern urbanization has created a city/country divide, with
Stockholm and Malm\"o innovating and the rural areas becoming relic
dialects. These two exceptions will be discussed more in the next
section.

\subsection{Dialect Regions}
% so which is it? A city/country divide or is just that the
% traditional areas were Right All Along.
% I guess upon reflection, it's probably the latter...

According to dialectology, Sweden does not have strong dialect
boundaries, but it still has some traditional dialect areas. However,
these are loosely defined and do not have sharp borders; the Eastern
area is centered around Stockholm, the Western around G\"oteborg, the
Southern around Malm\"o, and the Northern area covers the north of
Sweden. In addition, the island of Gotland forms a separate area. The
MDS maps and consensus tree maps reproduce these areas with varying
degrees of fidelity.

For example, in the consensus tree figure \ref{map-consensus-5-1000}, the
cyan cluster corresponds to the Northern and Western dialect areas,
the orange cluster corresponds to the Eastern area, and the red/yellow
cluster corresponds to the Southern. There is a question that arises
from this grouping, though; why should the northern and western areas
appear in the consensus tree as one group? It looks as if the
consensus tree map makes it more important that they differ from the
East and South than that the differ from each other. The MDS maps
reinforce this point; they show that the western sites and northern sites
do in fact differ quite a bit. However, because the eastern and
southern sites are so close, a clustering technique, like consensus
trees, with exclusive group membership will put distant sites in the
same group.

The boundary between the Sk\aa{}ne and Blekinge is quite abrupt,
presumably mirroring the former Danish border that existed until the
end of the Middle Ages. This contradicts Hallberg, who explicitly
mentions that dialectology research finds no border there, and
that the strongest north/south division more closely approximates
\quotecite{leinonen08} diagonal boundary in map
\ref{leinonen-factors-3-4} below.

There are three possible explanations for this: first, there could be
statistical, accumulative evidence which Swedish dialectologists have
missed; second, the distribution of Swediasyn interview sites may be
too sparse to reflect the real border; in particular, there are very
few in Sm\aa{}land; third, the dialect landscape may have changed
since the prevailing dialectology opinion was established.  The last
explanation is attractive, since the Swedia corpus is around 50 years
newer than newest dialectology studies. However, this is an old
boundary: it mirrors the Sweden-Denmark political border that existed
over 400 years ago. It would be odd for it to disappear for over 350
years and re-appear just before 2000. Instead, I believe the first
explanation is more likely: Leinonen's results, in addition to
reproducing the boundary described by Hallberg, also place a boundary
at the same location as these syntactic results. This boundary is
visible in factors 2 (figure \ref{leinonen-factors-1-2}) and 5 (figure
\ref{leinonen-factors-5-6}) of her factor analysis based on the
phonology data of Swedia, discussed below in section
\ref{discussion-chapter-phonological-dialectometry}. Both Leinonen's
method and mine are capable of detecting distributional patterns that
are difficult to see from manual analysis. For example, in the
previuos chapter, I showed that the trigrams AB-AV-AB, despite
appearing in all interview sites, was more common in central Swedish
cluster A.

% The second explanation may also be true, considering that Leinonen's
% results reproduce the boundary that Hallberg mentions, but it does not
% invalidate the boundary that the syntactic measure detects, since her
% results reproduce it as well.

% remember, these produce a
%   city/country divide, a boundary at Jamshog, and something kind of
%   near Goteborg.

\subsection{Dialect Features}

The literature for Swedish syntactic dialectology is not extensive,
largely because there is not much syntactic dialectology for any
language. As a result, I will compare my results to two papers,
\namecite{delsing03} and \namecite{rosenkvist07}. The first paper is a
survey of syntactic dialectology from the late 19th and early 20th
centuries. In the same volume, other papers analyze specific phenomena
in more detail; the survey is mostly concerned with the dialect differences
and distributions rather than the syntactic analysis. The second is an
analysis of the South Swedish Apparent Cleft.

\namecite{delsing03} surveys a number of dialectology studies. These
studies date from the height of the field in Sweden, from circa
1880--1930, which Delsing at times augments with modern data. It is
worth noting that the Swedia data in the comparison was collected
around 2000, so there were likely changes in the dialects in the
intervening 70--120 years. This is particularly true in the northern
dialect areas, where improved travel and communication have
leveled the dialects considerably \cite{hallberg05}.

However, comparison to the phenomena in the survey may still yield
interesting results, so for each
phenomenon I will start with a summary of the phenomenon for Swedish
dialects: its geographic distribution and its linguistic
realizations. Then I will match the geographic distribution with
Swediasyn interview sites and represent the phenomenon in terms of the
feature sets developed in this dissertation. For this initial
analysis, trigram features are used because they are simple. This
matches Delsing's survey descriptions, which are for the most part
surface-oriented; other papers in the same volume with his survey
analyze the phenomena in more detail.

With the target sites and features defined, it is straightforward to count the
number of occurrences of each feature in each site and compare the
two. If the predicted dialect phenomenon is reflected in the data,
then the sites associated with the phenomenon will have more
occurrences of the target features than the non-associated sites. This
difference is precisely what the distance measures use.

This method is inadequate for two reasons: first, the translation of
linguistic analysis to feature representation will not be perfect and
may miss some valid instances of the linguistic phenomenon. Second,
more importantly, the differences are not yet checked for statistical
significance. As such, the comparison can only be suggestive;
checking for statistical significance will have to wait for future
work.

% As an aside, much of this missing information IS available to me, so I
% could look manually. But none of it made it through to the distance
% measures, and this analysis compares the way the distance measures
% make the decisions with the way that linguists make their
% decisions. So I have to use only the information that the distance
% measures used.

The maps reproduced here are taken from Delsing's survey.

\subsubsection{``Partitive'' Article}

Northern Sweden uses the suffixed article much more than the rest of
Sweden. The reason, Delsing says, is that some uses of the suffixed
article are not definite in the north; they have a partitive function,
similar to the partitive article in French, which is not present in
the rest of the country. See figure \ref{partitive-article} for an
example.

\begin{figure}
  \gll H\"a finns vattne d\"ari hinken. \\
  Here found water-the in bucket-the \\
  \trans `There is water in the bucket.'
  \caption{Suffix marking for partitive}
  \label{partitive-article}
\end{figure}

Unfortunately, the part-of-speech tag set used for this dissertation
is quite coarse; it does not record whether nouns are marked with the
definite suffix. Therefore, there is no way for the distance measure
to tell the difference between suffixed dialect usage and bare
standard usage.

\subsubsection{Proper-Noun Articles}

\begin{figure}
  \includegraphics[scale=0.7]{dialektboka-karta3}
  \caption{Proper-Noun Articles}
  \label{indefinite-article-proper-noun-map}
\end{figure}

\begin{figure}
 \gll En Bjurstr\"om ha aff\"arn. \\
  A Bjurstr\"om has the-store. \\
  \trans `Bjurstr\"om has a store.'
  \caption{Indefinite Article for Proper Nouns: First Names}
  \label{indefinite-article-proper-noun}
\end{figure}

In Northern Scandinavia, first names are preceded by an indefinite
article, and sometimes last names as well. The indefinite article also
precedes kinship terms that are used as proper names, for example
``Mother'' or ``Grandfather''. An example is given in figure
\ref{indefinite-article-proper-noun}. Standard Swedish does not
include this feature. In Sweden, this feature is found along the
border with Norway as well as Northern Sweden. In the Swediasyn data, this
includes the interview sites K\"ola, Indal, and Anundsj\"o---the dark
area in figure \ref{indefinite-article-proper-noun-map}, there labeled
``Prepropriell artikel''.

Unlike the partitive article suffix, this feature is easy to detect
with a coarse part-of-speech tag set. Specifically, it can be represented as the
bigram EN-PN (indefinite article-proper noun), which can be used as a
search term in the trigram feature set. The same EN-PN sequence is
expected for leaf-head paths, since the indefinite article depends on
proper noun. The phrase-structure-rule features should
look something like NP$\to$EN-PN.

Occurrences of the EN-PN bigram in the trigram feature set for
Leksand, Indal and K\"ola agree with the linguistic analysis: a rate
of 0.00007 versus 0.00006. Unfortunately, this result cannot be
trusted because the rate of occurrence for both regions is so rare, as
well as so close between the two regions. The only conclusion that can
be drawn is that the hypothesis is not yet disproved.

\subsubsection{Possessives and the article}

\begin{figure}
  \includegraphics[scale=0.7]{dialektboka-karta4}
  \caption{Proper-Noun Articles}
  \label{possessive-plus-article-map}
\end{figure}

\begin{figure}
 \gll naboens den stribede kat \\
  Neighbors' the striped cat \\
  \trans `The neighbors' striped cat'
  \caption{Simultaneous possessive and determiner in noun phrase in
    Danish, and at one time Southwest Sweden}
  \label{possessive-plus-article-example}
\end{figure}

In Swedish, and in the other Scandinavian countries, there is a good
deal of variation in the handling of possessives with articles. In
Swedish, normally only one is allowed in a noun phrase: either a
possessive or a determiner, but not both. However, in Danish and the
Danish-influenced areas of Sweden, both are allowed in certain cases:
for example, when the possessive and determiner are separated from the
noun by an adjective. Delsing gives an example from Danish, shown here
in figure \ref{possessive-plus-article-example}.  This pattern also
exists in the southwest corner of Sweden, very near to Denmark. In
figure \ref{possessive-plus-article-map}, this area is shaded
left-to-right diagonally; it includes the interview site Bara. In
addition, this pattern alternates with the standard Swedish pattern on
the island of Gotland (cross-hatched on the map), which includes the
interview sites Fole, F\aa{}r\"o and Sproge.

This pattern can be detected by analyzing the per-site recall for the
4-grams PO-PO-AJ-NN, PR-PO-AJ-NN and NN-PO-AJ-NN. The first is the
sequence pronoun-pronoun-adjective-noun, for example {\it mitt det
  gamla huset} ``My the old house-the''. The second starts with a
proper name, such as {\it Pers} ``Per's'', and the third starts with a
noun, such as {\it naboens} ``neighbor's''. These three 4-tag sequence
can be encoded as trigrams by breaking them into two pieces. This
allows them to be searched for in that the distance measure would have
encountered them.

In addition to this pattern, there is a second in the north of
Sweden. Here, it is simply that possessive personal pronouns are
allowed both before and after the noun. This pattern includes the
interview sites Indal and Anundsj\"o and is covered in the next
section.

Searching Bara, in the southwest of Sweden, for the previously
mentioned trigram patterns does not find them: the rate of occurrence
is 0.00289 inside Bara but 0.000341 outside. It should be higher in
Bara. However, Delsing, writing in 2003, mentions that residents of
Sk\aa{}ne that he has asked do not recognize this form either, so it is
possible that it has fallen out of use in the 70 years or so since it
was last reported.

Executing a similar search for the alternation of standard
Swedish with the possessive pronoun pattern in the Gotland sites (F\aa{}r\"o,
Fole and Sproge), the standard Swedish trigrams PO-AJ-NN, PR-AJ-NN and
NN-AJ-NN show similar results: 0.00441 in Gotland, 0.00495 outside
Gotland. This is opposite the predicted direction.

The final region in figure \ref{possessive-plus-article-map}, in northern
Sweden, which includes Indal and Anundsj\"o, is actually more
complicated than can be captured by the part-of-speech tags used here;
this region allows possessive proper nouns to occur with
suffix-determiner nouns. But this can occur in either order: for
example, both ``Pers huset'' and ``huset Pers'' is allowed. Although
both ``Pers hus'' and ``Pers huset'' produce identical tags (PN-NN),
trigrams do encode order, so the unusual order in ``huset Pers'' can be
searched for. Since both orders should be present in this northern
area, it should overuse bigrams like NN-PN (noun-proper noun) relative
to the rest of Sweden.

Searching for the bigrams NN-PN (noun-proper noun) and NN-PO
(noun-pronoun) shows a usage rate of 0.02532 for Indal and Anundsj\"o
and a rate of 0.02438 for the rest of Sweden. This is the expected
direction, but the rate of usage is very similar between the two
regions. The comparison is really too close to make a prediction
because the difference is not likely to be significant.

% \subsubsection{Pronominal Possessives}

% In Swedish, as well all of mainland Scandinavian, another possessive
% construction is the reflexive genitive, which consists of a
% noun-reflexive-noun sequence. An example is given in figure
% \ref{genitive-reflexive-normal-example}. However, this construction
% does not allow pronouns: the sequence noun-reflexive-pronoun is not
% allowed (see figure \ref{genitive-reflexive-pronoun-example}).

% \begin{figure}
%   {\it Per sitt hus} \\
%   Per its house
% ``Per's house'' \\
% \caption{Standard Swedish genitive reflexive construction}
% \label{genitive-reflexive-normal-example}
% \end{figure}

% \begin{figure}
%   {\it han sitt hus} \\
%   his its house
% ``his house'' \\
% \caption{Pronominal genitive reflexive construction}
% \label{genitive-reflexive-pronoun-example}
% \end{figure}

% However, this construction is allowed in NORTHERN SWEDEN.
% Oops, actually I think this whole section is whole throwaway intro to
% something else. Boooooo.

% However, this is not allowed with possessive pronouns:
% *{\it han sitt hus}.

% The prepositional genitive
% behaves the same way: {\it huset till Per} ``Per's
% house'' (gloss: house-the of Per) is legal but *{\it huset till meg}
% ``my house'' (gloss: house-the of me) is not.

% There is an exception for kinship words, which I don't understand
% yet. But somehow ``far min'' is different (maybe just because it's not
% ``min far''?)

% So basically standard Swedish allows trigrams sequences like NN-PO-NN
% ({\it Per sitt hus}) but not PO-PO-NN ({\it han sitt hus}). It also
% allows sequence like NN-PR-NN ({\it huset till Per}) but not NN-PR-PO
% ({\it huset till meg}).

% Does not work (is too close to call): 0.02229 vs 0.2429

% Reversing the bigram, looking for PO-NN in the south gives
% Works (but is still super close): 0.04243 vs 0.03998

% It looks like one set just uses more nouns than the others or
% something. Conclusion: inconclusive, leaning toward no---it looks like
% they're the same.

\subsubsection{Proper Noun Possessives}

\begin{figure}
 \gll Huset hans Per \\
  The-house his Per \\
  \trans `Per's house'
  \caption{Possessive formed of Possessive Pronoun and Proper Noun}
  \label{proper-noun-post-possessive}
\end{figure}

\begin{figure}
  \includegraphics[scale=0.7]{dialektboka-karta6}
  \caption{Proper-Noun Possessives}
  \label{proper-noun-post-possessive-map}
\end{figure}

In addition to the post-nominal possessive pattern of the previous
section, there is a variant that is common in Norway. Here, the
sequence is noun-possessive pronoun-proper noun. An example
of this pattern is given in figure \ref{proper-noun-post-possessive}.

This pattern overlaps slightly into Sweden, covering the interview
site K\"ola. The distribution is given in figure
\ref{proper-noun-post-possessive-map}. Note that the northern area
with small stripes is the same as in figure
\ref{possessive-plus-article-map}, and the northern area with thin
stripes has no matching sites. The area of interest is the one with
larger, thick stripes that covers the majority of Norway.

This phenomenon maps to a trigram NN-PO-PN: noun-pronoun-proper
noun. The occurrence rate of this trigram in K\"ola to the rest of
Sweden is 0 vs 0.00001. This is the wrong direction, and the value is
so low that it is probably noise. There are two possible causes for
this essentially zero result: either neither region has this feature
or there is not sufficient data to tell.

\subsubsection{Noun possessives}

Delsing mentions briefly that central Sweden, including \"Alvdalen and
V\"asterdalarna, uses the dative form of nouns for the
s-genitive. However, the part-of-speech tag set used here does not
distinguish between dative and other cases on nouns, so it is not
possible to represent this phenomenon in a way that the distance
measures could have used.

\subsubsection{Double indefinite}

In northern Sweden and northern Norway, indefinite articles are used
both before and after adjectives when modifying nouns. In map
\ref{double-indefinite-map}, this is the area covered by dark
diagonals, labeled ``Postadjektivisk artikel''. Delsing also calls
this the ``double indefinite''; for an example, see figure
\ref{double-indefinite-example}. One indefinite article is used after
each adjective, even for multiple adjectives, so {\it en stor en bil}
(a large car) but also {\it en stor en fin en bil} (a large fine car).

\begin{figure}
 \gll en stor en bil \\
  a large a car \\
  \trans `A large car'
  \caption{Double Indefinite}
  \label{double-indefinite-example}
\end{figure}

\begin{figure}
  \includegraphics[scale=0.7]{dialektboka-karta8}
  \caption{Double indefinite (post-adjectival articles)}
  \label{double-indefinite-map}
\end{figure}

In central Sweden, a similar pattern occurs, but the article is not
perceived as independent. Instead it is perceived as a suffix of the
adjective. In other words, the above example is perceived as {\it en
  stor-en bil} instead. According to Delsing, there is a difference in
intonation compared to the North Swedish construction, which does not
stress the intermediate articles nor co-ordinate them morphologically
as would be expected with a suffix. Unfortunately, this pattern
appears identical to the ordinary Swedish case given the course
part-of-speech tag set in use.  In contrast, the first pattern is
quite easy to represent with trigrams: the 4-gram EN-AJ-EN-NN and the
6-gram EN-AJ-EN-AJ-EN-NN---alternating series of indefinite articles
and adjectives ended by a noun. These larger n-grams can be broken
into the trigrams EN-AJ-EN and AJ-EN-NN in order to search for them in
the Swedia-based data.

The northern pattern includes the interview sites Anundsj\"o and
Indal. When measured, these trigrams occur at a rate of 0.00054 there
versus the rest of Sweden, which has a rate of 0.00012. From this we
can conclude that this is a rare phenomenon, but one that happens in
the north about 4 times more often than in the rest of Sweden.

\subsubsection{Double Definite}

\begin{figure}
 \gll det store huset \\
  The large the-house \\
  \trans `The large house'
  \caption{Double definite (Sweden and Norway)}
  \label{double-definite-example}
\end{figure}
\begin{figure}
 \gll det store hus \\
  The large house\\
  \trans `The large house'
  \caption{Single Indefinite (Denmark)}
  \label{single-definite-example}
\end{figure}
\begin{figure}
  \gll gamla h\'usid \\
  old house-the \\
  \trans `The old house'
  \caption{Single definite suffix (Iceland)}
  \label{single-definite-suffix-example}
\end{figure}
\begin{figure}
 \gll storhuset \\
  large-house-the \\
  \trans `The large house'
  \caption{Single definite suffix with combined adjective (Northern Sweden)}
  \label{adjective-single-definite-suffix-example}
\end{figure}

\begin{figure}
  \includegraphics[scale=0.7]{dialektboka-karta9}
  \caption{Double definite (and combined adjectives)}
  \label{double-definite-map}
\end{figure}

Double-definite with adjectives is standard in Sweden and Norway,
where there is a definite article as well as a definite suffix on the
noun (see figure \ref{double-definite-example}). This is not the case
in Denmark (figure \ref{single-definite-example}), where the definite
suffix disappears in case of a definite article, nor in Iceland, where
the definite is suffix-only and there is no article (figure
\ref{single-definite-suffix-example}).

However, in North Sweden, there is a fourth option, where the
adjective combines with the noun into a single word (figure
\ref{adjective-single-definite-suffix-example}). Delsing gives
examples like {\it storhuset} (the big house) and {\it
  stor-svart-gamm-katta} (the big, black, old cat), in which a series
of adjectives appear prefixed to a noun without their usual
morphological inflection. In Norrland, Delsing finds that this
construction is used almost to the exclusion of the normal Swedish
one. Further south, the two co-exist.


Therefore, since the annotation scheme does not differentiate between
a combined noun like {\it storhuset} and a normal noun like {\it
  huset}, the better way to detect the region difference is to count
the rate of normal trigrams like PO-AJ-NN (pronoun-adjective-noun);
this is the feature type that occurs rarely or not at all in the
north. If the region division in map \ref{double-definite-map} is
detected, then northern Sweden will have a lower rate of occurrence of
these standard trigrams.

As before, the two northern sites are Indal and Anundsj\"o. The rate
of PO-AJ-NN in this region is 0.00152, compared to 0.00216 for the
rest of Sweden. This difference is in the right direction, and it is
larger than most of the other comparisons here. However, like the
other comparisons, it has not been checked for significance so it is
currently only suggestive.

\subsubsection{Rosenkvist's Analysis of the South Swedish Apparent Cleft}

\begin{figure}
 \gll Det \"ar som han har missuppfattat. \\
  it is {\it som} he has misunderstood \\
  \trans `He has misunderstood.'
  \caption{Apparent Cleft}
  \label{apparent-cleft-example1}
\end{figure}

\begin{figure}
 \gll Det \"ar bara som han finner p\aa{}. \\
  it is only {\it som} he finds-on \\
  \trans `He just makes it up.'
  \caption{Apparent Cleft with adverb expressing speaker attitude}
  \label{apparent-cleft-example2}
\end{figure}


\namecite{rosenkvist07} analyzes a phenomenon he calls the South
Swedish Apparent Cleft. It involves an embedded clause, similar to a
cleft, but with no clefted constituent. Instead, the subordinating
conjunction {\it som} is directly preceded either by the verb or an
adverb expressing speaker attitude. The subject of of the {\it
  som}-clause must be a pronoun, though Rosenkvist notes that this may
be a pragmatic, not a syntactic, restriction. The two main variants
are given in figures \ref{apparent-cleft-example1} and
\ref{apparent-cleft-example2}, but the apparent cleft is also found in
yes/no questions and embedded clauses.

Unfortunately, Rosenkvist does not give a comprehensive syntactic
analysis of the apparent cleft. This means that a translation to our
feature set based on his description will necessarily be
surface-oriented in the same way this his analysis and results are
surface-oriented.

Accordingly, translating the sequences like {\it Det \"ar som han
  \ldots} gives the 4-gram PO-AV-UK-PO, and {\it Det \"ar bara som han
  \ldots} gives the 5-gram PO-AV-AB-UK-PO (pronoun-be
verb-adverb-subordinating conjunction-pronoun). Although these
part-of-speech sequences can obviously appear in other contexts, they
should appear more in the region that has apparent clefts than in the
region that does not. Converting these sequences to trigrams is
straightforward, producing 5 unique trigrams of interest, which the
distances measures should also have used to obtain their distances.

Rosenkvist captures the geographical distribution of the apparent
cleft in two ways. He first consults two collections of Swedish
novels, using the authors' birthplaces as proxies for their
dialect. Second, he uses the results of a questionnaire that he issued
to university students at several Swedish universities: Stockholm,
Gothenburg, Lund and Ume\aa{}.

Using author birthplace as a proxy for dialect, the apparent cleft can
be seen throughout southern and middle Sweden---this includes all the
interview sites except \AA{}rsunda, Indal and Anundsj\"o. However, based
on the survey results, the apparent cleft is only accepted by speakers
from Halland, Sm\aa{}land and Sk\aa{}ne. This includes the interview sites
Frilles\aa{}s, V\aa{}xtorp, Ankarsrum, Tors\aa{}s, Bara, L\"oderup, Norra
Rorum and \"Ossj\"o.

Therefore, the test for this comparison is the occurrence rates
for the 5 trigrams based on the two common forms Rosenkvist gives as
examples, with two variations: one region division based on author
birthplaces and one region division based on the student survey. The
southern region in both cases should have more occurrences of the
target trigrams.

For the larger cleft region division based on author birthplaces, the
comparison goes in the expected direction: a rate of 0.02430 in the
south and 0.02427 in the north. But these rates are so close to identical
that they should not be regarded as different. For the smaller division
based on the student survey, the comparison goes in the opposite
direction: 0.02264 in the south and 0.02491 in the north. Again, this
is not much of a difference.

With such a small difference, it is not possible to draw any
conclusions or even suggest whether the distance measures consistently
notice this difference. One problem is that it hard to capture a
phenomenon like this with trigrams, where the surface form is only
subtly different from that produced by other syntactic structures. A
more complete syntactic analysis of the phenomenon is needed so that
more advanced feature sets from dialectometry can be used to compare
to the results from dialectology.

\subsection{Conclusion}

The dialect constructions surveyed here do not support the agreement
of the new dialectometry results with existing dialectology results
nearly as well as the previous sections which compared the results at
a less detailed level. The larger problem is that no good method yet
exists for doing so; the differences were in some cases large enough
to be suggestive, but without significance testing, it is not possible
to know that they are reliable. It is possible that the small
differences are significant, and already being used by the distance
measures to distinguish regions; after all, the aggregation of many
small differences is the inherent in the working of the statistical
approach in this dissertation.

\section{Comparison to Phonological Dialectometry}
\label{discussion-chapter-phonological-dialectometry}
\begin{figure}
  \includegraphics[scale=0.4]{leinonen-factors-1-2}
  \caption{Factors 1 and 2 of Swedish vowels}
  \label{leinonen-factors-1-2}
\end{figure}

\begin{figure}
  \includegraphics[scale=0.4]{leinonen-factors-3-4}
  \caption{Factors 3 and 4 of Swedish vowels}
  \label{leinonen-factors-3-4}
\end{figure}

\begin{figure}
  \includegraphics[scale=0.4]{leinonen-factors-5-6}
  \caption{Factors 5 and 6 of Swedish vowels}
  \label{leinonen-factors-5-6}
\end{figure}

The comparison to phonological dialectometry is currently difficult in
two ways. First, there are few statistical methods in phonological
dialectometry. I proposed a simple Bayesian method \cite{sanders06}
and \namecite{hinrichs07} proposed two more complex methods, one
vector-based and the other from information theory. However, these
methods are less effective on small corpora than Levenshtein distance
and have not gained traction in the field.  Second, even comparing
results only, there has been little Swedish dialectometry to date. To
my knowledge, the only paper at the time of this writing is
\namecite{leinonen08}; its method is more similar to
\quotecite{spruit08} approach to syntax. It uses factor analysis to
characterize the distribution of nine phonological variables across
Sweden, but does not cluster the sites based on these
variables. However, the overall regions can still be compared. I
compare Leinonen's individual feature maps to my composite cluster and
MDS maps.

In addition, Leinonen's dissertation, currently unpublished, will
cover phonological dialectometry of Sweden comprehensively. In future
work, a better comparison should be possible, since both dissertations
are based on the same corpus.

Looking at Leinonen's first two maps, reproduced here as figure
\ref{leinonen-factors-1-2}, we see patterns similar to the
city/countryside difference from the syntactic results: in the first diagram,
Stockholm and Uppsala differ from the rest of the country, and in the
second Stockholm, Uppsala and Malm\"o areas all differ.

In Leinonen's third and fourth maps (figure \ref{leinonen-factors-3-4}),
there is a north/south divide roughly half way between Stockholm and
Malm\"o. This boundary generally reflects the north/south
gradient from my results. However, the phonological boundary is stronger and more
localized than numerous small syntactic ones, such as those seen in
the composite cluster map \ref{map-composite-5-1000}. It is closer to
the diagonal north/south boundary mentioned by \namecite{hallberg05}.

The fifth map (figure \ref{leinonen-factors-5-6}) is more specific
than the previous four; most of the sites are blue, but there are a
few in the south that are much yellower than the rest. These are the
same three sites that form the red cluster in figure \ref{red-cluster}
from the consensus tree results in chapter\ref{results-chapter}:
J\"amshog, \"Ossj\"o and Tors\aa{}s. The sixth map, however, shows a
clear east/west divide that is not reflected in my data.

Although this region-to-region comparison is not precise, it provides
hope that a quantitative comparison between the two result sets will
support high agreement with statistical evidence. The level of
agreement between the phonological results and syntactic results is
quite high. Of the six variables Leinonen illustrates with the maps in
figures \ref{leinonen-factors-5-6} -- \ref{leinonen-factors-3-4}, all
but one reflect some aspect of the combined syntactic results. The
exact overlap between Leinonen's fifth variable and the red cluster
from the consensus tree results is surprising for statistical methods.

\section{Comparison to Syntactic Dialectometry}

In the progression from dialectology of Swedish to phonological
dialectometry of Swedish and finally to syntactic dialectometry, there
is less and less existing literature. To my knowledge, this
dissertation is the first treatment of syntactic dialectometry for
Swedish. Even outside Swedish, very little syntactic dialectometry
exists. Besides \quotecite{spruit08} dissertation, based on Goebl's
limited-data techniques, statistical work is limited to Nerbonne and
Wiersma's work on Finnish \cite{nerbonne06} and \cite{wiersma09}, and
my work on English \cite{sanders07} and \cite{sanders08b}.

This dissertation is the first to show that a statistical measure
designed for syntax can find distances between dialect regions. It
directly addresses the shortcomings of the previous work, which showed
that a statistical measure could detect significant differences, but failed to
produce dialect distances. It evaluates
parameter variations, establishing which combinations of feature set,
distance measure and corpus size produce valid and useful results,
taking into account a number of practical considerations, such as
amount of existing annotation.

This dissertation shows that fairly small sites, on the order of
6,000--10,000 words, can produce significant distances. This contrasts
with previous work; the significant distances between English sites
were for much larger sizes: the ICE data for London had over 200,000
words, and Scotland over 25,000. The conclusion should be that when
the sites consist of properly collected dialect speech, the size
required to detect distance drops considerably. The Swediasyn corpus
captures dialect speech in a way that the ICE does not; the Swediasyn
contains interviews in homes, while the majority of the ICE is
interviews of students and professors at University College London.

In addition, the syntactic results of this dissertation agree closely
with the phonological results of \namecite{leinonen08}. Although
agreement of syntax and phonology is not necessarily a prediction when
looking for dialect regions, it is not surprising---circumstantial
evidence that a new method is valid because it agrees with an existing
one. This contrasts strongly with the English work, which found no
significant correlation of syntactic distance with phonological
distance. It may be that using the same corpus for both Swedish
studies was the key difference; the two English corpora's ages differed by
almost 50 years.

This dissertation agrees more closely with dialectology than previous
work. Although the English study reproduced the north/south divide
well known in British dialectology, it did not produce any more
detailed regions. In contrast, this study reproduced all of the
Swedish dialect regions. With respect to individual phenomena,
however, the feature comparison was inconclusive; a few results were
positive, but most were very close to zero. There are two problems: the
corpora once again differ in age---most of Swedish dialectology dates
from around 1900 while the Swediasyn was collected in 2000---as well as a
lack of significance testing. The small feature differences found
may well be significant, since the nature of statistical
methods is to accumulate many small differences, but it is not
possible to tell without a test.

Significance testing for precise feature analysis is future work, but
this is not necessarily a problem. For phonological dialectometry,
which began with Kessler's paper on Irish \cite{kessler95}, extraction
of specific features did not begin until much later, two to three
years after Heeringa's dissertation on the subject \cite{heeringa04},
with such work as \quotecite{prokic07}.
% I published similar work but it sucked and was only ever a
% presentation at a medical conference
In any case, \namecite{wiersma09} mentions a method for features of
individual regions that could be adapted to comparisons between a pair
of regions.

\chapter{Conclusion}
\label{conclusion-chapter}

The previous chapter discussed the impact of this work with respect to
previous work in various fields. In particular, it provided a picture
of how it advanced syntactic dialectometry. This chapter briefly
covers avenues of future work to which this work leads. This future work falls
into two categories: syntactic dialectometry and Swedish
dialectology.

\section{Future Work}
% TODO: Cite Shannon in methods chapter
% TODO: To the future work section, add:
% 1. Extract deps from CFG parses from Berkeley
% 2. Label dep features with both arc and POS tags interleaved in the
% proper order.
% 3. Better tag set. (duh, probably already have this one)
% 4. Non-linear feature set combination.
% Some other stuff from end of results chapter? (I think is only on
% paper)
% TODO: Various kinds of tag backoff; for example, to bigrams or coarser node
% tags.

% -- Other stuff left to do in results chapter --- %%
% TODO: Remove R.app's captions in favour of mine.
% TODO: Remove R.app's x-scale (y-scale) too
% TODO: CITE this, I think it's a Pieter Klieweg paper
% TODO: Get a whole example sentence probably. Ugh. People want so
% much context!
% TODO: Also format these examples properly.
% TODO: Get the full sentence either from jones or flenser


Some avenues of future work are obvious; Swediasyn is part of the
larger Nodalida project to create a syntactic dialect corpus for all
Scandinavian languages. And Swediasyn is itself not a complete
transcription of Swedia; for example, it does not include any of
Swedish-speaking Finland yet \cite{johennessen09}. Unfortunately, this work depends on
others since I do not speak any Scandinavian language natively. Once
these corpora are complete, they will provide a more complete picture
of syntactic variation over the entire Scandinavian language area.

With regard to feature sets, it is interesting that trigrams perform better
than the more complicated feature sets. From a linguists' point of
view, this is disturbing: why should the flattest representation of
syntax perform the best? This performance difference also
discourages others from developing even more complicated and
linguistically interesting feature sets. The reason for trigrams'
performance is likely because of the amount of automatic annotation
that is a prerequisite for the complex features developed
here. Trigrams rely on an automatic part-of-speech tagger, while
leaf-ancestor paths rely on an automatic parser that uses automatic
part-of-speech tags from that same tagger.

To enable more complex feature sets, manual annotation is needed. But
this is labor intensive. Failing that, improved automatic annotation
is needed, although this still usually implies some manual annotation
in the form of a seed corpus for bootstrapping
\cite{blitzer07,mcdonald06}. Bootstrapping should help automatic
parsing of dialect interviews: not only does the subject matter of an
interview differ from the typical newspaper training corpus, the
syntactic features where the dialect differs from the standard
language are precisely those that are hardest to parse. Giving a
machine parser a sample of dialect speech as training would allow it
to identify some of these features. For example, in the case of the
possible double modals discussed at the end of chapter
\ref{results-chapter}, the part of speech tagger never saw the tokens
\textit{``skulla kunna''} juxtaposed in the training. If both words
were not part of a closed class, it is likely that the tagger would
not produce the correct tag for this pair. The same problem applies to
parser, but because syntactic training is even more sparse, the parser
is less likely to to have seen similar structures in
training. The parser is correspondingly less likely to produce a
double modal structure without having seen it in training.

Processing of features is another area for future work: normalization
is the first half of this problem. The current sentence-level
normalizations function well for aggregate comparisons like cluster
maps, but for individual feature comparison, the overuse normalization
tends to rank highly features that may just be noise from the
annotation error. On the other hand, without the overuse
normalization, only very common features are high ranked. This makes
it hard to notice the unique features of a dialect that do not occur
much. A compromise that takes frequency into account to some extent is
needed, so that rare features can be highly ranked without introducing
noise from annotation errors.

The other half of the feature-processing problem is a test for
significance when comparing two regions. This would make sure that
comparisons to the dialectology literature are significant in the
future. \namecite{wiersma09} provides a similar method for testing
significance of individual features in a single region, so it should
be easy to modify this to work for comparisons between two regions.

Finally, another obvious extension of this work is a quantitative
comparison of these results on Swedish to the results in Leinonen's
upcoming dissertation on Swedish phonological dialectometry. Given the
agreement between these results and her published work, it is likely
that the correlation will be high. This comparison should be fairly
easy since both results use the same dialect corpus as a basis.

\section{Conclusion}

This dissertation establishes that statistical methods are useful
direction for syntactic dialectometry. Its results show that
significant differences can be obtained with dialect corpora. This
much had been accomplished by previous work. However, this work goes
on to establish that even smaller interviews of dialect speakers are
sufficient to produce significant distances, and investigates
variations on both feature set and distance measure. It shows that a
syntactic measure can reproduce the traditional regions of
dialectometry, and that it can produce agreement with a phonological
measure. Its comparison to individual dialect phenomena is
inconclusive, but opens an avenue for future investigation, and more
importantly, future development of methods to compare and rank
individual features.

Future directions based on this work are twofold. First, with a
statistical method established for syntax, dialectometry can begin to
investigate the syntactic features of other languages. Second, in
Swedish, this work and future work similar to it can contribute to
dialectology in general; syntax has been relatively neglected in
Swedish dialectology. As Swediasyn and Nodalida are completed, the
automatic analysis detailed in this dissertation can provide a quick
analysis of new data, and point linguists toward interesting dialect
features.

In conclusion, this dissertation has answered the questions of
agreement with dialectometry and best parameter configuration for
practical measurements, as well as agreement with phonological
dialectometry. It has established statistical methods for syntactic
dialectometry, pointing the way for future syntactic dialect studies,
future expansion of statistical methods in dialectometry, and future
syntactic analysis of Swedish.

%%% Local Variables: 
%%% mode: latex
%%% TeX-master: "dissertation.tex"
%%% End: 


\backmatter

\begin{center}
\section*{Nathan C. Sanders}
\end{center}
% \begin{tabular}{c|c|c} % re-arrange and maybe put home page on there?
% Electronic Mail & Address & Telephone \\
% \hline
% ncsander@indiana.edu & 4780 E St. Rd. 45 & 812-606-9785 \\
% sanders\_n@yahoo.com & Bloomington, IN 47408  & \\
% \end{tabular}

\subsection*{QUALIFICATIONS}
\begin{description}
  \item[Dialect Classification/Dialectology] Extraction of
    linguistic, human-interpretable features
  \item[Machine learning] Statistical (e.g. unsupervised clustering) \\
    Symbolic (e.g. learning and learnability in Optimality theory)
  \item[Parsing] Adaptation of computer science parsers to learner
    natural language
  \item[Programming] Fluent in Python, Haskell, Java/C\#, and Scheme \\
    Experience in C++, Perl, Visual Basic, Javascript, F\#, and
    Common Lisp \\
    Experience in prototyping and web programming \\
    Three-time competitor, ACM International Collegiate Programming
    Contest
\end{description}
% See http://www.sandersn.com/research.html for papers
% and code in the above areas.
\subsection*{EDUCATION}
\begin{description}
\item [PhD student in Linguistics, minor Computer Science] Indiana
  University, 2010
\item [MA Computational Linguistics] Indiana University, 2006, GPA 3.98
\item[BA Computer Science] minors in French and Spanish,
		College of the Ozarks, summa cum laude May 2004
\end{description}
\subsection*{EXPERIENCE}
\begin{description}
\item[August 2009--November 2009]
  SDET Intern, Microsoft, Bing Search Infrastructure --- Wrote
  software to suggest tests to developers at checkin. Tested
  the distributed computation system forming the infrastructure of
  Bing.
\item[August 2005--May 2009]
  Research Assistant, IU School of Medicine --- Investigated linguistic
  aspects of cochlear implant user development. Developed novel distance
  measure and compared it to existing measures and human judgments.
\item[May 2005--August 2005]
  Application Developer, IU Archives of Traditional Music --- Wrote Java
  application to gather cataloging data for videos. Worked with librarians and
  ethnomusicologists to gather archival metadata requirements.
\item[August 2004--May 2005]
  Web Programmer, IU Overseas Studies --- Maintained database and its
  web interface. Developed new features on request and maintained
  office computers.
\item[Summer 2004, Summer 2003]
  Lead Programmer, Everyware Inc --- Implemented web framework
  for interactive site building. Rewrote existing plug-in
  architecture and extended existing interface.
\item[Summer 2002]
  Intern Programmer, SIL International --- Programmed import
  process for translation editor in C++, with a pre-processor in Python.
\item[Spring 2001--Spring 2004]
  Lab Assistant, College of the Ozarks Foreign Language Lab ---
  Tutored students in Spanish and French. Maintained lab
  computers and created tracking databases.
\end{description}
\subsection*{PUBLICATIONS}

% no idea why I have to indent *twice*
\indent\indent Sanders, N. C. 2009. Phonological distance measures. \emph{Journal
    of Quantitative Linguistics}, 43:96--114.

  Sanders, N. C. 2008. Cluster analysis of phonological distance
  measures of cochlear implant users. In \emph{Proceedings of the
    Tenth International Conference on Cochlear Implants and Other
    Implantable Auditory Technologies}, 113.

 Sanders, N. C. 2007. Measuring Syntactic Difference in British
  English. In \emph{Proceeding of the ACL 2007 Student Research
    Workshop}, 1--6, Prague, Czech Republic, June.

  Sanders, N. C. and Chin, S. B. 2006. Phonological distance measures for cochlear implant users. \emph{Wiener Medizinische Wochenschrift 156}, [Suppl 119] 8

 Sanders, N. C.  2004. Compiler error detection techniques applied to natural language processing. In \emph{Proceedings of the Thirty-fifth SIGCSE Technical Symposium on Computer Science Education}. Association for Computing Machinery, 515

\bibliographystyle{robbib}
\bibliography{central}
\end{document}
