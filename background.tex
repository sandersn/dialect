\chapter{Introduction}
% Its results show that significant
% distances can be obtained with dialect corpora. This much had been
% accomplished by previous work. However, this work goes on to establish
% that even smaller corpora are sufficient, and investigates variations
% on both feature set and distance measure. It shows that a syntactic
% measure can reproduce the traditional regions of dialectometry, and
% that a syntactic measure can produce substantial agreement with a
% phonological measure. Its comparison to individual dialect phenomena is
% inconclusive, but opens an avenue for future investigation, and more
% importantly, future development of methods to compare and rank
% individual features.

This dissertation establishes the utility and reliability of a
statistical method for syntactic dialectometry. It is a continuation of my previous
work \cite{sanders07}, \cite{sanders08b} and earlier work by
\namecite{nerbonne06}, the first statistical measure of syntax
distance. These pioneering studies explored this measure, but failed
to compare it to established results from syntactic dialectology. establish that it reproduces the results of syntactic dialectology. This
dissertation 

Dialectometry has existed as a field since
\namecite{seguy73} and is a sub-field of dialectology
\cite{chambers98}; recently, computational methods have come to
dominate dialectometry, but they are limited in focus compared to
previous work; most have explored phonological distance only, while
earlier methods integrated phonological, lexical, and syntactic data.

Dialectology is the study of linguistic variation. % in space / over
% distance / other variables.
Its goal is to characterize the linguistic features that
separate two language varieties. Dialectometry is a subfield of
dialectology that uses mathematically sophisticated methods to extract
and combine linguistic features. In recent years it has
been associated with computational linguistic work, most of which
has focused on phonology, starting with
\namecite{kessler95}, followed by \namecite{nerbonne97} and
\namecite{nerbonne01}. \namecite{heeringa04} provides a comprehensive
review of phonological distance in dialectometry as well as some new
methods.

In dialectometry, a distance measure can be defined in two parts:
first, a method of decomposing the linguistic data into minimal,
linguistically meaningful features, and second, a method of combining
the features in a mathematically and linguistically sound way. Figure
\ref{abstract-distance-measure-model} gives an overview of how the
model works. Input consists of two corpora; each item in each corpus
is decomposed into a set of features extracted by $f(s)$. The
resulting corpora are then compared by $d(S,T)$, which combines the
corpora into a single number: the distance.

\begin{figure}
\[\xymatrix@C=1pc{
 \textrm{Corpus} \ar@{>}[d]|{f(s)} &
  S = s_o,s_1,\ldots
  \ar@{>}[d] % \ar@<2ex>[d] \ar@<-2ex>[d]
  &&
  T = t_o,t_1,\ldots
  \ar@{>}[d] % \ar@<2ex>[d] \ar@<-2ex>[d]
  \\
 *\txt{Decomposition} \ar@{>}[d]|{d(S,T)} &
 *{\begin{array}{c}
     \left[ + f_o, +f_1 \ldots \right], \\
     \left[ - f_o, +f_1 \ldots \right], \\
     \ldots \\ \end{array}}
 \ar@{>}[dr]
 &&
 *{\begin{array}{c}
     \left[ + f_o, -f_1 \ldots \right], \\
     \left[ + f_o, -f_1 \ldots \right], \\
     \ldots \\ \end{array}}
 \ar@{>}[dl]  \\
 \textrm{Combination} &
 & \textrm{Distance} & \\
} \]
\label{abstract-distance-measure-model}
\caption{Abstract Distance Measure Model : $f \circ d$}
\end{figure}

Dialectometry has focused on phonological distance measures, while
syntactic measures have remained undeveloped. The most important
reason for this focus is that it is easier to define a distance
measure on phonology. In phonology, it is easy to collect corpora
consisting of identical word sets. Then these words decompose to segments and,
if necessary, segments further decompose to phonological
features. This decomposition is straightforward and based on
\namecite{chomsky68}. For combination, string alignment, or Levenshtein
distance \cite{lev65}, is a well-understood algorithm used for
measuring changes between any two sequences of characters taken from a
common alphabet. Levenshtein distance is simple mathematically, and
has the additional advantage that its intermediate data structures are
easy to interpret as the linguistic processes of epenthesis, deletion and
metathesis. These things are not possible
with syntactic distance: neither matched sentence collections nor
straightforward functions for decomposition and combinations.

A secondary reason for dialectometry's focus on phonology is that it
is inherited from dialectology's focus on phonology.
% (TODO:Cite?)
This might be solely due to the history of dialectology as a field, but it is
likely that more phonological than syntactic differences exist between
dialects, due to historically greater standardization
of syntax via the written form of language. Phonological
dialect features are less likely to be stigmatized and suppressed by a
standard dialect than syntactic ones.
% (TODO:Cite, probably
% Trudgill and Chambers something like '98, maybe where they talk about
% what aspects of dialects are noticed and stigmatized).
Whatever the reason, much less dialectology work on syntax is
available for comparison with new dialectometry results.

\subsection{Problems}
% TODO:Or "Gaps in the field" or "Places to improve upon" or something

Because of the preceding two reasons, syntax is a relatively
undeveloped area in dialectometry. Currently, the literature lacks a
generally accepted syntax measure. Unfortunately, approaching the
problem by copying phonology is not a good solution; there are real
differences between syntax and phonology that mean phonological
approaches do not apply. For example, there are fewer differences to be
found in syntax, and they occur more sparsely.
% (TODO: Back this up either with reasoning or citation).
However, dialectology has traditionally worked with fairly small
corpora. This suffices for phonology, because
it is easy to extract good features and there are many
consistent differences between corpora. For syntax, though, it is not possible
% (TODO: Weasel a bit)
to identify reliable features in small corpora.

There are two approaches that have been proposed to remedy this. The
first, proposed by \namecite{spruit08} for analyzing the Syntactic
Atlas of the Dutch Dialects \cite{barbiers05}, is to continue using
small dialectology corpora and manually extract features so that only
the most salient features are used. Then a sophisticated method of
combination such as Goebl's Weighted Identity Value (WIV), described
below and by \namecite{goebl06}, can be used to produce a
distance. WIV is more complex mathematically than Levenshtein
distance, and operates on any type of linguistic feature. However, manual feature
extraction is not feasible in knowledge-poor or time-constrained
environments. It is also subject to bias from the
dialectologist. Since the best manually extracted features are those that capture
the difference between two dialects, the best-known features are most
likely to become the best manual features, passing over the rarely
occurring and obscure features that might actually be the best
indicators of a particular dialect.

This approach ignores the specific properties of the syntax distance
problem. It is easy to define features for syntactic structure. This
proposal covers part-of-speech trigrams, leaf-ancestor paths, and
dependency paths over nodes, but many variations on these features are
possible, such as lexical trigrams, lexicalized leaf-ancestor paths,
or dependency paths over dependency arc labels. Methods from other
syntactic work in computational linguistics could apply too: supertags
\cite{joshi94}, convolution kernels \cite{collins01} or any number of
simpler features such as tree height, number of nodes, or number of
words. The problem is not finding a feature set. The problem is
finding a good feature set. Small corpora hamper this search by making
statistical significance difficult to achieve, especially since
syntactic dialect differences are expected to be less frequent than
phonological ones. Fortunately, syntactic corpora are typically larger
than phonological corpora because the annotation work is easier; much
of the syntactic annotation can be generated automatically and then
corrected manually.

Even with a feature set defined, a distance measure still requires a
method of combining features. One such method, a simple statistical
measure called $R$, has been proposed by \namecite{nerbonne06} based
on work by \namecite{kessler01}. At present, however, $R$ has not been
adequately shown to detect dialect differences. A small body of work
suggests that it does, but as yet there has not been a satisfying
correlation of its results with phonology or, as with phonological
distance, with existing results from the dialectology literature on
syntax.

Nerbonne \& Wiersma's first paper used part-of-speech trigram features
as a proxy for syntactic information and $R$ for syntax distance
together with a test for statistical significance\cite{nerbonne06}.
Their experiment compared two generations of Norwegian L2 speakers of
English.  They found that the two generations were significantly
different, although they had to normalize the trigram counts to
account for differences in sentence length and complexity. However,
showing that two generations of speakers are significantly different
with respect to $R$ does not necessarily imply that the same will be
true for other types of language varieties. Specifically, for this
dissertation, the success of $R$ on generational differences does not
imply success on dialect differences.

I addressed this problem \cite{sanders08b} by measuring $R$ between
the nine Government Office Regions of England, using the International
Corpus of English Great Britain \cite{nelson02}. Speakers were classified by
birthplace. I also introduced Sampson's leaf-ancestor paths as
a feature set \cite{sampson00}. I found statistically
significant differences between most corpora, using both trigrams and
leaf-ancestor paths as features. However, $R$'s distances were not
significantly correlated with Levenshtein distances. Nor did I
show any qualitative similarities between known syntactic dialect
features and the high-ranked features used by $R$ in producing its
distance. As a result, it is not clear whether the significant $R$ distances
correlate with dialectometric phonological distance or with known
features found by dialectologists.

% NOTE: 2-d stuff is not the primary problem, since we can't compare
% trees to trees anyway. The primary problem is comparing two corpora
% full of differing sentences. A secondary problem arises to make sure
% that the 2-d-extracted features aren't skewed one way or another. I
% guess I need to come up with a general justification for the
% normalizing and smoothing code from Nerbonne & Wiersma

% Additional problems: phonology is 1-dimensional, with one obvious way
% to decompose words into segments and segments into features. Syntax is
% 2-dimensional, so the decomposition must take several more factors
% into account so that the features it produces are
% useful and comparable to each other. And those features are \ldots

% Overview : Goal, Variables, Method
%   Contribution
% Literature Review
%   : (including theoretical background)
%   Draw hypotheses from earlier studies
% Method
%   :
%   Experiment section as 'Corpus' section

% Goal: To extend existing measurement methods. To measure them
% better. To measure them on more complete data.

\section{Overview of the Dissertation}

The problem outlined in the previous section is that dialectometry
lacks a statistical method for syntax, one that is tailored for syntax
but at the same time does not require the linguist to specify ad-hoc
features manually. There has been early work providing such a method,
but so far no comprehensive treatment to show that it is a reliable
extension (into syntax)(?). This dissertation addresses the lack
directly by applying the method to a dialect corpus, then comparing
the results to existing syntactic dialectology literature of Swedish,
as well as phonological work using established dialectometry
methods. In addition, it tests variations of the experimental
parameters in order to identify the highest-performing parameters. In
summary, this analysis allows future dialectometry studies to include
syntactic as well as phonological analyses, having an idea of the best
method and parameters to use.

There are three research questions that must be answered to determine
the reliability of this measure. They are given in chapter
\ref{questions-chapter}. First, does the measure agree with the
results of dialectology? Previous work has not addressed this
question, but it is crucial that a new measure reproduce the results
from previous linguistic work. To answer this question, the results
Swedish dialect distance results will be processed in a number of ways
so that they are comparable to previous dialect work on Swedish in
multiple ways.

Second, which parameter variations produce the best agreement with
dialectology work? Both the distance measure and feature set can be
varied, as well as a number of other parameter settings, mostly
dealing with controlling for the effects of corpus size. The distance
measures include simple measures like $R$, which is a sum of differences, more
complex variants such as Jensen-Shannon divergence, which is a sum of
logarithmic differences, and cosine similarity, which models each
corpus as a vector in high-dimensional space and finds the angle
between two corpus vectors. Feature sets can be even more varied,
although all the feature sets discussed here assume that the word is the
basic unit of syntactic analysis and that words are naturally grouped
into sentences. Some example feature sets are part-of-speech trigrams,
which are simply triples of parts of speech. Leaf-ancestor paths and
leaf-head paths use the syntactic structure of the sentence, with
leaf-ancestor paths based on constituent grammars (phrase-structure
grammars) and leaf-head paths based on dependency grammars.

Third, does the measure agree with the results of phonological
dialectometry? Agreement is not required; phonological and syntactic
dialect boundaries may disagree, but they are more likely to agree
than disagree, so if the two dialectometric measurements agree, then
this inspires confidence on the new method based on the old method's
reliability.

To answer the three research questions, I start with the statistical
method described in the previous section with the parameter variations
described above in chapter \ref{methods-chapter}. To make sure that
the results are comparable to previous dialectology, I use the dialect
corpus Swediasyn, which is a transcription of interviews recorded
throughout Sweden in various villages. The interviewees were balanced
between older and younger men and women. To generate features from the
Swediasyn, a good deal of processing is required; the corpus is a
transcription with no syntactic annotation. To annotate the Swediasyn,
I use a number of automatic annotators, trained on Talbanken, a corpus
of spoken and written Swedish. However, Talbanken does not include
dialect sources, so some amount of error is expected during the
annotation process. After annotation, feature generation is
straightforward: transformation of parse trees and other annotations.

After measuring distances between the interview sites, a number of
analytic methods are applied to the distances so that they can be
compared to dialectology work. The methods are a test of significance,
a test of correlation, cluster dendrograms and consensus trees,
composite cluster maps, multi-dimensional scaling, and feature
ranking. The tests of significance and correlation represent the
distances' trustworthiness and ability to match dialectology's
assumptions, respectively. The consensus trees, and multi-dimensional
scaling both produce maps. These maps allow the linguist to visually
compare the results with traditional region maps. In the same way,
composite cluster maps allow visual comparison of the results to
isogloss bundles from dialectology. Finally, feature ranking allows
the linguist to view the features that contribute most to separating
two regions. These features can be compared to the dialect phenomena cataloged
by dialectologists.
% Note: There are no isogloss bundles in Swedish dialectology. But
% whatever. (Also I don't really use feature ranking to compare to
% dialectology)

The results in chapter \ref{results-chapter} are presented in the same
order as their corresponding analysis appear in \ref{methods-chapter}.
The dissertation concludes with discussion in chapter
\ref{discussion-chapter}. Here, I compare the results to the dialectology
and phonological dialectometry of Swedish. Then I discuss the relation
of this work to previous work in syntactic dialectology, detailing its
contribution to the field. I finish by presenting avenues for future
work: with a statistical measure of dialect distance, dialectometry
can analyze syntactic features as well as phonological and lexical
ones, producing more complete analyses.

%%% Local Variables: 
%%% mode: latex
%%% TeX-master: "dissertation.tex"
%%% End: 
