\chapter{Questions}
% TODO: Rewrite and merge the following question/hypothesis paragraph pairs
% H1 - organization is all wrong still
The state of syntax measures in dialectometry described above leaves
several research questions unresolved. The most important for this
proposal is whether $R$ is a good measure of syntax
distance. Specifically, have the ambiguous results of previous
research been a shortcoming of $R$, differences between phonological
and syntactic corpora, or differences between phonological and
syntactic dialect boundaries?

To investigate this, I propose Hypothesis 1: the features found by
dialectologists will agree with the highly ranked features used by $R$
for classification. I will test Hypothesis 1 by comparing $R$'s
results to the syntactic dialectology literature on Swedish. In
addition, Hypothesis 1B states that the regions of Sweden accepted by
dialectology will be reproduced by the $R$. For example, my
previous research on British English reproduced the well-known North
England-South England dialect regions. However, this research will eliminate the
corpus variability in that research \cite{sanders08b} that resulted in
the confounding factors mentioned above, meaning that more precise
results, such as specific identifying features, should be detectable as well.

%H2 - Dad didn't understand that this is other features to be fed into
%R not replacement of R entire.
A secondary question, relevant once a useful syntax distance measure
is established, is what input features cause $R$ to produce the best
results.  Previous work has shown that leaf-ancestor paths provide a
small advantage over part-of-speech trigrams, presumably by capturing
syntactic structure higher in the parse tree. Additional possible
feature sets include variations on the previously investigated
trigrams and leaf-ancestor paths, along with various kinds of backoff,
for example, to bigrams or coarser node tags. Features from dependency
parses may be useful, too, in capturing non-local dependencies that
can be captured neither by trigrams nor leaf-ancestor paths.

Therefore, I propose Hypothesis 2: better input features
for $R$ will produce more accurate syntax
distances. These features can be discovered by comparing performance
of a number of different feature sets on a fixed corpus. In addition,
combinations of successful features will produce even better
performance.
% This sentence is either redundant or should appear earlier.
The quality of a set of features can be
measured by its sensitivity---the number of significant distances it
finds---and the similarity of the highly ranked features $R$ produces
to those found by dialectologists.

% H3
A third question is whether $R$ agrees with phonological distance
measures like Levenshtein distance. Unlike agreement with traditional
dialectology, there is no {\it a priori} reason to expect agreement
between phonology and syntax in delineating dialect
boundaries. However, agreement with phonological distance would be
further evidence for the suitability of $R$ as a syntactic distance
measure.

Therefore, I propose Hypothesis 3: a phonology corpus and syntax corpus
constructed from the same data will provide better correlation between
phonology and syntax distance measures than a phonology corpus and
syntax corpus drawn from different data sources. I will test this
hypothesis by comparing results to my previous work on British English
phonological and syntactic corpora; there, no significant correlation
was found between the regions extracted from the two corpora drawn
from different data. If significant correlation is found by using the
same set of data for both corpora, it indicates that phonology and
syntax boundaries do coincide but that the agreement is weak enough to
be lost when using corpora collected from different populations.

Finally, if $R$ is found to be a bad measure of syntax distance, this
dissertation will propose and evaluate alternative syntactic distance
measures. Specifically, $R$ is one way to aggregate features that are
created by decomposing sentences. It treats features as atomic, and
does not manipulate them in any syntax-specific ways. As such, $R$ is
not much different than Goebl's WIV. This may not be a problem if the
decomposition methods used to generate features adequately capture
dialect differences in independent, atomic features. If dialect
differences cannot be captured by independent, atomic features, then a
more syntax-specific method of combining features will be needed
instead. Alternatively, a more complex statistical measure may be
useful, taking the basic idea of $R$ and increasing its
sensitivity. For example, Kullbeck-Leibler divergence, like $R$,
provides a dissimilarity that is intuitively similar to distance.

%%% Local Variables: 
%%% mode: latex
%%% TeX-master: "dissertation.tex"
%%% End: 
