\documentclass[11pt]{article}
\begin{document}
\begin{center}
Announcing the\\
Final Examination of \\
Nathan Sanders \\
for the \\
Degree of Doctor of Philosophy in Linguistics \\
June 11th, 2010, 10 AM (ha) \\
Room 317, Memorial Hall
\end{center}

Dissertation: A Statistical Measure for Syntactic Dialectometry (of Swedish) \\

This dissertation investigates a statistical measure for syntactic
dialectometry, using grammatical features of Swedish dialects to
measure how much they differ. It analyzes variant feature sets and
measures to determine which provide the most accurate results. The
results are compared to existing Swedish syntactic dialectology and
phonological dialectometry.

The statistical measure requires two functions: the first extracts
syntactic features from an interview corpus, and the second compares
the features from two such corpora. It does this by comparing the
number of each feature type between the two corpora; the sum of
differences is the total distance. Several variants of both functions
are defined: the feature extraction functions vary in complexity from
surface-oriented parts of speech up to subtrees of the syntax
tree. The comparison functions also vary in complexity; the simplest
is described above, while the more complex are based on information
theoretic measures.

The distances are compared to previous work in three ways: agreement
with (1) overall Swedish dialect geography, (2) specific dialect
regions and (3) specific dialect phenomena. For traditional
dialectology, the overall geography agrees moderately well, the
specific regions are very similar, and agreement with specific
phenomena is inconclusive. For phonological dialectometry, the
agreement is very good in all three areas.

This dissertation shows that a statistical measure of syntactic
dialect distance is a useful addition to dialectometry, extending the
field beyond phonology and the lexicon. It shows that the measure's
results on Swedish agree with those of related fields. And it provides
a path for future work by finding the best variations of the
measure, given various practical conditions.

\begin{minipage}[b]{0.5\linewidth}\centering
{\it Outline of Current Studies} \\

Major: Linguistics \\
Minor: Computer Science
\end{minipage}
\begin{minipage}[b]{0.5\linewidth}\centering
{\it Educational Career}

Ph D, Indiana University, 2010
MA, Indiana University 2006
BA, College of the Ozarks, 2004
\end{minipage}

\begin{center}
Committee in Charge \\
Assistant Professor Sandra K\"ubler, Chairperson \\
(812-855-3268), Linguistics \\
Assistant Professor Markus Dickinson \\
(812-856-2535), Linguistics \\
Associate Professor Michael Gasser \\
(812-855-7078), Computer Science \\
Professor Steven Franks \\
(812-855-8169), Linguistics and Slavic Languages and Literatures
\end{center}
\begin{center}
Approved: \hrulefill \\
Sandra K\"ubler
\end{center}

(Any member of the Graduate Faculty may attend. As a courtesy, please
notify the Committee Chairperson in advance.)

\end{document}

%%% Local Variables: 
%%% mode: latex
%%% TeX-master: 
%%% End: 
